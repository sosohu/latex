\documentclass{beamer}
\usetheme{Frankfurt}

%%%
\usepackage{color}
\usepackage{xcolor}
\definecolor{keywordcolor}{rgb}{0.8,0.1,0.5}
\usepackage{listings}
\lstset{breaklines}%这条命令可以让LaTeX自动将长的代码行换行排版
\lstset{extendedchars=false}%这一条命令可以解决代码跨页时,章节标题,页眉等汉字不显示的问题
\lstset{language=C++, %用于设置语言为C++
	keywordstyle=\color{keywordcolor} \bfseries, %设置关键词
	identifierstyle=,
	basicstyle=\ttfamily, 
	commentstyle=\color{blue} \textit,
	stringstyle=\ttfamily, 
	showstringspaces=false,
	%frame=shadowbox, %边框
	captionpos=b
}
%%%
\usepackage{CJK}
\hypersetup{CJKbookmarks=true} %解决section不能使用中文的问题

%%树
\usepackage{graphics,graphicx}
\usepackage{pstricks,pst-node,pst-tree}
\usepackage{caption}
\setbeamertemplate{caption}[numbered]
\def\dedge{\ncline[linestyle=dashed]}
%%

\begin{document} %申明文档的开始
\begin{CJK}{UTF8}{gbsn}     %CJK:支持中文

	\title{红黑树}
	\author{胡庆海}
	\date{}

    \begin{frame} %beamer里重要的概念,每个frame定义一张page
        \titlepage    % maketitle
    \end{frame}


    \begin{frame}
        \frametitle{主要内容}
        \tableofcontents
    \end{frame}

	\section{红黑树的定义}
	\begin{frame}
        \frametitle{红黑树的定义}
		%\framesubtitle{五个限定条件}  
        \begin{definition}
            红黑树:\ \ \ 满足以下五条性质的二叉搜索树称为红黑树 \
			\begin{itemize}
				\item[1]  任意节点的颜色不是红色就是黑色
				\item[2]  根节点颜色必须是黑色
				\item[3]  叶子节点(NIL)颜色必须是黑色
				\item[4]  红色节点的儿子节点必须是黑色
				\item[5]  对于每个节点,从该节点开始到其后代的任何叶子节点路径黑长度相同
			\end{itemize}
        \end{definition}
	\end{frame}

	\begin{frame}
        \frametitle{红黑树的定义}
		\framesubtitle{举例}  

		如下图,树B是红黑树,而树A不是,因为违反规则4和规则5
		\pause

		\begin{columns} %排版
		\begin{column}{0.5\textwidth}
		\centering
		\begin{figure}
		\pstree[levelsep=25pt,treesep=6pt]{\Tcircle[fillcolor=lightgray,fillstyle=solid]{1}}
		{% levelsep指不同层之间距离, treesep兄弟之间距离
		 \pstree{\Tcircle[fillcolor=lightgray,fillstyle=solid]{2}}{
		 	\pstree{\Tcircle[fillcolor=lightgray,fillstyle=solid]{4}}{
				\Tcircle[fillcolor=lightgray,fillstyle=solid]{}
				\Tcircle[fillcolor=lightgray,fillstyle=solid]{}
			}
		 	\pstree{\Tcircle[fillcolor=lightgray,fillstyle=solid]{5}}{
				\Tcircle[fillcolor=lightgray,fillstyle=solid]{}
				\Tcircle[fillcolor=lightgray,fillstyle=solid]{}
			}
		 }
		 \pstree{\Tcircle[fillcolor=red,fillstyle=solid]{3}}{
		 	\pstree{\Tcircle[fillcolor=lightgray,fillstyle=solid]{6}}{
		 		\pstree{\Tcircle[fillcolor=red,fillstyle=solid]{8}}{
					\Tcircle[fillcolor=lightgray,fillstyle=solid]{}
					\Tcircle[fillcolor=lightgray,fillstyle=solid]{}
				}
				%\textmd{\textbf{\ \ \ z}}
		 		\pstree{\Tcircle[fillcolor=red,fillstyle=solid]{9}}{
					\Tcircle[fillcolor=lightgray,fillstyle=solid]{}
					\Tcircle[fillcolor=lightgray,fillstyle=solid]{}
				}
			 }
		 	\pstree{\Tcircle[fillcolor=lightgray,fillstyle=solid]{7}}{
				\Tcircle[fillcolor=lightgray,fillstyle=solid]{}
				\Tcircle[fillcolor=lightgray,fillstyle=solid]{}
			}
		 }
		}
		\caption{树A}
		\end{figure}
		\end{column}

		\pause

		\begin{column}{0.5\textwidth}
		\centering
		\begin{figure}
		\pstree[levelsep=25pt,treesep=6pt]{\Tcircle[fillcolor=lightgray,fillstyle=solid]{1}}
		{% levelsep指不同层之间距离, treesep兄弟之间距离
		 \pstree{\Tcircle[fillcolor=red,fillstyle=solid]{2}\tlput{red}}{
		 	\pstree{\Tcircle[fillcolor=lightgray,fillstyle=solid]{4}}{
				\Tcircle[fillcolor=lightgray,fillstyle=solid]{}
				\Tcircle[fillcolor=lightgray,fillstyle=solid]{}
			}
		 	\pstree{\Tcircle[fillcolor=lightgray,fillstyle=solid]{5}}{
				\Tcircle[fillcolor=lightgray,fillstyle=solid]{}
				\Tcircle[fillcolor=lightgray,fillstyle=solid]{}
			}
		 }
		 \pstree{\Tcircle[fillcolor=red,fillstyle=solid]{3}}{
		 	\pstree{\Tcircle[fillcolor=lightgray,fillstyle=solid]{6}}{
		 		\pstree{\Tcircle[fillcolor=red,fillstyle=solid]{8}}{
					\Tcircle[fillcolor=lightgray,fillstyle=solid]{}
					\Tcircle[fillcolor=lightgray,fillstyle=solid]{}
				}
				%\textmd{\textbf{\ \ \ z}}
		 		\pstree{\Tcircle[fillcolor=red,fillstyle=solid]{9}}{
					\Tcircle[fillcolor=lightgray,fillstyle=solid]{}
					\Tcircle[fillcolor=lightgray,fillstyle=solid]{}
				}
			 }
		 	\pstree{\Tcircle[fillcolor=lightgray,fillstyle=solid]{7}}{
				\Tcircle[fillcolor=lightgray,fillstyle=solid]{}
				\Tcircle[fillcolor=lightgray,fillstyle=solid]{}
			}
		 }
		}
		\caption{树B}
		\end{figure}
		\end{column}

		\end{columns}

		\footnote{灰色节点代表节点颜色为黑,红色节点代表颜色为红,小尺寸节点代表黑色叶子节点,白色节点代表颜色有多重,绿色节点代表一个子树}
	\end{frame}

	\section{红黑树性质}
    \begin{frame}
        \frametitle{红黑树性质}\pause
        \begin{itemize}
         \item[1] 任意一个节点其左右两个子树的深度之差不会超过一倍 \pause
		 \item[2] 红黑树是平衡树,对于有n个节点的二叉树其高度为$lg_n$
        \end{itemize}
    \end{frame}

	\section{红黑树的基本操作}

	\subsection{红黑树的查找}
    \begin{frame}
		\frametitle{红黑树的查找}
		红黑树本身是属于二叉搜索树,所以其搜索方法与一半BST搜索方法一致
    \end{frame}

	\subsection{红黑树的插入}
    \begin{frame}
		\frametitle{红黑树的插入}
		对于红黑树的插入,我们首先使用BST的插入算法找到插入点并插入,然后设置其颜色为红色,
	再根据其具体情况进行修改,主要情况有以下四种
		\begin{itemize}
			\item[1] 插入节点的父节点是黑色
			\item[2] 插入节点的父节点是红色,叔叔节点也是红色
			\item[3] 插入节点的父节点是红色,叔叔节点也是黑色,插入节点是父节点的右子
			\item[4] 插入节点的父节点是红色,叔叔节点也是红色,插入节点是父节点的左子
		\end{itemize}
    \end{frame}

	\begin{frame}
		\frametitle{插入情况1}
		由于此时没有破坏红黑树的任何性质,所以算法结束

		\begin{columns} %排版
		\begin{column}{0.5\textwidth}
		\centering
		\begin{figure}
		\pstree[levelsep=25pt,treesep=6pt]{\Tcircle[fillcolor=lightgray,fillstyle=solid]{1}}
		{% levelsep指不同层之间距离, treesep兄弟之间距离
		 	\pstree{\Tcircle[fillcolor=lightgray,fillstyle=solid]{2}}{
				\Tcircle[fillcolor=lightgray,fillstyle=solid]{}
				\Tcircle[fillcolor=lightgray,fillstyle=solid]{}
			}
		 	\pstree{\Tcircle[fillcolor=red,fillstyle=solid]{3}}{
				\Tcircle[fillcolor=lightgray,fillstyle=solid]{}
				\Tcircle[fillcolor=lightgray,fillstyle=solid]{}
			}
		}
		%\ncline[linestyle=dashed]{top}{bot}
		\caption{插入节点3}
		\end{figure}
		\end{column}

		\pause

		\begin{column}{0.5\textwidth}
		\centering
		\begin{figure}
		\pstree[levelsep=25pt,treesep=6pt]{\Tcircle[fillcolor=lightgray,fillstyle=solid]{1}}
		{% levelsep指不同层之间距离, treesep兄弟之间距离
		 	\pstree{\Tcircle[fillcolor=lightgray,fillstyle=solid]{2}}{
				\Tcircle[fillcolor=lightgray,fillstyle=solid]{}
				\Tcircle[fillcolor=lightgray,fillstyle=solid]{}
			}
		 	\pstree{\Tcircle[fillcolor=red,fillstyle=solid]{3}}{
				\Tcircle[fillcolor=lightgray,fillstyle=solid]{}
				\Tcircle[fillcolor=lightgray,fillstyle=solid]{}
			}
		}
		\caption{调整之后}
		\end{figure}
		\end{column}

		\end{columns}
	\end{frame}

	\begin{frame}
		\frametitle{插入情况2}
		父节点和叔叔节点都是红节点,那么爷爷节点肯定是黑节点,这样只需要把父节点和叔叔节点染成
	黑色,爷爷节点染成红色,问题就变成调整爷爷节点了

		\begin{columns} %排版
		\begin{column}{0.5\textwidth}
		\centering
		\begin{figure}
		\pstree[levelsep=25pt,treesep=6pt]{\Tcircle[fillcolor=lightgray,fillstyle=solid]{1}}
		{% levelsep指不同层之间距离, treesep兄弟之间距离
		 \pstree{\Tcircle[fillcolor=red,fillstyle=solid]{2}}{
		 	\pstree{\Tcircle[fillcolor=lightgray,fillstyle=solid]{4}}{
				\Tcircle[fillcolor=green,fillstyle=solid]{A}
				\Tcircle[fillcolor=green,fillstyle=solid]{B}
			}
		 	\pstree{\Tcircle[fillcolor=red,fillstyle=solid]{5}}{
				\Tcircle[fillcolor=green,fillstyle=solid]{C}
				\Tcircle[fillcolor=green,fillstyle=solid]{D}
			}
		 }
		 \pstree{\Tcircle[fillcolor=red,fillstyle=solid]{3}}{
			\Tcircle[fillcolor=green,fillstyle=solid]{E}
			\Tcircle[fillcolor=green,fillstyle=solid]{F}
		 }
		}
		\caption{插入节点5}
		\end{figure}
		\end{column}

		\pause

		\begin{column}{0.5\textwidth}
		\centering
		\begin{figure}
		\pstree[levelsep=25pt,treesep=6pt]{\Tcircle[fillcolor=red,fillstyle=solid]{1}}
		{% levelsep指不同层之间距离, treesep兄弟之间距离
		 \pstree{\Tcircle[fillcolor=lightgray,fillstyle=solid]{2}}{
		 	\pstree{\Tcircle[fillcolor=lightgray,fillstyle=solid]{4}}{
				\Tcircle[fillcolor=green,fillstyle=solid]{A}
				\Tcircle[fillcolor=green,fillstyle=solid]{B}
			}
		 	\pstree{\Tcircle[fillcolor=red,fillstyle=solid]{5}}{
				\Tcircle[fillcolor=green,fillstyle=solid]{C}
				\Tcircle[fillcolor=green,fillstyle=solid]{D}
			}
		 }
		 \pstree{\Tcircle[fillcolor=lightgray,fillstyle=solid]{3}}{
			\Tcircle[fillcolor=green,fillstyle=solid]{E}
			\Tcircle[fillcolor=green,fillstyle=solid]{F}
		 }
		}
		\caption{调整之后,继续调整1}
		\end{figure}
		\end{column}

		\end{columns}
	\end{frame}

	\begin{frame}
		\frametitle{插入情况3}
		父节点是红节点,叔叔节点是黑节点,且插入节点是父节点的右子,那么可以以父节点为旋转点做左旋,转化为情况4

		\begin{columns} %排版
		\begin{column}{0.5\textwidth}
		\centering
		\begin{figure}
		\pstree[levelsep=25pt,treesep=6pt]{\Tcircle[fillcolor=lightgray,fillstyle=solid]{1}}
		{% levelsep指不同层之间距离, treesep兄弟之间距离
		 \pstree{\Tcircle[fillcolor=red,fillstyle=solid]{2}}{
		 	\pstree{\Tcircle[fillcolor=lightgray,fillstyle=solid]{4}}{
				\Tcircle[fillcolor=green,fillstyle=solid]{A}
				\Tcircle[fillcolor=green,fillstyle=solid]{B}
			}
		 	\pstree{\Tcircle[fillcolor=red,fillstyle=solid]{5}}{
		 		\pstree{\Tcircle[fillcolor=lightgray,fillstyle=solid]{7}}{
				\Tcircle[fillcolor=green,fillstyle=solid]{C}
				\Tcircle[fillcolor=green,fillstyle=solid]{D}
				}
		 		\pstree{\Tcircle[fillcolor=lightgray,fillstyle=solid]{8}}{
				\Tcircle[fillcolor=green,fillstyle=solid]{E}
				\Tcircle[fillcolor=green,fillstyle=solid]{F}
				}
			}
		 }
		 \pstree{\Tcircle[fillcolor=lightgray,fillstyle=solid]{3}}{
			\Tcircle[fillcolor=green,fillstyle=solid]{G}
			\Tcircle[fillcolor=green,fillstyle=solid]{H}
		 }
		}
		\caption{插入节点5}
		\end{figure}
		\end{column}

		\pause

		\begin{column}{0.5\textwidth}
		\centering
		\begin{figure}
		\pstree[levelsep=25pt,treesep=6pt]{\Tcircle[fillcolor=lightgray,fillstyle=solid]{1}}
		{% levelsep指不同层之间距离, treesep兄弟之间距离
		 \pstree{\Tcircle[fillcolor=red,fillstyle=solid]{5}}{
		 	\pstree{\Tcircle[fillcolor=red,fillstyle=solid]{2}}{
		 		\pstree{\Tcircle[fillcolor=lightgray,fillstyle=solid]{4}}{
					\Tcircle[fillcolor=green,fillstyle=solid]{A}
					\Tcircle[fillcolor=green,fillstyle=solid]{B}
				}
		 		\pstree{\Tcircle[fillcolor=lightgray,fillstyle=solid]{7}}{
					\Tcircle[fillcolor=green,fillstyle=solid]{C}
					\Tcircle[fillcolor=green,fillstyle=solid]{D}
				}
			}
		 	\pstree{\Tcircle[fillcolor=lightgray,fillstyle=solid]{8}}{
				\Tcircle[fillcolor=green,fillstyle=solid]{E}
				\Tcircle[fillcolor=green,fillstyle=solid]{F}
			}
		 }
		 \pstree{\Tcircle[fillcolor=lightgray,fillstyle=solid]{3}}{
			\Tcircle[fillcolor=green,fillstyle=solid]{G}
			\Tcircle[fillcolor=green,fillstyle=solid]{H}
		 }
		}
		\caption{调整之后,继续调整2}
		\end{figure}
		\end{column}

		\end{columns}
	\end{frame}

	\begin{frame}
		\frametitle{插入情况4}
		父节点是红节点,叔叔节点是黑节点,且插入节点是父节点的左子,那么可以以爷爷节点为旋转点右旋并交换父节点和爷爷节点的颜色,最终调整完毕,退出算法

		\begin{columns} %排版
		\begin{column}{0.5\textwidth}
		\centering
		\begin{figure}
		\pstree[levelsep=25pt,treesep=6pt]{\Tcircle[fillcolor=lightgray,fillstyle=solid]{1}}
		{% levelsep指不同层之间距离, treesep兄弟之间距离
		 \pstree{\Tcircle[fillcolor=red,fillstyle=solid]{2}}{
		 	\pstree{\Tcircle[fillcolor=red,fillstyle=solid]{4}}{
				\Tcircle[fillcolor=green,fillstyle=solid]{A}
				\Tcircle[fillcolor=green,fillstyle=solid]{B}
			}
		 	\pstree{\Tcircle[fillcolor=lightgray,fillstyle=solid]{5}}{
				\Tcircle[fillcolor=green,fillstyle=solid]{C}
				\Tcircle[fillcolor=green,fillstyle=solid]{D}
			}
		 }
		 \pstree{\Tcircle[fillcolor=lightgray,fillstyle=solid]{3}}{
			\Tcircle[fillcolor=green,fillstyle=solid]{E}
			\Tcircle[fillcolor=green,fillstyle=solid]{F}
		 }
		}
		\caption{插入节点4}
		\end{figure}
		\end{column}

		\pause

		\begin{column}{0.5\textwidth}
		\centering
		\begin{figure}
		\pstree[levelsep=25pt,treesep=6pt]{\Tcircle[fillcolor=lightgray,fillstyle=solid]{2}}
		{% levelsep指不同层之间距离, treesep兄弟之间距离
		 \pstree{\Tcircle[fillcolor=red,fillstyle=solid]{4}}{
				\Tcircle[fillcolor=green,fillstyle=solid]{A}
				\Tcircle[fillcolor=green,fillstyle=solid]{B}
			}
		 \pstree{\Tcircle[fillcolor=red,fillstyle=solid]{1}}{
		 	\pstree{\Tcircle[fillcolor=lightgray,fillstyle=solid]{5}}{
				\Tcircle[fillcolor=green,fillstyle=solid]{C}
				\Tcircle[fillcolor=green,fillstyle=solid]{D}
			}
		 	\pstree{\Tcircle[fillcolor=lightgray,fillstyle=solid]{3}}{
				\Tcircle[fillcolor=green,fillstyle=solid]{E}
				\Tcircle[fillcolor=green,fillstyle=solid]{F}
			}
		 }
		}
		\caption{调整之后}
		\end{figure}
		\end{column}

		\end{columns}
	\end{frame}

	\subsection{红黑树的删除}
    \begin{frame}
		\frametitle{红黑树的删除}
		红黑树的删除操作先是同一般BST一样操作,然后对于顶替上来的那个节点增加一重黑色,再分情况来处理,主要有以下几种情况
		\begin{itemize}
		 \item[1] 该节点颜色为R+B
		 \item[2] 该节点颜色为B+B,但是是根节点
		 \item[3] 该节点颜色为B+B,兄弟节点为红色节点
		 \item[4] 该节点颜色为B+B,兄弟节点为黑色节点,兄弟节点的两个孩子都是黑色
		 \item[5] 该节点颜色为B+B, 兄弟节点为黑色节点,兄弟节点的左子为红,右子为黑
		 \item[6] 该节点颜色为B+B, 兄弟节点为黑色节点,兄弟节点的右子为红
		\end{itemize}
    \end{frame}

	\begin{frame}
		\frametitle{删除情况1}
		直接将此节点颜色设为黑色
		\begin{columns} %排版

		\begin{column}{0.3\textwidth}
		\centering
		\begin{figure}
		\pstree[levelsep=25pt,treesep=6pt]{\Tcircle[fillcolor=lightgray,fillstyle=solid]{1}}
		{% levelsep指不同层之间距离, treesep兄弟之间距离
		 \pstree{\Tcircle[fillcolor=lightgray,fillstyle=solid]{2}}{
			\Tcircle[fillcolor=lightgray,fillstyle=solid]{}
		 	\pstree{\Tcircle[fillcolor=red,fillstyle=solid]{4}}{
				\Tcircle[fillcolor=green,fillstyle=solid]{A}
				\Tcircle[fillcolor=green,fillstyle=solid]{B}
			 }
		 }
		 \pstree{\Tcircle[fillcolor=lightgray,fillstyle=solid]{3}}{
			\Tcircle[fillcolor=green,fillstyle=solid]{C}
			\Tcircle[fillcolor=green,fillstyle=solid]{D}
		 }
		}
		\caption{删除节点2}
		\end{figure}
		\end{column}

		\pause

		\begin{column}{0.3\textwidth}
		\centering
		\begin{figure}
		\pstree[levelsep=25pt,treesep=6pt]{\Tcircle[fillcolor=lightgray,fillstyle=solid]{1}}
		{% levelsep指不同层之间距离, treesep兄弟之间距离
		 \pstree{\Tcircle[fillcolor=white,fillstyle=solid]{4}\tlput{r+b}}{
			\Tcircle[fillcolor=green,fillstyle=solid]{A}
			\Tcircle[fillcolor=green,fillstyle=solid]{B}
		 }
		 \pstree{\Tcircle[fillcolor=lightgray,fillstyle=solid]{3}}{
			\Tcircle[fillcolor=green,fillstyle=solid]{C}
			\Tcircle[fillcolor=green,fillstyle=solid]{D}
		 }
		}
		\caption{调整节点4}
		\end{figure}
		\end{column}

		\pause

		\begin{column}{0.3\textwidth}
		\centering
		\begin{figure}
		\pstree[levelsep=25pt,treesep=6pt]{\Tcircle[fillcolor=lightgray,fillstyle=solid]{1}}
		{% levelsep指不同层之间距离, treesep兄弟之间距离
		 \pstree{\Tcircle[fillcolor=lightgray,fillstyle=solid]{4}}{
			\Tcircle[fillcolor=green,fillstyle=solid]{A}
			\Tcircle[fillcolor=green,fillstyle=solid]{B}
		 }
		 \pstree{\Tcircle[fillcolor=lightgray,fillstyle=solid]{3}}{
			\Tcircle[fillcolor=green,fillstyle=solid]{C}
			\Tcircle[fillcolor=green,fillstyle=solid]{D}
		 }
		}
		\caption{调整之后}
		\end{figure}
		\end{column}

		\end{columns}
	\end{frame}

	\begin{frame}
		\frametitle{删除情况2}
		同样,直接将此节点颜色设为黑色
		\begin{columns} %排版

		\pause
		\begin{column}{0.3\textwidth}
		\centering
		\begin{figure}
		\pstree[levelsep=25pt,treesep=6pt]{\Tcircle[fillcolor=lightgray,fillstyle=solid]{1}}
		{% levelsep指不同层之间距离, treesep兄弟之间距离
		 \Tcircle[fillcolor=lightgray,fillstyle=solid]{}
		 \pstree{\Tcircle[fillcolor=lightgray,fillstyle=solid]{2}}{
			\Tcircle[fillcolor=green,fillstyle=solid]{C}
			\Tcircle[fillcolor=green,fillstyle=solid]{D}
		 }
		}
		\caption{删除根节点1}
		\end{figure}
		\end{column}

		\pause
		\begin{column}{0.3\textwidth}
		\centering
		\begin{figure}
		\pstree[levelsep=25pt,treesep=6pt]{\Tcircle[fillcolor=white,fillstyle=solid]{2}}
		{% levelsep指不同层之间距离, treesep兄弟之间距离
			\Tcircle[fillcolor=green,fillstyle=solid]{C}
			\Tcircle[fillcolor=green,fillstyle=solid]{D}
		}\textmd{b+b}
		\caption{调整节点2}
		\end{figure}
		\end{column}

		\pause
		\begin{column}{0.3\textwidth}
		\centering
		\begin{figure}
		\pstree[levelsep=25pt,treesep=6pt]{\Tcircle[fillcolor=lightgray,fillstyle=solid]{2}}
		{% levelsep指不同层之间距离, treesep兄弟之间距离
			\Tcircle[fillcolor=green,fillstyle=solid]{C}
			\Tcircle[fillcolor=green,fillstyle=solid]{D}
		}
		\caption{调整之后}
		\end{figure}
		\end{column}
	
		\end{columns}
	\end{frame}

	\begin{frame}
		\frametitle{删除情况3}
		当前节点是B+B,兄弟节点是红节点,则以父节点为旋转点左旋,并且交换父节点和兄弟节点的颜色,将问题转化为后面的情况
		\begin{columns} %排版

		\pause
		\begin{column}{0.5\textwidth}
		\centering
		\begin{figure}
		\pstree[levelsep=25pt,treesep=6pt]{\Tcircle[fillcolor=lightgray,fillstyle=solid]{1}}
		{% levelsep指不同层之间距离, treesep兄弟之间距离
		 \pstree{\Tcircle[fillcolor=white,fillstyle=solid]{2}^{b+b}}{
			\Tcircle[fillcolor=green,fillstyle=solid]{A}
			\Tcircle[fillcolor=green,fillstyle=solid]{B}
		 }
		 \pstree{\Tcircle[fillcolor=red,fillstyle=solid]{3}}{
			\Tcircle[fillcolor=green,fillstyle=solid]{C}
			\Tcircle[fillcolor=green,fillstyle=solid]{D}
		 }
		}
		\caption{调整节点2}
		\end{figure}
		\end{column}

		\pause
		\begin{column}{0.5\textwidth}
		\centering
		\begin{figure}
		\pstree[levelsep=25pt,treesep=6pt]{\Tcircle[fillcolor=lightgray,fillstyle=solid]{3}}
		{% levelsep指不同层之间距离, treesep兄弟之间距离
		 \pstree{\Tcircle[fillcolor=red,fillstyle=solid]{1}}{
		 	\pstree{\Tcircle[fillcolor=white,fillstyle=solid]{2}^{b+b}}{
				\Tcircle[fillcolor=green,fillstyle=solid]{A}
				\Tcircle[fillcolor=green,fillstyle=solid]{B}
			 }
			\Tcircle[fillcolor=green,fillstyle=solid]{C}
		 }
		\Tcircle[fillcolor=green,fillstyle=solid]{D}
		}
		\caption{调整之后,继续调整节点2}
		\end{figure}
		\end{column}
	
		\end{columns}
	\end{frame}

	\begin{frame}
		\frametitle{删除情况4}
		当前节点是B+B,兄弟节点是黑节点,且兄弟节点的孩子都是黑节点,这种情况下可以把兄弟节点换成红色,那么左右两边都会缺一个黑色,则可把节点多的那重黑色转移给父节点,问题就变成父节点的问题了
		\begin{columns} %排版

		\pause
		\begin{column}{0.5\textwidth}
		\centering
		\begin{figure}
		\pstree[levelsep=25pt,treesep=6pt]{\Tcircle[fillcolor=lightgray,fillstyle=solid]{1}}
		{% levelsep指不同层之间距离, treesep兄弟之间距离
		 \pstree{\Tcircle[fillcolor=white,fillstyle=solid]{2}^{b+b}}{
			\Tcircle[fillcolor=green,fillstyle=solid]{A}
			\Tcircle[fillcolor=green,fillstyle=solid]{B}
		 }
		 \pstree{\Tcircle[fillcolor=lightgray,fillstyle=solid]{3}}{
		 	\pstree{\Tcircle[fillcolor=lightgray,fillstyle=solid]{4}}{
				\Tcircle[fillcolor=green,fillstyle=solid]{C}
				\Tcircle[fillcolor=green,fillstyle=solid]{D}
			 }
		 	\pstree{\Tcircle[fillcolor=lightgray,fillstyle=solid]{5}}{
				\Tcircle[fillcolor=green,fillstyle=solid]{E}
				\Tcircle[fillcolor=green,fillstyle=solid]{F}
			 }
		 }
		}
		\caption{调整节点2}
		\end{figure}
		\end{column}

		\pause
		\begin{column}{0.5\textwidth}
		\centering
		\begin{figure}
		b+b
		\pstree[levelsep=25pt,treesep=6pt]{\Tcircle[fillcolor=white,fillstyle=solid]{1}}
		{% levelsep指不同层之间距离, treesep兄弟之间距离
		 \pstree{\Tcircle[fillcolor=lightgray,fillstyle=solid]{2}}{
			\Tcircle[fillcolor=green,fillstyle=solid]{A}
			\Tcircle[fillcolor=green,fillstyle=solid]{B}
		 }
		 \pstree{\Tcircle[fillcolor=red,fillstyle=solid]{3}}{
		 	\pstree{\Tcircle[fillcolor=lightgray,fillstyle=solid]{4}}{
				\Tcircle[fillcolor=green,fillstyle=solid]{C}
				\Tcircle[fillcolor=green,fillstyle=solid]{D}
			 }
		 	\pstree{\Tcircle[fillcolor=lightgray,fillstyle=solid]{5}}{
				\Tcircle[fillcolor=green,fillstyle=solid]{E}
				\Tcircle[fillcolor=green,fillstyle=solid]{F}
			 }
		 }
		}
		\caption{调整之后,继续调整节点1}
		\end{figure}
		\end{column}
	
		\end{columns}
	\end{frame}

	\begin{frame}
		\frametitle{删除情况5}
		当前节点是B+B,兄弟节点是黑节点,且兄弟节点的孩子左子是红,右子是黑,这种情况下可以右旋它的兄弟节点,把问题转化为情况6
		\begin{columns} %排版

		\pause
		\begin{column}{0.5\textwidth}
		\centering
		\begin{figure}
		\pstree[levelsep=25pt,treesep=6pt]{\Tcircle[fillcolor=lightgray,fillstyle=solid]{1}}
		{% levelsep指不同层之间距离, treesep兄弟之间距离
		 \pstree{\Tcircle[fillcolor=white,fillstyle=solid]{2}^{b+b}}{
			\Tcircle[fillcolor=green,fillstyle=solid]{A}
			\Tcircle[fillcolor=green,fillstyle=solid]{B}
		 }
		 \pstree{\Tcircle[fillcolor=lightgray,fillstyle=solid]{3}}{
		 	\pstree{\Tcircle[fillcolor=red,fillstyle=solid]{4}}{
				\Tcircle[fillcolor=green,fillstyle=solid]{C}
				\Tcircle[fillcolor=green,fillstyle=solid]{D}
			 }
		 	\pstree{\Tcircle[fillcolor=lightgray,fillstyle=solid]{5}}{
				\Tcircle[fillcolor=green,fillstyle=solid]{E}
				\Tcircle[fillcolor=green,fillstyle=solid]{F}
			 }
		 }
		}
		\caption{调整节点2}
		\end{figure}
		\end{column}

		\pause
		\begin{column}{0.5\textwidth}
		\centering
		\begin{figure}
		\pstree[levelsep=25pt,treesep=6pt]{\Tcircle[fillcolor=lightgray,fillstyle=solid]{1}}
		{% levelsep指不同层之间距离, treesep兄弟之间距离
		 \pstree{\Tcircle[fillcolor=white,fillstyle=solid]{2}^{b+b}}{
			\Tcircle[fillcolor=green,fillstyle=solid]{A}
			\Tcircle[fillcolor=green,fillstyle=solid]{B}
		 }
		 \pstree{\Tcircle[fillcolor=lightgray,fillstyle=solid]{4}}{
			\Tcircle[fillcolor=green,fillstyle=solid]{C}
		 	\pstree{\Tcircle[fillcolor=red,fillstyle=solid]{3}}{
				\Tcircle[fillcolor=green,fillstyle=solid]{D}
		 		\pstree{\Tcircle[fillcolor=lightgray,fillstyle=solid]{5}}{
					\Tcircle[fillcolor=green,fillstyle=solid]{E}
					\Tcircle[fillcolor=green,fillstyle=solid]{F}
			 	}
			 }
		 }
		}
		\caption{调整之后,继续调整节点2}
		\end{figure}
		\end{column}
	
		\end{columns}
	\end{frame}

	\begin{frame}
		\frametitle{删除情况6}
		当前节点是B+B,兄弟节点是黑节点,且兄弟节点的孩子右子是红,这种情况下可以左旋父节点,然后交换父节点和兄弟节点的颜色,并把兄弟节点的右子涂成黑色,则问题得以解决,退出算法
		\begin{columns} %排版

		\pause
		\begin{column}{0.5\textwidth}
		\centering
		\begin{figure}
		\pstree[levelsep=25pt,treesep=6pt]{\Tcircle[fillcolor=lightgray,fillstyle=solid]{1}}
		{% levelsep指不同层之间距离, treesep兄弟之间距离
		 \pstree{\Tcircle[fillcolor=white,fillstyle=solid]{2}^{b+b}}{
			\Tcircle[fillcolor=green,fillstyle=solid]{A}
			\Tcircle[fillcolor=green,fillstyle=solid]{B}
		 }
		 \pstree{\Tcircle[fillcolor=lightgray,fillstyle=solid]{3}}{
		 	\pstree{\Tcircle[fillcolor=red,fillstyle=solid]{4}}{
				\Tcircle[fillcolor=green,fillstyle=solid]{C}
				\Tcircle[fillcolor=green,fillstyle=solid]{D}
			 }
		 	\pstree{\Tcircle[fillcolor=red,fillstyle=solid]{5}}{
				\Tcircle[fillcolor=green,fillstyle=solid]{E}
				\Tcircle[fillcolor=green,fillstyle=solid]{F}
			 }
		 }
		}
		\caption{调整节点2}
		\end{figure}
		\end{column}

		\pause
		\begin{column}{0.5\textwidth}
		\centering
		\begin{figure}
		\pstree[levelsep=25pt,treesep=6pt]{\Tcircle[fillcolor=lightgray,fillstyle=solid]{3}}
		{% levelsep指不同层之间距离, treesep兄弟之间距离
		 \pstree{\Tcircle[fillcolor=lightgray,fillstyle=solid]{1}}{
		 	\pstree{\Tcircle[fillcolor=lightgray,fillstyle=solid]{2}}{
				\Tcircle[fillcolor=green,fillstyle=solid]{A}
				\Tcircle[fillcolor=green,fillstyle=solid]{B}
			 }
		 	\pstree{\Tcircle[fillcolor=red,fillstyle=solid]{4}}{
				\Tcircle[fillcolor=green,fillstyle=solid]{C}
				\Tcircle[fillcolor=green,fillstyle=solid]{D}
			 }
		 }
		 \pstree{\Tcircle[fillcolor=lightgray,fillstyle=solid]{5}}{
			\Tcircle[fillcolor=green,fillstyle=solid]{E}
			\Tcircle[fillcolor=green,fillstyle=solid]{F}
		 }
		}
		\caption{调整之后,算法结束}
		\end{figure}
		\end{column}
	
		\end{columns}
	\end{frame}


	\section{参考文献}
	\begin{frame}
		\frametitle{参考文献}
		[1] http://blog.csdn.net/v\_july\_v/article/details/6105630\newline
		[2] 算法导论\newline
		[3] STL源码剖析
	\end{frame}


	\begin{frame}
		\frametitle{致谢}
		Thanks
	\end{frame}


\end{CJK}
\end{document}
