\ifx allfiles undefined
\documentclass{article}
\usepackage{CJK}
\usepackage{verbatim}

%%%代码
\usepackage{color}
\usepackage{xcolor}
\definecolor{keywordcolor}{rgb}{0.8,0.1,0.5}
\usepackage{listings}
\lstset{breaklines}%这条命令可以让LaTeX自动将长的代码行换行排版
\lstset{extendedchars=false}%这一条命令可以解决代码跨页时,章节标题,页眉等汉字不显示的问题
\lstset{language=C++, %用于设置语言为C++
    keywordstyle=\color{keywordcolor} \bfseries, %设置关键词
    identifierstyle=,
    basicstyle=\ttfamily, 
    commentstyle=\color{blue} \textit,
    stringstyle=\ttfamily, 
    showstringspaces=false,
    %frame=shadowbox, %边框
    captionpos=b
}
%%%

%\hypersetup{CJKbookmarks=true} %解决section不能使用中文的问题

\begin{document}
\begin{CJK}{UTF8}{gbsn}     %CJK:支持中文

\else
    
\begin{description}
    \item{\textbf{问题}}: Implement int sqrt(int x). \textit{(leetcode 69)}
    \item{\textbf{二分查找}} : \fbox{时间复杂度O(lgn) , 空间复杂度O(1)}
    \\二分搜索,夹逼方法.这题其实也可以使用牛顿迭代法做.
    \begin{lstlisting}
int sqrt(int x) {
	if(x < 2)	return x;
	int low = 1, high = x/2;
	while(low <= high){
		int mid = (low + high) / 2;
		if(mid == x/mid)	return mid;
		if(mid > x/mid)	high = mid - 1;
		else	low = mid + 1;
	}
	return high;
}
    \end{lstlisting}
    \item{\textbf{牛顿迭代法}} : \fbox{时间复杂度O(???) , 空间复杂度O(1)}
    \begin{lstlisting}
int sqrt(int x) {
	if(x < 2) return x;
	int pos = x/2;
	while(!(pos <= x/pos && (pos+1) >= x/(pos+1))){
		pos = pos/2 + x/(2*pos);
	}
	if((pos+1) == x/(pos+1))	return pos+1;
	return pos;
}
    \end{lstlisting}
    \textit{}
\end{description}

\fi

\ifx allfiles undefined
\end{CJK}
\end{document}
\fi
