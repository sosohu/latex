\ifx allfiles undefined
\documentclass{article}
\usepackage{CJK}
\usepackage{verbatim}

%%%代码
\usepackage{color}
\usepackage{xcolor}
\definecolor{keywordcolor}{rgb}{0.8,0.1,0.5}
\usepackage{listings}
\lstset{breaklines}%这条命令可以让LaTeX自动将长的代码行换行排版
\lstset{extendedchars=false}%这一条命令可以解决代码跨页时,章节标题,页眉等汉字不显示的问题
\lstset{language=C++, %用于设置语言为C++
    keywordstyle=\color{keywordcolor} \bfseries, %设置关键词
    identifierstyle=,
    basicstyle=\ttfamily, 
    commentstyle=\color{blue} \textit,
    stringstyle=\ttfamily, 
    showstringspaces=false,
    %frame=shadowbox, %边框
    captionpos=b
}
%%%

%\hypersetup{CJKbookmarks=true} %解决section不能使用中文的问题

\begin{document}
\begin{CJK}{UTF8}{gbsn}     %CJK:支持中文

\else
    
\begin{description}
    \item{\textbf{问题}}: Given a sorted array and a target value, return the index if the target is found. If not, return the index where it would be if it were inserted in order.You may assume no duplicates in the array.\textit{(leetcode 35)}
	\item{\textbf{举例}}:\\
(1,3,5,6), 5 → 2\\
(1,3,5,6), 2 → 1\\
(1,3,5,6), 7 → 4\\
(1,3,5,6), 0 → 0\\
    \item{\textbf{二分查找}} : \fbox{时间复杂度O(lgn) , 空间复杂度O(1)}
    \\其实就是lower\_bound函数
    \begin{lstlisting}
int binary_search(int A[], int begin, int end, int target){
	if(begin > end){
		return begin;
	}
	int mid = (begin + end) / 2;
	if(A[mid] == target)	return mid;
	if(A[mid] > target)	return binary_search(A, begin, mid-1, target);
	return binary_search(A, mid+1, end, target);
}

int searchInsert(int A[], int n, int target) {
	return binary_search(A, 0, n-1, target);
}
    \end{lstlisting}
\end{description}

\fi

\ifx allfiles undefined
\end{CJK}
\end{document}
\fi
