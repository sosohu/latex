\ifx allfiles undefined
\documentclass{article}
\usepackage{CJK}
\usepackage{verbatim}

%%%代码
\usepackage{color}
\usepackage{xcolor}
\definecolor{keywordcolor}{rgb}{0.8,0.1,0.5}
\usepackage{listings}
\lstset{breaklines}%这条命令可以让LaTeX自动将长的代码行换行排版
\lstset{extendedchars=false}%这一条命令可以解决代码跨页时,章节标题,页眉等汉字不显示的问题
\lstset{language=C++, %用于设置语言为C++
    keywordstyle=\color{keywordcolor} \bfseries, %设置关键词
    identifierstyle=,
    basicstyle=\ttfamily, 
    commentstyle=\color{blue} \textit,
    stringstyle=\ttfamily, 
    showstringspaces=false,
    %frame=shadowbox, %边框
    captionpos=b
}
%%%

%\hypersetup{CJKbookmarks=true} %解决section不能使用中文的问题

\begin{document}
\begin{CJK}{UTF8}{gbsn}     %CJK:支持中文

\else
    
\begin{description}
    \item{\textbf{问题}}: Implement pow(x, n). \textit{(leetcode 50)}
    \item{\textbf{二分查找}} : \fbox{时间复杂度O(lgn), 空间复杂度O(1)}
    \\和除法几乎如出一辙,都是切分成很多个$$2^i$$,再用二分法
    \begin{lstlisting}
// return x^(count)
double pow2n(double x, int& n){
	int count = 1;
	while(count < n>>1){
		x = x * x;
		count = count<<1;
	}
	n = n - count;
	return x;
}

double pow(double x, int n) {
	if(n == 0)	return 1;
	if(n < 0)	return n == INT_MIN? 1.0/(x * pow(x, INT_MAX)) : 1.0/pow(x, -n);
	double result = 1.0;
	while(n) result = result * pow2n(x, n);	
	return result;
}
    \end{lstlisting}
\end{description}

\fi

\ifx allfiles undefined
\end{CJK}
\end{document}
\fi
