\ifx allfiles undefined
\documentclass{article}
\usepackage{CJK}
\usepackage{verbatim}

%%%代码
\usepackage{color}
\usepackage{xcolor}
\definecolor{keywordcolor}{rgb}{0.8,0.1,0.5}
\usepackage{listings}
\lstset{breaklines}%这条命令可以让LaTeX自动将长的代码行换行排版
\lstset{extendedchars=false}%这一条命令可以解决代码跨页时,章节标题,页眉等汉字不显示的问题
\lstset{language=C++, %用于设置语言为C++
    keywordstyle=\color{keywordcolor} \bfseries, %设置关键词
    identifierstyle=,
    basicstyle=\ttfamily, 
    commentstyle=\color{blue} \textit,
    stringstyle=\ttfamily, 
    showstringspaces=false,
    %frame=shadowbox, %边框
    captionpos=b
}
%%%

%\hypersetup{CJKbookmarks=true} %解决section不能使用中文的问题

\begin{document}
\begin{CJK}{UTF8}{gbsn}     %CJK:支持中文

\else
    
\begin{description}
    \item{\textbf{问题}}:Given a sorted array of integers, find the starting and ending position of a given target value.\textit{(leetcode 34)}
	\item{\textbf{举例}}:\\
Given [5, 7, 7, 8, 8, 10] and target value 8,\\
return [3, 4].\\
	\item{\textbf{Note}}:\\
Your algorithm's runtime complexity must be in the order of O(log n).\\
If the target is not found in the array, return [-1, -1].

    \item{\textbf{二分查找}} : \fbox{时间复杂度O(lgn), 空间复杂度O(1)}
    \\我这里的方法是先通过二分查找随便找到一个位置,然后再找其上下界.其实可以直接编写类似STL库中的lower\_bound和upper\_bound就可以了-.-
    \begin{lstlisting}
int binary_sarch(int A[], int begin, int end, int target){
	if(begin > end)	return -1;
	int mid = (begin + end) / 2;
	if(A[mid] == target)	return mid;
	if(A[mid] > target)	return binary_sarch(A, begin, mid-1, target);
	return binary_sarch(A, mid+1, end, target);
}

int binary_sarchNo(int A[], int begin, int end, int target, bool left){
	if(begin > end)	return left? begin : end;
	int mid = (begin + end) / 2;
	int new_begin, new_end;
	if(A[mid] == target){
		new_begin = left? begin : mid + 1;
		new_end = left? mid - 1 : end;
	}else if(A[mid] > target){
		return binary_sarchNo(A, begin, mid - 1, target, left);
	}else{
		return binary_sarchNo(A, mid+1, end, target, left);
	}
	return binary_sarchNo(A, new_begin, new_end, target, left);
}

vector<int> searchRange(int A[], int n, int target) {
	vector<int> result{-1, -1};
	int pos = binary_sarch(A, 0, n-1, target);
	if(pos != -1){
		result[0] = pos == 0? 0 : binary_sarchNo(A, 0, pos, target, true);
		result[1] = pos == n-1? n-1 : binary_sarchNo(A, pos, n-1, target, false);
	}
	return result;
}
    \end{lstlisting}
    \textit{}
\end{description}

\fi

\ifx allfiles undefined
\end{CJK}
\end{document}
\fi
