\ifx allfiles undefined
\documentclass{article}
\usepackage{CJK}
\usepackage{verbatim}

%%%代码
\usepackage{color}
\usepackage{xcolor}
\definecolor{keywordcolor}{rgb}{0.8,0.1,0.5}
\usepackage{listings}
\lstset{breaklines}%这条命令可以让LaTeX自动将长的代码行换行排版
\lstset{extendedchars=false}%这一条命令可以解决代码跨页时,章节标题,页眉等汉字不显示的问题
\lstset{language=C++, %用于设置语言为C++
    keywordstyle=\color{keywordcolor} \bfseries, %设置关键词
    identifierstyle=,
    basicstyle=\ttfamily, 
    commentstyle=\color{blue} \textit,
    stringstyle=\ttfamily, 
    showstringspaces=false,
    %frame=shadowbox, %边框
    captionpos=b
}
%%%

%\hypersetup{CJKbookmarks=true} %解决section不能使用中文的问题

\begin{document}
\begin{CJK}{UTF8}{gbsn}     %CJK:支持中文

\else
    
\begin{description}
    \item{\textbf{问题}}: \\
Suppose a sorted array is rotated at some pivot unknown to you beforehand.\textit{(leetcode 33)}
    \item{\textbf{举例}}: \\
(i.e., 0 1 2 4 5 6 7 might become 4 5 6 7 0 1 2).
    \item{\textbf{Note}}: \\
You are given a target value to search. If found in the array return its index, otherwise return -1.\\
You may assume no duplicate exists in the array.\\
    \item{\textbf{二分查找}} : \fbox{时间复杂度O(lgn) , 空间复杂度O(1)}
    \\同样采用二分查找,需要注意的就是如果查找范围已经是排序好的,需要改变查找策略.
    \begin{lstlisting}
int binay_search(int A[], int begin, int end, int target){
	if(begin > end)	return -1;
	int mid = (begin + end) / 2;
	if(A[mid] == target)	return mid;
	if(A[mid] < target)	return binay_search(A, mid+1, end, target);
	else	return binay_search(A, begin, mid-1, target);
}

int search_aux(int A[], int begin, int end, int target){
	if(begin > end)	return -1;
	if(begin == end)	return A[begin] == target? begin : -1;
	if(A[end] > A[begin])	return binay_search(A, begin, end, target);
	int mid = (begin + end) / 2;
	if(target == A[mid])	return mid;
	if(A[mid] > A[begin]){
		if(target < A[mid] && target >= A[begin])
			return binay_search(A, begin, mid-1, target);
		return search_aux(A, mid+1, end, target);
	}else if(A[mid] == A[begin]){
		return A[end] == target? end : -1;
	}else{
		if(target > A[mid] && target <= A[end])
			return binay_search(A, mid+1, end, target);
		return search_aux(A, begin, mid-1, target);
	}
}

int search(int A[], int n, int target) {
	return search_aux(A, 0, n-1, target);
}
    \end{lstlisting}
    \textit{}
\end{description}

\fi

\ifx allfiles undefined
\end{CJK}
\end{document}
\fi
