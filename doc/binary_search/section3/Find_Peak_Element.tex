\ifx allfiles undefined
\documentclass{article}
\usepackage{CJK}
\usepackage{verbatim}

%%%代码
\usepackage{color}
\usepackage{xcolor}
\definecolor{keywordcolor}{rgb}{0.8,0.1,0.5}
\usepackage{listings}
\lstset{breaklines}%这条命令可以让LaTeX自动将长的代码行换行排版
\lstset{extendedchars=false}%这一条命令可以解决代码跨页时,章节标题,页眉等汉字不显示的问题
\lstset{language=C++, %用于设置语言为C++
    keywordstyle=\color{keywordcolor} \bfseries, %设置关键词
    identifierstyle=,
    basicstyle=\ttfamily, 
    commentstyle=\color{blue} \textit,
    stringstyle=\ttfamily, 
    showstringspaces=false,
    %frame=shadowbox, %边框
    captionpos=b
}
%%%

%\hypersetup{CJKbookmarks=true} %解决section不能使用中文的问题

\begin{document}
\begin{CJK}{UTF8}{gbsn}     %CJK:支持中文

\else
    
\begin{description}
    \item{\textbf{问题}}: \\
A peak element is an element that is greater than its neighbors.\\
Given an input array where num[i] ≠ num[i+1], find a peak element and return its index.\\
The array may contain multiple peaks, in that case return the index to any one of the peaks is fine.\\
You may imagine that num[-1] = num[n] = -∞.\\
\textit{(leetcode 162)}
	\item{\textbf{举例}}:\\
For example, in array [1, 2, 3, 1], 3 is a peak element and your function should return the index number 2.
    \item{\textbf{二分搜索}} : \fbox{时间复杂度O(lgn) , 空间复杂度O(1)}
    \\这题同样采用切分,然后根据丢掉一部分选择范围,我们对于选择范围[s, e],取中间点m = (s + e)/2,判断A[m-1], A[m]和A[m+1]的关系然后选择下一次迭代范围为[s,m-1]还是[m+1,e],然后不断的缩小范围. 
    \begin{lstlisting}
bool isPeak(const vector<int> &num, int pos){
	return pos == 0?  num[pos] > num[pos+1] : num[pos] > num[pos-1];
}

int binary_sarch(const vector<int> &num, int begin, int end){
	if(begin >= end)	return end;
	int  mid = (begin + end) / 2;
	if(num[mid] > num[mid-1] && num[mid] > num[mid+1])
		return mid;
	if(num[mid] >= num[begin] && num[mid] >= num[end]){
		if(num[mid] < num[mid-1])
			return binary_sarch(num, begin, mid-1);
		return binary_sarch(num, mid+1, end);
	}
	if(num[mid] < num[begin])
		return binary_sarch(num, begin, mid-1);
	else
		return binary_sarch(num, mid+1, end);
}

int findPeakElement(const vector<int> &num) {
	int n = num.size();
	if(n <= 1)	return n-1;	
	if(isPeak(num, 0))	return 0;
	if(isPeak(num, n-1))	return n - 1;
	return binary_sarch(num, 1, n-2);
}
    \end{lstlisting}
    \textit{}
\end{description}

\fi

\ifx allfiles undefined
\end{CJK}
\end{document}
\fi
