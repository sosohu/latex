    
\begin{description}
    \item{\textbf{问题}}:\\
Given an array of size n, find the majority element. The majority element is the element that appears more than ⌊ n/2 ⌋ times.\\
\\
You may assume that the array is non-empty and the majority element always exist in the array.\\
\textit{(leetcode 169)}
    \item{\textbf{记数法}} : \fbox{时间复杂度O(n), 空间复杂度O(1)}
    \\这题我最先了解到的解法就是记数法,因为主元素出现的次数多于总个数的一半,所以我们可以放着两个椅子,然后让元素去争抢:对于每个新加入的元素,如果椅子上有与它相同的元素,那么该椅子记数+1;如果没有,此时如果有空闲椅子,则该元素抢占该椅子,然后置此椅子记数为1;如果没有空闲椅子,那么舍弃该元素,并且两个椅子的记数都-1,如果此时某个椅子记数为0,则置此椅子为空闲.
	\\最后,椅子上记数较大的那个元素就是主元素.
	\\这个方法的证明也很简单,我们可以试想一下,由于主元素个数超过一半,又有两个椅子,所以主元素必然在最终的一个之一,而且另一个椅子的记数也绝不会超过它.
    \begin{lstlisting}
int majorityElement(vector<int>& nums) {
	if(nums.size() == 0)	return -1;
	int p1 = nums[0], p2, c1 = 1, c2 = 0;
	for(int i = 1; i < nums.size(); i++){
		if(c1 && nums[i] == p1)	c1++;
		else	if(c2 && nums[i] == p2)	c2++;
		else	if(!c1){
			p1 = nums[i];
			c1++;
		}
		else	if(!c2){
			p2 = nums[i];
			c2++;
		}else{
			c1--;
			c2--;
		}
	}
	return c1 > c2? p1 : p2;
}
    \end{lstlisting}
    \item{\textbf{位操作}} : \fbox{时间复杂度O(n), 空间复杂度O(1)}
	\\这题还可以这么考虑:我们对于数组中每个元素分析它们的每个位,由于主元素个数超半,所以每个位的0,1个数必然被主元素控制,如果主元素该位为1那么1的个数必然超过一半,反之亦然.所以我们可以基于这个原则设计算法.
	\begin{lstlisting}
int majorityElement(vector<int>& nums) {
	int result = 0, n = nums.size();
	for(int i = 0; i < 32; i++){
		int bit = 0;
		for(int j = 0; j < n; j++)
			bit += ((nums[j] & (0x1 << i))? 1 : 0);
		result += ((bit > n/2? 1 : 0) << i);
	}
	return result;
}
	\end{lstlisting}
\end{description}
