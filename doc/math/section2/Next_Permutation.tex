    
\begin{description}
    \item{\textbf{问题}}:\\
Implement next permutation, which rearranges numbers into the lexicographically next greater permutation of numbers.\\
\\
If such arrangement is not possible, it must rearrange it as the lowest possible order (ie, sorted in ascending order).\\
\\
The replacement must be in-place, do not allocate extra memory.\\
\textit{(leetcode 31)}
    \item{\textbf{举例}}:\\
Inputs are in the left-hand column and its corresponding outputs are in the right-hand column.\\
1,2,3 $\rightarrow$ 1,3,2 \\
3,2,1 $\rightarrow$ 1,2,3 \\
1,1,5 $\rightarrow$ 1,5,1
    \item{\textbf{组合数学}} : \fbox{时间复杂度O(n), 空间复杂度O(1)}
    \\这是组合数学中的经典问题,假设我们现在的数是$a_1,a_2,...a_n$,我们找该序列最后一个极大点,设为$a_k$,我们知道$a_{k-1} \textless a_k$, $a_k \textgreater a_{k+1} \textgreater ... \textgreater a_n$,另外设$a_m$是$a_k, ..., a_n$中大于$a_{k-1}$的最小的那个,那么下一个数就是$a_1, a_2, ... a_m, a_k, a_{k+1}, ... , a_{m-1}, a_{k-1}, a_{m+1}, ... , a_n$
    \begin{lstlisting}
void nextPermutation(vector<int> &num) {
	int pos = num.size() - 1;
	while(pos > 0 && num[pos - 1] >= num[pos])	pos--;
	if(pos != 0){
		int pre = pos - 1;
		auto index = lower_bound(num.begin() + pos, num.end(), num[pre], greater<int>());
		swap(num[pre], num[index - num.begin() - 1]);
	}
	reverse(num.begin() + pos, num.end());
}
    \end{lstlisting}
\end{description}
