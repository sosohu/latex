    
\begin{description}
    \item{\textbf{问题}}:\\
Given two numbers represented as strings, return multiplication of the numbers as a string.\\
\textit{(leetcode 43)}
    \item{\textbf{Note}}:\\
Note: The numbers can be arbitrarily large and are non-negative.
    \item{\textbf{大数}} : \fbox{时间复杂度O(n*m), 空间复杂度O(n*m)}
    \\这题是典型的大数问题,对于大数问题一般采用vector或者string来存储大数,需要注意高低位的顺序,其运算方式完全按照小学学习的加减乘除计算方法计算就可以了.
    \begin{lstlisting}
// num1[0]为低位,num2[0]为低位, num1 = num1 + (num2 << left)
void bigAdd(string &num1, string &num2, int left){
	string fill = string(left, '0');
	num2 = fill + num2;
	int cin = 0;
	for(int i = 0; i < num1.size(); i++){
		int out = cin + (num1[i] - '0') + (num2[i] - '0');
		num1[i] = (out % 10) + '0';
		cin = out / 10;
	}
}

// num1[0]为低位
void bigMul(string &result, const string &num1, const char num2){
	int cin = 0;
	int mul = num2 - '0';
	for(int i = 0; i < num1.size(); i++){
		int out = cin + mul * (num1[i] - '0');
		result[i] = (out % 10) + '0';
		cin = out / 10;
	}
	if(cin)	result[num1.size()] = cin + '0';
}

string multiply(string num1, string num2) {
	int n1 = num1.size(), n2 = num2.size();
	if(n1 < n2)	return multiply(num2, num1);
	reverse(num1.begin(), num1.end());
	reverse(num2.begin(), num2.end());
	int n = n1 + n2;
	vector<string> line(n2, string(n, '0'));
	string result(n, '0');
	for(int i = 0; i < n2; i++){
		bigMul(line[i], num1, num2[i]);
		bigAdd(result, line[i], i);
	}
	reverse(result.begin(), result.end());
	int nozero = 0;
	while(nozero < result.size() && result[nozero] == '0')
		nozero++;
	return nozero == result.size()? "0" : result.substr(nozero, result.size() - nozero);
}
    \end{lstlisting}
\end{description}
