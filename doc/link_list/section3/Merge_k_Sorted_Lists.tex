\ifx allfiles undefined
\documentclass{article}
\usepackage{CJK}
\usepackage{verbatim}

%%%代码
\usepackage{color}
\usepackage{xcolor}
\definecolor{keywordcolor}{rgb}{0.8,0.1,0.5}
\usepackage{listings}
\lstset{breaklines}%这条命令可以让LaTeX自动将长的代码行换行排版
\lstset{extendedchars=false}%这一条命令可以解决代码跨页时,章节标题,页眉等汉字不显示的问题
\lstset{language=C++, %用于设置语言为C++
    keywordstyle=\color{keywordcolor} \bfseries, %设置关键词
    identifierstyle=,
    basicstyle=\ttfamily, 
    commentstyle=\color{blue} \textit,
    stringstyle=\ttfamily, 
    showstringspaces=false,
    %frame=shadowbox, %边框
    captionpos=b
}
%%%

%\hypersetup{CJKbookmarks=true} %解决section不能使用中文的问题

\begin{document}
\begin{CJK}{UTF8}{gbsn}     %CJK:支持中文

\else
    
\begin{description}
    \item{\textbf{问题}}: Merge k sorted linked lists and return it as one sorted list. Analyze and describe its complexity. \textit{(leetcode 23)}
    \item{\textbf{}} : \fbox{时间复杂度 , 空间复杂度O}
    \\这道题其实是Merge two list的延伸,我们在Merge两个链表时候,每次都是比较一下取较小的入新链表,当有k个链表的时候,我们也是取这k个节点最小的那个,然后用它的next替换它,再进行下次选取.我们想一下,最naive的做法是,每次这k个都比较一遍,那么这个每取一个的时间复杂度就是O(k),整个的时间复杂度就是O(k*k*n). 其实这里面的过程有一个信息被我们遗漏了,或者说是没有被利用好,那就是当我们取出这k个节点最小那个时候,其实除了最小的那个其他的节点之间大小关系也有比较过,比如是节点s和t,然后新替入的那个进来之后,再来一次取最小值,这时候其实s和t的大小关系我们是已知的,但是我们没有利用,而是有可能还会取比较一次! 为了避免这个重复比较,我们想到使用最小堆的方法: 把这个k个节点放入最小堆,然后取堆顶点,再加入新节点,整个操作只需要O(lgk)的时间复杂度,所以整个的时间复杂度降为O(k*nlgk)
    \begin{lstlisting}
ListNode *mergeKLists(vector<ListNode*> &lists) {
	multimap<int, int> data;
	ListNode ret(0) , *pos = &ret;
	for(int i = 0; i < lists.size(); i++){
		if(lists[i])
			data.insert(make_pair(lists[i]->val, i));
	}
	while(!data.empty()){
		int index = data.begin()->second;
		data.erase(data.begin());
		pos = pos->next = lists[index];
		lists[index] = lists[index]->next;
		if(lists[index])
			data.insert(make_pair(lists[index]->val, index));
	}
	return ret.next;
}
    \end{lstlisting}
	上面没有使用堆,但是使用的是tree map,它其实可以用来模拟堆操作,整个的时间复杂度和用堆是一样的.
\end{description}

\fi

\ifx allfiles undefined
\end{CJK}
\end{document}
\fi
