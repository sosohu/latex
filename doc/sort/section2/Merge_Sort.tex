\qquad Merge Sort这里分为递归和迭代两种实现,关于迭代的实现可以参考博客\href{https://sosohu.github.io/algorithm/2015/04/14/%E5%BD%92%E5%B9%B6%E6%8E%92%E5%BA%8F%E7%9A%84%E8%BF%AD%E4%BB%A3%E5%86%99%E6%B3%95.html}{归并排序的迭代写法}    
\begin{lstlisting}
//----------------------递归版merge sort---------
int min(int a, int b){
	return a < b? a : b;
}

void merge_aux(int A[], int begin, int end){
	if(begin >= end)	return;
	int mid = (begin + end)/2;
	merge_aux(A, begin, mid);
	merge_aux(A, mid+1, end);
	int result[end - begin + 1];
	int i = begin, j = mid+1, k = 0;
	while(i <= mid || j <= end){
		int first = i > mid? INT_MAX : A[i];
		int second = j > end? INT_MAX : A[j];
		if(first < second)	i++;
		else	j++;
		result[k++] = min(first, second);
	}
	for(int i = 0; i < k; i++)
		A[begin + i] = result[i];
}

void merge_sort(int A[], int n){
	merge_aux(A, 0, n-1);
}
//-----------------------------------------------
//----------------------迭代版merge sort---------
void merge_iteration(vector<int> &mix, vector<int> &cur){
	vector<int> sum;
	int i = 0, j = 0;
	while(i < mix.size() || j < cur.size()){
		int left = i == mix.size()? INT_MAX : mix[i];
		int right = j == cur.size()? INT_MAX : cur[j];
		if(left < right) i++;
		else	j++;
		sum.push_back(min(left, right));
	}
	cur.clear();
	mix = sum;
}

void merge_sort(int A[], int n){
	if(n <= 1)	return;
	vector<vector<int> > bucket(64, vector<int>());
	for(int i = 0; i < n; i++){
		int index = 0;
		vector<int> mix(1, A[i]);
		while(!bucket[index].empty()){
			merge_iteration(mix, bucket[index]);
			index++;
		}
		bucket[index] = mix;
	}
	vector<int> mix;
	for(int i = 0; i < 64; i++){
		if(!bucket[i].empty()){
			merge_iteration(mix, bucket[i]);
		}
	}
	for(int i = 0; i < mix.size(); i++)
		A[i] = mix[i];
}
\end{lstlisting}
