    
\begin{description}
    \item{\textbf{问题}}:\\
Given a triangle, find the minimum path sum from top to bottom. Each step you may move to adjacent numbers on the row below.\textit{(leetcode 120)}
    \item{\textbf{举例}}:\\
given the following triangle\\
\\
     2, \\
    3,4, \\
   6,5,7, \\
  4,1,8,3 \\
\\
The minimum path sum from top to bottom is 11 (i.e., 2 + 3 + 5 + 1 = 11). \\

    \item{\textbf{Note}}:Bonus point if you are able to do this using only O(n) extra space, where n is the total number of rows in the triangle.
    \item{\textbf{DP}} : \fbox{时间复杂度O($n^2$) , 空间复杂度O(n)}
    \\设动态规划变量为dp[i][j]表示处于triangle[i][j]到顶点的最短距离,那么target=min\{dp[n-1][j]\}
	\\dp的递推关系式:
$$
dp[i][j] =
\begin{cases} 
triangle[i][j] & i = 0  \\
dp[i-1][j] & j = 0  \\
dp[i-1][j-1] & j = triangle[i].size() - 1 \\
min(dp[i-1][j-1], dp[i-1][j]) + triangle[i][j] & other 
\end{cases}
$$
    \begin{lstlisting}
int minimumTotal(vector<vector<int> > &triangle) {
	int n = triangle.size();
	if( n == 0)	return 0;
	vector<int> odp(triangle[0]);
	vector<int> ndp(triangle[0]);
	for(int i = 1; i < n; i++){
		odp = ndp;
		ndp.clear();
		ndp.resize(i+1);
		for(int j = 0; j < triangle[i].size(); j++){
			ndp[j] = triangle[i][j] + min((j == 0? INT_MAX : odp[j-1]), 
							(j == triangle[i].size()-1? INT_MAX : odp[j]));
		}
	}
	auto iter = min_element(ndp.begin(), ndp.end());
	return *iter;
}
    \end{lstlisting}
	\textit{我们可以看到dp[i]行变量只依赖dp[i-1]变量,所以可以使用滚动数组将O($n^2$)空间复杂度降为O(n)}
\end{description}

