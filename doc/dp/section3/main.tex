\ifx allfiles undefined
\documentclass{article}
\usepackage{CJK}

%%%代码
\usepackage{color}
\usepackage{xcolor}
\definecolor{keywordcolor}{rgb}{0.8,0.1,0.5}
\usepackage{listings}
\lstset{breaklines}%这条命令可以让LaTeX自动将长的代码行换行排版
\lstset{extendedchars=false}%这一条命令可以解决代码跨页时,章节标题,页眉等汉字不显示的问题
\lstset{language=C++, %用于设置语言为C++
    keywordstyle=\color{keywordcolor} \bfseries, %设置关键词
    identifierstyle=,
    basicstyle=\ttfamily, 
    commentstyle=\color{blue} \textit,
    stringstyle=\ttfamily, 
    showstringspaces=false,
    %frame=shadowbox, %边框
    captionpos=b
}
%%%

%\hypersetup{CJKbookmarks=true} %解决section不能使用中文的问题

\begin{document}
\begin{CJK}{UTF8}{gbsn}     %CJK:支持中文

\else
    
\qquad动态规划算法还有很多应用场景,下面介绍几种常见的题目.
\\
\fi
\subsection{Longest Valid Parentheses}
    
\begin{description}
    \item{\textbf{问题}}:\\
Given a string containing just the characters '(' and ')', find the length of the longest valid (well-formed) parentheses substring.\\
For "(()", the longest valid parentheses substring is "()", which has length = 2.\\
Another example is ")()())", where the longest valid parentheses substring is "()()", which has length = 4. \textit{(leetcode 32)}
    \item{\textbf{DP}} : \fbox{时间复杂度O(n) , 空间复杂度O(n)}
    \\设动态规划变量为dp[i]表示以s[i]为开始的最长匹配串长度.以")()())"为例,dp[i]分别为0,4,0,2,0,0. 所以target = max\{dp[i]\}即为4.\\
	\\dp的递推关系式:
$$
dp[i] =
\begin{cases} 
0 & s[i]=')' \,||\, i=n-1  \\
2+dp[i+2] & s[i,i+1]="()" \\
0 & s[i,i+1]="((" \,\And\, i+dp[i+1]=n-1\\
0 & s[i,i+1]="((" \,\And\, s[i+dp[i+1]] = '('\\
2+dp[i+1]+dp[i+dp[i+1]+2] & s[i,i+1]="((" \,\And\, s[i+dp[i+1]] = ')'
\end{cases}
$$
    \begin{lstlisting}
int longestValidParentheses(string s) {
	int size = s.size();
	vector<int> dp(size + 1, 0);
	int max = 0;
	for(int i = size - 2; i >= 0; i--){
		if(s[i] == '('){
			if(s[i+1] == ')'){
				dp[i] = 2 + dp[i+2];
			}else{
				if(i + dp[i+1] < size - 1){
					if(s[i + dp[i+1] + 1] == ')')
						dp[i] = 2 + dp[i+1] + dp[i + dp[i+1] + 2];
				}
			}
		}
		if(max < dp[i])
			max = dp[i];
	}
	return max;
}
    \end{lstlisting}
    \textit{}
\end{description}


\subsection{Unique Paths}
    
\begin{description}
    \item{\textbf{问题}}:\\
A robot is located at the top-left corner of a m x n grid (marked 'Start' in the diagram below).\\
The robot can only move either down or right at any point in time. The robot is trying to reach the bottom-right corner of the grid (marked 'Finish' in the diagram below).\\
How many possible unique paths are there?\textit{(leetcode 62)}
    \item{\textbf{DP}} : \fbox{时间复杂度O(n*m) , 空间复杂度O(n*m)}
    \\设动态规划变量为dp[i][j]表示从起点到i,j位置的路径数目,那么target = dp[n-1][m-1]\\
	\\dp的递推关系式:
$$
dp[i][j] =
\begin{cases} 
1 & i = 0 \,||\, j = 0 \\
dp[i-1][j]+dp[i][j-1] & other
\end{cases}
$$
    \begin{lstlisting}
int uniquePaths(int m, int n) {
	if(m < 1 || n < 1)	return 0;
	vector<vector<int> > dp(m, vector<int>(n, 1));
	for(int i = 1; i < m; i++)
		for(int j = 1; j < n; j++)
			dp[i][j] = dp[i-1][j] + dp[i][j-1];
	return dp[m-1][n-1];
}
    \end{lstlisting}
\end{description}


\subsection{Unique Paths II}
    
\begin{description}
    \item{\textbf{问题}}:\\
Follow up for "Unique Paths":\\
Now consider if some obstacles are added to the grids. How many unique paths would there be?\\
An obstacle and empty space is marked as 1 and 0 respectively in the grid.\textit{(leetcode 63)}
    \item{\textbf{DP}} : \fbox{时间复杂度O(n*m) , 空间复杂度O(n*m)}
    \\设动态规划变量为dp[i][j]表示从起点到i,j位置的路径数目,那么target = dp[n-1][m-1]\\
	\\dp的递推关系式:
$$
dp[i][j] =
\begin{cases} 
0 & A[i][j] = 1 \\
1 & A[0][0] = 0 \,\And\, i = 0 \,\And\, j = 0 \\
dp[i][j-1] & A[i][j] = 0 \,\And\, i = 0 \,\And\, j != 0 \\
dp[i-1][j] & A[i][j] = 0 \,\And\, i != 0 \,\And\, j = 0 \\
dp[i-1][j]+dp[i][j-1] &  A[i][j] = 0 \,\And\, i != 0 \,\And\, j != 0
\end{cases}
$$
    \begin{lstlisting}
int uniquePathsWithObstacles(vector<vector<int> > &obstacleGrid) {
	int m = obstacleGrid.size();
	int n = obstacleGrid[0].size();
	vector<vector<int> > dp(m, vector<int>(n, 1));
	for(int i = 0; i < m; i++)
		for(int j = 0; j < n; j++){
			if(obstacleGrid[i][j])	dp[i][j] = 0;
			else{
				if(i + j != 0)
					dp[i][j] = (i > 0? dp[i-1][j] : 0) + (j > 0? dp[i][j-1] : 0);
			}
		}
	return dp[m-1][n-1];
}
    \end{lstlisting}
	\textit{这题和Unique Paths唯一的区别就是注意路障不能选}
\end{description}


\subsection{Minimum Path Sum}
    
\begin{description}
    \item{\textbf{问题}}:\\
Given a m x n grid filled with non-negative numbers, find a path from top left to bottom right which minimizes the sum of all numbers along its path.\textit{(leetcode 64)}
    \item{\textbf{Note}}:\\
You can only move either down or right at any point in time.
    \item{\textbf{DP}} : \fbox{时间复杂度O(n*m) , 空间复杂度O(n*m)}
    \\设动态规划变量为dp[i][j]表示从起点到i,j位置的最短路径长度,那么target = dp[n-1][m-1]\\
	\\dp的递推关系式:
$$
dp[i][j] =
\begin{cases} 
A[0][0] & i = 0 \,\And\, j = 0 \\
A[i][j]+dp[i][j-1] & i = 0 \,\And\, j != 0 \\
A[i][j]+dp[i-1][j] & i != 0 \,\And\, j = 0 \\
min(dp[i-1][j], dp[i][j-1]) + A[i][j] & other
\end{cases}
$$
    \begin{lstlisting}
int minPathSum(vector<vector<int> > &grid) {
	int n = grid.size();
	int m = grid[0].size();
	vector<vector<int> > dp(grid);
	for(int i = 0; i < n; i++)
		for(int j = 0; j < m; j++){
			if(i + j != 0){
				dp[i][j] = grid[i][j] + 
						min((i > 0? dp[i-1][j] : INT_MAX), (j > 0? dp[i][j-1] : INT_MAX));
			}
		}
	return dp[n-1][m-1];
}
    \end{lstlisting}
\end{description}


\subsection{Climbing Stairs}
    
\begin{description}
    \item{\textbf{问题}}:\\
You are climbing a stair case. It takes n steps to reach to the top.\\
Each time you can either climb 1 or 2 steps. In how many distinct ways can you climb to the top?\textit{(leetcode 70)}
    \item{\textbf{DP}} : \fbox{时间复杂度O(n) , 空间复杂度O(1)}
    \\其实本题就是著名的斐波那契数列问题,设动态规划变量为dp[i]为到阶梯i的所有可能的路数,那么target=dp[n-1].
	\\dp的递推关系式:
$$
dp[i] =
\begin{cases} 
1 & i = 0 \\
2 & i = 1 \\
dp[i-2]+dp[i-1] & other
\end{cases}
$$
    \begin{lstlisting}
int climbStairs(int n) {
	if(n <= 1)	return n;
	int fn1 = 2, fn2 = 1, fn = 2;
	for(int i = 3; i <= n; i++){
		fn = fn1 + fn2;
		fn2 = fn1;
		fn1 = fn;
	}
	return fn;
}
    \end{lstlisting}
	\textit{虽然我们递推公式中是使用dp[i]一个数组来解释,但是因为dp[i]只依赖dp[i-1]和dp[i-2],那么可以压缩状态存储空间到O(1),这是节约空间的一种常用手段,像轮转数组等等}
\end{description}


\subsection{Scramble String}
    
\begin{description}
    \item{\textbf{问题}}:\\
Given a string s1, we may represent it as a binary tree by partitioning it to two non-empty substrings recursively.\\
Below is one possible representation of s1 = "great":\\
\\
    great\\
   /     $\backslash$ \\
  gr    eat \\
 /  $\backslash$    /   $\backslash$ \\
g   r  e   at \\
           /  $\backslash$ \\
          a   t \\
To scramble the string, we may choose any non-leaf node and swap its two children. \\
\\
For example, if we choose the node "gr" and swap its two children, it produces a scrambled string "rgeat".\\
\\
    rgeat \\
   /     $\backslash$ \\
  rg    eat \\
 /  $\backslash$    /   $\backslash$ \\
r   g  e   at \\
           /  $\backslash$ \\
          a   t \\
We say that "rgeat" is a scrambled string of "great".\\
\\
Similarly, if we continue to swap the children of nodes "eat" and "at", it produces a scrambled string "rgtae".\\
\\
    rgtae\\
   /     $\backslash$ \\
  rg    tae \\
 /  $\backslash$    /   $\backslash$ \\
r   g  ta  e \\
       /  $\backslash$ \\
      t   a \\
We say that "rgtae" is a scrambled string of "great". \\
\\
Given two strings s1 and s2 of the same length, determine if s2 is a scrambled string of s1.\textit{(leetcode 87)}
    \item{\textbf{DP}} : \fbox{时间复杂度O($n^4$) , 空间复杂度O($n^3$)}
	\\我们假设dp[i][j][l]表示$s1_i, ... s1_{i+l-1}$和$s2_j, ... s2_{j+l-1}$是否互为可转置的.\\
	\\dp的递推关系式:
$$
dp[i][j][l] =
\begin{cases} 
s1[i] == s2[j] & l = 1 \\
\prod_{k=1}^{l-1}(dp[i][j][k] \And dp[i+k][j+k][l-k])||(dp[i][j+l-k][k] \And dp[i+k][j][l-k]) & other
\end{cases}
$$
其中上面的求积符号代表或运算
    \begin{lstlisting}
bool isScramble(string s1, string s2) {
	int n = s1.size();
	if(n == 0 || s1.size() != s2.size() ) return false;
	vector<vector<vector<bool> > > dp(n, vector<vector<bool> >(n, vector<bool>(n+1, false)));
	for(int i = 0; i < n; i++)
		for(int j = 0; j < n; j++)
			dp[i][j][1] = s1[i] == s2[j];
	for(int l = 2; l < n+1; l++){
		for(int i = 0; i <= n - l; i++)
			for(int j = 0; j <= n - l; j++){
				for(int k = 1; k < l; k++){
					dp[i][j][l] = dp[i][j][l] 
								|| (dp[i][j][k] && dp[i+k][j+k][l-k])
								|| (dp[i][j+l-k][k] && dp[i+k][j][l-k]);
				}
			}
	}
	return dp[0][0][n];
}
    \end{lstlisting}
	\textit{这道题困难之处在于敢不敢这么简单粗暴的递推}
\end{description}



\ifx allfiles undefined
\end{CJK}
\end{document}
\fi
