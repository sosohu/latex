    
\begin{description}
    \item{\textbf{问题}}:\\
Given s1, s2, s3, find whether s3 is formed by the interleaving of s1 and s2.\textit{(leetcode 97)}

    \item{\textbf{举例}}:\\
Given:\\
s1 = "aabcc",\\
s2 = "dbbca",\\
\\
When s3 = "aadbbcbcac", return true.\\
When s3 = "aadbbbaccc", return false.\\
    \item{\textbf{DP}} : \fbox{时间复杂度O(n*m) , 空间复杂度O(n*m)}
	\\我们假设dp[i][j]表示s1[0..i-1]与s2[0...j-1]是否能交叉构造出s3[0..i+j-1]\\
	\\dp的递推关系式:
$$
dp[i][j] =
\begin{cases} 
s2[0..j-1] == s3[0..j-1] & i = 0 \\
s1[0..i-1] == s3[0..i-1] & j = 0 \\
(s1[i-1] == s3[i+j-1] \And dp[i-1][j]) || (s2[j-1] == s3[i+j-1] \And dp[i][j-1]) & other
\end{cases}
$$
    \begin{lstlisting}
bool isInterleave(string s1, string s2, string s3) {
	int n1 = s1.size(), n2 = s2.size(), n3 = s3.size();
	if(n1 + n2 != n3) return false;
	vector<vector<int> > dp(n1+1, vector<int>(n2+1, false));
	dp[0][0] = true;
	for(int i = 0; i < n1; i++)
		if(s1[i] == s3[i])	dp[i+1][0] = true;
		else	break;
	for(int i = 0; i < n2; i++)
		if(s2[i] == s3[i])	dp[0][i+1] = true;
		else	break;
	for(int i = 0; i < n1; i++)
		for(int j = 0; j < n2; j++){
			dp[i+1][j+1] = (s1[i] == s3[i+j+1] && dp[i][j+1])
							|| (s2[j] == s3[i+j+1] && dp[i+1][j]);
		}
	return dp[n1][n2];
}
    \end{lstlisting}
\end{description}

