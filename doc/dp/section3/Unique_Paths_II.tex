    
\begin{description}
    \item{\textbf{问题}}:\\
Follow up for "Unique Paths":\\
Now consider if some obstacles are added to the grids. How many unique paths would there be?\\
An obstacle and empty space is marked as 1 and 0 respectively in the grid.\textit{(leetcode 63)}
    \item{\textbf{DP}} : \fbox{时间复杂度O(n*m) , 空间复杂度O(n*m)}
    \\设动态规划变量为dp[i][j]表示从起点到i,j位置的路径数目,那么target = dp[n-1][m-1]\\
	\\dp的递推关系式:
$$
dp[i][j] =
\begin{cases} 
0 & A[i][j] = 1 \\
1 & A[0][0] = 0 \,\And\, i = 0 \,\And\, j = 0 \\
dp[i][j-1] & A[i][j] = 0 \,\And\, i = 0 \,\And\, j != 0 \\
dp[i-1][j] & A[i][j] = 0 \,\And\, i != 0 \,\And\, j = 0 \\
dp[i-1][j]+dp[i][j-1] &  A[i][j] = 0 \,\And\, i != 0 \,\And\, j != 0
\end{cases}
$$
    \begin{lstlisting}
int uniquePathsWithObstacles(vector<vector<int> > &obstacleGrid) {
	int m = obstacleGrid.size();
	int n = obstacleGrid[0].size();
	vector<vector<int> > dp(m, vector<int>(n, 1));
	for(int i = 0; i < m; i++)
		for(int j = 0; j < n; j++){
			if(obstacleGrid[i][j])	dp[i][j] = 0;
			else{
				if(i + j != 0)
					dp[i][j] = (i > 0? dp[i-1][j] : 0) + (j > 0? dp[i][j-1] : 0);
			}
		}
	return dp[m-1][n-1];
}
    \end{lstlisting}
	\textit{这题和Unique Paths唯一的区别就是注意路障不能选}
\end{description}

