    
\begin{description}
    \item{\textbf{问题}}: \\
Say you have an array for which the ith element is the price of a given stock on day i. \\
\\
Design an algorithm to find the maximum profit. You may complete at most k transactions. \\
\textit{(leetcode 188)}
    \item{\textbf{Note}}: \\
You may not engage in multiple transactions at the same time (ie, you must sell the stock before you buy again).
    \item{\textbf{DP}} : \fbox{时间复杂度O(kn), 空间复杂度O(n)}
    \\首先需要考虑一种特殊情况: $k > price.size() / 2$, 这种情况就是等同于你可以购买无数次,也就是你需要把每个增长区间都计算进去就可以了.这样就可以在O(n)时间解决这种情况而不是O(kn).
	\\对于其他情况,我们假设left[i]表示prices[i]的左边连续最小值的位置,dp[i][p]表示至多购买p次并且以prices[i]为最后一次卖出点的最大利润,r[i][p]表示至多购买p次并且最后一次卖出点在prices[i]之前的最大利润,所以target = r[n-1][k]
	\\我们可以有递推公式
$$
left[i] =
\begin{cases} 
0 & i = 0 \\
left[i-1] & prices[i] >= prices[i-1] \\
i & prices[i] < prices[i-1] \\
\end{cases}
$$
$$
dp[i][p] =
\begin{cases} 
prices[i] - prices[0] & left[i] = 0 \\
max(prices[i] - prices[j] + r[j-1][p-1], prices[i] - prices[j-1] + dp[j-1][p]) & other \\
\end{cases}
$$
\\其中$j = left[i]$
$$
r[i][p] =
\begin{cases} 
0 & p = 0 || r = 0 \\
max(r[i-1][p], dp[i][p]) & other \\
\end{cases}
$$
\\从递推公式可以看出,dp和r可以状态压缩以节省空间
    \begin{lstlisting}
int maxProfit(int k, vector<int>& prices) {
	int n = prices.size();
	// simple case, very important
	if (k > n/2){ 
		int ans = 0;
		for (int i=1; i<n; ++i){
			ans += max(prices[i] - prices[i-1],0);
		}
		return ans;
	}
	k = min(k, n/2);
	if(n <= 1)	return 0;
	vector<int> dp(n, 0), r(n, 0), left(n,0);
	for(int i = 1; i < n; i++){
		if(prices[i] >= prices[i-1]){
			left[i] = left[i-1];
		}
		else{
			left[i] = i;
		}
	}
	for(int p = 1; p <= k; p++){
		for(int i = 1; i < n; i++){
			int j = left[i];
			if(j == 0)	dp[i] = prices[i] - prices[0];
			else
				dp[i] = max(prices[i] - prices[j-1] + dp[j-1],
							prices[i] - prices[j] + r[j-1]);
		}
		for(int i = 1; i < n; i++)
			r[i] = max(r[i-1], dp[i]);
	}
	return r[n-1];
}
    \end{lstlisting}
\end{description}

