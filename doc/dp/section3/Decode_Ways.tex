    
\begin{description}
    \item{\textbf{问题}}:\\
A message containing letters from A-Z is being encoded to numbers using the following mapping:\\
\\
'A' $\rightarrow$ 1\\
'B' $\rightarrow$ 2\\
...\\
'Z' $\rightarrow$ 26\\
Given an encoded message containing digits, determine the total number of ways to decode it.\textit{(leetcode 91)}

    \item{\textbf{举例}}:\\
Given encoded message "12", it could be decoded as "AB" (1 2) or "L" (12).\\
The number of ways decoding "12" is 2.\\
    \item{\textbf{DP}} : \fbox{时间复杂度O(n) , 空间复杂度O(n)}
	\\本题有两个需要注意的地方一个是检测非法输入,比如00这样的. 另外就是算出多少种解密方法.
\begin{itemize}
	\item{检测非法输入}: 非法输入的检测使用NFA自动机来实现,这样是最优美和最简单的,状态转移图如下,其中起始状态为0,非法状态为3:\\ \\ \\
$
\psmatrix[mnode=circle,colsep=1]
0 & 1 \\
3 & 2 \\
\endpsmatrix
$
\psset{shortput=nab,arrows=->,labelsep=3pt, nodesep=3pt}
\small
\ncline{1,1}{2,1}
\nbput*{0}
\ncarc[arcangle=45]{1,1}{1,2}
\naput*{1-2}
\ncline{1,1}{2,2}
\nbput*[npos=.4]{3-9}
\nccircle{1,2}{.3cm}
\nbput*{1-2}
\ncline{1,2}{1,1}^{0}
\ncarc[arcangle=45]{1,2}{2,2}
\naput*{3-9}
\nccircle{2,2}{.3cm}
\nbput*{3-9}
\ncarc[arcangle=45]{2,2}{1,2}
\naput*[npos=.2]{1-2}
\ncline{2,2}{2,1}^{0}
	\item{计算解密数}:设dp[i]表示s[0..i-1]可解密的数目,那么我们有递推公式如下:\\
$$
dp[i] =
\begin{cases} 
1 & i = 0 \\
dp[i-2] & i != 0 \And s[i-1] = '0' \\
dp[i-1] + dp[i-2] & i > 1 \And s[i-2] = '1' || (s[i-2] = '2' \And s[i-1] <= '6') \\
dp[i-1] & other
\end{cases}
$$
\end{itemize}
    \begin{lstlisting}
bool isValid(string &s){
	int size = s.size();
	int dfa[4][10] = {
		{3,1,1,2,2,2,2,2,2,2},
		{0,1,1,2,2,2,2,2,2,2},
		{3,1,1,2,2,2,2,2,2,2},
		{3,3,3,3,3,3,3,3,3,3}
	};
	int status = 0;
	for(int i = 0; i < size; i++)
		status = dfa[status][s[i]-'0'];
	return status != 3;
}

int numDecodings(string s) {
	int size = s.size();
	if(size == 0 || s[0] == '0' || !isValid(s))	return 0;
	vector<int> dp(size+1, 1);
	for(int i = 1; i < size; i++){
		if(s[i] == '0')
			dp[i+1] = dp[i-1];
		else
			if(s[i-1] == '1' || (s[i-1] == '2' && s[i] <= '6'))
				dp[i+1] = dp[i] + dp[i-1];
			else
				dp[i+1] = dp[i];
	}
	return dp[size];
}
    \end{lstlisting}
	\textit{这题递推公式很简单,但是考虑到非法输入比较麻烦,最简单的做法就是先检测一边,使用有限自动机是最优雅的检测手段}
\end{description}

