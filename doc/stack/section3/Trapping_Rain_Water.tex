    
\begin{description}
    \item{\textbf{问题}}:\\
Given n non-negative integers representing an elevation map where the width of each bar is 1, compute how much water it is able to trap after raining.\\
\textit{(leetcode 42)}
    \item{\textbf{举例}}:\\
Given [0,1,0,2,1,0,1,3,2,1,2,1], return 6.
\includegraphics{stack/section3/rainwatertrap.eps}
    \item{\textbf{栈}} : \fbox{时间复杂度O(n), 空间复杂度O(n)}
    \\这道题可以这样想,依次遍历$a_1, a_2, ..., a_n$如果$a_i \le a_{i-1}$那么加入到栈中;反之,进行弹栈处理,因为$a_i > a_{i-1} \le a_{i-2}$所以可以计算$a_{i-1}$处的积水,依次类推.
    \begin{lstlisting}
int trap(vector<int>& height) {
	if(height.size() <= 1)	return 0;
	stack<pair<int,int> > s;
	s.push(make_pair(height[0], 0));
	int sum = 0;
	pair<int,int> cur;
	for(int i = 1; i < height.size(); i++){
		while(height[i] > s.top().first){
			cur = s.top();
			s.pop();
			if(!s.empty())	
				sum += (min(height[i], s.top().first) - cur.first) 
						* (i - s.top().second - 1);
			else	break;
		}
		s.push(make_pair(height[i], i));
	}
	return sum;
}
    \end{lstlisting}
    \item{\textbf{DP}} : \fbox{时间复杂度O(n), 空间复杂度O(n)}
    \\其实栈的方法比较麻烦,这里推荐一种更简单直接的做法,我们来考虑每个元素$a_i$它能盛放的水量:$V_i$,我们可以轻易的知道$V_i = min(left[i], right[i])$,其中$left[i],right[i]$分别是$a_i$左右连续最大值.
	\\那么,我们可以通过动态规划计算$left_i和right_i$:
$$
left[i] =
\begin{cases} 
a_0 & i = 0 \\
max(a_i, left[i-1]) & other
\end{cases}
$$
$$
right[i] =
\begin{cases} 
a_{n-1} & i = n-1 \\
max(a_i, right[i+1]) & other
\end{cases}
$$
	\\最终$target = \sum_{i=0}^{n-1}V_i$
    \begin{lstlisting}
int trap(int A[], int n) {
	vector<int> left(n, 0), right(n, 0);
	left[0] = A[0];
	right[n-1] = A[n-1];
	for(int i = 1; i < n; i++){
		A[i] = A[i] < left[i-1]? left[i-1] : A[i];
	}

	for(int i = n - 2; i >= 0; i--){
		A[i] = A[i] < right[i+1]? right[i+1] : A[i];
	}

	int sum = 0;
	for(int i = 0; i < n; i++){
		min = left[i] < right[i]? left[i] : right[i];
		sum = sum + min - A[i];
	}
	return sum;
}
    \end{lstlisting}
\end{description}
