    
\begin{description}
    \item{\textbf{问题}}:\\
Evaluate the value of an arithmetic expression in Reverse Polish Notation.\\
\\
Valid operators are +, -, *, /. Each operand may be an integer or another expression.\\
\textit{(leetcode 150)}
    \item{\textbf{举例}}:\\
  $["2", "1", "+", "3", "*"] \rightarrow ((2 + 1) * 3) -> 9$
  $["4", "13", "5", "/", "+"] \rightarrow (4 + (13 / 5)) -> 6$
    \item{\textbf{栈}} : \fbox{时间复杂度O(n), 空间复杂度O(n)}
    \\逆波兰式是典型的栈式问题.
    \begin{lstlisting}
bool scanNum(const string &str, int& num){
	if(str.size() == 1 && (str[0] == '+' || str[0] == '-' 
		|| str[0] == '*' || str[0] == '/') )
		return false;
	bool isMunis = false;
	if(str[0] == '-')	isMunis = true;
	num = 0;
	for(int i = 0; i < str.size(); i++)
		if(str[i] != '+' && str[i] != '-')
			num = num*10 + str[i] - '0';
	num = isMunis? -num : num;
	return true;
}

int evalRPN(vector<string>& tokens) {
	int result = 0, num = 0;
	if(tokens.size() == 0)	return 0;
	scanNum(tokens[0], num);
	if(tokens.size() <= 2)	return num;
	stack<int> s;
	s.push(num);
	scanNum(tokens[1], num);
	s.push(num);
	for(int i = 2; i < tokens.size(); i++){
		if(!scanNum(tokens[i], num)){
			int first = s.top(); s.pop();
			int second = s.top(); s.pop();
			switch(tokens[i][0]){
				case '+': s.push(second + first); break;
				case '-': s.push(second - first); break;
				case '*': s.push(second * first); break;
				case '/': s.push(second / first); break;
			}
		}else{
			s.push(num);
		}
	}
	result = s.top();
	return result;
}
    \end{lstlisting}
\end{description}
