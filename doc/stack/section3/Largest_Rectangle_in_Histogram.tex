    
\begin{description}
    \item{\textbf{问题}}:\\
Given n non-negative integers representing the histogram's bar height where the width of each bar is 1, find the area of largest rectangle in the histogram.\\
\textit{(leetcode 84)}
    \item{\textbf{举例}}:\\
Given height = $[2,1,5,6,2,3]$, \\
Above is a histogram where width of each bar is 1, given height = $[2,1,5,6,2,3]$.\\
\includegraphics{stack/section3/histogram.eps} \\
The largest rectangle is shown in the shaded area, which has area = 10 unit. \\
\includegraphics{stack/section3/histogram_area.eps} \\
return 10.
    \item{\textbf{栈}} : \fbox{时间复杂度O(n), 空间复杂度O(n)}
    \\我们可以这样考虑这个问题,我么依次扫描$a_1, a_2, ..., a_n$如果$a_i \ge a_{i-1}$就将$a_i$加入到栈中;反之,我们就弹栈处理$a_{i-1}$,因为$a_{i-2} < a_{i-1} > a_i$所以我们可以计算以$a_{i-1}$为基准的histogram大小,依次类推.
    \begin{lstlisting}
int largestRectangleArea(vector<int> &height) {
	if(height.size() == 0)	return 0;
	if(height.size() == 1)	return height[0];
	stack<pair<int, int> > s;
	s.push(make_pair(height[0], 0));
	int result = INT_MIN;
	for(int i = 1; i < height.size(); i++){
		while(height[i] < s.top().first){
			pair<int, int> cur = s.top();
			s.pop();
			if(!s.empty()){
				if(result < cur.first * (i - s.top().second - 1))
					result = cur.first * (i - s.top().second - 1);
			}else{
				if(result < cur.first * i)
					result = cur.first * i;
				break;
			}
		}
		s.push(make_pair(height[i], i));
	}
	int index = s.top().second;
	while(!s.empty()){
		pair<int, int> cur = s.top();
		s.pop();
		if(!s.empty()){
			if(result < cur.first * (index - s.top().second))
				result = cur.first * (index - s.top().second);
		}else{
			if(result < cur.first * (index + 1))
				result = cur.first * (index + 1);
		}
	}
	return result;
}
    \end{lstlisting}
    \item{\textbf{DP}} : \fbox{时间复杂度O(n), 空间复杂度O(n)}
	\\同样类似前面的Traip Water问题,分别计算$left[i]和right[i]$即可
    \begin{lstlisting}
int largestRectangleArea_1st(vector<int> &height) {
	int len = height.size();
	if(len == 0)	return 0;

	vector<int> left(len, 0), right(len, 0);

	for(int i = 1; i < len; i++){
		for(int j = i - 1; j >=0; j--){
			if(height[i] <= height[j]){
				left[i] = left[i] + left[j] + 1;
				j = j - left[j];
			}else{
				break;
			}
		}
	}
	for(int i = len - 2; i >= 0; i--){
		for(int j = i+1; j < len; j++){
			if(height[i] <= height[j]){
				right[i] = right[i] + right[j] + 1;
				j = j + right[j];
			}else{
				break;
			}
		}
	}
	int data = 0, tmp;

	for(int i = 0; i < len; i++){
		tmp = height[i] * (left[i] + right[i] + 1);
		if(data < tmp){
			data = tmp;
		}
	}
	return data;
}
    \end{lstlisting}
\end{description}
