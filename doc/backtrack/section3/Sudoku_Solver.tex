    
\begin{description}
    \item{\textbf{问题}}: \\
Write a program to solve a Sudoku puzzle by filling the empty cells.\\
Empty cells are indicated by the character '.'.\\
You may assume that there will be only one unique solution.\\
\textit{(leetcode 37)}
	\item{\textbf{举例}}:\\
\includegraphics{backtrack/section3/250px-Sudoku-by-L2G-20050714.svg.eps}
A sudoku puzzle...
\includegraphics{backtrack/section3/250px-Sudoku-by-L2G-20050714_solution.svg.eps}
    \item{\textbf{回溯}} : \fbox{时间复杂度O(9!8!7!...1!) , 空间复杂度O(1)}
    \\时间复杂度也是远远没有O(9!8!7!...1!)这么多
    \begin{lstlisting}
unordered_set<char> row[9]; 
unordered_set<char> col[9]; 
unordered_set<char> area[9]; 

bool dfs(vector<vector<char> > &board, int i, int j){
	if(i == 9)	return true;
	if(board[i][j] != '.'){
		i = j == 8? i+1 : i;
		j = j == 8? 0 : j+1;
		return dfs(board, i, j);
	}else{
		for(int k = 1; k <= 9; k++){
			char cur = '0' + k;
			if(!row[i].count(cur) && !col[j].count(cur) 
				&& !area[(i/3)*3 + j/3].count(cur)){
				row[i].insert(cur);
				col[j].insert(cur);
				area[(i/3)*3 + j/3].insert(cur);
				board[i][j] = cur;
				i = j == 8? i+1 : i;
				j = j == 8? 0 : j+1;
				if(dfs(board, i, j))	return true;
				i = j == 0? i-1 : i;
				j = j == 0? 8 : j-1;
				row[i].erase(cur);
				col[j].erase(cur);
				area[(i/3)*3 + j/3].erase(cur);
			}
		}
		board[i][j] = '.';
		return false;
	}
}

void solveSudoku(vector<vector<char> > &board) {
	for(int i = 0; i < 9; i++){
		row[i].clear();
		col[i].clear();
		area[i].clear();
	}
	for(int i = 0; i < 9; i++)
		for(int j = 0; j < 9; j++){
			if(board[i][j] != '.'){
				row[i].insert(board[i][j]);
				col[j].insert(board[i][j]);
				area[(i/3)*3+j/3].insert(board[i][j]);
			}
		}
	dfs(board, 0, 0);
}
    \end{lstlisting}
\end{description}
