    
\begin{description}
    \item{\textbf{问题}}:\\
The gray code is a binary numeral system where two successive values differ in only one bit.\\
Given a non-negative integer n representing the total number of bits in the code, print the sequence of gray code. A gray code sequence must begin with 0.\\
\textit{(leetcode 89)}
    \item{\textbf{举例}}:\\
Given n = 2, return [0,1,3,2]. Its gray code sequence is:\\
\\
00 - 0\\
01 - 1\\
11 - 3\\
10 - 2
    \item{\textbf{Note}}:\\
For a given n, a gray code sequence is not uniquely defined.\\
For example, [0,2,3,1] is also a valid gray code sequence according to the above definition.\\
For now, the judge is able to judge based on one instance of gray code sequence. Sorry about that.
    \item{\textbf{回溯}} : \fbox{时间复杂度O($2^n$), 空间复杂度O($2^n$)}
    \\这道题其实和回溯没多大关系,我的解法就是从第1位开始确定数据,接着确定第二位,...第三位... 第k位的数据就是第k-1位数据集合并上第k-1位数据集合的反序集合.从这个角度上看也算是回溯吧.
    \begin{lstlisting}
vector<int> grayCode(int n) {
	vector<int> result;
	if(n < 0 || n > 31)	return result;
	if(n == 0) return vector<int>(1, 0);
	vector<int> last{0, 1};
	result = last;
	for(int i = 1; i < n; i++){
		last = result;
		reverse(last.begin(), last.end());
		for(int j = 0; j < last.size(); j++){
			last[j] |= (0x1 << i);
			result.push_back(last[j]);
		}
	}
	return result;
}
    \end{lstlisting}
\end{description}
