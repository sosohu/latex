
\qquad回溯法可以解决很多需要枚举才能确定解的问题,一般来说会有$a_1, a_2, ..., a_n$的值需要确定,且这些$a_i$还要满足某种约束. 我们采用的办法是: 从$a_1$开始分别枚举它的所有可能值,然后采用DFS的做法,依次确定$a_2, ....$,如果确定到$a_n$,那么这组枚举解是合理解,就放入到搜索结果中,如果枚举到某个$a_i$发现它已经没有可用解时候,就需要回溯.

\qquad另外,我的编码风格是使用result存放最终解,而trace保存一路DFS下来的状态信息,每次进入一个新节点时候trace要压入新节点的状态,每次离开该节点时候就需要pop刚才压入的状态.
\subsection{Letter Combinations of a Phone Number}
    
\begin{description}
    \item{\textbf{问题}}:\\
Given a digit string, return all possible letter combinations that the number could represent.\\
A mapping of digit to letters (just like on the telephone buttons) is given below.\\

\includegraphics{backtrack/section3/LetterCombinations.eps}

\textit{(leetcode 17)}
	\item{\textbf{举例}}:\\
Input:Digit string "23"\\
Output: ["ad", "ae", "af", "bd", "be", "bf", "cd", "ce", "cf"].\\
	\item{\textbf{Note}}:\\
Although the above answer is in lexicographical order, your answer could be in any order you want.
    \item{\textbf{回溯}} : \fbox{时间复杂度O($3^n$), 空间复杂度O(n)}
    \begin{lstlisting}
string table[] = {" ","", "abc", "def", "ghi", "jkl", "mno", "pqrs", "tuv", "wxyz"};

void dfs(vector<string> &result, string& digits, string& trace, int pos){
	if(pos == digits.size()){
		result.push_back(trace);
		return;
	}
	string& all = table[digits[pos] - '0'];
	for(int i = 0; i <  all.size(); i++){
		trace.push_back(all[i]);
		dfs(result, digits, trace, pos+1);
		trace.erase(trace.size() - 1);
	}
}

vector<string> letterCombinations(string digits) {
	vector<string> result;
	int n = digits.size();
	if(n == 0)	return result;
	string trace;
	dfs(result, digits, trace, 0);	
	return result;
}
    \end{lstlisting}
    \textit{}
\end{description}

\subsection{Generate Parentheses}
    
\begin{description}
    \item{\textbf{问题}}:\\
Given n pairs of parentheses, write a function to generate all combinations of well-formed parentheses.\textit{(leetcode 22)}
	\item{\textbf{举例}}:\\
Given n = 3, a solution set is:\\
"((()))", "(()())", "(())()", "()(())", "()()()"
    \item{\textbf{回溯}} : \fbox{时间复杂度O($2^n$) , 空间复杂度O(n)}
    \\这个时间复杂度也是会远低于O($2^n$)
    \begin{lstlisting}
void dfs(vector<string> &result, int n, string& trace, int left){
	if(left == -1 || (n == 0 && left != 0))	return;
	if(n == 0){
		result.push_back(trace);
		return ;
	}
	trace.push_back('(');
	dfs(result, n-1, trace, left+1);
	trace[trace.size()-1] = ')';
	dfs(result, n-1, trace, left - 1);
	trace.erase(trace.size()-1);
}

vector<string> generateParenthesis(int n) {
	vector<string> result;
	string trace;
	int left = 0;
	dfs(result, 2*n, trace, left);
	return result;
}
    \end{lstlisting}
\end{description}

\subsection{Sudoku Solver}
    
\begin{description}
    \item{\textbf{问题}}: \\
Write a program to solve a Sudoku puzzle by filling the empty cells.\\
Empty cells are indicated by the character '.'.\\
You may assume that there will be only one unique solution.\\
\textit{(leetcode 37)}
	\item{\textbf{举例}}:\\
\includegraphics{backtrack/section3/250px-Sudoku-by-L2G-20050714.svg.eps}
A sudoku puzzle...
\includegraphics{backtrack/section3/250px-Sudoku-by-L2G-20050714_solution.svg.eps}
    \item{\textbf{回溯}} : \fbox{时间复杂度O(9!8!7!...1!) , 空间复杂度O(1)}
    \\时间复杂度也是远远没有O(9!8!7!...1!)这么多
    \begin{lstlisting}
unordered_set<char> row[9]; 
unordered_set<char> col[9]; 
unordered_set<char> area[9]; 

bool dfs(vector<vector<char> > &board, int i, int j){
	if(i == 9)	return true;
	if(board[i][j] != '.'){
		i = j == 8? i+1 : i;
		j = j == 8? 0 : j+1;
		return dfs(board, i, j);
	}else{
		for(int k = 1; k <= 9; k++){
			char cur = '0' + k;
			if(!row[i].count(cur) && !col[j].count(cur) 
				&& !area[(i/3)*3 + j/3].count(cur)){
				row[i].insert(cur);
				col[j].insert(cur);
				area[(i/3)*3 + j/3].insert(cur);
				board[i][j] = cur;
				i = j == 8? i+1 : i;
				j = j == 8? 0 : j+1;
				if(dfs(board, i, j))	return true;
				i = j == 0? i-1 : i;
				j = j == 0? 8 : j-1;
				row[i].erase(cur);
				col[j].erase(cur);
				area[(i/3)*3 + j/3].erase(cur);
			}
		}
		board[i][j] = '.';
		return false;
	}
}

void solveSudoku(vector<vector<char> > &board) {
	for(int i = 0; i < 9; i++){
		row[i].clear();
		col[i].clear();
		area[i].clear();
	}
	for(int i = 0; i < 9; i++)
		for(int j = 0; j < 9; j++){
			if(board[i][j] != '.'){
				row[i].insert(board[i][j]);
				col[j].insert(board[i][j]);
				area[(i/3)*3+j/3].insert(board[i][j]);
			}
		}
	dfs(board, 0, 0);
}
    \end{lstlisting}
\end{description}

\subsection{Combination Sum}
    
\begin{description}
    \item{\textbf{问题}}:\\
Given a set of candidate numbers (C) and a target number (T), find all unique combinations in C where the candidate numbers sums to T.\\
The same repeated number may be chosen from C unlimited number of times.\\
\textit{(leetcode 39)}
    \item{\textbf{举例}}:\\
Given candidate set 2,3,6,7 and target 7, \\
A solution set is: \\
(7) \\
(2, 2, 3) 
	\item{\textbf{Note}}:\\
All numbers (including target) will be positive integers.\\
Elements in a combination (a1, a2, … , ak) must be in non-descending order. (ie, a1 ≤ a2 ≤ … ≤ ak).\\
The solution set must not contain duplicate combinations.
    \item{\textbf{回溯}} : \fbox{时间复杂度O($2^n$), 空间复杂度O(n)}
    \\这题就是对于每个值包含还是不包含两种选择,然后进行回溯算法.这里需要注意的就是遇到一个值有多个的情况,这里因为每个值可以用无数次,所以就可以直接把原数组数据去重就可以了,具体办法可以每次枚举完$a_i$后前向枚举下一个第一个不等于$a_i$的那个.
    \begin{lstlisting}
void backtrack(vector<int> &num, int target, int pos, 
			vector<vector<int> > &result, vector<int> &track){
	if(target == 0){
		result.push_back(track);
		return;
	}
	if(pos == -1)	return;
	if(num[pos] <= target){
		track.push_back(num[pos]);
		backtrack(num, target - num[pos], pos, result, track);
		track.pop_back();
	}
	pos--;
	while(pos >= 0 && num[pos] == num[pos+1]) pos--;
	if(pos >= 0){
		backtrack(num, target, pos, result, track);
	}
}

vector<vector<int> > combinationSum(vector<int> &nums, int target) {
	sort(nums.begin(), nums.end(), greater<int>());
	vector<vector<int> > result;
	vector<int> track;
	backtrack(nums, target, nums.size()-1, result, track);
	return result;
}
    \end{lstlisting}
\end{description}

\subsection{Combination Sum II}
    
\begin{description}
    \item{\textbf{问题}}:\\
Given a collection of candidate numbers (C) and a target number (T), find all unique combinations in C where the candidate numbers sums to T.\\
Each number in C may only be used once in the combination.\\
\textit{(leetcode 40)}
    \item{\textbf{Note}}:\\
All numbers (including target) will be positive integers.\\
Elements in a combination (a1, a2, … , ak) must be in non-descending order. (ie, a1 ≤ a2 ≤ … ≤ ak).\\
The solution set must not contain duplicate combinations.\\
    \item{\textbf{举例}}:\\
For example, given candidate set 10,1,2,7,6,1,5 and target 8, \\
A solution set is: \\
(1, 7) \\
(1, 2, 5)\\ 
(2, 6) \\
(1, 1, 6)\\ 
    \item{\textbf{回溯}} : \fbox{时间复杂度O($2^n$) , 空间复杂度O(n)}
    \\这题就是对于每个值包含还是不包含两种选择,然后进行回溯算法.这里需要注意的就是遇到一个值有多个的情况,这里通常最简单的处理办法就是进行浓缩,把原数组改成一个个点对,例如(1,1,1,2,3,3)改成((1,3),(2,1),(3,2))这样再每次选择时候消耗一下记数,这样就不会出现重复的结果了.
    \begin{lstlisting}
void backtrack(vector<int> &key, vector<int> &count, int target,
			int pos, vector<vector<int> > &result, vector<int> &track){
	if(target == 0){
		result.push_back(track);
		return;
	}
	if(pos == -1)	return;
	if(count[pos] > 0 && key[pos] <= target){
		track.push_back(key[pos]);
		count[pos]--;
		backtrack(key, count, target - key[pos], pos, result, track);
		count[pos]++;
		track.pop_back();
	}
	backtrack(key, count, target, pos-1, result, track);
}

vector<vector<int> > combinationSum2(vector<int> &num, int target) {
	sort(num.begin(), num.end(), greater<int>());
	vector<int> key, count, track;
	vector<vector<int> > result;
	if(num.size() == 0)	return result;
	int last = 0;
	for(int i = 1; i < num.size(); i++){
		if(num[i] != num[i-1]){
			key.push_back(num[last]);
			count.push_back(i - last);
			last = i;
		}
	}
	key.push_back(num[last]);
	count.push_back(num.size() - last);
	backtrack(key, count, target, key.size() - 1, result, track);
	return result;
}
    \end{lstlisting}
\end{description}

\subsection{Permutations}
    
\begin{description}
    \item{\textbf{问题}}:Given a collection of numbers, return all possible permutations.\textit{(leetcode 46)}
	\item{\textbf{举例}}:\\
(1,2,3) have the following permutations:\\
(1,2,3), (1,3,2), (2,1,3), (2,3,1), (3,1,2), and (3,2,1).
    \item{\textbf{排序}} : \fbox{时间复杂度O(n!), 空间复杂度O(n)}
    \\这题本应该使用回溯法求解,特别是对于没有重复元素情况可以非常简单的使用回溯法求解,但是这道题又是经典的组合数学问题: 我们把这些数的组合看成一个数,例如(1,2,3)看成123,(3,2,1)看成321,那么我们可以从最小的数开始逐渐增大这个数直到算到最大的数,那么所有的组合也就求出来了.那么怎么根据现在的数求出下一个更大一点的数的? 假设我们现在的数是$a_1,a_2,...a_n$,我们找该序列最后一个极大点,设为$a_k$,我们知道$a_{k-1} \textless a_k$, $a_k \textgreater a_{k+1} \textgreater ... \textgreater a_n$,另外设$a_m$是$a_k, ..., a_n$中大于$a_{k-1}$的最小的那个,那么下一个数就是$a_1, a_2, ... a_m, a_k, a_{k+1}, ... , a_{m-1}, a_{k-1}, a_{m+1}, ... , a_n$
    \begin{lstlisting}
vector<vector<int> > permute(vector<int> &num) {
	vector<int> data(num);
	sort(data.begin(), data.end());
	vector<vector<int> > result;
	while(1){
		int peak = data.size() - 1;
		while(peak > 0 && data[peak] <= data[peak-1])	peak--;
		result.push_back(data);
		if(peak == 0)	break;
		int before_peak = peak - 1;
		vector<int>::iterator next_peak = lower_bound(data.begin() + peak, data.end(), 
									data[before_peak], greater<int>());
		next_peak--;
		swap(data[before_peak], *next_peak);
		reverse(data.begin() + peak, data.end());
	}
	return result;
}
    \end{lstlisting}
\end{description}

\subsection{Permutations II}
    
\begin{description}
    \item{\textbf{问题}}:Given a collection of numbers that might contain duplicates, return all possible unique permutations.\textit{(leetcode 47)}
	\item{\textbf{举例}}:\\
(1,1,2) have the following unique permutations:\\
(1,1,2), (1,2,1), and (2,1,1).
    \item{\textbf{排序}} : \fbox{时间复杂度O(n!), 空间复杂度O(n)}
    \\这题同样还是采用转为大数处理,这里就可以看到,如果采用回溯法,那么还要想办法去掉由于有重复元素带来的问题(压缩原数组变成(元素,记数)对).而这个方法则不需要,是不是很飘逸呢.
    \begin{lstlisting}
vector<vector<int> > permuteUnique(vector<int>& nums) {
	vector<int> data(nums);
	sort(data.begin(), data.end());
	vector<vector<int> > result;
	while(1){
		int peak = data.size() - 1;
		while(peak > 0 && data[peak] <= data[peak-1])	peak--;
		result.push_back(data);
		if(peak == 0)	break;
		int before_peak = peak - 1;
		vector<int>::iterator next_peak = lower_bound(data.begin() + peak, data.end(), 
									data[before_peak], greater<int>());
		next_peak--;
		swap(data[before_peak], *next_peak);
		reverse(data.begin() + peak, data.end());
	}
	return result;
}
    \end{lstlisting}
\end{description}

