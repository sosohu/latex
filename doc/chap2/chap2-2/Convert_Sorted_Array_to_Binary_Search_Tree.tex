\ifx allfiles undefined
\documentclass{article}
\usepackage{CJK}
\usepackage{verbatim}

%%%代码
\usepackage{color}
\usepackage{xcolor}
\definecolor{keywordcolor}{rgb}{0.8,0.1,0.5}
\usepackage{listings}
\lstset{breaklines}%这条命令可以让LaTeX自动将长的代码行换行排版
\lstset{extendedchars=false}%这一条命令可以解决代码跨页时,章节标题,页眉等汉字不显示的问题
\lstset{language=C++, %用于设置语言为C++
	keywordstyle=\color{keywordcolor} \bfseries, %设置关键词
	identifierstyle=,
	basicstyle=\ttfamily, 
	commentstyle=\color{blue} \textit,
	stringstyle=\ttfamily, 
	showstringspaces=false,
	%frame=shadowbox, %边框
	captionpos=b
}
%%%

%\hypersetup{CJKbookmarks=true} %解决section不能使用中文的问题

\begin{document}
\begin{CJK}{UTF8}{gbsn}     %CJK:支持中文

\else
	
\begin{description}
	\item{\textbf{问题}}: Given an array where elements are sorted in ascending order, convert it to a height balanced BST. \textit{(leetcode 108)}
	\item{\textbf{递归}} : \fbox{时间复杂度O(n), 空间复杂度O(n)}
	\\建立平衡的二叉树,那么我们每次取数组的中间位置那个元素为根节点,然后它之前的部分创建左子树,之后的部分创建右子树,那么很容易就可以使用递归实现.
	\begin{lstlisting}
TreeNode* dfs(vector<int> &num, int start, int end){
	if(start > end)	return NULL;
	if(start == end)	return (new TreeNode(num[start]));
	int mid = (start + end) / 2;
	TreeNode* root = new TreeNode(num[mid]);
	root->left = dfs(num, start, mid - 1);
	root->right = dfs(num, mid + 1, end);
	return root;
}

TreeNode *sortedArrayToBST(vector<int> &num) {	
	int size = num.size();
	return dfs(num, 0, size - 1);
}
	\end{lstlisting}
	\textit{因为是数组,所以可以很方便的找到中位数,但是如果是链表,则需要使用一些小技巧}
\end{description}

\fi

\ifx allfiles undefined
\end{CJK}
\end{document}
\fi
