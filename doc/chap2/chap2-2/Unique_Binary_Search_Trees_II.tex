\ifx allfiles undefined
\documentclass{article}
\usepackage{CJK}
\usepackage{verbatim}

%%%代码
\usepackage{color}
\usepackage{xcolor}
\definecolor{keywordcolor}{rgb}{0.8,0.1,0.5}
\usepackage{listings}
\lstset{breaklines}%这条命令可以让LaTeX自动将长的代码行换行排版
\lstset{extendedchars=false}%这一条命令可以解决代码跨页时,章节标题,页眉等汉字不显示的问题
\lstset{language=C++, %用于设置语言为C++
	keywordstyle=\color{keywordcolor} \bfseries, %设置关键词
	identifierstyle=,
	basicstyle=\ttfamily, 
	commentstyle=\color{blue} \textit,
	stringstyle=\ttfamily, 
	showstringspaces=false,
	%frame=shadowbox, %边框
	captionpos=b
}
%%%

%\hypersetup{CJKbookmarks=true} %解决section不能使用中文的问题

\begin{document}
\begin{CJK}{UTF8}{gbsn}     %CJK:支持中文

\else
	
\begin{description}
	\item{\textbf{问题}}: Given n, generate all structurally unique BST's (binary search trees) that store values 1...n. \textit{(leetcode 95)}
	\item{\textbf{}} : \fbox{时间复杂度 TODO , 空间复杂度 TODO}
	\\每次对于构造序列(i,i+1,...,j),切分j-i+1次,然后分别递归构造.
	\begin{lstlisting}
vector<TreeNode*> generate(int start, int end){
	if(start == end){
		return vector<TreeNode*>(1, new TreeNode(start));
	}
	vector<TreeNode*> data;
	if(start > end){
		data.push_back(NULL);
		return data;
	}
	for(int i = start; i <= end; i++){
		TreeNode* root;
		vector<TreeNode*> left = generate(start, i - 1);
		vector<TreeNode*> right = generate(i + 1, end);
		for(int j = 0; j < left.size(); j++){
			for(int k = 0; k < right.size(); k++){
				root = new TreeNode(i);
				root->left = left[j];
				root->right = right[k];
				data.push_back(root);
			}
		}
	}
	return data;
}

vector<TreeNode*> generateTrees(int n) {
	return generate(1, n);
}
	\end{lstlisting}
\end{description}

\fi

\ifx allfiles undefined
\end{CJK}
\end{document}
\fi
