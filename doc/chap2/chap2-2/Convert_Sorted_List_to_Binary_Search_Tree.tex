\ifx allfiles undefined
\documentclass{article}
\usepackage{CJK}
\usepackage{verbatim}

%%%代码
\usepackage{color}
\usepackage{xcolor}
\definecolor{keywordcolor}{rgb}{0.8,0.1,0.5}
\usepackage{listings}
\lstset{breaklines}%这条命令可以让LaTeX自动将长的代码行换行排版
\lstset{extendedchars=false}%这一条命令可以解决代码跨页时,章节标题,页眉等汉字不显示的问题
\lstset{language=C++, %用于设置语言为C++
	keywordstyle=\color{keywordcolor} \bfseries, %设置关键词
	identifierstyle=,
	basicstyle=\ttfamily, 
	commentstyle=\color{blue} \textit,
	stringstyle=\ttfamily, 
	showstringspaces=false,
	%frame=shadowbox, %边框
	captionpos=b
}
%%%

%\hypersetup{CJKbookmarks=true} %解决section不能使用中文的问题

\begin{document}
\begin{CJK}{UTF8}{gbsn}     %CJK:支持中文

\else
	
\begin{description}
	\item{\textbf{问题}}: Given a singly linked list where elements are sorted in ascending order, convert it to a height balanced BST. \textit{(leetcode 109)}
	\item{\textbf{递归}} : \fbox{时间复杂度O(n), 空间复杂度O(n)}
	\\前面使用数组构造BST树,我们可以看到每次需要求出它的中间的那个数,然后以它创建根节点,但是对于有序链表来说,找到中位数起码要花O(n)时间,那么这样算下来整个程序需要O($nlgn$)的时间!这似乎和数组的O(n)差别比较大。我们可以想一下,摒弃这种自上而下的思维,来一次自下而上的方法:我们先构建左子树,构建完了之后访问的最后一点节点是不是就是根节点的前驱?这样我们记下这个前驱,然后一记next是不是就求出我们梦寐以求的中位数那个节点!然后再next一下,构建右子树,看看发生了什么?我们一边建树一边就得到中间节点,所以就省掉了找中间节点的那个时间!
	\begin{lstlisting}
TreeNode* DFS(ListNode* &head, int n){
	if(n == 0)	return NULL;
	TreeNode *root;
	if(n == 1){
		root = new TreeNode(head->val);
		head = head->next;
		return root;
	}
	TreeNode *left = DFS(head, n/2);// head travel the list
	root = new TreeNode(head->val);
	head = head->next; 
	TreeNode *right = DFS(head, n - 1 - n/2 );
	root->left = left;
	root->right = right;
	return root;
}

TreeNode *sortedListToBST(ListNode *head) {
	if(!head)	return NULL;
	int count = 0;
	ListNode *pos = head;
	while(pos){//compute the length of the list
		pos = pos->next;
		count++;
	}
	return DFS(head, count);
}
	\end{lstlisting}
	\textit{这里使用自下而上的做法很具有普遍性,我们从上面看得不到的东西,从下面可以积累到}
\end{description}

\fi

\ifx allfiles undefined
\end{CJK}
\end{document}
\fi
