\ifx allfiles undefined
\documentclass{article}
\usepackage{CJK}
\usepackage{verbatim}

%%%代码
\usepackage{color}
\usepackage{xcolor}
\definecolor{keywordcolor}{rgb}{0.8,0.1,0.5}
\usepackage{listings}
\lstset{breaklines}%这条命令可以让LaTeX自动将长的代码行换行排版
\lstset{extendedchars=false}%这一条命令可以解决代码跨页时,章节标题,页眉等汉字不显示的问题
\lstset{language=C++, %用于设置语言为C++
	keywordstyle=\color{keywordcolor} \bfseries, %设置关键词
	identifierstyle=,
	basicstyle=\ttfamily, 
	commentstyle=\color{blue} \textit,
	stringstyle=\ttfamily, 
	showstringspaces=false,
	%frame=shadowbox, %边框
	captionpos=b
}
%%%

%\hypersetup{CJKbookmarks=true} %解决section不能使用中文的问题

\begin{document}
\begin{CJK}{UTF8}{gbsn}     %CJK:支持中文

\else
	
\begin{description}
	\item{\textbf{问题}}: Given postorder and inorder traversal of a tree, construct the binary tree. \textit{(leetcode 106)}
	\item{\textbf{Note}}: You may assume that duplicates do not exist in the tree.
	\item{\textbf{递归}} : \fbox{时间复杂度O($nlgn$) , 空间复杂度O($lgn$)}
	\\后序序列的最后一个元素肯定是树的根节点,而且使用这个值在中序序列中查找,找到的那个位置之前的必然是左子树,之后的必然是右子树,所以根据这个特点就可以很容易的使用递归的做法解题。
	\begin{lstlisting}
TreeNode* recursion(vector<int> &inorder, vector<int> &postorder,
					int s_in, int e_in, int s_po, int e_po){
	//inorder[s_in, e_in],postorder[s_po, e_po]
	if(s_in > e_in)	return NULL;
	if(s_in == e_in) return (new TreeNode(inorder[s_in]));
	TreeNode *root = new TreeNode(postorder[e_po]);
	int split_in, split_po;
	for(split_in = s_in; split_in <= e_in; split_in++){
		if(postorder[e_po] == inorder[split_in])
			break;
	}
	split_po = s_po + split_in - s_in;
	root->left = recursion(inorder, postorder, s_in, split_in - 1, s_po , split_po - 1 );
	root->right = recursion(inorder, postorder, split_in + 1, e_in, split_po, e_po - 1);
	return root;
}

TreeNode *buildTree(vector<int> &inorder, vector<int> &postorder) {
	if(inorder.size() != postorder.size() || inorder.size() == 0)
		return NULL;
	return recursion(inorder, postorder, 0, inorder.size() - 1, 0, postorder.size() - 1);
}
	\end{lstlisting}
	\textit{}
\end{description}

\fi

\ifx allfiles undefined
\end{CJK}
\end{document}
\fi
