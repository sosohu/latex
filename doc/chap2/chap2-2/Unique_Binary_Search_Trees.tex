\ifx allfiles undefined
\documentclass{article}
\usepackage{CJK}
\usepackage{verbatim}

%%%代码
\usepackage{color}
\usepackage{xcolor}
\definecolor{keywordcolor}{rgb}{0.8,0.1,0.5}
\usepackage{listings}
\lstset{breaklines}%这条命令可以让LaTeX自动将长的代码行换行排版
\lstset{extendedchars=false}%这一条命令可以解决代码跨页时,章节标题,页眉等汉字不显示的问题
\lstset{language=C++, %用于设置语言为C++
    keywordstyle=\color{keywordcolor} \bfseries, %设置关键词
    identifierstyle=,
    basicstyle=\ttfamily, 
    commentstyle=\color{blue} \textit,
    stringstyle=\ttfamily, 
    showstringspaces=false,
    %frame=shadowbox, %边框
    captionpos=b
}
%%%

%\hypersetup{CJKbookmarks=true} %解决section不能使用中文的问题

\begin{document}
\begin{CJK}{UTF8}{gbsn}     %CJK:支持中文

\else
    
\begin{description}
    \item{\textbf{问题}}: Given n, how many structurally unique BST's (binary search trees) that store values 1...n? \textit{(leetcode 96)}
    \\这道题虽然是一个二叉树的问题,但是其实是数学归纳法的问题,我们可以得到递推公式:
    \begin{equation}
    \begin{split}
    S(0) = 1
    \\S(n) = \sum_{i = 0}^{n - 1}{S(i)*S(n-1-i)}
    \end{split}
    \end{equation}
    \\这是因为对于有n个元素的BST(1,2,...,n),我们考虑由谁来作为根节点,如果以i+1为根节点,那么左子树为(1,2,...,i),右子树为(i+2,i+3,...,n),所以可以得到上面的递推关系.
    \item{\textbf{迭代}} : \fbox{时间复杂度O($n^2$) , 空间复杂度O(n)}
    \begin{lstlisting}
int numTrees(int n) {
    if(!n)    return 0;
    vector<int> num(n+1, 0);
    num[1] = 1;
    num[2] = 2;
    for(int i = 3; i <= n; i++){
        num[i] = 2*num[i-1];
        for(int j = 1; j < i; j++){
            num[i] = num[i] + num[j]*num[i-1-j];
        }
    }
    return num[n];
}
    \end{lstlisting}
    \textit{}
\end{description}

\fi

\ifx allfiles undefined
\end{CJK}
\end{document}
\fi
