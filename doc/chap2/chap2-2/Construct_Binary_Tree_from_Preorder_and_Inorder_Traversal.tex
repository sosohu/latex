\ifx allfiles undefined
\documentclass{article}
\usepackage{CJK}
\usepackage{verbatim}

%%%代码
\usepackage{color}
\usepackage{xcolor}
\definecolor{keywordcolor}{rgb}{0.8,0.1,0.5}
\usepackage{listings}
\lstset{breaklines}%这条命令可以让LaTeX自动将长的代码行换行排版
\lstset{extendedchars=false}%这一条命令可以解决代码跨页时,章节标题,页眉等汉字不显示的问题
\lstset{language=C++, %用于设置语言为C++
	keywordstyle=\color{keywordcolor} \bfseries, %设置关键词
	identifierstyle=,
	basicstyle=\ttfamily, 
	commentstyle=\color{blue} \textit,
	stringstyle=\ttfamily, 
	showstringspaces=false,
	%frame=shadowbox, %边框
	captionpos=b
}
%%%

%\hypersetup{CJKbookmarks=true} %解决section不能使用中文的问题

\begin{document}
\begin{CJK}{UTF8}{gbsn}     %CJK:支持中文

\else
	
\begin{description}
	\item{\textbf{问题}}: Given preorder and inorder traversal of a tree, construct the binary tree. \textit{(leetcode 105)}
	\item{\textbf{Note}}: You may assume that duplicates do not exist in the tree.
	\item{\textbf{递归}} : \fbox{时间复杂度O($nlgn$) , 空间复杂度O($lgn$)}
	\\前序序列的第一个元素肯定是树的根节点,而且使用这个值在中序序列中查找,找到的那个位置之前的必然是左子树,之后的必然是右子树,所以根据这个特点就可以很容易的使用递归的做法解题。
	\begin{lstlisting}
TreeNode* detail(vector<int> &inorder, vector<int> &preorder, 
				int in_start, int in_end, int pre_start, int pre_end){
//inorder[in_start, in_end], preorder[pre_start, pre_end]
	if(in_start > in_end || pre_start > pre_end) return NULL;
	int val = preorder[pre_start];

	TreeNode* father = new TreeNode(val);
	TreeNode *left = NULL, *right = NULL;
		
	int in_pos = in_start, pre_pos = pre_start;
	for(; in_pos <= in_end; in_pos++, pre_pos++){
		if(val == inorder[in_pos])
			break;
	}

	left = detail(inorder, preorder, in_start, in_pos-1, pre_start+1, pre_pos);
	right = detail(inorder, preorder, in_pos+1, in_end, pre_pos+1, pre_end);

	father->left = left;
	father->right = right;

	return father;
}

TreeNode *buildTree(vector<int> &preorder, vector<int> &inorder) {
	int size = inorder.size();
	if(size == 0) return NULL;
		return detail(inorder, preorder, 0, size - 1, 0, size - 1);
}
	\end{lstlisting}
	\textit{}
\end{description}

\fi

\ifx allfiles undefined
\end{CJK}
\end{document}
\fi
