\ifx allfiles undefined
\documentclass{article}
\usepackage{CJK}
\usepackage{verbatim}

%%%代码
\usepackage{color}
\usepackage{xcolor}
\definecolor{keywordcolor}{rgb}{0.8,0.1,0.5}
\usepackage{listings}
\lstset{breaklines}%这条命令可以让LaTeX自动将长的代码行换行排版
\lstset{extendedchars=false}%这一条命令可以解决代码跨页时,章节标题,页眉等汉字不显示的问题
\lstset{language=C++, %用于设置语言为C++
    keywordstyle=\color{keywordcolor} \bfseries, %设置关键词
    identifierstyle=,
    basicstyle=\ttfamily, 
    commentstyle=\color{blue} \textit,
    stringstyle=\ttfamily, 
    showstringspaces=false,
    %frame=shadowbox, %边框
    captionpos=b
}
%%%

%\hypersetup{CJKbookmarks=true} %解决section不能使用中文的问题

\begin{document}
\begin{CJK}{UTF8}{gbsn}     %CJK:支持中文

\else
    
\begin{description}
    \item{\textbf{问题}}: \\
Implement an iterator over a binary search tree (BST). Your iterator will be initialized with the root node of a BST. \\
\\
Calling next() will return the next smallest number in the BST. \\
\textit{(leetcode 173)}
    \item{\textbf{Note}}: \\
next() and hasNext() should run in average O(1) time and uses O(h) memory, where h is the height of the tree.
    \item{\textbf{Stack}} : \fbox{时间复杂度O(1), 空间复杂度O(lgn)}
    \\这里使用一个栈来保存上级节点,每个next和hasNext的操作平均时间复杂度是O(1)
    \begin{lstlisting}
class BSTIterator {
private:
	TreeNode* root;
	stack<TreeNode*> path;
public:
    BSTIterator(TreeNode *root) : root(root){
		TreeNode* iter = root;
		while(iter){
			path.push(iter);
			iter = iter->left;
		}
    }

    /** @return whether we have a next smallest number */
    bool hasNext() {
		return !path.empty();
    }

    /** @return the next smallest number */
    int next() {
        TreeNode *cur = path.top();
		TreeNode *iter = cur->right;
		path.pop();
		while(iter){
			path.push(iter);
			iter = iter->left;
		}
		return cur->val;
    }
};
    \end{lstlisting}
\end{description}

\fi

\ifx allfiles undefined
\end{CJK}
\end{document}
\fi
