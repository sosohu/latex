\ifx allfiles undefined
\documentclass{article}
\usepackage{CJK}
\usepackage{verbatim}

%%%代码
\usepackage{color}
\usepackage{xcolor}
\definecolor{keywordcolor}{rgb}{0.8,0.1,0.5}
\usepackage{listings}
\lstset{breaklines}%这条命令可以让LaTeX自动将长的代码行换行排版
\lstset{extendedchars=false}%这一条命令可以解决代码跨页时,章节标题,页眉等汉字不显示的问题
\lstset{language=C++, %用于设置语言为C++
	keywordstyle=\color{keywordcolor} \bfseries, %设置关键词
	identifierstyle=,
	basicstyle=\ttfamily, 
	commentstyle=\color{blue} \textit,
	stringstyle=\ttfamily, 
	showstringspaces=false,
	%frame=shadowbox, %边框
	captionpos=b
}
%%%

%\hypersetup{CJKbookmarks=true} %解决section不能使用中文的问题

\begin{document}
\begin{CJK}{UTF8}{gbsn}     %CJK:支持中文

\else
	
\begin{description}
	\item{\textbf{问题}}: Follow up for problem "Populating Next Right Pointers in Each Node".What if the given tree could be any binary tree? Would your previous solution still work?
	\textbf{Note}: You may only use constant extra space. \textit{(leetcode 117)}
	\item{\textbf{迭代}} : \fbox{时间复杂度O(n) , 空间复杂度O(1)}
	\\这里其实和上一题一样,只不过多了一些判断条件.
	\begin{lstlisting}
void connect(TreeLinkNode *root) {
	if(!root)	return;
	TreeLinkNode *cur = root, *next, *last;
	cur->next = NULL;
	do{
		next = NULL;
		while(cur){
			if(cur->left){
				if(next){
					last->next = cur->left;
					last = last->next;
				}
				else{
					last = cur->left;
					next = last;
				}
			}
			if(cur->right){
				if(next){
					last->next = cur->right;
					last = last->next;
				}
				else{
					last = cur->left;
					next = last;
				}
			}
			cur = cur->next;
		}
		last->next = NULL;
		cur = next;
	}while(cur);
}
	\end{lstlisting}
	\textit{}
\end{description}

\fi

\ifx allfiles undefined
\end{CJK}
\end{document}
\fi
