\ifx allfiles undefined
\documentclass{article}
\usepackage{CJK}
\usepackage{verbatim}

%%%代码
\usepackage{color}
\usepackage{xcolor}
\definecolor{keywordcolor}{rgb}{0.8,0.1,0.5}
\usepackage{listings}
\lstset{breaklines}%这条命令可以让LaTeX自动将长的代码行换行排版
\lstset{extendedchars=false}%这一条命令可以解决代码跨页时,章节标题,页眉等汉字不显示的问题
\lstset{language=C++, %用于设置语言为C++
	keywordstyle=\color{keywordcolor} \bfseries, %设置关键词
	identifierstyle=,
	basicstyle=\ttfamily, 
	commentstyle=\color{blue} \textit,
	stringstyle=\ttfamily, 
	showstringspaces=false,
	%frame=shadowbox, %边框
	captionpos=b
}
%%%

%\hypersetup{CJKbookmarks=true} %解决section不能使用中文的问题

\begin{document}
\begin{CJK}{UTF8}{gbsn}     %CJK:支持中文

\else
	
\begin{description}
	\item{\textbf{问题}}: Given a binary tree, flatten it to a linked list in-place by the pre-order. \textit{(leetcode 114)}
	\\这是一个基于先序遍历的问题,所以可以使用递归和迭代的方法.
	\item{\textbf{递归}} : \fbox{时间复杂度O(n) , 空间复杂度O($lgn$)}
	\begin{lstlisting}
void flatten(TreeNode *root) {
	TreeNode *tail;
	recursion(root, tail);
}

TreeNode* recursion(TreeNode *root, TreeNode* &tail){
	if(!root)	return NULL;
	TreeNode *next = NULL;
	tail = root;
	if(root->left)
		next = recursion(root->left, tail);
	if(root->right)
		tail->right = recursion(root->right, tail);
	root->left = NULL;
	if(next)
		root->right = next;
	return root;
}
	\end{lstlisting}
	\item{\textbf{迭代}} : \fbox{时间复杂度O(n) , 空间复杂度O($lgn$)}
	\\这里就是完完全全的迭代版前序遍历,这里使用了栈,同样你也可以使用Mirror算法.
	\begin{lstlisting}
void flatten(TreeNode *root) {
	if(!root)	return;
	stack<TreeNode*> s;
	s.push(root);
	TreeNode *last = NULL, *cur;
	while(!s.empty()){
		cur = s.top();
		s.pop();
		if(last)
			last->right = cur;
		if(cur->right)
			s.push(cur->right);
		if(cur->left)
			s.push(cur->left);
		cur->left = NULL;
		last = cur;
	}
	last->right = NULL;
}
	\end{lstlisting}
\end{description}

\fi

\ifx allfiles undefined
\end{CJK}
\end{document}
\fi
