\ifx allfiles undefined
\documentclass{article}
\usepackage{CJK}
\usepackage{verbatim}

%%%代码
\usepackage{color}
\usepackage{xcolor}
\definecolor{keywordcolor}{rgb}{0.8,0.1,0.5}
\usepackage{listings}
\lstset{breaklines}%这条命令可以让LaTeX自动将长的代码行换行排版
\lstset{extendedchars=false}%这一条命令可以解决代码跨页时,章节标题,页眉等汉字不显示的问题
\lstset{language=C++, %用于设置语言为C++
	keywordstyle=\color{keywordcolor} \bfseries, %设置关键词
	identifierstyle=,
	basicstyle=\ttfamily, 
	commentstyle=\color{blue} \textit,
	stringstyle=\ttfamily, 
	showstringspaces=false,
	%frame=shadowbox, %边框
	captionpos=b
}
%%%

%\hypersetup{CJKbookmarks=true} %解决section不能使用中文的问题

\begin{document}
\begin{CJK}{UTF8}{gbsn}     %CJK:支持中文

\else
	
\begin{description}
	\item{\textbf{问题}}: Given a binary tree:
	\begin{lstlisting}
struct TreeLinkNode {
    TreeLinkNode *left;
    TreeLinkNode *right;
	TreeLinkNode *next;
}
	\end{lstlisting}
	Populate each next pointer to point to its next right node. If there is no next right node, the next pointer should be set to NULL. 
	\\\textbf{Note}:
	\\You may only use constant extra space.
	\\You may assume that it is a perfect binary tree (ie, all leaves are at the same level, and every parent has two children).
	\textit{(leetcode 116)}
	\\其实就是一个很简单的BFS过程.
	\item{\textbf{迭代}} : \fbox{时间复杂度O(n), 空间复杂度O(1)}
	\\因为是满二叉树,所以每次在上一层建立这一层的next,然后再到这一层来,这样就不需要队列,使用常数的空间复杂度.
	\begin{lstlisting}
void connect(TreeLinkNode *root) {
	if(!root)	return;
	TreeLinkNode *cur = root, *next;
	cur->next = NULL;
	while(cur->left){
		next = cur->left;
		while(cur){
			cur->left->next = cur->right;
			cur->right->next = cur->next? cur->next->left : NULL;
			cur = cur->next;
		}
		cur = next;
	}
}
	\end{lstlisting}
\end{description}

\fi

\ifx allfiles undefined
\end{CJK}
\end{document}
\fi
