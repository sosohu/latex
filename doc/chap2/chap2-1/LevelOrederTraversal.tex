\ifx allfiles undefined
\documentclass{article}
\usepackage{CJK}
\usepackage{verbatim}

%%%代码
\usepackage{color}
\usepackage{xcolor}
\definecolor{keywordcolor}{rgb}{0.8,0.1,0.5}
\usepackage{listings}
\lstset{breaklines}%这条命令可以让LaTeX自动将长的代码行换行排版
\lstset{extendedchars=false}%这一条命令可以解决代码跨页时,章节标题,页眉等汉字不显示的问题
\lstset{language=C++, %用于设置语言为C++
	keywordstyle=\color{keywordcolor} \bfseries, %设置关键词
	identifierstyle=,
	basicstyle=\ttfamily, 
	commentstyle=\color{blue} \textit,
	stringstyle=\ttfamily, 
	showstringspaces=false,
	%frame=shadowbox, %边框
	captionpos=b
}
%%%

%\hypersetup{CJKbookmarks=true} %解决section不能使用中文的问题

\begin{document}
\begin{CJK}{UTF8}{gbsn}     %CJK:支持中文

\else
	
\begin{description}
	\item{\textbf{问题}}: Given a binary tree, return the level order traversal of its nodes' values. (ie, from left to right, level by level). \textit{(leetcode 145)}
	\item{\textbf{队列}} : \fbox{时间复杂度O(n) , 空间复杂度O(w), w为二叉树最大宽度}
	\\使用队列,上一层进入队列,然后添加下一层,直到不再有节点进入队列
	\begin{lstlisting}
vector<vector<int> > levelOrder(TreeNode *root) {
	vector<vector<int> > data;
	if(!root)	return data;
	queue<TreeNode*> cur;
	cur.push(root);
	int size = 0;
	TreeNode* now;
	while(!cur.empty()){
		vector<int> tmp;
		size = cur.size();
		for(int i = 0; i < size; i++){
			now = cur.front();
			cur.pop();
			tmp.push_back(now->val);
			if(now->left)
				cur.push(now->left);
			if(now->right)
				cur.push(now->right);
		}
		data.push_back(tmp);
	}
	return data;
}
	\end{lstlisting}
	\textit{其实层次遍历除了这个先上而下,先左而右的顺序以外还有很多顺序,但是都可以通过这个顺序来转换,所以就不再仔细讨论}
\end{description}

\fi

\ifx allfiles undefined
\end{CJK}
\end{document}
\fi
