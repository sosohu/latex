\ifx allfiles undefined
\documentclass{article}
\usepackage{CJK}
\usepackage{verbatim}

%%%代码
\usepackage{color}
\usepackage{xcolor}
\definecolor{keywordcolor}{rgb}{0.8,0.1,0.5}
\usepackage{listings}
\lstset{breaklines}%这条命令可以让LaTeX自动将长的代码行换行排版
\lstset{extendedchars=false}%这一条命令可以解决代码跨页时,章节标题,页眉等汉字不显示的问题
\lstset{language=C++, %用于设置语言为C++
	keywordstyle=\color{keywordcolor} \bfseries, %设置关键词
	identifierstyle=,
	basicstyle=\ttfamily, 
	commentstyle=\color{blue} \textit,
	stringstyle=\ttfamily, 
	showstringspaces=false,
	%frame=shadowbox, %边框
	captionpos=b
}
%%%

%\hypersetup{CJKbookmarks=true} %解决section不能使用中文的问题

\begin{document}
\begin{CJK}{UTF8}{gbsn}     %CJK:支持中文

\else
	
%二叉树的先序遍历
\begin{description}
	\item{\textbf{问题}}: Given a binary tree, return the preorder traversal of its nodes' values.(\textit{leetcode 144})
	\item{\textbf{递归}} : \fbox{时间复杂度O(n) , 空间复杂度O($lgn$)}
	\\前序遍历的递归写法,非常简单,只需要先访问跟节点,再递归的执行左子树和右子树
	\begin{lstlisting}
void BiTree::PreOrderTraversal(TreeNode* root){
	if(!root)	return;
	cout<<root->val<<endl;
	if(root->left)
		PreOrderTraversal(root->left);
	if(root->right)
		PreOrderTraversal(root->right);
}
	\end{lstlisting}
	\item{\textbf{栈式迭代}} : \fbox{时间复杂度O(n) , 空间复杂度O($lgn$)}
	\\使用栈来模拟递归过程也是很显而易见的,具体做法就是先访问根节点,然后先让右孩子入栈接着是左孩子,然后左孩子出栈后重复这个过程
	\begin{lstlisting}
void BiTree::PreOrderTraversal(TreeNode* root){
	if(!root)	return;
	stack<TreeNode*> s;
	TreeNode *cur;
	s.push(root);
	while(!s.empty()){
		cur = s.top();
		s.pop();
		cout<<cur->val<<endl;
		if(cur->right)
			s.push(cur->right);
		if(cur->left)
			s.push(cur->left);
	}
}
	\end{lstlisting}
	\item{\textbf{Mirror迭代}} : \fbox{时间复杂度O(n) , 空间复杂度O(1)}
	\\Mirror迭代法是经过Lee介绍过来的,非常的迷人,它的做法就是在遍历的过程中,访问了当前节点之后,先找当前节点的前驱并让此前驱的右孩子指向它,再访问它的左孩子并重复这个过程。在此之后会访问到它前驱然后再次回到当前节点,此时再次试图建立前驱,发现已经建立了,这就说明当前节点左边已经全部遍历完,则继续访问当前节点的右边节点,不断的重复此过程。
	\begin{lstlisting}
void BiTree::PreOrderTraversal(TreeNode* root){
	if(!root)	return;
	TreeNode *curr = root, *next;
	while(curr){
		next = curr->left;
		if(!next){
			cout<<curr->val<<endl;
			curr = curr->right;
			continue;
		}
		while(next->right && next->right != curr){
			next = next->right;
		}
		if(next->right == curr){
			next->right = NULL;
			curr = curr->right;
		}else{
			cout<<curr->val<<endl;
			next->right = curr;
			curr = curr->left;
		}
	}
}
	\end{lstlisting}
	\textit{这个Mirror算法一旦掌握后,威力无穷,你可以用它方便的建立二叉树前序索引并且遇到那些要求用迭代来实现的二叉树问题也可以很快的写出来}
\end{description}

\fi

\ifx allfiles undefined
\end{CJK}
\end{document}
\fi
