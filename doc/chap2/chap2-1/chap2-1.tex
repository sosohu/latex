\ifx allfiles undefined
\documentclass{article}
\usepackage{CJK}

%%%代码
\usepackage{color}
\usepackage{xcolor}
\definecolor{keywordcolor}{rgb}{0.8,0.1,0.5}
\usepackage{listings}
\lstset{breaklines}%这条命令可以让LaTeX自动将长的代码行换行排版
\lstset{extendedchars=false}%这一条命令可以解决代码跨页时,章节标题,页眉等汉字不显示的问题
\lstset{language=C++, %用于设置语言为C++
    keywordstyle=\color{keywordcolor} \bfseries, %设置关键词
    identifierstyle=,
    basicstyle=\ttfamily, 
    commentstyle=\color{blue} \textit,
    stringstyle=\ttfamily, 
    showstringspaces=false,
    %frame=shadowbox, %边框
    captionpos=b
}
%%%

%\hypersetup{CJKbookmarks=true} %解决section不能使用中文的问题

\begin{document}
\begin{CJK}{UTF8}{gbsn}     %CJK:支持中文

\else
    
%二叉树的遍历
\qquad二叉树的遍历是解决很多二叉树问题的基础,它的递归写法和非递归写法更是需要都要掌握的,这里的遍历就是将树的节点都依次访问一遍,因为访问顺序的问题,就可以分为前序,中序,后序以及层次等很多种遍历的方法。
\subsection{PreOrederTraversal}
\ifx allfiles undefined
\documentclass{article}
\usepackage{CJK}
\usepackage{verbatim}

%%%代码
\usepackage{color}
\usepackage{xcolor}
\definecolor{keywordcolor}{rgb}{0.8,0.1,0.5}
\usepackage{listings}
\lstset{breaklines}%这条命令可以让LaTeX自动将长的代码行换行排版
\lstset{extendedchars=false}%这一条命令可以解决代码跨页时,章节标题,页眉等汉字不显示的问题
\lstset{language=C++, %用于设置语言为C++
    keywordstyle=\color{keywordcolor} \bfseries, %设置关键词
    identifierstyle=,
    basicstyle=\ttfamily, 
    commentstyle=\color{blue} \textit,
    stringstyle=\ttfamily, 
    showstringspaces=false,
    %frame=shadowbox, %边框
    captionpos=b
}
%%%

%\hypersetup{CJKbookmarks=true} %解决section不能使用中文的问题

\begin{document}
\begin{CJK}{UTF8}{gbsn}     %CJK:支持中文

\else
    
%二叉树的先序遍历
\begin{description}
    \item{\textbf{问题}}: Given a binary tree, return the preorder traversal of its nodes' values.(\textit{leetcode 144})
    \item{\textbf{递归}} : \fbox{时间复杂度O(n) , 空间复杂度O($lgn$)}
    \\前序遍历的递归写法,非常简单,只需要先访问跟节点,再递归的执行左子树和右子树
    \begin{lstlisting}
void BiTree::PreOrderTraversal(TreeNode* root){
    if(!root)    return;
    cout<<root->val<<endl;
    if(root->left)
        PreOrderTraversal(root->left);
    if(root->right)
        PreOrderTraversal(root->right);
}
    \end{lstlisting}
    \item{\textbf{栈式迭代}} : \fbox{时间复杂度O(n) , 空间复杂度O($lgn$)}
    \\使用栈来模拟递归过程也是很显而易见的,具体做法就是先访问根节点,然后先让右孩子入栈接着是左孩子,然后左孩子出栈后重复这个过程
    \begin{lstlisting}
void BiTree::PreOrderTraversal(TreeNode* root){
    if(!root)    return;
    stack<TreeNode*> s;
    TreeNode *cur;
    s.push(root);
    while(!s.empty()){
        cur = s.top();
        s.pop();
        cout<<cur->val<<endl;
        if(cur->right)
            s.push(cur->right);
        if(cur->left)
            s.push(cur->left);
    }
}
    \end{lstlisting}
    \item{\textbf{Mirror迭代}} : \fbox{时间复杂度O(n) , 空间复杂度O(1)}
    \\Mirror迭代法是经过Lee介绍过来的,非常的迷人,它的做法就是在遍历的过程中,访问了当前节点之后,先找当前节点的前驱并让此前驱的右孩子指向它,再访问它的左孩子并重复这个过程。在此之后会访问到它前驱然后再次回到当前节点,此时再次试图建立前驱,发现已经建立了,这就说明当前节点左边已经全部遍历完,则继续访问当前节点的右边节点,不断的重复此过程。
    \begin{lstlisting}
void BiTree::PreOrderTraversal(TreeNode* root){
    if(!root)    return;
    TreeNode *curr = root, *next;
    while(curr){
        next = curr->left;
        if(!next){
            cout<<curr->val<<endl;
            curr = curr->right;
            continue;
        }
        while(next->right && next->right != curr){
            next = next->right;
        }
        if(next->right == curr){
            next->right = NULL;
            curr = curr->right;
        }else{
            cout<<curr->val<<endl;
            next->right = curr;
            curr = curr->left;
        }
    }
}
    \end{lstlisting}
    \textit{这个Mirror算法一旦掌握后,威力无穷,你可以用它方便的建立二叉树前序索引并且遇到那些要求用迭代来实现的二叉树问题也可以很快的写出来}
\end{description}

\fi

\ifx allfiles undefined
\end{CJK}
\end{document}
\fi

\subsection{InOrederTraversal}
\ifx allfiles undefined
\documentclass{article}
\usepackage{CJK}
\usepackage{verbatim}

%%%代码
\usepackage{color}
\usepackage{xcolor}
\definecolor{keywordcolor}{rgb}{0.8,0.1,0.5}
\usepackage{listings}
\lstset{breaklines}%这条命令可以让LaTeX自动将长的代码行换行排版
\lstset{extendedchars=false}%这一条命令可以解决代码跨页时,章节标题,页眉等汉字不显示的问题
\lstset{language=C++, %用于设置语言为C++
	keywordstyle=\color{keywordcolor} \bfseries, %设置关键词
	identifierstyle=,
	basicstyle=\ttfamily, 
	commentstyle=\color{blue} \textit,
	stringstyle=\ttfamily, 
	showstringspaces=false,
	%frame=shadowbox, %边框
	captionpos=b
}
%%%

%\hypersetup{CJKbookmarks=true} %解决section不能使用中文的问题

\begin{document}
\begin{CJK}{UTF8}{gbsn}     %CJK:支持中文

\else
	
%XXX 问题
\begin{description}
	\item{\textbf{问题}}: Given a binary tree, return the inorder traversal of its nodes' values. \textit{(leetcode 94)}
	\item{\textbf{递归}} : \fbox{时间复杂度O(n) , 空间复杂度O($lgn$)}
	\\
	\begin{lstlisting}
void BiTree::InOrderTraversal(TreeNode* root){
	if(!root)	return;
	if(root->left)
		InOrderTraversal(root->left);
	cout<<root->val<<endl;
	if(root->right)
		InOrderTraversal(root->right);
	\end{lstlisting}
	\item{\textbf{栈式迭代}} : \fbox{时间复杂度O(n) , 空间复杂度O($lgn$)}
	\\这里需要说一下的是,数据结构那本书上写了两种栈式迭代方法,这是其中之一,使用两重循环的那个
	\begin{lstlisting}
void BiTree::InOrderTraversal(TreeNode* root){
	vector<int> data;
	if(!root)	return data;
	stack<TreeNode*> s;
	TreeNode *pos = root;
	while(!s.empty() || pos){
		while(pos){
			s.push(pos);
			pos = pos->left;
		}
		pos = s.top();
		s.pop();
		std::cout<<pos->val<<std::endl;
		pos = pos->right;  //这个非常重要
	}
}
	\end{lstlisting}
	\item{\textbf{栈式迭代}} : \fbox{时间复杂度O(n) , 空间复杂度O($lgn$)}
	\\这里需要说一下的是,数据结构那本书上写了两种栈式迭代方法,这是其中之二,使用一重循环,实际上是一样的
	\begin{lstlisting}
void BiTree::InOrderTraversal(TreeNode* root){
	vector<int> data;
	if(!root)	return data;
	stack<TreeNode*> s;
	TreeNode *pos = root;
	while(!s.empty() || pos){
		if(pos){
			s.push(pos);
			pos = pos->left;
		}else{
			pos = s.top();
			s.pop();
			std::cout<<pos->val<<std::endl;
			pos = pos->right;  //这个非常重要
		}
	}
}
	\end{lstlisting}
	\item{\textbf{Mirror迭代}} : \fbox{时间复杂度O(n) , 空间复杂度O(1)}
	\\这里Mirror方法和前序的Mirror方法基本一样,唯一的区别就是打印当前值的时机
	\begin{lstlisting}
void BiTree::InOrderTraversal(TreeNode* root){
	if(!root)	return;
	TreeNode *curr = root, *next;
	while(curr){
		next = curr->left;
		if(!next){
			cout<<curr->val<<endl;
			curr = curr->right;
			continue;
		}
		while(next->right && next->right != curr){
			next = next->right;
		}
		if(next->right == curr){
			next->right = NULL;
			cout<<curr->val<<endl;
			curr = curr->right;
		}else{
			next->right = curr;
			curr = curr->left;
		}
	}
}
	\end{lstlisting}
\end{description}

\fi

\ifx allfiles undefined
\end{CJK}
\end{document}
\fi

\subsection{PostOrederTraversal}
\ifx allfiles undefined
\documentclass{article}
\usepackage{CJK}
\usepackage{verbatim}

%%%代码
\usepackage{color}
\usepackage{xcolor}
\definecolor{keywordcolor}{rgb}{0.8,0.1,0.5}
\usepackage{listings}
\lstset{breaklines}%这条命令可以让LaTeX自动将长的代码行换行排版
\lstset{extendedchars=false}%这一条命令可以解决代码跨页时,章节标题,页眉等汉字不显示的问题
\lstset{language=C++, %用于设置语言为C++
	keywordstyle=\color{keywordcolor} \bfseries, %设置关键词
	identifierstyle=,
	basicstyle=\ttfamily, 
	commentstyle=\color{blue} \textit,
	stringstyle=\ttfamily, 
	showstringspaces=false,
	%frame=shadowbox, %边框
	captionpos=b
}
%%%

%\hypersetup{CJKbookmarks=true} %解决section不能使用中文的问题

\begin{document}
\begin{CJK}{UTF8}{gbsn}     %CJK:支持中文

\else
	
\begin{description}
	\item{\textbf{问题}}: Given a binary tree, return the postorder traversal of its nodes' values. \textit{(leetcode 145)}
	\item{\textbf{递归}} : \fbox{时间复杂度O(n) , 空间复杂度O($lgn$)}
	\\
	\begin{lstlisting}
void BiTree::PostOrderTraversal(TreeNode* root){
	if(!root)	return;
	if(root->left)
		PostOrderTraversal(root->left);
	if(root->right)
		PostOrderTraversal(root->right);
	cout<<root->val<<endl;
}
	\end{lstlisting}
	\item{栈式迭代}: \fbox{时间复杂度O(n), 空间复杂度O($lgn$)}
	\\这里判断一个节点的孩子是否被访问过的方法是:记录上一次打印的节点,如果上一次打印的节点是它的孩子节点,那么必然它的所有孩子及其子树都访问完了,换句话说该访问它本身了.
	\begin{lstlisting}
void BiTree::PostOrderTraversal(TreeNode* root){
	if(!root)	return;
	stack<TreeNode*> s;
	TreeNode *pos = root, *last = root;
	s.push(root);
	while(!s.empty()){
		pos = s.top();
		if(pos->left == last || pos->right == last || (!pos->left && !pos->right)){
		//孩子已经打印完毕或者根本就没有孩子
			cout<<pos->val<<endl;
			last = pos;
			s.pop();
		}else{
			if(pos->right){
				s.push(pos->right);
			}
			if(pos->left){
				s.push(pos->right);
			}
		}
	}
}
	\end{lstlisting}
	\textit{这里没有出现万众期待的Mirror算法,主要是后序使用Mirror很复杂,我暂时还没有想到怎么实现\^\_\^}
\end{description}

\fi

\ifx allfiles undefined
\end{CJK}
\end{document}
\fi

\subsection{LevelOrderTraversal}
\ifx allfiles undefined
\documentclass{article}
\usepackage{CJK}
\usepackage{verbatim}

%%%代码
\usepackage{color}
\usepackage{xcolor}
\definecolor{keywordcolor}{rgb}{0.8,0.1,0.5}
\usepackage{listings}
\lstset{breaklines}%这条命令可以让LaTeX自动将长的代码行换行排版
\lstset{extendedchars=false}%这一条命令可以解决代码跨页时,章节标题,页眉等汉字不显示的问题
\lstset{language=C++, %用于设置语言为C++
	keywordstyle=\color{keywordcolor} \bfseries, %设置关键词
	identifierstyle=,
	basicstyle=\ttfamily, 
	commentstyle=\color{blue} \textit,
	stringstyle=\ttfamily, 
	showstringspaces=false,
	%frame=shadowbox, %边框
	captionpos=b
}
%%%

%\hypersetup{CJKbookmarks=true} %解决section不能使用中文的问题

\begin{document}
\begin{CJK}{UTF8}{gbsn}     %CJK:支持中文

\else
	
\begin{description}
	\item{\textbf{问题}}: Given a binary tree, return the level order traversal of its nodes' values. (ie, from left to right, level by level). \textit{(leetcode 145)}
	\item{\textbf{队列}} : \fbox{时间复杂度O(n) , 空间复杂度O(w), w为二叉树最大宽度}
	\\使用队列,上一层进入队列,然后添加下一层,直到不再有节点进入队列
	\begin{lstlisting}
vector<vector<int> > levelOrder(TreeNode *root) {
	vector<vector<int> > data;
	if(!root)	return data;
	queue<TreeNode*> cur;
	cur.push(root);
	int size = 0;
	TreeNode* now;
	while(!cur.empty()){
		vector<int> tmp;
		size = cur.size();
		for(int i = 0; i < size; i++){
			now = cur.front();
			cur.pop();
			tmp.push_back(now->val);
			if(now->left)
				cur.push(now->left);
			if(now->right)
				cur.push(now->right);
		}
		data.push_back(tmp);
	}
	return data;
}
	\end{lstlisting}
	\textit{其实层次遍历除了这个先上而下,先左而右的顺序以外还有很多顺序,但是都可以通过这个顺序来转换,所以就不再仔细讨论}
\end{description}

\fi

\ifx allfiles undefined
\end{CJK}
\end{document}
\fi

\subsection{Recover Binary Search Tree }
\ifx allfiles undefined
\documentclass{article}
\usepackage{CJK}
\usepackage{verbatim}

%%%代码
\usepackage{color}
\usepackage{xcolor}
\definecolor{keywordcolor}{rgb}{0.8,0.1,0.5}
\usepackage{listings}
\lstset{breaklines}%这条命令可以让LaTeX自动将长的代码行换行排版
\lstset{extendedchars=false}%这一条命令可以解决代码跨页时,章节标题,页眉等汉字不显示的问题
\lstset{language=C++, %用于设置语言为C++
	keywordstyle=\color{keywordcolor} \bfseries, %设置关键词
	identifierstyle=,
	basicstyle=\ttfamily, 
	commentstyle=\color{blue} \textit,
	stringstyle=\ttfamily, 
	showstringspaces=false,
	%frame=shadowbox, %边框
	captionpos=b
}
%%%

%\hypersetup{CJKbookmarks=true} %解决section不能使用中文的问题

\begin{document}
\begin{CJK}{UTF8}{gbsn}     %CJK:支持中文

\else
	
\begin{description}
	\item{\textbf{问题}}: Two elements of a binary search tree (BST) are swapped by mistake. Recover the tree without changing its structure. \textit{(leetcode 99)}
	\\BST树的中序遍历是排序好的,那么可以通过中序遍历来看一下哪两个地方发生了交换,发生交换的地方必然是前面那个数比后面的大,只要在遍历过程记录这个位置就可以了.
	\item{\textbf{递归}} : \fbox{时间复杂度O(n), 空间复杂度O($lgn$)}
	\\这段代码很有技巧,需要细细品读. 对于1,5,3,4,2这个序列: fisrt是5, second是2;对于1,3,2,4,5这个序列: first是3, second是2.
	\begin{lstlisting}
void dfs(TreeNode* root, int& last, TreeNode* &first, TreeNode* &second){
	if(root->left){
		dfs(root->left, last, first, second);
	}
	if(root->val < last){
		second = root;
	}
	if(!second){
		first = root;
	}
	last = root->val;
	if(root->right){
		dfs(root->right, last, first, second);
	}
}

void recoverTree(TreeNode* root) {
	if(!root)	return;
	TreeNode *first = NULL, *second = NULL;
	int min = (-1)<<31;
	dfs(root, min, first, second);
	int tmp = first->val;
	first->val = second->val;
	second->val = tmp;
}
	\end{lstlisting}
	\item{\textbf{迭代}} : \fbox{时间复杂度O(n), 空间复杂度O($lgn$)}
	\begin{lstlisting}
void recoverTree(TreeNode* root) {
	if(!root)	return;
	TreeNode *first = NULL, *second = NULL, *cur;
	int last = INT_MIN;
	stack<TreeNode*> s;
	s.push(root);
	while(!s.empty()){
		cur = s.top();
		while(cur){
			s.push(cur->left);
			cur = cur->left;
		}
		s.pop();
		if(!s.empty()){
			cur = s.top();
			if(cur->val < last){
				second = cur;
			}
			if(!second){
				first = cur;
			}
			last = cur->val;
			s.pop();
			s.push(cur->right);
		}
	}
	int tmp = first->val;
	first->val = second->val;
	second->val = tmp;
}
	\end{lstlisting}
\end{description}

\fi

\ifx allfiles undefined
\end{CJK}
\end{document}
\fi


\fi

\ifx allfiles undefined
\end{CJK}
\end{document}
\fi
