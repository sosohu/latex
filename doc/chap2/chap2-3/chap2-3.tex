\ifx allfiles undefined
\documentclass{article}
\usepackage{CJK}

%%%代码
\usepackage{color}
\usepackage{xcolor}
\definecolor{keywordcolor}{rgb}{0.8,0.1,0.5}
\usepackage{listings}
\lstset{breaklines}%这条命令可以让LaTeX自动将长的代码行换行排版
\lstset{extendedchars=false}%这一条命令可以解决代码跨页时,章节标题,页眉等汉字不显示的问题
\lstset{language=C++, %用于设置语言为C++
	keywordstyle=\color{keywordcolor} \bfseries, %设置关键词
	identifierstyle=,
	basicstyle=\ttfamily, 
	commentstyle=\color{blue} \textit,
	stringstyle=\ttfamily, 
	showstringspaces=false,
	%frame=shadowbox, %边框
	captionpos=b
}
%%%

%\hypersetup{CJKbookmarks=true} %解决section不能使用中文的问题

\begin{document}
\begin{CJK}{UTF8}{gbsn}     %CJK:支持中文

\else
	
%二叉树的属性
\subsection{Validate Binary Search Tree}
\ifx allfiles undefined
\documentclass{article}
\usepackage{CJK}
\usepackage{verbatim}

%%%代码
\usepackage{color}
\usepackage{xcolor}
\definecolor{keywordcolor}{rgb}{0.8,0.1,0.5}
\usepackage{listings}
\lstset{breaklines}%这条命令可以让LaTeX自动将长的代码行换行排版
\lstset{extendedchars=false}%这一条命令可以解决代码跨页时,章节标题,页眉等汉字不显示的问题
\lstset{language=C++, %用于设置语言为C++
	keywordstyle=\color{keywordcolor} \bfseries, %设置关键词
	identifierstyle=,
	basicstyle=\ttfamily, 
	commentstyle=\color{blue} \textit,
	stringstyle=\ttfamily, 
	showstringspaces=false,
	%frame=shadowbox, %边框
	captionpos=b
}
%%%

%\hypersetup{CJKbookmarks=true} %解决section不能使用中文的问题

\begin{document}
\begin{CJK}{UTF8}{gbsn}     %CJK:支持中文

\else
	
\begin{description}
	\item{\textbf{问题}}: Given a binary tree, determine if it is a valid binary search tree (BST). \textit{(leetcode 98)}
	\\判断一个二叉树是否是合法的BST,我们可以想到BST树的中序序列是非减序列,于是我们可以使用中序遍历这颗二叉树,在遍历的过程中查看是否有反常的数据.
	\\当然,根据上面说的三种中序遍历的方法,这里同样有三种解法.
	\item{\textbf{递归}} : \fbox{时间复杂度O(n) , 空间复杂度O($lgn$)}
	\begin{lstlisting}
bool dfs(TreeNode *root, int& up){
	if(!root)	return true;
	if(root->left){
		bool left =  dfs(root->left, up);
		if(!left) return false;
	}
	if(root->val <= up && (MIN || up != (-1)<<31)){ 
	    return false;
	}	
    if(root->val == (-1)<<31)
		MIN = true;
	up = root->val;
	if(root->right){
		bool right = dfs(root->right, up);
		if(!right)	return false;
	}
	return true;
}

bool isValidBST(TreeNode *root) {
	if(!root)	return true;
	int up = (-1)<<31;
	MIN = false;
	return dfs(root, up);
}
	\end{lstlisting}
	\textit{这里可以看到一些边界条件的判断,显得有点复杂,其实就是简单的中序遍历}
	\item{\textbf{栈式迭代}} : \fbox{时间复杂度O(n) , 空间复杂度O($lgn$)}
	\begin{lstlisting}
bool isValidBST(TreeNode *root) {
	if(!root)	return false;
	stack<TreeNode*> s;
	TreeNode *p = root;
	s.push(root);
	while(p->left){
		s.push(p->left);
		p = p->left;
	}
	int last = p->val;
	s.pop();
	if(p->right){
		s.push(p->right);
		p = p->right;
		while(p->left){
			s.push(p->left);
			p = p->left;
		}
	}
	while(!s.empty()){
		p = s.top();
		s.pop();
		if(last >= p->val) return false;
		last = p->val;
		if(p->right){
			s.push(p->right);
			p = p->right;
			while(p->left){
				s.push(p->left);
				p = p->left;
			}
		}
	}
	return true;
}
	\end{lstlisting}
	\item{\textbf{Mirror迭代}} : \fbox{时间复杂度O(n) , 空间复杂度O(1)}
	\\这里使用Mirror建立线索然后进行中序遍历,在中序遍历的同时进行判断
	\begin{lstlisting}
bool isValidBST(TreeNode *root) {
	if(!root)	return true;
	TreeNode *curr = root, *next;
	int last = INT_MIN;
	bool isFirst = true;
	bool ret = true;
	while(curr){
		if(!curr->left){
			if(!isFirst && curr->val <= last){
				ret = false;
			}
			if(isFirst)
				isFirst = false;
			last = curr->val;
			curr = curr->right;
			continue;
		}
		next = curr->left;
		while(next->right){
			if(next->right == curr)	break;
			next = next->right;
		}
		if(next->right == curr){
			next->right = NULL;
			if(!isFirst && curr->val <= last){
				ret = false;
			}
			if(isFirst)
				isFirst = false;
			last = curr->val;
			curr = curr->right;
		}else{
			next->right = curr;
			curr = curr->left;
		}
	}
	return ret;
}
	\end{lstlisting}
	\textit{有了Mirror算法,是不是你已经爱上它了,再也不用栈这么麻烦了,不过有一点需要注意的是一旦你使用Mirror算法,那么必须保证把整个树全遍历一遍,不能中途退出,因为那样树的结构被改变了}
\end{description}

\fi

\ifx allfiles undefined
\end{CJK}
\end{document}
\fi

\subsection{Symmetric Tree}
\ifx allfiles undefined
\documentclass{article}
\usepackage{CJK}
\usepackage{verbatim}

%%%代码
\usepackage{color}
\usepackage{xcolor}
\definecolor{keywordcolor}{rgb}{0.8,0.1,0.5}
\usepackage{listings}
\lstset{breaklines}%这条命令可以让LaTeX自动将长的代码行换行排版
\lstset{extendedchars=false}%这一条命令可以解决代码跨页时,章节标题,页眉等汉字不显示的问题
\lstset{language=C++, %用于设置语言为C++
	keywordstyle=\color{keywordcolor} \bfseries, %设置关键词
	identifierstyle=,
	basicstyle=\ttfamily, 
	commentstyle=\color{blue} \textit,
	stringstyle=\ttfamily, 
	showstringspaces=false,
	%frame=shadowbox, %边框
	captionpos=b
}
%%%

%\hypersetup{CJKbookmarks=true} %解决section不能使用中文的问题

\begin{document}
\begin{CJK}{UTF8}{gbsn}     %CJK:支持中文

\else
	
\begin{description}
	\item{\textbf{问题}}: Given a binary tree, check whether it is a mirror of itself (ie, symmetric around its center). \textit{(leetcode 101)}
	\\这是求证树是不是自身Mirror(成镜像).
	\item{\textbf{队列}} : \fbox{时间复杂度O(n) , 空间复杂度O(w), w为树的最大宽度}
	\begin{lstlisting}
bool isSymmetric(TreeNode* root){
	if(!root) return true;
	deque<TreeNode*> left(1, root->left), right(1, root->right);
	TreeNode *l, *r;
	while(!left.empty() && !right.empty()){
		l = left.front();
		r = right.front();
		left.pop_front();
		right.pop_front();
		if(!l && !r)	continue;
		if(!l || !r || l->val != r->val)	return false;
		left.push_back(l->left);
		left.push_back(l->right);
		right.push_back(r->right);
		right.push_back(r->left);
	}
	return true;
}
	\end{lstlisting}
	\item{\textbf{递归}} : \fbox{时间复杂度O(n) , 空间复杂度O($lgn$)}
	\\这里是把一棵树的对称问题看成两棵树的对称问题
	\begin{lstlisting}
bool recursion(TreeNode* root, TreeNode* symm){
	if(!root && !symm)
		return true;
	if(!root || !symm)	return false;
	if(root->val != symm->val)	return false;
	if(root == symm)	return recursion(root->left, symm->right);
	return recursion(root->left, symm->right) && recursion(root->right, symm->left);
}

bool isSymmetric(TreeNode* root){
	if(!root) return true;
	return recursion(root, root);
}
	\end{lstlisting}
	\textit{\\这里还可以延伸出一个问题: 求一个二叉树的镜像树}
\end{description}

\fi

\ifx allfiles undefined
\end{CJK}
\end{document}
\fi

\subsection{Maximum Depth of Binary Tree}
\ifx allfiles undefined
\documentclass{article}
\usepackage{CJK}
\usepackage{verbatim}

%%%代码
\usepackage{color}
\usepackage{xcolor}
\definecolor{keywordcolor}{rgb}{0.8,0.1,0.5}
\usepackage{listings}
\lstset{breaklines}%这条命令可以让LaTeX自动将长的代码行换行排版
\lstset{extendedchars=false}%这一条命令可以解决代码跨页时,章节标题,页眉等汉字不显示的问题
\lstset{language=C++, %用于设置语言为C++
	keywordstyle=\color{keywordcolor} \bfseries, %设置关键词
	identifierstyle=,
	basicstyle=\ttfamily, 
	commentstyle=\color{blue} \textit,
	stringstyle=\ttfamily, 
	showstringspaces=false,
	%frame=shadowbox, %边框
	captionpos=b
}
%%%

%\hypersetup{CJKbookmarks=true} %解决section不能使用中文的问题

\begin{document}
\begin{CJK}{UTF8}{gbsn}     %CJK:支持中文

\else
	
\begin{description}
	\item{\textbf{问题}}: Given a binary tree, find its maximum depth.\textit{(leetcode 104)}
	\\从根节点来看,它的深度就是左右子树深度较大的那个+1,所以很自然的想到递归
	\item{\textbf{递归}} : \fbox{时间复杂度O(n) , 空间复杂度O($lgn$)}
	\\递归代码十分简洁
	\begin{lstlisting}
int maxDepth(TreeNode *root){
	if(!root)   return 0;
	int left = maxDepth(root->left);
	int right = maxDepth(root->right);
	return left < right? right + 1 : left + 1;
}
	\end{lstlisting}
	\qquad除了递归,其实这道题能不能用迭代的做法呢?答案是肯定的,最初你可能会想到用两个栈,一个栈存放节点,一个栈存放深度,其实可以把这个两者打包成一个pair,使用一个栈就可以啦
	\item{\textbf{迭代}} : \fbox{时间复杂度O(n) , 空间复杂度O($lgn$)}
	\begin{lstlisting}
int maxDepth(TreeNode *root) {
	if(!root)	return 0;
	stack<pair<TreeNode*, int> > s;
	s.push(make_pair(root, 1));
	pair<TreeNode*, int> curr;
	int result = INT_MIN;
	while(!s.empty()){
		curr = s.top();
		s.pop();
		if(!curr.first->left && !curr.first->right){
			if(result < curr.second)
				result = curr.second;
			continue;
		}
		if(curr.first->left){
			s.push(make_pair(curr.first->left, curr.second + 1));
		}
		if(curr.first->right){
			s.push(make_pair(curr.first->right, curr.second + 1));
		}
	}
	return result;
}
	\end{lstlisting}
\end{description}

\fi

\ifx allfiles undefined
\end{CJK}
\end{document}
\fi

\subsection{Minimum Depth of Binary Tree}
\ifx allfiles undefined
\documentclass{article}
\usepackage{CJK}
\usepackage{verbatim}

%%%代码
\usepackage{color}
\usepackage{xcolor}
\definecolor{keywordcolor}{rgb}{0.8,0.1,0.5}
\usepackage{listings}
\lstset{breaklines}%这条命令可以让LaTeX自动将长的代码行换行排版
\lstset{extendedchars=false}%这一条命令可以解决代码跨页时,章节标题,页眉等汉字不显示的问题
\lstset{language=C++, %用于设置语言为C++
	keywordstyle=\color{keywordcolor} \bfseries, %设置关键词
	identifierstyle=,
	basicstyle=\ttfamily, 
	commentstyle=\color{blue} \textit,
	stringstyle=\ttfamily, 
	showstringspaces=false,
	%frame=shadowbox, %边框
	captionpos=b
}
%%%

%\hypersetup{CJKbookmarks=true} %解决section不能使用中文的问题

\begin{document}
\begin{CJK}{UTF8}{gbsn}     %CJK:支持中文

\else
	
\begin{description}
	\item{\textbf{问题}}: Given a binary tree, find its minimum depth. The minimum depth is the number of nodes along the shortest path from the root node down to the nearest leaf node. \textit{(leetcode 111)}
	\item{\textbf{递归}} : \fbox{时间复杂度O(n), 空间复杂度O($lgn$)}
	\\自下而上的递归,非常的简单
	\begin{lstlisting}
int minDepth(TreeNode *root) {
	if(!root)	return 0;
	if(!root->left && !root->right)	return 1;
	if(!root->left)
		return minDepth(root->right) + 1;
	if(!root->right)
		return minDepth(root->left) + 1;
	return min(minDepth(root->left), minDepth(root->right)) + 1;
}
	\end{lstlisting}
	\item{\textbf{DFS}} : \fbox{时间复杂度O(n) , 空间复杂度O($lgn$)}
	\\这也是递归,但是是一种自上而下的递归,可以进行剪枝而不必把整个树都访问一遍
	\begin{lstlisting}
void dfs(TreeNode *root, int &result, int depth){
	if(result < depth + 1) return;
	if(!root->left && !root->right){
		result = depth + 1;
		return;
	}
	if(root->left)
		dfs(root->left, result, depth + 1);
	if(root->right)
		dfs(root->right, result, depth + 1);
}

	int minDepth(TreeNode *root) {
	if(!root)	return 0;
	int result = INT_MAX;
	dfs(root, result, 0);
	return result;
}
	\end{lstlisting}
	\item{\textbf{迭代}} : \fbox{时间复杂度O(n) , 空间复杂度O($lgn$)}
	\\同样我们也可以剪枝
	\begin{lstlisting}
int minDepth(TreeNode *root) {
	if(!root)	return 0;
	stack<pair<TreeNode*, int> > s;
	s.push(make_pair(root, 1));
	int	 result = -((1<<31) + 1);
	TreeNode *node;
	int depth;
	while(!s.empty()){
		node = s.top().first;
		depth = s.top().second;
		s.pop();
		if(result < depth)	continue;
		if(!node->left && !node->right)
			result = depth;
	
		if(node->left )
			s.push(make_pair(node->left, depth + 1));
		if(node->right)
			s.push(make_pair(node->right, depth + 1));
	}
	return result;
}
	\end{lstlisting}
\end{description}

\fi

\ifx allfiles undefined
\end{CJK}
\end{document}
\fi

\subsection{Balanced Binary Tree}
\ifx allfiles undefined
\documentclass{article}
\usepackage{CJK}
\usepackage{verbatim}

%%%代码
\usepackage{color}
\usepackage{xcolor}
\definecolor{keywordcolor}{rgb}{0.8,0.1,0.5}
\usepackage{listings}
\lstset{breaklines}%这条命令可以让LaTeX自动将长的代码行换行排版
\lstset{extendedchars=false}%这一条命令可以解决代码跨页时,章节标题,页眉等汉字不显示的问题
\lstset{language=C++, %用于设置语言为C++
	keywordstyle=\color{keywordcolor} \bfseries, %设置关键词
	identifierstyle=,
	basicstyle=\ttfamily, 
	commentstyle=\color{blue} \textit,
	stringstyle=\ttfamily, 
	showstringspaces=false,
	%frame=shadowbox, %边框
	captionpos=b
}
%%%

%\hypersetup{CJKbookmarks=true} %解决section不能使用中文的问题

\begin{document}
\begin{CJK}{UTF8}{gbsn}     %CJK:支持中文

\else
	
\begin{description}
	\item{\textbf{问题}}: Given a binary tree, determine if it is height-balanced. \textit{(leetcode 110)}
	\item{\textbf{递归}} : \fbox{时间复杂度O(n) , 空间复杂度O($lgn$)}
	\\先判断左子树是否高度平衡并返回左子树高度,再判断右子树是否高度平衡,再返回右子树高度,根据左右子树高度再判断当前树是否平衡.
	\begin{lstlisting}
bool dfs(TreeNode *root, int &hight){
	if(!root){
		hight = 0;
		return true;
	}
	int left, right;
	bool is_left = dfs(root->left, left);
	bool is_right = dfs(root->right, right);
	hight = left > right? left + 1 : right + 1;
	return is_left && is_right && (abs(left - right) < 2);
}

bool isBalanced(TreeNode *root) {
	int hight;
	return dfs(root, hight);
}
	\end{lstlisting}
\end{description}

\fi

\ifx allfiles undefined
\end{CJK}
\end{document}
\fi

\subsection{Path Sum}
\ifx allfiles undefined
\documentclass{article}
\usepackage{CJK}
\usepackage{verbatim}

%%%代码
\usepackage{color}
\usepackage{xcolor}
\definecolor{keywordcolor}{rgb}{0.8,0.1,0.5}
\usepackage{listings}
\lstset{breaklines}%这条命令可以让LaTeX自动将长的代码行换行排版
\lstset{extendedchars=false}%这一条命令可以解决代码跨页时,章节标题,页眉等汉字不显示的问题
\lstset{language=C++, %用于设置语言为C++
	keywordstyle=\color{keywordcolor} \bfseries, %设置关键词
	identifierstyle=,
	basicstyle=\ttfamily, 
	commentstyle=\color{blue} \textit,
	stringstyle=\ttfamily, 
	showstringspaces=false,
	%frame=shadowbox, %边框
	captionpos=b
}
%%%

%\hypersetup{CJKbookmarks=true} %解决section不能使用中文的问题

\begin{document}
\begin{CJK}{UTF8}{gbsn}     %CJK:支持中文

\else
	
\begin{description}
	\item{\textbf{问题}}: Given a binary tree and a sum, determine if the tree has a root-to-leaf path such that adding up all the values along the path equals the given sum. \textit{(leetcode 112)}
	\item{\textbf{递归}} : \fbox{时间复杂度O(n) , 空间复杂度O($lgn$)}
	\\先减去当前节点的值,剩余的值再分别递归求解左右子树
	\begin{lstlisting}
bool hasPathSum(TreeNode *root, int sum) {
	if(root == NULL) return false;
	int val = root->val;
	sum = sum - val;
	if(sum == 0 && !root->left && !root->right)
		return true;

	bool left = false; 
	if(root->left){
		left = hasPathSum(root->left, sum);	
	}
	if(left) return true;
	if(root->right)
		return hasPathSum(root->right, sum);
	return false;
}
	\end{lstlisting}
	\qquad仿照最大/小深度问题,可以在迭代栈加上附加路径和这个信息,那么实现起来就非常简单了
	\item{\textbf{迭代}} : \fbox{时间复杂度O(n) , 空间复杂度O($lgn$)}
	\begin{lstlisting}
bool hasPathSum(TreeNode *root, int sum) {
	if(!root)	return false;
	stack<pair<TreeNode*, int> > s;
	s.push(make_pair(root, root->val));
	TreeNode *node;
	int e;
	while(!s.empty()){
		node = s.top().first;
		e = s.top().second;
		s.pop();
		if(!node->left && !node->right && e == sum)
			return true;
		if(node->left)
			s.push(make_pair(node->left, e + node->left->val));
		if(node->right)
			s.push(make_pair(node->right, e + node->right->val));
	}
	return false;
}
	\end{lstlisting}
	\qquad可以看得出,迭代栈加入附件信息是一个常用的技巧,需要深刻理解和掌握.
\end{description}

\fi

\ifx allfiles undefined
\end{CJK}
\end{document}
\fi

\subsection{Path Sum II}
\ifx allfiles undefined
\documentclass{article}
\usepackage{CJK}
\usepackage{verbatim}

%%%代码
\usepackage{color}
\usepackage{xcolor}
\definecolor{keywordcolor}{rgb}{0.8,0.1,0.5}
\usepackage{listings}
\lstset{breaklines}%这条命令可以让LaTeX自动将长的代码行换行排版
\lstset{extendedchars=false}%这一条命令可以解决代码跨页时,章节标题,页眉等汉字不显示的问题
\lstset{language=C++, %用于设置语言为C++
    keywordstyle=\color{keywordcolor} \bfseries, %设置关键词
    identifierstyle=,
    basicstyle=\ttfamily, 
    commentstyle=\color{blue} \textit,
    stringstyle=\ttfamily, 
    showstringspaces=false,
    %frame=shadowbox, %边框
    captionpos=b
}
%%%

%\hypersetup{CJKbookmarks=true} %解决section不能使用中文的问题

\begin{document}
\begin{CJK}{UTF8}{gbsn}     %CJK:支持中文

\else
    
\begin{description}
    \item{\textbf{问题}}: Given a binary tree and a sum, find all root-to-leaf paths where each path's sum equals the given sum. \textit{(leetcode 113)}
    \\这题是Path Sum问题的延续,解法其实是一模一样的.
    \item{\textbf{递归}} : \fbox{时间复杂度O(n) , 空间复杂度O(n)}
    \\这里有个技巧需要注意的就是,可以使用一个引用参数cur来记录跟节点到当前节点这条路径中的所有值,我们在进入某个节点后cur要push这个节点,在离开这个节点后就要pop这个节点
    \begin{lstlisting}
vector<vector<int> > pathSum(TreeNode *root, int sum){
    vector<int> cur;
    vector<vector<int> > result;
    recursion(root, cur, result, sum);
    return result;
}

void recursion(TreeNode *root, vector<int> &cur, vector<vector<int> > &result, int sum){
    if(!root)    return;
    cur.push_back(root->val);
    if(!root->left && !root->right && root->val == sum)
        result.push_back(cur);
    if(root->left)
        recursion(root->left, cur, result, sum - root->val);
    if(root->right)
        recursion(root->right, cur, result, sum - root->val);
    cur.pop_back();
}
    \end{lstlisting}
    \qquad我们知道Path Sum有迭代的做法,但是那种通过栈附加信息的做法对于我们这题还要求求出路径的问题不是很适合,所以就没有写出这种做法了,你有什么好想法么?
\end{description}

\fi

\ifx allfiles undefined
\end{CJK}
\end{document}
\fi

\subsection{Binary Tree Maximum Path Sum}
\ifx allfiles undefined
\documentclass{article}
\usepackage{CJK}
\usepackage{verbatim}

%%%代码
\usepackage{color}
\usepackage{xcolor}
\definecolor{keywordcolor}{rgb}{0.8,0.1,0.5}
\usepackage{listings}
\lstset{breaklines}%这条命令可以让LaTeX自动将长的代码行换行排版
\lstset{extendedchars=false}%这一条命令可以解决代码跨页时,章节标题,页眉等汉字不显示的问题
\lstset{language=C++, %用于设置语言为C++
	keywordstyle=\color{keywordcolor} \bfseries, %设置关键词
	identifierstyle=,
	basicstyle=\ttfamily, 
	commentstyle=\color{blue} \textit,
	stringstyle=\ttfamily, 
	showstringspaces=false,
	%frame=shadowbox, %边框
	captionpos=b
}
%%%

%\hypersetup{CJKbookmarks=true} %解决section不能使用中文的问题

\begin{document}
\begin{CJK}{UTF8}{gbsn}     %CJK:支持中文

\else
	
\begin{description}
	\item{\textbf{问题}}: Given a binary tree, find the maximum path sum.The path may start and end at any node in the tree. \textit{(leetcode 124)}
	\item{\textbf{递归}} : \fbox{时间复杂度 , 空间复杂度O}
	\\这里先说约定一些叫法:
	\begin{itemize}
	\item[1] 最大路径和: max\{树中任意两节点之间的路径和\}
	\item[2] 最大到根路径和: max\{树中任意节点到根节点的路径和\}
	\end{itemize}
	\qquad这里我们知道,这个路径肯定是有个拐点,那么这个拐点肯定是某个子树的根节点,所以我们只要递归的求每个子树的过根最大路径和,然后在这些路径和中选出最大的那个就可以了.
	
	\qquad过根最大路径和怎么求呢?可以先分别求出左右子树的最大到根路径和,根据这个两个路径和与根节点则可以求出当前树的最大路径和.
	\begin{lstlisting}
int dfs(TreeNode *root, int &result){
	if(!root)	return 0;
	if(!root->left && !root->right){
		if(result < root->val)
			result = root->val;
		return root->val > 0? root->val : 0;
	}
	int left = 0, right = 0;
	if(root->left){
		left = dfs(root->left, result);
	}
	if(root->right){
		right = dfs(root->right, result);
	}
	int cur = root->val + left + right;
	if(result < cur)
		result = cur;
	int add = left > right ? left : right;
	if(add < 0)	return root->val > 0? root->val : 0;
	return root->val + add > 0 ? root->val + add : 0;
}

int maxPathSum(TreeNode *root) {
	int result = INT_MIN;
	dfs(root, result);
	return result;
}
	\end{lstlisting}
	\textit{}
\end{description}

\fi

\ifx allfiles undefined
\end{CJK}
\end{document}
\fi


\fi

\ifx allfiles undefined
\end{CJK}
\end{document}
\fi
