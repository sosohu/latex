\ifx allfiles undefined
\documentclass{article}
\usepackage{CJK}
\usepackage{verbatim}

%%%代码
\usepackage{color}
\usepackage{xcolor}
\definecolor{keywordcolor}{rgb}{0.8,0.1,0.5}
\usepackage{listings}
\lstset{breaklines}%这条命令可以让LaTeX自动将长的代码行换行排版
\lstset{extendedchars=false}%这一条命令可以解决代码跨页时,章节标题,页眉等汉字不显示的问题
\lstset{language=C++, %用于设置语言为C++
	keywordstyle=\color{keywordcolor} \bfseries, %设置关键词
	identifierstyle=,
	basicstyle=\ttfamily, 
	commentstyle=\color{blue} \textit,
	stringstyle=\ttfamily, 
	showstringspaces=false,
	%frame=shadowbox, %边框
	captionpos=b
}
%%%

%\hypersetup{CJKbookmarks=true} %解决section不能使用中文的问题

\begin{document}
\begin{CJK}{UTF8}{gbsn}     %CJK:支持中文

\else
	
\begin{description}
	\item{\textbf{问题}}: Given a binary tree, find its maximum depth.\textit{(leetcode 104)}
	\\从根节点来看,它的深度就是左右子树深度较大的那个+1,所以很自然的想到递归
	\item{\textbf{递归}} : \fbox{时间复杂度O(n) , 空间复杂度O($lgn$)}
	\\递归代码十分简洁
	\begin{lstlisting}
int maxDepth(TreeNode *root){
	if(!root)   return 0;
	int left = maxDepth(root->left);
	int right = maxDepth(root->right);
	return left < right? right + 1 : left + 1;
}
	\end{lstlisting}
	\qquad除了递归,其实这道题能不能用迭代的做法呢?答案是肯定的,最初你可能会想到用两个栈,一个栈存放节点,一个栈存放深度,其实可以把这个两者打包成一个pair,使用一个栈就可以啦
	\item{\textbf{迭代}} : \fbox{时间复杂度O(n) , 空间复杂度O($lgn$)}
	\begin{lstlisting}
int maxDepth(TreeNode *root) {
	if(!root)	return 0;
	stack<pair<TreeNode*, int> > s;
	s.push(make_pair(root, 1));
	pair<TreeNode*, int> curr;
	int result = INT_MIN;
	while(!s.empty()){
		curr = s.top();
		s.pop();
		if(!curr.first->left && !curr.first->right){
			if(result < curr.second)
				result = curr.second;
			continue;
		}
		if(curr.first->left){
			s.push(make_pair(curr.first->left, curr.second + 1));
		}
		if(curr.first->right){
			s.push(make_pair(curr.first->right, curr.second + 1));
		}
	}
	return result;
}
	\end{lstlisting}
\end{description}

\fi

\ifx allfiles undefined
\end{CJK}
\end{document}
\fi
