\ifx allfiles undefined
\documentclass{article}
\usepackage{CJK}
\usepackage{verbatim}

%%%代码
\usepackage{color}
\usepackage{xcolor}
\definecolor{keywordcolor}{rgb}{0.8,0.1,0.5}
\usepackage{listings}
\lstset{breaklines}%这条命令可以让LaTeX自动将长的代码行换行排版
\lstset{extendedchars=false}%这一条命令可以解决代码跨页时,章节标题,页眉等汉字不显示的问题
\lstset{language=C++, %用于设置语言为C++
    keywordstyle=\color{keywordcolor} \bfseries, %设置关键词
    identifierstyle=,
    basicstyle=\ttfamily, 
    commentstyle=\color{blue} \textit,
    stringstyle=\ttfamily, 
    showstringspaces=false,
    %frame=shadowbox, %边框
    captionpos=b
}
%%%

%\hypersetup{CJKbookmarks=true} %解决section不能使用中文的问题

\begin{document}
\begin{CJK}{UTF8}{gbsn}     %CJK:支持中文

\else
    
\begin{description}
    \item{\textbf{问题}}: Given a binary tree and a sum, determine if the tree has a root-to-leaf path such that adding up all the values along the path equals the given sum. \textit{(leetcode 112)}
    \item{\textbf{递归}} : \fbox{时间复杂度O(n) , 空间复杂度O($lgn$)}
    \\先减去当前节点的值,剩余的值再分别递归求解左右子树
    \begin{lstlisting}
bool hasPathSum(TreeNode *root, int sum) {
    if(root == NULL) return false;
    int val = root->val;
    sum = sum - val;
    if(sum == 0 && !root->left && !root->right)
        return true;

    bool left = false; 
    if(root->left){
        left = hasPathSum(root->left, sum);    
    }
    if(left) return true;
    if(root->right)
        return hasPathSum(root->right, sum);
    return false;
}
    \end{lstlisting}
    \qquad仿照最大/小深度问题,可以在迭代栈加上附加路径和这个信息,那么实现起来就非常简单了
    \item{\textbf{迭代}} : \fbox{时间复杂度O(n) , 空间复杂度O($lgn$)}
    \begin{lstlisting}
bool hasPathSum(TreeNode *root, int sum) {
    if(!root)    return false;
    stack<pair<TreeNode*, int> > s;
    s.push(make_pair(root, root->val));
    TreeNode *node;
    int e;
    while(!s.empty()){
        node = s.top().first;
        e = s.top().second;
        s.pop();
        if(!node->left && !node->right && e == sum)
            return true;
        if(node->left)
            s.push(make_pair(node->left, e + node->left->val));
        if(node->right)
            s.push(make_pair(node->right, e + node->right->val));
    }
    return false;
}
    \end{lstlisting}
    \qquad可以看得出,迭代栈加入附件信息是一个常用的技巧,需要深刻理解和掌握. 另外,这道题还可以变形为求二叉树最远两个节点距离的问题,这就相当于把每个节点的权值都设为1,方法和这道题一样,而且还要比它简单.
\end{description}

\fi

\ifx allfiles undefined
\end{CJK}
\end{document}
\fi
