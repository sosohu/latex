\ifx allfiles undefined
\documentclass{article}
\usepackage{CJK}
\usepackage{verbatim}

%%%代码
\usepackage{color}
\usepackage{xcolor}
\definecolor{keywordcolor}{rgb}{0.8,0.1,0.5}
\usepackage{listings}
\lstset{breaklines}%这条命令可以让LaTeX自动将长的代码行换行排版
\lstset{extendedchars=false}%这一条命令可以解决代码跨页时,章节标题,页眉等汉字不显示的问题
\lstset{language=C++, %用于设置语言为C++
	keywordstyle=\color{keywordcolor} \bfseries, %设置关键词
	identifierstyle=,
	basicstyle=\ttfamily, 
	commentstyle=\color{blue} \textit,
	stringstyle=\ttfamily, 
	showstringspaces=false,
	%frame=shadowbox, %边框
	captionpos=b
}
%%%

%\hypersetup{CJKbookmarks=true} %解决section不能使用中文的问题

\begin{document}
\begin{CJK}{UTF8}{gbsn}     %CJK:支持中文

\else
	
\begin{description}
	\item{\textbf{问题}}: Given a binary tree containing digits from 0-9 only, each root-to-leaf path could represent a number.An example is the root-to-leaf path 1->2->3 which represents the number 123.Find the total sum of all root-to-leaf numbers. \textit{(leetcode 129)}
	\\其实这还是属于那种自上而下带有附加信息过来的问题,和最大/小深度问题是一样的.
	\item{\textbf{递归}} : \fbox{时间复杂度O(n), 空间复杂度O($lgn$)}
	\\从上面传来上面路径产生的数字,类似于传来上面路径的深度(最大深度问题)和上面路径的和(Path Sum问题)
	\begin{lstlisting}
void dfs(TreeNode *root, int track, int &sum){
	if(!root)	return;
	track = track*10 + root->val;
	if(!root->left && !root->right){
		sum = sum + track;
	}
	if(root->left)
		dfs(root->left, track, sum);
	if(root->right)
		dfs(root->right, track, sum);
}

int sumNumbers(TreeNode *root){
	int sum = 0;
	int track = 0;
	dfs(root, track, sum);
	return sum;
}
	\end{lstlisting}
	\qquad同样,类似与Path Sum这类问题,可以常用附加路径信息的栈实现迭代,具体做法就是当前节点左右孩子入栈时候当前节点的附加信息值*10加左右孩子节点的值然后作为左右孩子的附加信息值,这里就不再写出.
\end{description}

\fi

\ifx allfiles undefined
\end{CJK}
\end{document}
\fi
