\documentclass[oneside]{book}
\usepackage{CJK}
\usepackage{verbatim}
\usepackage{amsmath}
\usepackage{titlesec}
\usepackage{fancyhdr} %页眉页脚布局
\usepackage{tocloft} % 目录相关


%%%代码
\usepackage{color}
\usepackage{xcolor}
\definecolor{keywordcolor}{rgb}{0.8,0.1,0.5}
\usepackage{listings}
\lstset{extendedchars=false}%这一条命令可以解决代码跨页时,章节标题,页眉等汉字不显示的问题
\definecolor{darkgreen}{rgb}{0.0, 0.2, 0.13}
\definecolor{grey}{rgb}{0.55, 0.57, 0.67}
\definecolor{darkgray}{rgb}{0.66, 0.66, 0.66}
\lstset{language=C++, %用于设置语言为C++
    numbers=left,
    numberstyle=\ttfamily\scriptsize,
    backgroundcolor=\color{darkgray}, frame=single,framesep=5pt,framexleftmargin=8mm,%frameround=fttt,
    basicstyle=\ttfamily\small,
    keywordstyle=\ttfamily\bf\color{blue},
    ndkeywordstyle=\ttfamily\bf\color{brown},
    commentstyle=\color{darkgreen},
    identifierstyle=\ttfamily\color{black}\bfseries,
    stringstyle=\color{pink}\ttfamily,showstringspaces=false,
    breaklines=true,
    escapeinside=``
}
%%%

%\hypersetup{CJKbookmarks=true} %解决section不能使用中文的问题

\usepackage[colorlinks,linkcolor=black,anchorcolor=blue,citecolor=green,CJKbookmarks=true]{hyperref}
\begin{document}
\begin{CJK}{UTF8}{gbsn}     %CJK:支持中文

%%文章中章节等转化为中文
\renewcommand{\contentsname}{目录}
\renewcommand{\figurename}{图}
\renewcommand{\tablename}{表}
\titleformat{\chapter}{\centering\Huge\bfseries}{第\,\thechapter\,章}{1em}{}
%%

%%目录中章节
\renewcommand{\cftchapfont}{\bfseries}
\renewcommand{\cftchappagefont}{\bfseries}
\renewcommand{\cftchappresnum}{第}
\renewcommand{\cftchapaftersnum}{章:}
\renewcommand{\cftchapnumwidth}{4em}      %  add 'chapter' word before number
%%

%%布局
\pagestyle{fancy}
\renewcommand{\chaptermark}[1]{\markboth{\small 第\,\thechapter\,章\quad #1}{}}
\renewcommand{\sectionmark}[1]{\markright{\small\thesection\quad #1}{}}
\fancyhf{}
\fancyhead[ER]{\leftmark}
\fancyhead[OL]{\rightmark}
\fancyhead[EL,OR]{$\cdot$\ \thepage\ $\cdot$}
\renewcommand{\headrulewidth}{0.4pt}
%%

\title{我的解题报告}
\author{胡庆海}
\date{}

\maketitle

%\newpage
\tableofcontents

\chapter{链表}
\ifx allfiles undefined
\documentclass{article}
\usepackage{CJK}

%%%代码
\usepackage{color}
\usepackage{xcolor}
\definecolor{keywordcolor}{rgb}{0.8,0.1,0.5}
\usepackage{listings}
\lstset{breaklines}%这条命令可以让LaTeX自动将长的代码行换行排版
\lstset{extendedchars=false}%这一条命令可以解决代码跨页时,章节标题,页眉等汉字不显示的问题
\lstset{language=C++, %用于设置语言为C++
    keywordstyle=\color{keywordcolor} \bfseries, %设置关键词
    identifierstyle=,
    basicstyle=\ttfamily, 
    commentstyle=\color{blue} \textit,
    stringstyle=\ttfamily, 
    showstringspaces=false,
    %frame=shadowbox, %边框
    captionpos=b
}
%%%

%\hypersetup{CJKbookmarks=true} %解决section不能使用中文的问题

\begin{document}
\begin{CJK}{UTF8}{gbsn}     %CJK:支持中文

\else

\begin{lstlisting}
int main(int argc, char *argv[])
{
    int i;
    return 0;
}
\end{lstlisting}

\fi

\ifx allfiles undefined
\end{CJK}
\end{document}
\fi


\chapter{树}
\ifx allfiles undefined
\documentclass{article}
\usepackage{CJK}

%%%代码
\usepackage{color}
\usepackage{xcolor}
\definecolor{keywordcolor}{rgb}{0.8,0.1,0.5}
\usepackage{listings}
\lstset{breaklines}%这条命令可以让LaTeX自动将长的代码行换行排版
\lstset{extendedchars=false}%这一条命令可以解决代码跨页时,章节标题,页眉等汉字不显示的问题
\lstset{language=C++, %用于设置语言为C++
	keywordstyle=\color{keywordcolor} \bfseries, %设置关键词
	identifierstyle=,
	basicstyle=\ttfamily, 
	commentstyle=\color{blue} \textit,
	stringstyle=\ttfamily, 
	showstringspaces=false,
	%frame=shadowbox, %边框
	captionpos=b
}
%%%

%\hypersetup{CJKbookmarks=true} %解决section不能使用中文的问题

\begin{document}
\begin{CJK}{UTF8}{gbsn}     %CJK:支持中文

\else
	
	%树的基本问题分类
\section{二叉树的遍历}
\ifx allfiles undefined
\documentclass{article}
\usepackage{CJK}

%%%代码
\usepackage{color}
\usepackage{xcolor}
\definecolor{keywordcolor}{rgb}{0.8,0.1,0.5}
\usepackage{listings}
\lstset{breaklines}%这条命令可以让LaTeX自动将长的代码行换行排版
\lstset{extendedchars=false}%这一条命令可以解决代码跨页时,章节标题,页眉等汉字不显示的问题
\lstset{language=C++, %用于设置语言为C++
	keywordstyle=\color{keywordcolor} \bfseries, %设置关键词
	identifierstyle=,
	basicstyle=\ttfamily, 
	commentstyle=\color{blue} \textit,
	stringstyle=\ttfamily, 
	showstringspaces=false,
	%frame=shadowbox, %边框
	captionpos=b
}
%%%

%\hypersetup{CJKbookmarks=true} %解决section不能使用中文的问题

\begin{document}
\begin{CJK}{UTF8}{gbsn}     %CJK:支持中文

\else
	
	%二叉树的遍历
\subsection{PreOrederTraversal}
\ifx allfiles undefined
\documentclass{article}
\usepackage{CJK}
\usepackage{verbatim}

%%%代码
\usepackage{color}
\usepackage{xcolor}
\definecolor{keywordcolor}{rgb}{0.8,0.1,0.5}
\usepackage{listings}
\lstset{breaklines}%这条命令可以让LaTeX自动将长的代码行换行排版
\lstset{extendedchars=false}%这一条命令可以解决代码跨页时,章节标题,页眉等汉字不显示的问题
\lstset{language=C++, %用于设置语言为C++
    keywordstyle=\color{keywordcolor} \bfseries, %设置关键词
    identifierstyle=,
    basicstyle=\ttfamily, 
    commentstyle=\color{blue} \textit,
    stringstyle=\ttfamily, 
    showstringspaces=false,
    %frame=shadowbox, %边框
    captionpos=b
}
%%%

%\hypersetup{CJKbookmarks=true} %解决section不能使用中文的问题

\begin{document}
\begin{CJK}{UTF8}{gbsn}     %CJK:支持中文

\else
    
%二叉树的先序遍历
\begin{description}
    \item{\textbf{问题}}: Given a binary tree, return the preorder traversal of its nodes' values.(\textit{leetcode 144})
    \item{\textbf{递归}} : \fbox{时间复杂度O(n) , 空间复杂度O($lgn$)}
    \\前序遍历的递归写法,非常简单,只需要先访问跟节点,再递归的执行左子树和右子树
    \begin{lstlisting}
void BiTree::PreOrderTraversal(TreeNode* root){
    if(!root)    return;
    cout<<root->val<<endl;
    if(root->left)
        PreOrderTraversal(root->left);
    if(root->right)
        PreOrderTraversal(root->right);
}
    \end{lstlisting}
    \item{\textbf{栈式迭代}} : \fbox{时间复杂度O(n) , 空间复杂度O($lgn$)}
    \\使用栈来模拟递归过程也是很显而易见的,具体做法就是先访问根节点,然后先让右孩子入栈接着是左孩子,然后左孩子出栈后重复这个过程
    \begin{lstlisting}
void BiTree::PreOrderTraversal(TreeNode* root){
    if(!root)    return;
    stack<TreeNode*> s;
    TreeNode *cur;
    s.push(root);
    while(!s.empty()){
        cur = s.top();
        s.pop();
        cout<<cur->val<<endl;
        if(cur->right)
            s.push(cur->right);
        if(cur->left)
            s.push(cur->left);
    }
}
    \end{lstlisting}
    \item{\textbf{Mirror迭代}} : \fbox{时间复杂度O(n) , 空间复杂度O(1)}
    \\Mirror迭代法是经过Lee介绍过来的,非常的迷人,它的做法就是在遍历的过程中,访问了当前节点之后,先找当前节点的前驱并让此前驱的右孩子指向它,再访问它的左孩子并重复这个过程。在此之后会访问到它前驱然后再次回到当前节点,此时再次试图建立前驱,发现已经建立了,这就说明当前节点左边已经全部遍历完,则继续访问当前节点的右边节点,不断的重复此过程。
    \begin{lstlisting}
void BiTree::PreOrderTraversal(TreeNode* root){
    if(!root)    return;
    TreeNode *curr = root, *next;
    while(curr){
        next = curr->left;
        if(!next){
            cout<<curr->val<<endl;
            curr = curr->right;
            continue;
        }
        while(next->right && next->right != curr){
            next = next->right;
        }
        if(next->right == curr){
            next->right = NULL;
            curr = curr->right;
        }else{
            cout<<curr->val<<endl;
            next->right = curr;
            curr = curr->left;
        }
    }
}
    \end{lstlisting}
    \textit{这个Mirror算法一旦掌握后,威力无穷,你可以用它方便的建立二叉树前序索引并且遇到那些要求用迭代来实现的二叉树问题也可以很快的写出来}
\end{description}

\fi

\ifx allfiles undefined
\end{CJK}
\end{document}
\fi

\subsection{InOrederTraversal}
\ifx allfiles undefined
\documentclass{article}
\usepackage{CJK}
\usepackage{verbatim}

%%%代码
\usepackage{color}
\usepackage{xcolor}
\definecolor{keywordcolor}{rgb}{0.8,0.1,0.5}
\usepackage{listings}
\lstset{breaklines}%这条命令可以让LaTeX自动将长的代码行换行排版
\lstset{extendedchars=false}%这一条命令可以解决代码跨页时,章节标题,页眉等汉字不显示的问题
\lstset{language=C++, %用于设置语言为C++
	keywordstyle=\color{keywordcolor} \bfseries, %设置关键词
	identifierstyle=,
	basicstyle=\ttfamily, 
	commentstyle=\color{blue} \textit,
	stringstyle=\ttfamily, 
	showstringspaces=false,
	%frame=shadowbox, %边框
	captionpos=b
}
%%%

%\hypersetup{CJKbookmarks=true} %解决section不能使用中文的问题

\begin{document}
\begin{CJK}{UTF8}{gbsn}     %CJK:支持中文

\else
	
%XXX 问题
\begin{description}
	\item{\textbf{问题}}: Given a binary tree, return the inorder traversal of its nodes' values. \textit{(leetcode 94)}
	\item{\textbf{递归}} : \fbox{时间复杂度O(n) , 空间复杂度O($lgn$)}
	\\
	\begin{lstlisting}
void BiTree::InOrderTraversal(TreeNode* root){
	if(!root)	return;
	if(root->left)
		InOrderTraversal(root->left);
	cout<<root->val<<endl;
	if(root->right)
		InOrderTraversal(root->right);
	\end{lstlisting}
	\item{\textbf{栈式迭代}} : \fbox{时间复杂度O(n) , 空间复杂度O($lgn$)}
	\\这里需要说一下的是,数据结构那本书上写了两种栈式迭代方法,这是其中之一,使用两重循环的那个
	\begin{lstlisting}
void BiTree::InOrderTraversal(TreeNode* root){
	vector<int> data;
	if(!root)	return data;
	stack<TreeNode*> s;
	TreeNode *pos = root;
	while(!s.empty() || pos){
		while(pos){
			s.push(pos);
			pos = pos->left;
		}
		pos = s.top();
		s.pop();
		std::cout<<pos->val<<std::endl;
		pos = pos->right;  //这个非常重要
	}
}
	\end{lstlisting}
	\item{\textbf{栈式迭代}} : \fbox{时间复杂度O(n) , 空间复杂度O($lgn$)}
	\\这里需要说一下的是,数据结构那本书上写了两种栈式迭代方法,这是其中之二,使用一重循环,实际上是一样的
	\begin{lstlisting}
void BiTree::InOrderTraversal(TreeNode* root){
	vector<int> data;
	if(!root)	return data;
	stack<TreeNode*> s;
	TreeNode *pos = root;
	while(!s.empty() || pos){
		if(pos){
			s.push(pos);
			pos = pos->left;
		}else{
			pos = s.top();
			s.pop();
			std::cout<<pos->val<<std::endl;
			pos = pos->right;  //这个非常重要
		}
	}
}
	\end{lstlisting}
	\item{\textbf{Mirror迭代}} : \fbox{时间复杂度O(n) , 空间复杂度O(1)}
	\\这里Mirror方法和前序的Mirror方法基本一样,唯一的区别就是打印当前值的时机
	\begin{lstlisting}
void BiTree::InOrderTraversal(TreeNode* root){
	if(!root)	return;
	TreeNode *curr = root, *next;
	while(curr){
		next = curr->left;
		if(!next){
			cout<<curr->val<<endl;
			curr = curr->right;
			continue;
		}
		while(next->right && next->right != curr){
			next = next->right;
		}
		if(next->right == curr){
			next->right = NULL;
			cout<<curr->val<<endl;
			curr = curr->right;
		}else{
			next->right = curr;
			curr = curr->left;
		}
	}
}
	\end{lstlisting}
\end{description}

\fi

\ifx allfiles undefined
\end{CJK}
\end{document}
\fi


\fi

\ifx allfiles undefined
\end{CJK}
\end{document}
\fi

\newpage
\section{二叉树的建立}
\ifx allfiles undefined
\documentclass{article}
\usepackage{CJK}

%%%代码
\usepackage{color}
\usepackage{xcolor}
\definecolor{keywordcolor}{rgb}{0.8,0.1,0.5}
\usepackage{listings}
\lstset{breaklines}%这条命令可以让LaTeX自动将长的代码行换行排版
\lstset{extendedchars=false}%这一条命令可以解决代码跨页时,章节标题,页眉等汉字不显示的问题
\lstset{language=C++, %用于设置语言为C++
	keywordstyle=\color{keywordcolor} \bfseries, %设置关键词
	identifierstyle=,
	basicstyle=\ttfamily, 
	commentstyle=\color{blue} \textit,
	stringstyle=\ttfamily, 
	showstringspaces=false,
	%frame=shadowbox, %边框
	captionpos=b
}
%%%

%\hypersetup{CJKbookmarks=true} %解决section不能使用中文的问题

\begin{document}
\begin{CJK}{UTF8}{gbsn}     %CJK:支持中文

\else
	
%二叉树的建立

\fi

\ifx allfiles undefined
\end{CJK}
\end{document}
\fi

\newpage
\section{二叉树的属性}
\ifx allfiles undefined
\documentclass{article}
\usepackage{CJK}

%%%代码
\usepackage{color}
\usepackage{xcolor}
\definecolor{keywordcolor}{rgb}{0.8,0.1,0.5}
\usepackage{listings}
\lstset{breaklines}%这条命令可以让LaTeX自动将长的代码行换行排版
\lstset{extendedchars=false}%这一条命令可以解决代码跨页时,章节标题,页眉等汉字不显示的问题
\lstset{language=C++, %用于设置语言为C++
	keywordstyle=\color{keywordcolor} \bfseries, %设置关键词
	identifierstyle=,
	basicstyle=\ttfamily, 
	commentstyle=\color{blue} \textit,
	stringstyle=\ttfamily, 
	showstringspaces=false,
	%frame=shadowbox, %边框
	captionpos=b
}
%%%

%\hypersetup{CJKbookmarks=true} %解决section不能使用中文的问题

\begin{document}
\begin{CJK}{UTF8}{gbsn}     %CJK:支持中文

\else
	
%二叉树的属性
\subsection{Validate Binary Search Tree}
\ifx allfiles undefined
\documentclass{article}
\usepackage{CJK}
\usepackage{verbatim}

%%%代码
\usepackage{color}
\usepackage{xcolor}
\definecolor{keywordcolor}{rgb}{0.8,0.1,0.5}
\usepackage{listings}
\lstset{breaklines}%这条命令可以让LaTeX自动将长的代码行换行排版
\lstset{extendedchars=false}%这一条命令可以解决代码跨页时,章节标题,页眉等汉字不显示的问题
\lstset{language=C++, %用于设置语言为C++
	keywordstyle=\color{keywordcolor} \bfseries, %设置关键词
	identifierstyle=,
	basicstyle=\ttfamily, 
	commentstyle=\color{blue} \textit,
	stringstyle=\ttfamily, 
	showstringspaces=false,
	%frame=shadowbox, %边框
	captionpos=b
}
%%%

%\hypersetup{CJKbookmarks=true} %解决section不能使用中文的问题

\begin{document}
\begin{CJK}{UTF8}{gbsn}     %CJK:支持中文

\else
	
\begin{description}
	\item{\textbf{问题}}: Given a binary tree, determine if it is a valid binary search tree (BST). \textit{(leetcode 98)}
	\\判断一个二叉树是否是合法的BST,我们可以想到BST树的中序序列是非减序列,于是我们可以使用中序遍历这颗二叉树,在遍历的过程中查看是否有反常的数据.
	\\当然,根据上面说的三种中序遍历的方法,这里同样有三种解法.
	\item{\textbf{递归}} : \fbox{时间复杂度O(n) , 空间复杂度O($lgn$)}
	\begin{lstlisting}
bool dfs(TreeNode *root, int& up){
	if(!root)	return true;
	if(root->left){
		bool left =  dfs(root->left, up);
		if(!left) return false;
	}
	if(root->val <= up && (MIN || up != (-1)<<31)){ 
	    return false;
	}	
    if(root->val == (-1)<<31)
		MIN = true;
	up = root->val;
	if(root->right){
		bool right = dfs(root->right, up);
		if(!right)	return false;
	}
	return true;
}

bool isValidBST(TreeNode *root) {
	if(!root)	return true;
	int up = (-1)<<31;
	MIN = false;
	return dfs(root, up);
}
	\end{lstlisting}
	\textit{这里可以看到一些边界条件的判断,显得有点复杂,其实就是简单的中序遍历}
	\item{\textbf{栈式迭代}} : \fbox{时间复杂度O(n) , 空间复杂度O($lgn$)}
	\begin{lstlisting}
bool isValidBST(TreeNode *root) {
	if(!root)	return false;
	stack<TreeNode*> s;
	TreeNode *p = root;
	s.push(root);
	while(p->left){
		s.push(p->left);
		p = p->left;
	}
	int last = p->val;
	s.pop();
	if(p->right){
		s.push(p->right);
		p = p->right;
		while(p->left){
			s.push(p->left);
			p = p->left;
		}
	}
	while(!s.empty()){
		p = s.top();
		s.pop();
		if(last >= p->val) return false;
		last = p->val;
		if(p->right){
			s.push(p->right);
			p = p->right;
			while(p->left){
				s.push(p->left);
				p = p->left;
			}
		}
	}
	return true;
}
	\end{lstlisting}
	\item{\textbf{Mirror迭代}} : \fbox{时间复杂度O(n) , 空间复杂度O(1)}
	\\这里使用Mirror建立线索然后进行中序遍历,在中序遍历的同时进行判断
	\begin{lstlisting}
bool isValidBST(TreeNode *root) {
	if(!root)	return true;
	TreeNode *curr = root, *next;
	int last = INT_MIN;
	bool isFirst = true;
	bool ret = true;
	while(curr){
		if(!curr->left){
			if(!isFirst && curr->val <= last){
				ret = false;
			}
			if(isFirst)
				isFirst = false;
			last = curr->val;
			curr = curr->right;
			continue;
		}
		next = curr->left;
		while(next->right){
			if(next->right == curr)	break;
			next = next->right;
		}
		if(next->right == curr){
			next->right = NULL;
			if(!isFirst && curr->val <= last){
				ret = false;
			}
			if(isFirst)
				isFirst = false;
			last = curr->val;
			curr = curr->right;
		}else{
			next->right = curr;
			curr = curr->left;
		}
	}
	return ret;
}
	\end{lstlisting}
	\textit{有了Mirror算法,是不是你已经爱上它了,再也不用栈这么麻烦了,不过有一点需要注意的是一旦你使用Mirror算法,那么必须保证把整个树全遍历一遍,不能中途退出,因为那样树的结构被改变了}
\end{description}

\fi

\ifx allfiles undefined
\end{CJK}
\end{document}
\fi

\subsection{Symmetric Tree}
\ifx allfiles undefined
\documentclass{article}
\usepackage{CJK}
\usepackage{verbatim}

%%%代码
\usepackage{color}
\usepackage{xcolor}
\definecolor{keywordcolor}{rgb}{0.8,0.1,0.5}
\usepackage{listings}
\lstset{breaklines}%这条命令可以让LaTeX自动将长的代码行换行排版
\lstset{extendedchars=false}%这一条命令可以解决代码跨页时,章节标题,页眉等汉字不显示的问题
\lstset{language=C++, %用于设置语言为C++
	keywordstyle=\color{keywordcolor} \bfseries, %设置关键词
	identifierstyle=,
	basicstyle=\ttfamily, 
	commentstyle=\color{blue} \textit,
	stringstyle=\ttfamily, 
	showstringspaces=false,
	%frame=shadowbox, %边框
	captionpos=b
}
%%%

%\hypersetup{CJKbookmarks=true} %解决section不能使用中文的问题

\begin{document}
\begin{CJK}{UTF8}{gbsn}     %CJK:支持中文

\else
	
\begin{description}
	\item{\textbf{问题}}: Given a binary tree, check whether it is a mirror of itself (ie, symmetric around its center). \textit{(leetcode 101)}
	\\这是求证树是不是自身Mirror(成镜像).
	\item{\textbf{队列}} : \fbox{时间复杂度O(n) , 空间复杂度O(w), w为树的最大宽度}
	\begin{lstlisting}
bool isSymmetric(TreeNode* root){
	if(!root) return true;
	deque<TreeNode*> left(1, root->left), right(1, root->right);
	TreeNode *l, *r;
	while(!left.empty() && !right.empty()){
		l = left.front();
		r = right.front();
		left.pop_front();
		right.pop_front();
		if(!l && !r)	continue;
		if(!l || !r || l->val != r->val)	return false;
		left.push_back(l->left);
		left.push_back(l->right);
		right.push_back(r->right);
		right.push_back(r->left);
	}
	return true;
}
	\end{lstlisting}
	\item{\textbf{递归}} : \fbox{时间复杂度O(n) , 空间复杂度O($lgn$)}
	\\这里是把一棵树的对称问题看成两棵树的对称问题
	\begin{lstlisting}
bool recursion(TreeNode* root, TreeNode* symm){
	if(!root && !symm)
		return true;
	if(!root || !symm)	return false;
	if(root->val != symm->val)	return false;
	if(root == symm)	return recursion(root->left, symm->right);
	return recursion(root->left, symm->right) && recursion(root->right, symm->left);
}

bool isSymmetric(TreeNode* root){
	if(!root) return true;
	return recursion(root, root);
}
	\end{lstlisting}
	\textit{\\这里还可以延伸出一个问题: 求一个二叉树的镜像树}
\end{description}

\fi

\ifx allfiles undefined
\end{CJK}
\end{document}
\fi

\subsection{Maximum Depth of Binary Tree}
\ifx allfiles undefined
\documentclass{article}
\usepackage{CJK}
\usepackage{verbatim}

%%%代码
\usepackage{color}
\usepackage{xcolor}
\definecolor{keywordcolor}{rgb}{0.8,0.1,0.5}
\usepackage{listings}
\lstset{breaklines}%这条命令可以让LaTeX自动将长的代码行换行排版
\lstset{extendedchars=false}%这一条命令可以解决代码跨页时,章节标题,页眉等汉字不显示的问题
\lstset{language=C++, %用于设置语言为C++
	keywordstyle=\color{keywordcolor} \bfseries, %设置关键词
	identifierstyle=,
	basicstyle=\ttfamily, 
	commentstyle=\color{blue} \textit,
	stringstyle=\ttfamily, 
	showstringspaces=false,
	%frame=shadowbox, %边框
	captionpos=b
}
%%%

%\hypersetup{CJKbookmarks=true} %解决section不能使用中文的问题

\begin{document}
\begin{CJK}{UTF8}{gbsn}     %CJK:支持中文

\else
	
\begin{description}
	\item{\textbf{问题}}: Given a binary tree, find its maximum depth.\textit{(leetcode 104)}
	\\从根节点来看,它的深度就是左右子树深度较大的那个+1,所以很自然的想到递归
	\item{\textbf{递归}} : \fbox{时间复杂度O(n) , 空间复杂度O($lgn$)}
	\\递归代码十分简洁
	\begin{lstlisting}
int maxDepth(TreeNode *root){
	if(!root)   return 0;
	int left = maxDepth(root->left);
	int right = maxDepth(root->right);
	return left < right? right + 1 : left + 1;
}
	\end{lstlisting}
	\qquad除了递归,其实这道题能不能用迭代的做法呢?答案是肯定的,最初你可能会想到用两个栈,一个栈存放节点,一个栈存放深度,其实可以把这个两者打包成一个pair,使用一个栈就可以啦
	\item{\textbf{迭代}} : \fbox{时间复杂度O(n) , 空间复杂度O($lgn$)}
	\begin{lstlisting}
int maxDepth(TreeNode *root) {
	if(!root)	return 0;
	stack<pair<TreeNode*, int> > s;
	s.push(make_pair(root, 1));
	pair<TreeNode*, int> curr;
	int result = INT_MIN;
	while(!s.empty()){
		curr = s.top();
		s.pop();
		if(!curr.first->left && !curr.first->right){
			if(result < curr.second)
				result = curr.second;
			continue;
		}
		if(curr.first->left){
			s.push(make_pair(curr.first->left, curr.second + 1));
		}
		if(curr.first->right){
			s.push(make_pair(curr.first->right, curr.second + 1));
		}
	}
	return result;
}
	\end{lstlisting}
\end{description}

\fi

\ifx allfiles undefined
\end{CJK}
\end{document}
\fi

\subsection{Minimum Depth of Binary Tree}
\ifx allfiles undefined
\documentclass{article}
\usepackage{CJK}
\usepackage{verbatim}

%%%代码
\usepackage{color}
\usepackage{xcolor}
\definecolor{keywordcolor}{rgb}{0.8,0.1,0.5}
\usepackage{listings}
\lstset{breaklines}%这条命令可以让LaTeX自动将长的代码行换行排版
\lstset{extendedchars=false}%这一条命令可以解决代码跨页时,章节标题,页眉等汉字不显示的问题
\lstset{language=C++, %用于设置语言为C++
	keywordstyle=\color{keywordcolor} \bfseries, %设置关键词
	identifierstyle=,
	basicstyle=\ttfamily, 
	commentstyle=\color{blue} \textit,
	stringstyle=\ttfamily, 
	showstringspaces=false,
	%frame=shadowbox, %边框
	captionpos=b
}
%%%

%\hypersetup{CJKbookmarks=true} %解决section不能使用中文的问题

\begin{document}
\begin{CJK}{UTF8}{gbsn}     %CJK:支持中文

\else
	
\begin{description}
	\item{\textbf{问题}}: Given a binary tree, find its minimum depth. The minimum depth is the number of nodes along the shortest path from the root node down to the nearest leaf node. \textit{(leetcode 111)}
	\item{\textbf{递归}} : \fbox{时间复杂度O(n), 空间复杂度O($lgn$)}
	\\自下而上的递归,非常的简单
	\begin{lstlisting}
int minDepth(TreeNode *root) {
	if(!root)	return 0;
	if(!root->left && !root->right)	return 1;
	if(!root->left)
		return minDepth(root->right) + 1;
	if(!root->right)
		return minDepth(root->left) + 1;
	return min(minDepth(root->left), minDepth(root->right)) + 1;
}
	\end{lstlisting}
	\item{\textbf{DFS}} : \fbox{时间复杂度O(n) , 空间复杂度O($lgn$)}
	\\这也是递归,但是是一种自上而下的递归,可以进行剪枝而不必把整个树都访问一遍
	\begin{lstlisting}
void dfs(TreeNode *root, int &result, int depth){
	if(result < depth + 1) return;
	if(!root->left && !root->right){
		result = depth + 1;
		return;
	}
	if(root->left)
		dfs(root->left, result, depth + 1);
	if(root->right)
		dfs(root->right, result, depth + 1);
}

	int minDepth(TreeNode *root) {
	if(!root)	return 0;
	int result = INT_MAX;
	dfs(root, result, 0);
	return result;
}
	\end{lstlisting}
	\item{\textbf{迭代}} : \fbox{时间复杂度O(n) , 空间复杂度O($lgn$)}
	\\同样我们也可以剪枝
	\begin{lstlisting}
int minDepth(TreeNode *root) {
	if(!root)	return 0;
	stack<pair<TreeNode*, int> > s;
	s.push(make_pair(root, 1));
	int	 result = -((1<<31) + 1);
	TreeNode *node;
	int depth;
	while(!s.empty()){
		node = s.top().first;
		depth = s.top().second;
		s.pop();
		if(result < depth)	continue;
		if(!node->left && !node->right)
			result = depth;
	
		if(node->left )
			s.push(make_pair(node->left, depth + 1));
		if(node->right)
			s.push(make_pair(node->right, depth + 1));
	}
	return result;
}
	\end{lstlisting}
\end{description}

\fi

\ifx allfiles undefined
\end{CJK}
\end{document}
\fi

\subsection{Balanced Binary Tree}
\ifx allfiles undefined
\documentclass{article}
\usepackage{CJK}
\usepackage{verbatim}

%%%代码
\usepackage{color}
\usepackage{xcolor}
\definecolor{keywordcolor}{rgb}{0.8,0.1,0.5}
\usepackage{listings}
\lstset{breaklines}%这条命令可以让LaTeX自动将长的代码行换行排版
\lstset{extendedchars=false}%这一条命令可以解决代码跨页时,章节标题,页眉等汉字不显示的问题
\lstset{language=C++, %用于设置语言为C++
	keywordstyle=\color{keywordcolor} \bfseries, %设置关键词
	identifierstyle=,
	basicstyle=\ttfamily, 
	commentstyle=\color{blue} \textit,
	stringstyle=\ttfamily, 
	showstringspaces=false,
	%frame=shadowbox, %边框
	captionpos=b
}
%%%

%\hypersetup{CJKbookmarks=true} %解决section不能使用中文的问题

\begin{document}
\begin{CJK}{UTF8}{gbsn}     %CJK:支持中文

\else
	
\begin{description}
	\item{\textbf{问题}}: Given a binary tree, determine if it is height-balanced. \textit{(leetcode 110)}
	\item{\textbf{递归}} : \fbox{时间复杂度O(n) , 空间复杂度O($lgn$)}
	\\先判断左子树是否高度平衡并返回左子树高度,再判断右子树是否高度平衡,再返回右子树高度,根据左右子树高度再判断当前树是否平衡.
	\begin{lstlisting}
bool dfs(TreeNode *root, int &hight){
	if(!root){
		hight = 0;
		return true;
	}
	int left, right;
	bool is_left = dfs(root->left, left);
	bool is_right = dfs(root->right, right);
	hight = left > right? left + 1 : right + 1;
	return is_left && is_right && (abs(left - right) < 2);
}

bool isBalanced(TreeNode *root) {
	int hight;
	return dfs(root, hight);
}
	\end{lstlisting}
\end{description}

\fi

\ifx allfiles undefined
\end{CJK}
\end{document}
\fi


\fi

\ifx allfiles undefined
\end{CJK}
\end{document}
\fi

\newpage
\section{其他}
\ifx allfiles undefined
\documentclass{article}
\usepackage{CJK}
\usepackage{verbatim}

%%%代码
\usepackage{color}
\usepackage{xcolor}
\definecolor{keywordcolor}{rgb}{0.8,0.1,0.5}
\usepackage{listings}
\lstset{breaklines}%这条命令可以让LaTeX自动将长的代码行换行排版
\lstset{extendedchars=false}%这一条命令可以解决代码跨页时,章节标题,页眉等汉字不显示的问题
\lstset{language=C++, %用于设置语言为C++
    keywordstyle=\color{keywordcolor} \bfseries, %设置关键词
    identifierstyle=,
    basicstyle=\ttfamily, 
    commentstyle=\color{blue} \textit,
    stringstyle=\ttfamily, 
    showstringspaces=false,
    %frame=shadowbox, %边框
    captionpos=b
}
%%%

%\hypersetup{CJKbookmarks=true} %解决section不能使用中文的问题

\begin{document}
\begin{CJK}{UTF8}{gbsn}     %CJK:支持中文

\else

\subsection{Flatten Binary Tree to Linked List}    
\ifx allfiles undefined
\documentclass{article}
\usepackage{CJK}
\usepackage{verbatim}

%%%代码
\usepackage{color}
\usepackage{xcolor}
\definecolor{keywordcolor}{rgb}{0.8,0.1,0.5}
\usepackage{listings}
\lstset{breaklines}%这条命令可以让LaTeX自动将长的代码行换行排版
\lstset{extendedchars=false}%这一条命令可以解决代码跨页时,章节标题,页眉等汉字不显示的问题
\lstset{language=C++, %用于设置语言为C++
    keywordstyle=\color{keywordcolor} \bfseries, %设置关键词
    identifierstyle=,
    basicstyle=\ttfamily, 
    commentstyle=\color{blue} \textit,
    stringstyle=\ttfamily, 
    showstringspaces=false,
    %frame=shadowbox, %边框
    captionpos=b
}
%%%

%\hypersetup{CJKbookmarks=true} %解决section不能使用中文的问题

\begin{document}
\begin{CJK}{UTF8}{gbsn}     %CJK:支持中文

\else
    
\begin{description}
    \item{\textbf{问题}}: Given a binary tree, flatten it to a linked list in-place by the pre-order. \textit{(leetcode 114)}
    \\这是一个基于先序遍历的问题,所以可以使用递归和迭代的方法.
    \item{\textbf{递归}} : \fbox{时间复杂度O(n) , 空间复杂度O($lgn$)}
    \begin{lstlisting}
void flatten(TreeNode *root) {
    TreeNode *tail;
    recursion(root, tail);
}

TreeNode* recursion(TreeNode *root, TreeNode* &tail){
    if(!root)    return NULL;
    TreeNode *next = NULL;
    tail = root;
    if(root->left)
        next = recursion(root->left, tail);
    if(root->right)
        tail->right = recursion(root->right, tail);
    root->left = NULL;
    if(next)
        root->right = next;
    return root;
}
    \end{lstlisting}
    \item{\textbf{迭代}} : \fbox{时间复杂度O(n) , 空间复杂度O($lgn$)}
    \\这里就是完完全全的迭代版前序遍历,这里使用了栈,同样你也可以使用Mirror算法.
    \begin{lstlisting}
void flatten(TreeNode *root) {
    if(!root)    return;
    stack<TreeNode*> s;
    s.push(root);
    TreeNode *last = NULL, *cur;
    while(!s.empty()){
        cur = s.top();
        s.pop();
        if(last)
            last->right = cur;
        if(cur->right)
            s.push(cur->right);
        if(cur->left)
            s.push(cur->left);
        cur->left = NULL;
        last = cur;
    }
    last->right = NULL;
}
    \end{lstlisting}
\end{description}

\fi

\ifx allfiles undefined
\end{CJK}
\end{document}
\fi

\subsection{Populating Next Right Pointers in Each Node}    
\ifx allfiles undefined
\documentclass{article}
\usepackage{CJK}
\usepackage{verbatim}

%%%代码
\usepackage{color}
\usepackage{xcolor}
\definecolor{keywordcolor}{rgb}{0.8,0.1,0.5}
\usepackage{listings}
\lstset{breaklines}%这条命令可以让LaTeX自动将长的代码行换行排版
\lstset{extendedchars=false}%这一条命令可以解决代码跨页时,章节标题,页眉等汉字不显示的问题
\lstset{language=C++, %用于设置语言为C++
	keywordstyle=\color{keywordcolor} \bfseries, %设置关键词
	identifierstyle=,
	basicstyle=\ttfamily, 
	commentstyle=\color{blue} \textit,
	stringstyle=\ttfamily, 
	showstringspaces=false,
	%frame=shadowbox, %边框
	captionpos=b
}
%%%

%\hypersetup{CJKbookmarks=true} %解决section不能使用中文的问题

\begin{document}
\begin{CJK}{UTF8}{gbsn}     %CJK:支持中文

\else
	
\begin{description}
	\item{\textbf{问题}}: Given a binary tree:
	\begin{lstlisting}
struct TreeLinkNode {
    TreeLinkNode *left;
    TreeLinkNode *right;
	TreeLinkNode *next;
}
	\end{lstlisting}
	Populate each next pointer to point to its next right node. If there is no next right node, the next pointer should be set to NULL. 
	\\\textbf{Note}:
	\\You may only use constant extra space.
	\\You may assume that it is a perfect binary tree (ie, all leaves are at the same level, and every parent has two children).
	\textit{(leetcode 116)}
	\\其实就是一个很简单的BFS过程.
	\item{\textbf{迭代}} : \fbox{时间复杂度O(n), 空间复杂度O(1)}
	\\因为是满二叉树,所以每次在上一层建立这一层的next,然后再到这一层来,这样就不需要队列,使用常数的空间复杂度.
	\begin{lstlisting}
void connect(TreeLinkNode *root) {
	if(!root)	return;
	TreeLinkNode *cur = root, *next;
	cur->next = NULL;
	while(cur->left){
		next = cur->left;
		while(cur){
			cur->left->next = cur->right;
			cur->right->next = cur->next? cur->next->left : NULL;
			cur = cur->next;
		}
		cur = next;
	}
}
	\end{lstlisting}
\end{description}

\fi

\ifx allfiles undefined
\end{CJK}
\end{document}
\fi

\subsection{Populating Next Right Pointers in Each Node II}    
\ifx allfiles undefined
\documentclass{article}
\usepackage{CJK}
\usepackage{verbatim}

%%%代码
\usepackage{color}
\usepackage{xcolor}
\definecolor{keywordcolor}{rgb}{0.8,0.1,0.5}
\usepackage{listings}
\lstset{breaklines}%这条命令可以让LaTeX自动将长的代码行换行排版
\lstset{extendedchars=false}%这一条命令可以解决代码跨页时,章节标题,页眉等汉字不显示的问题
\lstset{language=C++, %用于设置语言为C++
	keywordstyle=\color{keywordcolor} \bfseries, %设置关键词
	identifierstyle=,
	basicstyle=\ttfamily, 
	commentstyle=\color{blue} \textit,
	stringstyle=\ttfamily, 
	showstringspaces=false,
	%frame=shadowbox, %边框
	captionpos=b
}
%%%

%\hypersetup{CJKbookmarks=true} %解决section不能使用中文的问题

\begin{document}
\begin{CJK}{UTF8}{gbsn}     %CJK:支持中文

\else
	
\begin{description}
	\item{\textbf{问题}}: Follow up for problem "Populating Next Right Pointers in Each Node".What if the given tree could be any binary tree? Would your previous solution still work?
	\textbf{Note}: You may only use constant extra space. \textit{(leetcode 117)}
	\item{\textbf{迭代}} : \fbox{时间复杂度O(n) , 空间复杂度O(1)}
	\\这里其实和上一题一样,只不过多了一些判断条件.
	\begin{lstlisting}
void connect(TreeLinkNode *root) {
	if(!root)	return;
	TreeLinkNode *cur = root, *next, *last;
	cur->next = NULL;
	do{
		next = NULL;
		while(cur){
			if(cur->left){
				if(next){
					last->next = cur->left;
					last = last->next;
				}
				else{
					last = cur->left;
					next = last;
				}
			}
			if(cur->right){
				if(next){
					last->next = cur->right;
					last = last->next;
				}
				else{
					last = cur->left;
					next = last;
				}
			}
			cur = cur->next;
		}
		last->next = NULL;
		cur = next;
	}while(cur);
}
	\end{lstlisting}
	\textit{}
\end{description}

\fi

\ifx allfiles undefined
\end{CJK}
\end{document}
\fi

\subsection{Binary Search Tree Iterator}    
\ifx allfiles undefined
\documentclass{article}
\usepackage{CJK}
\usepackage{verbatim}

%%%代码
\usepackage{color}
\usepackage{xcolor}
\definecolor{keywordcolor}{rgb}{0.8,0.1,0.5}
\usepackage{listings}
\lstset{breaklines}%这条命令可以让LaTeX自动将长的代码行换行排版
\lstset{extendedchars=false}%这一条命令可以解决代码跨页时,章节标题,页眉等汉字不显示的问题
\lstset{language=C++, %用于设置语言为C++
    keywordstyle=\color{keywordcolor} \bfseries, %设置关键词
    identifierstyle=,
    basicstyle=\ttfamily, 
    commentstyle=\color{blue} \textit,
    stringstyle=\ttfamily, 
    showstringspaces=false,
    %frame=shadowbox, %边框
    captionpos=b
}
%%%

%\hypersetup{CJKbookmarks=true} %解决section不能使用中文的问题

\begin{document}
\begin{CJK}{UTF8}{gbsn}     %CJK:支持中文

\else
    
\begin{description}
    \item{\textbf{问题}}: \\
Implement an iterator over a binary search tree (BST). Your iterator will be initialized with the root node of a BST. \\
\\
Calling next() will return the next smallest number in the BST. \\
\textit{(leetcode 173)}
    \item{\textbf{Note}}: \\
next() and hasNext() should run in average O(1) time and uses O(h) memory, where h is the height of the tree.
    \item{\textbf{Stack}} : \fbox{时间复杂度O(1), 空间复杂度O(lgn)}
    \\这里使用一个栈来保存上级节点,每个next和hasNext的操作平均时间复杂度是O(1)
    \begin{lstlisting}
class BSTIterator {
private:
	TreeNode* root;
	stack<TreeNode*> path;
public:
    BSTIterator(TreeNode *root) : root(root){
		TreeNode* iter = root;
		while(iter){
			path.push(iter);
			iter = iter->left;
		}
    }

    /** @return whether we have a next smallest number */
    bool hasNext() {
		return !path.empty();
    }

    /** @return the next smallest number */
    int next() {
        TreeNode *cur = path.top();
		TreeNode *iter = cur->right;
		path.pop();
		while(iter){
			path.push(iter);
			iter = iter->left;
		}
		return cur->val;
    }
};
    \end{lstlisting}
\end{description}

\fi

\ifx allfiles undefined
\end{CJK}
\end{document}
\fi


\fi

\ifx allfiles undefined
\end{CJK}
\end{document}
\fi

\newpage

\fi

\ifx allfiles undefined
\end{CJK}
\end{document}
\fi


\end{CJK}
\end{document}
