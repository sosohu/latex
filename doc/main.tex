\documentclass[oneside]{book}
\usepackage{CJK}
\usepackage{graphics,graphicx}
\usepackage{pstricks,pst-node,pst-tree}
\usepackage{verbatim}
\usepackage{amsmath}
\usepackage{titlesec}
\usepackage{fancyhdr} %页眉页脚布局
\usepackage{tocloft} % 目录相关


%%%代码
\usepackage{color}
\usepackage{xcolor}
\definecolor{keywordcolor}{rgb}{0.8,0.1,0.5}
\usepackage{listings}
\lstset{extendedchars=false}%这一条命令可以解决代码跨页时,章节标题,页眉等汉字不显示的问题
\definecolor{darkgreen}{rgb}{0.0, 0.2, 0.13}
\definecolor{grey}{rgb}{0.55, 0.57, 0.67}
\definecolor{darkgray}{rgb}{0.66, 0.66, 0.66}
\lstset{language=C++, %用于设置语言为C++
    numbers=left,
    numberstyle=\ttfamily\scriptsize,
    backgroundcolor=\color{darkgray}, frame=single,framesep=5pt,framexleftmargin=8mm,%frameround=fttt,
    basicstyle=\ttfamily\small,
    keywordstyle=\ttfamily\bf\color{blue},
    ndkeywordstyle=\ttfamily\bf\color{brown},
    commentstyle=\color{darkgreen},
    identifierstyle=\ttfamily\color{black}\bfseries,
    stringstyle=\color{pink}\ttfamily,showstringspaces=false,
    breaklines=true,
	tabsize=4,
    escapeinside=``
}

%%%

%\hypersetup{CJKbookmarks=true} %解决section不能使用中文的问题

\usepackage[colorlinks,linkcolor=black,anchorcolor=blue,citecolor=green,CJKbookmarks=true]{hyperref}
\begin{document}
\begin{CJK}{UTF8}{gbsn}     %CJK:支持中文

%%文章中章节等转化为中文
\renewcommand{\contentsname}{目录}
\renewcommand{\figurename}{图}
\renewcommand{\tablename}{表}
\titleformat{\chapter}{\centering\Huge\bfseries}{第\,\thechapter\,章}{1em}{}
%%

%%目录中章节
\renewcommand{\cftchapfont}{\bfseries}
\renewcommand{\cftchappagefont}{\bfseries}
\renewcommand{\cftchappresnum}{第}
\renewcommand{\cftchapaftersnum}{章:}
\renewcommand{\cftchapnumwidth}{4em}      %  add 'chapter' word before number
%%

%%布局
\pagestyle{fancy}
\renewcommand{\chaptermark}[1]{\markboth{\small 第\,\thechapter\,章\quad #1}{}}
\renewcommand{\sectionmark}[1]{\markright{\small\thesection\quad #1}{}}
\fancyhf{}
\fancyhead[ER]{\leftmark}
\fancyhead[OL]{\rightmark}
\fancyhead[EL,OR]{$\cdot$\ \thepage\ $\cdot$}
\renewcommand{\headrulewidth}{0.4pt}
%%

\title{我的解题报告}
\author{胡庆海}
\date{}

\maketitle

%\newpage
\tableofcontents

\chapter{链表}
\input{chap1/chap1.tex}

\chapter{树}
\input{chap2/chap2.tex}

\chapter{字符串}
\input{chap3/chap3.tex}

\end{CJK}
\end{document}
