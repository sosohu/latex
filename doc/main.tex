\documentclass[oneside]{book}
%\usepackage{CJK}
\usepackage{CJKutf8}
\usepackage{graphics,graphicx}
\usepackage{pstricks,pst-node,pst-tree}
\usepackage{verbatim}
\usepackage{amsmath}
\usepackage{titlesec}
\usepackage{fancyhdr} %页眉页脚布局
\usepackage{tocloft} % 目录相关


%%%代码
\usepackage{color}
\usepackage{xcolor}
\definecolor{keywordcolor}{rgb}{0.8,0.1,0.5}
\usepackage{listings}
\lstset{extendedchars=false}%这一条命令可以解决代码跨页时,章节标题,页眉等汉字不显示的问题
\definecolor{darkgreen}{rgb}{0.0, 0.2, 0.13}
\definecolor{grey}{rgb}{0.55, 0.57, 0.67}
\definecolor{darkgray}{rgb}{0.66, 0.66, 0.66}
\lstset{language=C++, %用于设置语言为C++
    numbers=left,
    numberstyle=\ttfamily\scriptsize,
    backgroundcolor=\color{darkgray}, frame=single,framesep=3pt,framexleftmargin=8mm,%frameround=fttt,
    basicstyle=\ttfamily\small,
    keywordstyle=\ttfamily\bf\color{blue},
    ndkeywordstyle=\ttfamily\bf\color{brown},
    commentstyle=\color{darkgreen},
    identifierstyle=\ttfamily\color{black}\bfseries,
    stringstyle=\color{pink}\ttfamily,showstringspaces=false,
    breaklines=true,
	tabsize=4,
    escapeinside=``
}

%%%

%\hypersetup{CJKbookmarks=true} %解决section不能使用中文的问题

\usepackage[top=1in, bottom=1in, left=1.25in, right=1.25in]{geometry} %页边距

%最好保证 hyperref 是最后加载的宏包
\usepackage[pdftex,bookmarksnumbered,colorlinks,linkcolor=black,anchorcolor=blue,citecolor=green,unicode=true]{hyperref}

\begin{document}
\begin{CJK}{UTF8}{gbsn}     %CJK:支持中文

%%文章中章节等转化为中文
\renewcommand{\contentsname}{目录}
\renewcommand{\figurename}{图}
\renewcommand{\tablename}{表}
\titleformat{\chapter}{\centering\Huge\bfseries}{第\,\thechapter\,章}{1em}{}
%%

%%目录中章节
\renewcommand{\cftchapfont}{\bfseries}
\renewcommand{\cftchappagefont}{\bfseries}
\renewcommand{\cftchappresnum}{第}
\renewcommand{\cftchapaftersnum}{章:}
\renewcommand{\cftchapnumwidth}{4em}      %  add 'chapter' word before number
%%

%%布局
\pagestyle{fancy}
\renewcommand{\chaptermark}[1]{\markboth{\small 第\,\thechapter\,章\quad #1}{}}
\renewcommand{\sectionmark}[1]{\markright{\small\thesection\quad #1}{}}
\fancyhf{}
\fancyhead[ER]{\leftmark}
\fancyhead[OL]{\rightmark}
\fancyhead[EL,OR]{$\cdot$\ \thepage\ $\cdot$}
\renewcommand{\headrulewidth}{0.4pt}
%%

\title{我的解题报告}
\author{胡庆海}
\date{}

\maketitle

%\newpage
\tableofcontents

\chapter{链表}
\ifx allfiles undefined
\documentclass{article}
\usepackage{CJK}

%%%代码
\usepackage{color}
\usepackage{xcolor}
\definecolor{keywordcolor}{rgb}{0.8,0.1,0.5}
\usepackage{listings}
\lstset{breaklines}%这条命令可以让LaTeX自动将长的代码行换行排版
\lstset{extendedchars=false}%这一条命令可以解决代码跨页时,章节标题,页眉等汉字不显示的问题
\lstset{language=C++, %用于设置语言为C++
    keywordstyle=\color{keywordcolor} \bfseries, %设置关键词
    identifierstyle=,
    basicstyle=\ttfamily, 
    commentstyle=\color{blue} \textit,
    stringstyle=\ttfamily, 
    showstringspaces=false,
    %frame=shadowbox, %边框
    captionpos=b
}
%%%

%\hypersetup{CJKbookmarks=true} %解决section不能使用中文的问题

\begin{document}
\begin{CJK}{UTF8}{gbsn}     %CJK:支持中文

\else

\begin{lstlisting}
int main(int argc, char *argv[])
{
    int i;
    return 0;
}
\end{lstlisting}

\fi

\ifx allfiles undefined
\end{CJK}
\end{document}
\fi


\chapter{树}
\ifx allfiles undefined
\documentclass{article}
\usepackage{CJK}

%%%代码
\usepackage{color}
\usepackage{xcolor}
\definecolor{keywordcolor}{rgb}{0.8,0.1,0.5}
\usepackage{listings}
\lstset{breaklines}%这条命令可以让LaTeX自动将长的代码行换行排版
\lstset{extendedchars=false}%这一条命令可以解决代码跨页时,章节标题,页眉等汉字不显示的问题
\lstset{language=C++, %用于设置语言为C++
	keywordstyle=\color{keywordcolor} \bfseries, %设置关键词
	identifierstyle=,
	basicstyle=\ttfamily, 
	commentstyle=\color{blue} \textit,
	stringstyle=\ttfamily, 
	showstringspaces=false,
	%frame=shadowbox, %边框
	captionpos=b
}
%%%

%\hypersetup{CJKbookmarks=true} %解决section不能使用中文的问题

\begin{document}
\begin{CJK}{UTF8}{gbsn}     %CJK:支持中文

\else
	
	%树的基本问题分类
\section{二叉树的遍历}
\ifx allfiles undefined
\documentclass{article}
\usepackage{CJK}

%%%代码
\usepackage{color}
\usepackage{xcolor}
\definecolor{keywordcolor}{rgb}{0.8,0.1,0.5}
\usepackage{listings}
\lstset{breaklines}%这条命令可以让LaTeX自动将长的代码行换行排版
\lstset{extendedchars=false}%这一条命令可以解决代码跨页时,章节标题,页眉等汉字不显示的问题
\lstset{language=C++, %用于设置语言为C++
	keywordstyle=\color{keywordcolor} \bfseries, %设置关键词
	identifierstyle=,
	basicstyle=\ttfamily, 
	commentstyle=\color{blue} \textit,
	stringstyle=\ttfamily, 
	showstringspaces=false,
	%frame=shadowbox, %边框
	captionpos=b
}
%%%

%\hypersetup{CJKbookmarks=true} %解决section不能使用中文的问题

\begin{document}
\begin{CJK}{UTF8}{gbsn}     %CJK:支持中文

\else
	
	%二叉树的遍历
\subsection{PreOrederTraversal}
\ifx allfiles undefined
\documentclass{article}
\usepackage{CJK}
\usepackage{verbatim}

%%%代码
\usepackage{color}
\usepackage{xcolor}
\definecolor{keywordcolor}{rgb}{0.8,0.1,0.5}
\usepackage{listings}
\lstset{breaklines}%这条命令可以让LaTeX自动将长的代码行换行排版
\lstset{extendedchars=false}%这一条命令可以解决代码跨页时,章节标题,页眉等汉字不显示的问题
\lstset{language=C++, %用于设置语言为C++
    keywordstyle=\color{keywordcolor} \bfseries, %设置关键词
    identifierstyle=,
    basicstyle=\ttfamily, 
    commentstyle=\color{blue} \textit,
    stringstyle=\ttfamily, 
    showstringspaces=false,
    %frame=shadowbox, %边框
    captionpos=b
}
%%%

%\hypersetup{CJKbookmarks=true} %解决section不能使用中文的问题

\begin{document}
\begin{CJK}{UTF8}{gbsn}     %CJK:支持中文

\else
    
%二叉树的先序遍历
\begin{description}
    \item{\textbf{问题}}: Given a binary tree, return the preorder traversal of its nodes' values.(\textit{leetcode 144})
    \item{\textbf{递归}} : \fbox{时间复杂度O(n) , 空间复杂度O($lgn$)}
    \\前序遍历的递归写法,非常简单,只需要先访问跟节点,再递归的执行左子树和右子树
    \begin{lstlisting}
void BiTree::PreOrderTraversal(TreeNode* root){
    if(!root)    return;
    cout<<root->val<<endl;
    if(root->left)
        PreOrderTraversal(root->left);
    if(root->right)
        PreOrderTraversal(root->right);
}
    \end{lstlisting}
    \item{\textbf{栈式迭代}} : \fbox{时间复杂度O(n) , 空间复杂度O($lgn$)}
    \\使用栈来模拟递归过程也是很显而易见的,具体做法就是先访问根节点,然后先让右孩子入栈接着是左孩子,然后左孩子出栈后重复这个过程
    \begin{lstlisting}
void BiTree::PreOrderTraversal(TreeNode* root){
    if(!root)    return;
    stack<TreeNode*> s;
    TreeNode *cur;
    s.push(root);
    while(!s.empty()){
        cur = s.top();
        s.pop();
        cout<<cur->val<<endl;
        if(cur->right)
            s.push(cur->right);
        if(cur->left)
            s.push(cur->left);
    }
}
    \end{lstlisting}
    \item{\textbf{Mirror迭代}} : \fbox{时间复杂度O(n) , 空间复杂度O(1)}
    \\Mirror迭代法是经过Lee介绍过来的,非常的迷人,它的做法就是在遍历的过程中,访问了当前节点之后,先找当前节点的前驱并让此前驱的右孩子指向它,再访问它的左孩子并重复这个过程。在此之后会访问到它前驱然后再次回到当前节点,此时再次试图建立前驱,发现已经建立了,这就说明当前节点左边已经全部遍历完,则继续访问当前节点的右边节点,不断的重复此过程。
    \begin{lstlisting}
void BiTree::PreOrderTraversal(TreeNode* root){
    if(!root)    return;
    TreeNode *curr = root, *next;
    while(curr){
        next = curr->left;
        if(!next){
            cout<<curr->val<<endl;
            curr = curr->right;
            continue;
        }
        while(next->right && next->right != curr){
            next = next->right;
        }
        if(next->right == curr){
            next->right = NULL;
            curr = curr->right;
        }else{
            cout<<curr->val<<endl;
            next->right = curr;
            curr = curr->left;
        }
    }
}
    \end{lstlisting}
    \textit{这个Mirror算法一旦掌握后,威力无穷,你可以用它方便的建立二叉树前序索引并且遇到那些要求用迭代来实现的二叉树问题也可以很快的写出来}
\end{description}

\fi

\ifx allfiles undefined
\end{CJK}
\end{document}
\fi

\subsection{InOrederTraversal}
\ifx allfiles undefined
\documentclass{article}
\usepackage{CJK}
\usepackage{verbatim}

%%%代码
\usepackage{color}
\usepackage{xcolor}
\definecolor{keywordcolor}{rgb}{0.8,0.1,0.5}
\usepackage{listings}
\lstset{breaklines}%这条命令可以让LaTeX自动将长的代码行换行排版
\lstset{extendedchars=false}%这一条命令可以解决代码跨页时,章节标题,页眉等汉字不显示的问题
\lstset{language=C++, %用于设置语言为C++
	keywordstyle=\color{keywordcolor} \bfseries, %设置关键词
	identifierstyle=,
	basicstyle=\ttfamily, 
	commentstyle=\color{blue} \textit,
	stringstyle=\ttfamily, 
	showstringspaces=false,
	%frame=shadowbox, %边框
	captionpos=b
}
%%%

%\hypersetup{CJKbookmarks=true} %解决section不能使用中文的问题

\begin{document}
\begin{CJK}{UTF8}{gbsn}     %CJK:支持中文

\else
	
%XXX 问题
\begin{description}
	\item{\textbf{问题}}: Given a binary tree, return the inorder traversal of its nodes' values. \textit{(leetcode 94)}
	\item{\textbf{递归}} : \fbox{时间复杂度O(n) , 空间复杂度O($lgn$)}
	\\
	\begin{lstlisting}
void BiTree::InOrderTraversal(TreeNode* root){
	if(!root)	return;
	if(root->left)
		InOrderTraversal(root->left);
	cout<<root->val<<endl;
	if(root->right)
		InOrderTraversal(root->right);
	\end{lstlisting}
	\item{\textbf{栈式迭代}} : \fbox{时间复杂度O(n) , 空间复杂度O($lgn$)}
	\\这里需要说一下的是,数据结构那本书上写了两种栈式迭代方法,这是其中之一,使用两重循环的那个
	\begin{lstlisting}
void BiTree::InOrderTraversal(TreeNode* root){
	vector<int> data;
	if(!root)	return data;
	stack<TreeNode*> s;
	TreeNode *pos = root;
	while(!s.empty() || pos){
		while(pos){
			s.push(pos);
			pos = pos->left;
		}
		pos = s.top();
		s.pop();
		std::cout<<pos->val<<std::endl;
		pos = pos->right;  //这个非常重要
	}
}
	\end{lstlisting}
	\item{\textbf{栈式迭代}} : \fbox{时间复杂度O(n) , 空间复杂度O($lgn$)}
	\\这里需要说一下的是,数据结构那本书上写了两种栈式迭代方法,这是其中之二,使用一重循环,实际上是一样的
	\begin{lstlisting}
void BiTree::InOrderTraversal(TreeNode* root){
	vector<int> data;
	if(!root)	return data;
	stack<TreeNode*> s;
	TreeNode *pos = root;
	while(!s.empty() || pos){
		if(pos){
			s.push(pos);
			pos = pos->left;
		}else{
			pos = s.top();
			s.pop();
			std::cout<<pos->val<<std::endl;
			pos = pos->right;  //这个非常重要
		}
	}
}
	\end{lstlisting}
	\item{\textbf{Mirror迭代}} : \fbox{时间复杂度O(n) , 空间复杂度O(1)}
	\\这里Mirror方法和前序的Mirror方法基本一样,唯一的区别就是打印当前值的时机
	\begin{lstlisting}
void BiTree::InOrderTraversal(TreeNode* root){
	if(!root)	return;
	TreeNode *curr = root, *next;
	while(curr){
		next = curr->left;
		if(!next){
			cout<<curr->val<<endl;
			curr = curr->right;
			continue;
		}
		while(next->right && next->right != curr){
			next = next->right;
		}
		if(next->right == curr){
			next->right = NULL;
			cout<<curr->val<<endl;
			curr = curr->right;
		}else{
			next->right = curr;
			curr = curr->left;
		}
	}
}
	\end{lstlisting}
\end{description}

\fi

\ifx allfiles undefined
\end{CJK}
\end{document}
\fi


\fi

\ifx allfiles undefined
\end{CJK}
\end{document}
\fi

\newpage
\section{二叉树的建立}
\ifx allfiles undefined
\documentclass{article}
\usepackage{CJK}

%%%代码
\usepackage{color}
\usepackage{xcolor}
\definecolor{keywordcolor}{rgb}{0.8,0.1,0.5}
\usepackage{listings}
\lstset{breaklines}%这条命令可以让LaTeX自动将长的代码行换行排版
\lstset{extendedchars=false}%这一条命令可以解决代码跨页时,章节标题,页眉等汉字不显示的问题
\lstset{language=C++, %用于设置语言为C++
	keywordstyle=\color{keywordcolor} \bfseries, %设置关键词
	identifierstyle=,
	basicstyle=\ttfamily, 
	commentstyle=\color{blue} \textit,
	stringstyle=\ttfamily, 
	showstringspaces=false,
	%frame=shadowbox, %边框
	captionpos=b
}
%%%

%\hypersetup{CJKbookmarks=true} %解决section不能使用中文的问题

\begin{document}
\begin{CJK}{UTF8}{gbsn}     %CJK:支持中文

\else
	
%二叉树的建立

\fi

\ifx allfiles undefined
\end{CJK}
\end{document}
\fi

\newpage
\section{二叉树的属性}
\ifx allfiles undefined
\documentclass{article}
\usepackage{CJK}

%%%代码
\usepackage{color}
\usepackage{xcolor}
\definecolor{keywordcolor}{rgb}{0.8,0.1,0.5}
\usepackage{listings}
\lstset{breaklines}%这条命令可以让LaTeX自动将长的代码行换行排版
\lstset{extendedchars=false}%这一条命令可以解决代码跨页时,章节标题,页眉等汉字不显示的问题
\lstset{language=C++, %用于设置语言为C++
	keywordstyle=\color{keywordcolor} \bfseries, %设置关键词
	identifierstyle=,
	basicstyle=\ttfamily, 
	commentstyle=\color{blue} \textit,
	stringstyle=\ttfamily, 
	showstringspaces=false,
	%frame=shadowbox, %边框
	captionpos=b
}
%%%

%\hypersetup{CJKbookmarks=true} %解决section不能使用中文的问题

\begin{document}
\begin{CJK}{UTF8}{gbsn}     %CJK:支持中文

\else
	
%二叉树的属性
\subsection{Validate Binary Search Tree}
\ifx allfiles undefined
\documentclass{article}
\usepackage{CJK}
\usepackage{verbatim}

%%%代码
\usepackage{color}
\usepackage{xcolor}
\definecolor{keywordcolor}{rgb}{0.8,0.1,0.5}
\usepackage{listings}
\lstset{breaklines}%这条命令可以让LaTeX自动将长的代码行换行排版
\lstset{extendedchars=false}%这一条命令可以解决代码跨页时,章节标题,页眉等汉字不显示的问题
\lstset{language=C++, %用于设置语言为C++
	keywordstyle=\color{keywordcolor} \bfseries, %设置关键词
	identifierstyle=,
	basicstyle=\ttfamily, 
	commentstyle=\color{blue} \textit,
	stringstyle=\ttfamily, 
	showstringspaces=false,
	%frame=shadowbox, %边框
	captionpos=b
}
%%%

%\hypersetup{CJKbookmarks=true} %解决section不能使用中文的问题

\begin{document}
\begin{CJK}{UTF8}{gbsn}     %CJK:支持中文

\else
	
\begin{description}
	\item{\textbf{问题}}: Given a binary tree, determine if it is a valid binary search tree (BST). \textit{(leetcode 98)}
	\\判断一个二叉树是否是合法的BST,我们可以想到BST树的中序序列是非减序列,于是我们可以使用中序遍历这颗二叉树,在遍历的过程中查看是否有反常的数据.
	\\当然,根据上面说的三种中序遍历的方法,这里同样有三种解法.
	\item{\textbf{递归}} : \fbox{时间复杂度O(n) , 空间复杂度O($lgn$)}
	\begin{lstlisting}
bool dfs(TreeNode *root, int& up){
	if(!root)	return true;
	if(root->left){
		bool left =  dfs(root->left, up);
		if(!left) return false;
	}
	if(root->val <= up && (MIN || up != (-1)<<31)){ 
	    return false;
	}	
    if(root->val == (-1)<<31)
		MIN = true;
	up = root->val;
	if(root->right){
		bool right = dfs(root->right, up);
		if(!right)	return false;
	}
	return true;
}

bool isValidBST(TreeNode *root) {
	if(!root)	return true;
	int up = (-1)<<31;
	MIN = false;
	return dfs(root, up);
}
	\end{lstlisting}
	\textit{这里可以看到一些边界条件的判断,显得有点复杂,其实就是简单的中序遍历}
	\item{\textbf{栈式迭代}} : \fbox{时间复杂度O(n) , 空间复杂度O($lgn$)}
	\begin{lstlisting}
bool isValidBST(TreeNode *root) {
	if(!root)	return false;
	stack<TreeNode*> s;
	TreeNode *p = root;
	s.push(root);
	while(p->left){
		s.push(p->left);
		p = p->left;
	}
	int last = p->val;
	s.pop();
	if(p->right){
		s.push(p->right);
		p = p->right;
		while(p->left){
			s.push(p->left);
			p = p->left;
		}
	}
	while(!s.empty()){
		p = s.top();
		s.pop();
		if(last >= p->val) return false;
		last = p->val;
		if(p->right){
			s.push(p->right);
			p = p->right;
			while(p->left){
				s.push(p->left);
				p = p->left;
			}
		}
	}
	return true;
}
	\end{lstlisting}
	\item{\textbf{Mirror迭代}} : \fbox{时间复杂度O(n) , 空间复杂度O(1)}
	\\这里使用Mirror建立线索然后进行中序遍历,在中序遍历的同时进行判断
	\begin{lstlisting}
bool isValidBST(TreeNode *root) {
	if(!root)	return true;
	TreeNode *curr = root, *next;
	int last = INT_MIN;
	bool isFirst = true;
	bool ret = true;
	while(curr){
		if(!curr->left){
			if(!isFirst && curr->val <= last){
				ret = false;
			}
			if(isFirst)
				isFirst = false;
			last = curr->val;
			curr = curr->right;
			continue;
		}
		next = curr->left;
		while(next->right){
			if(next->right == curr)	break;
			next = next->right;
		}
		if(next->right == curr){
			next->right = NULL;
			if(!isFirst && curr->val <= last){
				ret = false;
			}
			if(isFirst)
				isFirst = false;
			last = curr->val;
			curr = curr->right;
		}else{
			next->right = curr;
			curr = curr->left;
		}
	}
	return ret;
}
	\end{lstlisting}
	\textit{有了Mirror算法,是不是你已经爱上它了,再也不用栈这么麻烦了,不过有一点需要注意的是一旦你使用Mirror算法,那么必须保证把整个树全遍历一遍,不能中途退出,因为那样树的结构被改变了}
\end{description}

\fi

\ifx allfiles undefined
\end{CJK}
\end{document}
\fi

\subsection{Symmetric Tree}
\ifx allfiles undefined
\documentclass{article}
\usepackage{CJK}
\usepackage{verbatim}

%%%代码
\usepackage{color}
\usepackage{xcolor}
\definecolor{keywordcolor}{rgb}{0.8,0.1,0.5}
\usepackage{listings}
\lstset{breaklines}%这条命令可以让LaTeX自动将长的代码行换行排版
\lstset{extendedchars=false}%这一条命令可以解决代码跨页时,章节标题,页眉等汉字不显示的问题
\lstset{language=C++, %用于设置语言为C++
	keywordstyle=\color{keywordcolor} \bfseries, %设置关键词
	identifierstyle=,
	basicstyle=\ttfamily, 
	commentstyle=\color{blue} \textit,
	stringstyle=\ttfamily, 
	showstringspaces=false,
	%frame=shadowbox, %边框
	captionpos=b
}
%%%

%\hypersetup{CJKbookmarks=true} %解决section不能使用中文的问题

\begin{document}
\begin{CJK}{UTF8}{gbsn}     %CJK:支持中文

\else
	
\begin{description}
	\item{\textbf{问题}}: Given a binary tree, check whether it is a mirror of itself (ie, symmetric around its center). \textit{(leetcode 101)}
	\\这是求证树是不是自身Mirror(成镜像).
	\item{\textbf{队列}} : \fbox{时间复杂度O(n) , 空间复杂度O(w), w为树的最大宽度}
	\begin{lstlisting}
bool isSymmetric(TreeNode* root){
	if(!root) return true;
	deque<TreeNode*> left(1, root->left), right(1, root->right);
	TreeNode *l, *r;
	while(!left.empty() && !right.empty()){
		l = left.front();
		r = right.front();
		left.pop_front();
		right.pop_front();
		if(!l && !r)	continue;
		if(!l || !r || l->val != r->val)	return false;
		left.push_back(l->left);
		left.push_back(l->right);
		right.push_back(r->right);
		right.push_back(r->left);
	}
	return true;
}
	\end{lstlisting}
	\item{\textbf{递归}} : \fbox{时间复杂度O(n) , 空间复杂度O($lgn$)}
	\\这里是把一棵树的对称问题看成两棵树的对称问题
	\begin{lstlisting}
bool recursion(TreeNode* root, TreeNode* symm){
	if(!root && !symm)
		return true;
	if(!root || !symm)	return false;
	if(root->val != symm->val)	return false;
	if(root == symm)	return recursion(root->left, symm->right);
	return recursion(root->left, symm->right) && recursion(root->right, symm->left);
}

bool isSymmetric(TreeNode* root){
	if(!root) return true;
	return recursion(root, root);
}
	\end{lstlisting}
	\textit{\\这里还可以延伸出一个问题: 求一个二叉树的镜像树}
\end{description}

\fi

\ifx allfiles undefined
\end{CJK}
\end{document}
\fi

\subsection{Maximum Depth of Binary Tree}
\ifx allfiles undefined
\documentclass{article}
\usepackage{CJK}
\usepackage{verbatim}

%%%代码
\usepackage{color}
\usepackage{xcolor}
\definecolor{keywordcolor}{rgb}{0.8,0.1,0.5}
\usepackage{listings}
\lstset{breaklines}%这条命令可以让LaTeX自动将长的代码行换行排版
\lstset{extendedchars=false}%这一条命令可以解决代码跨页时,章节标题,页眉等汉字不显示的问题
\lstset{language=C++, %用于设置语言为C++
	keywordstyle=\color{keywordcolor} \bfseries, %设置关键词
	identifierstyle=,
	basicstyle=\ttfamily, 
	commentstyle=\color{blue} \textit,
	stringstyle=\ttfamily, 
	showstringspaces=false,
	%frame=shadowbox, %边框
	captionpos=b
}
%%%

%\hypersetup{CJKbookmarks=true} %解决section不能使用中文的问题

\begin{document}
\begin{CJK}{UTF8}{gbsn}     %CJK:支持中文

\else
	
\begin{description}
	\item{\textbf{问题}}: Given a binary tree, find its maximum depth.\textit{(leetcode 104)}
	\\从根节点来看,它的深度就是左右子树深度较大的那个+1,所以很自然的想到递归
	\item{\textbf{递归}} : \fbox{时间复杂度O(n) , 空间复杂度O($lgn$)}
	\\递归代码十分简洁
	\begin{lstlisting}
int maxDepth(TreeNode *root){
	if(!root)   return 0;
	int left = maxDepth(root->left);
	int right = maxDepth(root->right);
	return left < right? right + 1 : left + 1;
}
	\end{lstlisting}
	\qquad除了递归,其实这道题能不能用迭代的做法呢?答案是肯定的,最初你可能会想到用两个栈,一个栈存放节点,一个栈存放深度,其实可以把这个两者打包成一个pair,使用一个栈就可以啦
	\item{\textbf{迭代}} : \fbox{时间复杂度O(n) , 空间复杂度O($lgn$)}
	\begin{lstlisting}
int maxDepth(TreeNode *root) {
	if(!root)	return 0;
	stack<pair<TreeNode*, int> > s;
	s.push(make_pair(root, 1));
	pair<TreeNode*, int> curr;
	int result = INT_MIN;
	while(!s.empty()){
		curr = s.top();
		s.pop();
		if(!curr.first->left && !curr.first->right){
			if(result < curr.second)
				result = curr.second;
			continue;
		}
		if(curr.first->left){
			s.push(make_pair(curr.first->left, curr.second + 1));
		}
		if(curr.first->right){
			s.push(make_pair(curr.first->right, curr.second + 1));
		}
	}
	return result;
}
	\end{lstlisting}
\end{description}

\fi

\ifx allfiles undefined
\end{CJK}
\end{document}
\fi

\subsection{Minimum Depth of Binary Tree}
\ifx allfiles undefined
\documentclass{article}
\usepackage{CJK}
\usepackage{verbatim}

%%%代码
\usepackage{color}
\usepackage{xcolor}
\definecolor{keywordcolor}{rgb}{0.8,0.1,0.5}
\usepackage{listings}
\lstset{breaklines}%这条命令可以让LaTeX自动将长的代码行换行排版
\lstset{extendedchars=false}%这一条命令可以解决代码跨页时,章节标题,页眉等汉字不显示的问题
\lstset{language=C++, %用于设置语言为C++
	keywordstyle=\color{keywordcolor} \bfseries, %设置关键词
	identifierstyle=,
	basicstyle=\ttfamily, 
	commentstyle=\color{blue} \textit,
	stringstyle=\ttfamily, 
	showstringspaces=false,
	%frame=shadowbox, %边框
	captionpos=b
}
%%%

%\hypersetup{CJKbookmarks=true} %解决section不能使用中文的问题

\begin{document}
\begin{CJK}{UTF8}{gbsn}     %CJK:支持中文

\else
	
\begin{description}
	\item{\textbf{问题}}: Given a binary tree, find its minimum depth. The minimum depth is the number of nodes along the shortest path from the root node down to the nearest leaf node. \textit{(leetcode 111)}
	\item{\textbf{递归}} : \fbox{时间复杂度O(n), 空间复杂度O($lgn$)}
	\\自下而上的递归,非常的简单
	\begin{lstlisting}
int minDepth(TreeNode *root) {
	if(!root)	return 0;
	if(!root->left && !root->right)	return 1;
	if(!root->left)
		return minDepth(root->right) + 1;
	if(!root->right)
		return minDepth(root->left) + 1;
	return min(minDepth(root->left), minDepth(root->right)) + 1;
}
	\end{lstlisting}
	\item{\textbf{DFS}} : \fbox{时间复杂度O(n) , 空间复杂度O($lgn$)}
	\\这也是递归,但是是一种自上而下的递归,可以进行剪枝而不必把整个树都访问一遍
	\begin{lstlisting}
void dfs(TreeNode *root, int &result, int depth){
	if(result < depth + 1) return;
	if(!root->left && !root->right){
		result = depth + 1;
		return;
	}
	if(root->left)
		dfs(root->left, result, depth + 1);
	if(root->right)
		dfs(root->right, result, depth + 1);
}

	int minDepth(TreeNode *root) {
	if(!root)	return 0;
	int result = INT_MAX;
	dfs(root, result, 0);
	return result;
}
	\end{lstlisting}
	\item{\textbf{迭代}} : \fbox{时间复杂度O(n) , 空间复杂度O($lgn$)}
	\\同样我们也可以剪枝
	\begin{lstlisting}
int minDepth(TreeNode *root) {
	if(!root)	return 0;
	stack<pair<TreeNode*, int> > s;
	s.push(make_pair(root, 1));
	int	 result = -((1<<31) + 1);
	TreeNode *node;
	int depth;
	while(!s.empty()){
		node = s.top().first;
		depth = s.top().second;
		s.pop();
		if(result < depth)	continue;
		if(!node->left && !node->right)
			result = depth;
	
		if(node->left )
			s.push(make_pair(node->left, depth + 1));
		if(node->right)
			s.push(make_pair(node->right, depth + 1));
	}
	return result;
}
	\end{lstlisting}
\end{description}

\fi

\ifx allfiles undefined
\end{CJK}
\end{document}
\fi

\subsection{Balanced Binary Tree}
\ifx allfiles undefined
\documentclass{article}
\usepackage{CJK}
\usepackage{verbatim}

%%%代码
\usepackage{color}
\usepackage{xcolor}
\definecolor{keywordcolor}{rgb}{0.8,0.1,0.5}
\usepackage{listings}
\lstset{breaklines}%这条命令可以让LaTeX自动将长的代码行换行排版
\lstset{extendedchars=false}%这一条命令可以解决代码跨页时,章节标题,页眉等汉字不显示的问题
\lstset{language=C++, %用于设置语言为C++
	keywordstyle=\color{keywordcolor} \bfseries, %设置关键词
	identifierstyle=,
	basicstyle=\ttfamily, 
	commentstyle=\color{blue} \textit,
	stringstyle=\ttfamily, 
	showstringspaces=false,
	%frame=shadowbox, %边框
	captionpos=b
}
%%%

%\hypersetup{CJKbookmarks=true} %解决section不能使用中文的问题

\begin{document}
\begin{CJK}{UTF8}{gbsn}     %CJK:支持中文

\else
	
\begin{description}
	\item{\textbf{问题}}: Given a binary tree, determine if it is height-balanced. \textit{(leetcode 110)}
	\item{\textbf{递归}} : \fbox{时间复杂度O(n) , 空间复杂度O($lgn$)}
	\\先判断左子树是否高度平衡并返回左子树高度,再判断右子树是否高度平衡,再返回右子树高度,根据左右子树高度再判断当前树是否平衡.
	\begin{lstlisting}
bool dfs(TreeNode *root, int &hight){
	if(!root){
		hight = 0;
		return true;
	}
	int left, right;
	bool is_left = dfs(root->left, left);
	bool is_right = dfs(root->right, right);
	hight = left > right? left + 1 : right + 1;
	return is_left && is_right && (abs(left - right) < 2);
}

bool isBalanced(TreeNode *root) {
	int hight;
	return dfs(root, hight);
}
	\end{lstlisting}
\end{description}

\fi

\ifx allfiles undefined
\end{CJK}
\end{document}
\fi


\fi

\ifx allfiles undefined
\end{CJK}
\end{document}
\fi

\newpage
\section{其他}
\ifx allfiles undefined
\documentclass{article}
\usepackage{CJK}
\usepackage{verbatim}

%%%代码
\usepackage{color}
\usepackage{xcolor}
\definecolor{keywordcolor}{rgb}{0.8,0.1,0.5}
\usepackage{listings}
\lstset{breaklines}%这条命令可以让LaTeX自动将长的代码行换行排版
\lstset{extendedchars=false}%这一条命令可以解决代码跨页时,章节标题,页眉等汉字不显示的问题
\lstset{language=C++, %用于设置语言为C++
    keywordstyle=\color{keywordcolor} \bfseries, %设置关键词
    identifierstyle=,
    basicstyle=\ttfamily, 
    commentstyle=\color{blue} \textit,
    stringstyle=\ttfamily, 
    showstringspaces=false,
    %frame=shadowbox, %边框
    captionpos=b
}
%%%

%\hypersetup{CJKbookmarks=true} %解决section不能使用中文的问题

\begin{document}
\begin{CJK}{UTF8}{gbsn}     %CJK:支持中文

\else

\subsection{Flatten Binary Tree to Linked List}    
\ifx allfiles undefined
\documentclass{article}
\usepackage{CJK}
\usepackage{verbatim}

%%%代码
\usepackage{color}
\usepackage{xcolor}
\definecolor{keywordcolor}{rgb}{0.8,0.1,0.5}
\usepackage{listings}
\lstset{breaklines}%这条命令可以让LaTeX自动将长的代码行换行排版
\lstset{extendedchars=false}%这一条命令可以解决代码跨页时,章节标题,页眉等汉字不显示的问题
\lstset{language=C++, %用于设置语言为C++
    keywordstyle=\color{keywordcolor} \bfseries, %设置关键词
    identifierstyle=,
    basicstyle=\ttfamily, 
    commentstyle=\color{blue} \textit,
    stringstyle=\ttfamily, 
    showstringspaces=false,
    %frame=shadowbox, %边框
    captionpos=b
}
%%%

%\hypersetup{CJKbookmarks=true} %解决section不能使用中文的问题

\begin{document}
\begin{CJK}{UTF8}{gbsn}     %CJK:支持中文

\else
    
\begin{description}
    \item{\textbf{问题}}: Given a binary tree, flatten it to a linked list in-place by the pre-order. \textit{(leetcode 114)}
    \\这是一个基于先序遍历的问题,所以可以使用递归和迭代的方法.
    \item{\textbf{递归}} : \fbox{时间复杂度O(n) , 空间复杂度O($lgn$)}
    \begin{lstlisting}
void flatten(TreeNode *root) {
    TreeNode *tail;
    recursion(root, tail);
}

TreeNode* recursion(TreeNode *root, TreeNode* &tail){
    if(!root)    return NULL;
    TreeNode *next = NULL;
    tail = root;
    if(root->left)
        next = recursion(root->left, tail);
    if(root->right)
        tail->right = recursion(root->right, tail);
    root->left = NULL;
    if(next)
        root->right = next;
    return root;
}
    \end{lstlisting}
    \item{\textbf{迭代}} : \fbox{时间复杂度O(n) , 空间复杂度O($lgn$)}
    \\这里就是完完全全的迭代版前序遍历,这里使用了栈,同样你也可以使用Mirror算法.
    \begin{lstlisting}
void flatten(TreeNode *root) {
    if(!root)    return;
    stack<TreeNode*> s;
    s.push(root);
    TreeNode *last = NULL, *cur;
    while(!s.empty()){
        cur = s.top();
        s.pop();
        if(last)
            last->right = cur;
        if(cur->right)
            s.push(cur->right);
        if(cur->left)
            s.push(cur->left);
        cur->left = NULL;
        last = cur;
    }
    last->right = NULL;
}
    \end{lstlisting}
\end{description}

\fi

\ifx allfiles undefined
\end{CJK}
\end{document}
\fi

\subsection{Populating Next Right Pointers in Each Node}    
\ifx allfiles undefined
\documentclass{article}
\usepackage{CJK}
\usepackage{verbatim}

%%%代码
\usepackage{color}
\usepackage{xcolor}
\definecolor{keywordcolor}{rgb}{0.8,0.1,0.5}
\usepackage{listings}
\lstset{breaklines}%这条命令可以让LaTeX自动将长的代码行换行排版
\lstset{extendedchars=false}%这一条命令可以解决代码跨页时,章节标题,页眉等汉字不显示的问题
\lstset{language=C++, %用于设置语言为C++
	keywordstyle=\color{keywordcolor} \bfseries, %设置关键词
	identifierstyle=,
	basicstyle=\ttfamily, 
	commentstyle=\color{blue} \textit,
	stringstyle=\ttfamily, 
	showstringspaces=false,
	%frame=shadowbox, %边框
	captionpos=b
}
%%%

%\hypersetup{CJKbookmarks=true} %解决section不能使用中文的问题

\begin{document}
\begin{CJK}{UTF8}{gbsn}     %CJK:支持中文

\else
	
\begin{description}
	\item{\textbf{问题}}: Given a binary tree:
	\begin{lstlisting}
struct TreeLinkNode {
    TreeLinkNode *left;
    TreeLinkNode *right;
	TreeLinkNode *next;
}
	\end{lstlisting}
	Populate each next pointer to point to its next right node. If there is no next right node, the next pointer should be set to NULL. 
	\\\textbf{Note}:
	\\You may only use constant extra space.
	\\You may assume that it is a perfect binary tree (ie, all leaves are at the same level, and every parent has two children).
	\textit{(leetcode 116)}
	\\其实就是一个很简单的BFS过程.
	\item{\textbf{迭代}} : \fbox{时间复杂度O(n), 空间复杂度O(1)}
	\\因为是满二叉树,所以每次在上一层建立这一层的next,然后再到这一层来,这样就不需要队列,使用常数的空间复杂度.
	\begin{lstlisting}
void connect(TreeLinkNode *root) {
	if(!root)	return;
	TreeLinkNode *cur = root, *next;
	cur->next = NULL;
	while(cur->left){
		next = cur->left;
		while(cur){
			cur->left->next = cur->right;
			cur->right->next = cur->next? cur->next->left : NULL;
			cur = cur->next;
		}
		cur = next;
	}
}
	\end{lstlisting}
\end{description}

\fi

\ifx allfiles undefined
\end{CJK}
\end{document}
\fi

\subsection{Populating Next Right Pointers in Each Node II}    
\ifx allfiles undefined
\documentclass{article}
\usepackage{CJK}
\usepackage{verbatim}

%%%代码
\usepackage{color}
\usepackage{xcolor}
\definecolor{keywordcolor}{rgb}{0.8,0.1,0.5}
\usepackage{listings}
\lstset{breaklines}%这条命令可以让LaTeX自动将长的代码行换行排版
\lstset{extendedchars=false}%这一条命令可以解决代码跨页时,章节标题,页眉等汉字不显示的问题
\lstset{language=C++, %用于设置语言为C++
	keywordstyle=\color{keywordcolor} \bfseries, %设置关键词
	identifierstyle=,
	basicstyle=\ttfamily, 
	commentstyle=\color{blue} \textit,
	stringstyle=\ttfamily, 
	showstringspaces=false,
	%frame=shadowbox, %边框
	captionpos=b
}
%%%

%\hypersetup{CJKbookmarks=true} %解决section不能使用中文的问题

\begin{document}
\begin{CJK}{UTF8}{gbsn}     %CJK:支持中文

\else
	
\begin{description}
	\item{\textbf{问题}}: Follow up for problem "Populating Next Right Pointers in Each Node".What if the given tree could be any binary tree? Would your previous solution still work?
	\textbf{Note}: You may only use constant extra space. \textit{(leetcode 117)}
	\item{\textbf{迭代}} : \fbox{时间复杂度O(n) , 空间复杂度O(1)}
	\\这里其实和上一题一样,只不过多了一些判断条件.
	\begin{lstlisting}
void connect(TreeLinkNode *root) {
	if(!root)	return;
	TreeLinkNode *cur = root, *next, *last;
	cur->next = NULL;
	do{
		next = NULL;
		while(cur){
			if(cur->left){
				if(next){
					last->next = cur->left;
					last = last->next;
				}
				else{
					last = cur->left;
					next = last;
				}
			}
			if(cur->right){
				if(next){
					last->next = cur->right;
					last = last->next;
				}
				else{
					last = cur->left;
					next = last;
				}
			}
			cur = cur->next;
		}
		last->next = NULL;
		cur = next;
	}while(cur);
}
	\end{lstlisting}
	\textit{}
\end{description}

\fi

\ifx allfiles undefined
\end{CJK}
\end{document}
\fi

\subsection{Binary Search Tree Iterator}    
\ifx allfiles undefined
\documentclass{article}
\usepackage{CJK}
\usepackage{verbatim}

%%%代码
\usepackage{color}
\usepackage{xcolor}
\definecolor{keywordcolor}{rgb}{0.8,0.1,0.5}
\usepackage{listings}
\lstset{breaklines}%这条命令可以让LaTeX自动将长的代码行换行排版
\lstset{extendedchars=false}%这一条命令可以解决代码跨页时,章节标题,页眉等汉字不显示的问题
\lstset{language=C++, %用于设置语言为C++
    keywordstyle=\color{keywordcolor} \bfseries, %设置关键词
    identifierstyle=,
    basicstyle=\ttfamily, 
    commentstyle=\color{blue} \textit,
    stringstyle=\ttfamily, 
    showstringspaces=false,
    %frame=shadowbox, %边框
    captionpos=b
}
%%%

%\hypersetup{CJKbookmarks=true} %解决section不能使用中文的问题

\begin{document}
\begin{CJK}{UTF8}{gbsn}     %CJK:支持中文

\else
    
\begin{description}
    \item{\textbf{问题}}: \\
Implement an iterator over a binary search tree (BST). Your iterator will be initialized with the root node of a BST. \\
\\
Calling next() will return the next smallest number in the BST. \\
\textit{(leetcode 173)}
    \item{\textbf{Note}}: \\
next() and hasNext() should run in average O(1) time and uses O(h) memory, where h is the height of the tree.
    \item{\textbf{Stack}} : \fbox{时间复杂度O(1), 空间复杂度O(lgn)}
    \\这里使用一个栈来保存上级节点,每个next和hasNext的操作平均时间复杂度是O(1)
    \begin{lstlisting}
class BSTIterator {
private:
	TreeNode* root;
	stack<TreeNode*> path;
public:
    BSTIterator(TreeNode *root) : root(root){
		TreeNode* iter = root;
		while(iter){
			path.push(iter);
			iter = iter->left;
		}
    }

    /** @return whether we have a next smallest number */
    bool hasNext() {
		return !path.empty();
    }

    /** @return the next smallest number */
    int next() {
        TreeNode *cur = path.top();
		TreeNode *iter = cur->right;
		path.pop();
		while(iter){
			path.push(iter);
			iter = iter->left;
		}
		return cur->val;
    }
};
    \end{lstlisting}
\end{description}

\fi

\ifx allfiles undefined
\end{CJK}
\end{document}
\fi


\fi

\ifx allfiles undefined
\end{CJK}
\end{document}
\fi

\newpage

\fi

\ifx allfiles undefined
\end{CJK}
\end{document}
\fi


\chapter{字符串}
\ifx allfiles undefined
\documentclass{article}
\usepackage{CJK}

%%%代码
\usepackage{color}
\usepackage{xcolor}
\definecolor{keywordcolor}{rgb}{0.8,0.1,0.5}
\usepackage{listings}
\lstset{breaklines}%这条命令可以让LaTeX自动将长的代码行换行排版
\lstset{extendedchars=false}%这一条命令可以解决代码跨页时,章节标题,页眉等汉字不显示的问题
\lstset{language=C++, %用于设置语言为C++
    keywordstyle=\color{keywordcolor} \bfseries, %设置关键词
    identifierstyle=,
    basicstyle=\ttfamily, 
    commentstyle=\color{blue} \textit,
    stringstyle=\ttfamily, 
    showstringspaces=false,
    %frame=shadowbox, %边框
    captionpos=b
}
%%%

%\hypersetup{CJKbookmarks=true} %解决section不能使用中文的问题

\begin{document}
\begin{CJK}{UTF8}{gbsn}     %CJK:支持中文

\else

\section{库函数}
\ifx allfiles undefined
\documentclass{article}
\usepackage{CJK}

%%%代码
\usepackage{color}
\usepackage{xcolor}
\definecolor{keywordcolor}{rgb}{0.8,0.1,0.5}
\usepackage{listings}
\lstset{breaklines}%这条命令可以让LaTeX自动将长的代码行换行排版
\lstset{extendedchars=false}%这一条命令可以解决代码跨页时,章节标题,页眉等汉字不显示的问题
\lstset{language=C++, %用于设置语言为C++
    keywordstyle=\color{keywordcolor} \bfseries, %设置关键词
    identifierstyle=,
    basicstyle=\ttfamily, 
    commentstyle=\color{blue} \textit,
    stringstyle=\ttfamily, 
    showstringspaces=false,
    %frame=shadowbox, %边框
    captionpos=b
}
%%%

%\hypersetup{CJKbookmarks=true} %解决section不能使用中文的问题

\begin{document}
\begin{CJK}{UTF8}{gbsn}     %CJK:支持中文

\else
    
\qquad字符串的库函数有很多,虽然大多数算法上都不是很难,但是需要考虑很多细节问题,所以需要研究一下.关于字符串库函数需要注意的问题大致可以总结为以下几点:

\begin{itemize}
\item 输入参数是什么类型,该不该是const修饰
\item 输入的指针是否为NULL
\item 字符串处理完毕后,新产生的字符串末尾是否要加$'\backslash0'$
\item 要不要返回值,该返回什么类型的(对于要返回值的是为了实现链式操作)
\end{itemize}

\textbf{还有一些我觉得可能需要考虑的:}

\begin{itemize}
\item 有没有地址重叠
\item src和dest地址一样
\item 需不需要优化做法
\end{itemize}

\subsection{strlen}
\ifx allfiles undefined
\documentclass{article}
\usepackage{CJK}
\usepackage{verbatim}

%%%代码
\usepackage{color}
\usepackage{xcolor}
\definecolor{keywordcolor}{rgb}{0.8,0.1,0.5}
\usepackage{listings}
\lstset{breaklines}%这条命令可以让LaTeX自动将长的代码行换行排版
\lstset{extendedchars=false}%这一条命令可以解决代码跨页时,章节标题,页眉等汉字不显示的问题
\lstset{language=C++, %用于设置语言为C++
    keywordstyle=\color{keywordcolor} \bfseries, %设置关键词
    identifierstyle=,
    basicstyle=\ttfamily, 
    commentstyle=\color{blue} \textit,
    stringstyle=\ttfamily, 
    showstringspaces=false,
    %frame=shadowbox, %边框
    captionpos=b
}
%%%

%\hypersetup{CJKbookmarks=true} %解决section不能使用中文的问题

\begin{document}
\begin{CJK}{UTF8}{gbsn}     %CJK:支持中文

\else
    
\begin{description}
    \item{\textbf{问题}}: 求出字符串的长度.
    \item{\textbf{Care}} : \fbox{时间复杂度O(n), 空间复杂度O(1)}
    \\考虑输入指针不为空,输入参数使用const char
    \begin{lstlisting}
int strlen(const char *str){
	int len = 0;
	assert(str);  //判断不为NULL
	while(*str++ != '\0'){
		len++;
	}
	return len;
}
    \end{lstlisting}
\end{description}

\fi

\ifx allfiles undefined
\end{CJK}
\end{document}
\fi

\subsection{strcat}
\ifx allfiles undefined
\documentclass{article}
\usepackage{CJK}
\usepackage{verbatim}

%%%代码
\usepackage{color}
\usepackage{xcolor}
\definecolor{keywordcolor}{rgb}{0.8,0.1,0.5}
\usepackage{listings}
\lstset{breaklines}%这条命令可以让LaTeX自动将长的代码行换行排版
\lstset{extendedchars=false}%这一条命令可以解决代码跨页时,章节标题,页眉等汉字不显示的问题
\lstset{language=C++, %用于设置语言为C++
    keywordstyle=\color{keywordcolor} \bfseries, %设置关键词
    identifierstyle=,
    basicstyle=\ttfamily, 
    commentstyle=\color{blue} \textit,
    stringstyle=\ttfamily, 
    showstringspaces=false,
    %frame=shadowbox, %边框
    captionpos=b
}
%%%

%\hypersetup{CJKbookmarks=true} %解决section不能使用中文的问题

\begin{document}
\begin{CJK}{UTF8}{gbsn}     %CJK:支持中文

\else
    
\begin{description}
    \item{\textbf{问题}}: 将一个字符串连接到另一个字符串后面
    \item{\textbf{Care}} : \fbox{时间复杂度O(n), 空间复杂度O(1)}
	\\输入参数const性和非NULL,尾赋$'\backslash0'$, 有返回值
    \begin{lstlisting}
char *strcat(char *strDest, const char *strScr){
	assert(strDest && strScr);
	if(!strScr)	return strDest;
	char* p = strDest;
	while(*p){
		p++;
	}
	while(*strScr){
		*p++ = *strScr++;
	}
	*p = '\0';
	return strDest;
}
    \end{lstlisting}
\end{description}

\fi

\ifx allfiles undefined
\end{CJK}
\end{document}
\fi

\subsection{strcmp}
\ifx allfiles undefined
\documentclass{article}
\usepackage{CJK}
\usepackage{verbatim}

%%%代码
\usepackage{color}
\usepackage{xcolor}
\definecolor{keywordcolor}{rgb}{0.8,0.1,0.5}
\usepackage{listings}
\lstset{breaklines}%这条命令可以让LaTeX自动将长的代码行换行排版
\lstset{extendedchars=false}%这一条命令可以解决代码跨页时,章节标题,页眉等汉字不显示的问题
\lstset{language=C++, %用于设置语言为C++
    keywordstyle=\color{keywordcolor} \bfseries, %设置关键词
    identifierstyle=,
    basicstyle=\ttfamily, 
    commentstyle=\color{blue} \textit,
    stringstyle=\ttfamily, 
    showstringspaces=false,
    %frame=shadowbox, %边框
    captionpos=b
}
%%%

%\hypersetup{CJKbookmarks=true} %解决section不能使用中文的问题

\begin{document}
\begin{CJK}{UTF8}{gbsn}     %CJK:支持中文

\else
    
\begin{description}
    \item{\textbf{问题}}: 比较两个字符串是否完全相同.
    \item{\textbf{Care}} : \fbox{时间复杂度O(n), 空间复杂度O(1)}
	\\输入参数const性和非NULL,注意如何写的简洁.
    \begin{lstlisting}
int strcmp(const char *str1,const char *str2){
	if(str1 == str2)	return 0;
	assert(str1 && str2);
	//这样写很简洁
	while(*str1 && *str2 && (*str1 == *str2)){
		str1++;
		str2++;
	}
	return (*str1) - (*str2);
}
    \end{lstlisting}
\end{description}

\fi

\ifx allfiles undefined
\end{CJK}
\end{document}
\fi

\subsection{strcpy}
\ifx allfiles undefined
\documentclass{article}
\usepackage{CJK}
\usepackage{verbatim}

%%%代码
\usepackage{color}
\usepackage{xcolor}
\definecolor{keywordcolor}{rgb}{0.8,0.1,0.5}
\usepackage{listings}
\lstset{breaklines}%这条命令可以让LaTeX自动将长的代码行换行排版
\lstset{extendedchars=false}%这一条命令可以解决代码跨页时,章节标题,页眉等汉字不显示的问题
\lstset{language=C++, %用于设置语言为C++
    keywordstyle=\color{keywordcolor} \bfseries, %设置关键词
    identifierstyle=,
    basicstyle=\ttfamily, 
    commentstyle=\color{blue} \textit,
    stringstyle=\ttfamily, 
    showstringspaces=false,
    %frame=shadowbox, %边框
    captionpos=b
}
%%%

%\hypersetup{CJKbookmarks=true} %解决section不能使用中文的问题

\begin{document}
\begin{CJK}{UTF8}{gbsn}     %CJK:支持中文

\else
    
\begin{description}
    \item{\textbf{问题}}: 将源字符串完全拷贝给目的字符串.
    \item{\textbf{Care}} : \fbox{时间复杂度O(n), 空间复杂度O(1)}
	\\考虑输入参数的const性,输入参数是否合法,返回值,最后位置为$'\backslash0'$
    \begin{lstlisting}
char *strcpy(char *dest,const char *src){
	if(dest == src)	return dest;
	assert(dest && src);  //输入参数不为NULL
	char* address = dest;
	int lens = strlen(src);
	int lend = strlen(dest);
	if(src + lens > dest){ // 从后往前拷贝
		dest[lens] = '\0';   //当lens < lend就很重要
		for(int i = lens - 1; i >= 0; i--){
			dest[i] = src[i];
		}
	}else{ //从前往后拷贝
		for(int i = 0; i < lens; i++){
			dest[i] = src[i];
		}
		dest[lens] = '\0';  //当lens < lend就很重要
	}
	return address; //注意此函数有返回值
}
    \end{lstlisting}
	\textbf{至于src和dest地址重叠问题,libc里面并没有考虑,我觉得考虑一下还是很好的.}
\end{description}

\fi

\ifx allfiles undefined
\end{CJK}
\end{document}
\fi

\subsection{strncpy}
\ifx allfiles undefined
\documentclass{article}
\usepackage{CJK}
\usepackage{verbatim}

%%%代码
\usepackage{color}
\usepackage{xcolor}
\definecolor{keywordcolor}{rgb}{0.8,0.1,0.5}
\usepackage{listings}
\lstset{breaklines}%这条命令可以让LaTeX自动将长的代码行换行排版
\lstset{extendedchars=false}%这一条命令可以解决代码跨页时,章节标题,页眉等汉字不显示的问题
\lstset{language=C++, %用于设置语言为C++
    keywordstyle=\color{keywordcolor} \bfseries, %设置关键词
    identifierstyle=,
    basicstyle=\ttfamily, 
    commentstyle=\color{blue} \textit,
    stringstyle=\ttfamily, 
    showstringspaces=false,
    %frame=shadowbox, %边框
    captionpos=b
}
%%%

%\hypersetup{CJKbookmarks=true} %解决section不能使用中文的问题

\begin{document}
\begin{CJK}{UTF8}{gbsn}     %CJK:支持中文

\else
    
\begin{description}
    \item{\textbf{问题}}: 将源字符串拷贝前n个字符给目的字符串.
    \item{\textbf{Care}} : \fbox{时间复杂度O(n), 空间复杂度O(1)}
	\\注意事项同strcpy
    \begin{lstlisting}
char *strncpy(char *dest,const char *src, int n){
	assert(dest && src);
	char *address = dest;
	int lens = strlen(src);
	int lend = strlen(dest);
	if(src + lens > dest && src + n > dest){//从后往前拷贝
		dest[min(lens, n)] = '\0';
		for(int i = min(lens, n) - 1; i >= 0; i--){
			dest[i] = src[i];
		}
	}else{ //从前往后拷贝
		for(int i = 0; i < lens && i < n; i++){
			dest[i] = src[i];
		}
		dest[min(lens, n)] = '\0';
	}
	return address;
}
    \end{lstlisting}
	\textbf{注意不能仅仅因为src == dest就放弃拷贝.}
\end{description}

\fi

\ifx allfiles undefined
\end{CJK}
\end{document}
\fi

\subsection{memcpy}
\ifx allfiles undefined
\documentclass{article}
\usepackage{CJK}
\usepackage{verbatim}

%%%代码
\usepackage{color}
\usepackage{xcolor}
\definecolor{keywordcolor}{rgb}{0.8,0.1,0.5}
\usepackage{listings}
\lstset{breaklines}%这条命令可以让LaTeX自动将长的代码行换行排版
\lstset{extendedchars=false}%这一条命令可以解决代码跨页时,章节标题,页眉等汉字不显示的问题
\lstset{language=C++, %用于设置语言为C++
    keywordstyle=\color{keywordcolor} \bfseries, %设置关键词
    identifierstyle=,
    basicstyle=\ttfamily, 
    commentstyle=\color{blue} \textit,
    stringstyle=\ttfamily, 
    showstringspaces=false,
    %frame=shadowbox, %边框
    captionpos=b
}
%%%

%\hypersetup{CJKbookmarks=true} %解决section不能使用中文的问题

\begin{document}
\begin{CJK}{UTF8}{gbsn}     %CJK:支持中文

\else
    
\begin{description}
    \item{\textbf{问题}}: 将源地址开始的连续n个字节大小空间拷贝给给目的地址.
    \item{\textbf{Care}} : \fbox{时间复杂度O(n), 空间复杂度O(1)}
	\\注意输入参数是void*和const void*, 输出参数是void*,要保证输出参数非NULL
    \begin{lstlisting}
void *memcpy(void *dest, const void *src, size_t count){
	if(dest == src)	return dest;
	assert(src && dest);
	if(!src || !dest)	return NULL;
	unsigned char *d = (unsigned char*)dest, *s = (unsigned char*)src;
	while(count--){
		*d++ = *s++;
	}
	//不需要设置'\0',因为是内存拷贝
	return dest;
}
    \end{lstlisting}
	\textbf{libc使用了page copy, unsigned long copy做到快速拷贝}
\end{description}

\fi

\ifx allfiles undefined
\end{CJK}
\end{document}
\fi


\fi

\ifx allfiles undefined
\end{CJK}
\end{document}
\fi

\newpage
\section{字符串经典问题}
\ifx allfiles undefined
\documentclass{article}
\usepackage{CJK}

%%%代码
\usepackage{color}
\usepackage{xcolor}
\definecolor{keywordcolor}{rgb}{0.8,0.1,0.5}
\usepackage{listings}
\lstset{breaklines}%这条命令可以让LaTeX自动将长的代码行换行排版
\lstset{extendedchars=false}%这一条命令可以解决代码跨页时,章节标题,页眉等汉字不显示的问题
\lstset{language=C++, %用于设置语言为C++
    keywordstyle=\color{keywordcolor} \bfseries, %设置关键词
    identifierstyle=,
    basicstyle=\ttfamily, 
    commentstyle=\color{blue} \textit,
    stringstyle=\ttfamily, 
    showstringspaces=false,
    %frame=shadowbox, %边框
    captionpos=b
}
%%%

%\hypersetup{CJKbookmarks=true} %解决section不能使用中文的问题

\begin{document}
\begin{CJK}{UTF8}{gbsn}     %CJK:支持中文

\else
    
\qquad字符串的经典问题有很多,虽然很多解法都属于算法范畴,诸如动态规划等,这里还是归结为字符串的问题.

\subsection{strStr}
\ifx allfiles undefined
\documentclass{article}
\usepackage{CJK}
\usepackage{verbatim}

%%%代码
\usepackage{color}
\usepackage{xcolor}
\definecolor{keywordcolor}{rgb}{0.8,0.1,0.5}
\usepackage{listings}
\lstset{breaklines}%这条命令可以让LaTeX自动将长的代码行换行排版
\lstset{extendedchars=false}%这一条命令可以解决代码跨页时,章节标题,页眉等汉字不显示的问题
\lstset{language=C++, %用于设置语言为C++
    keywordstyle=\color{keywordcolor} \bfseries, %设置关键词
    identifierstyle=,
    basicstyle=\ttfamily, 
    commentstyle=\color{blue} \textit,
    stringstyle=\ttfamily, 
    showstringspaces=false,
    %frame=shadowbox, %边框
    captionpos=b
}
%%%

%\hypersetup{CJKbookmarks=true} %解决section不能使用中文的问题

\begin{document}
\begin{CJK}{UTF8}{gbsn}     %CJK:支持中文

\else
    
\begin{description}
    \item{\textbf{问题}}: Implement strStr(). Returns the index of the first occurrence of needle in haystack, or -1 if needle is not part of haystack. \textit{(leetcode 28)}
    \item{\textbf{KMP}} : \fbox{时间复杂度O(n), 空间复杂度O(m)}
    \\strstr属于字符串的库函数,放在这里讲完全是因为它是字符串问题中最有名的之一,其解法多样,这里推荐KMP解法,关于此解法的介绍可以参考我们ThreeCobblers主页上\href{https://github.com/ThreeCobblers/Paladin/blob/master/blog/string/KMP.md}{zhuoyuan的博文}
    \begin{lstlisting}
void gen_next(const char *p) {
    next[0] = -1;
    int i = 0;
    int j = -1;
    int lp = strlen(p);
    while(i < lp)
        if(j == -1 || p[i] == p[j]) i++, j++, next[i] = j;
        else j = next[j];
}
int kmp(const char *s, const char *p) {
    gen_next(p);
    int ls = strlen(s);
    int lp = strlen(p);
    int i = -1;
    int j = -1;
    while(i < ls && j < lp)
        if(j == -1 || s[i] == p[j]) i++, j++;
        else j = next[j];
    if(j == lp) return i - lp;
    return -1;
}
    \end{lstlisting}
    \textit{这段代码水很深,写的非常老道,需要好好揣测才可以明白,关于求next数组问题可以看上面说的那篇博客的介绍}
\end{description}

\fi

\ifx allfiles undefined
\end{CJK}
\end{document}
\fi

\subsection{atoi}
\ifx allfiles undefined
\documentclass{article}
\usepackage{CJK}
\usepackage{verbatim}

%%%代码
\usepackage{color}
\usepackage{xcolor}
\definecolor{keywordcolor}{rgb}{0.8,0.1,0.5}
\usepackage{listings}
\lstset{breaklines}%这条命令可以让LaTeX自动将长的代码行换行排版
\lstset{extendedchars=false}%这一条命令可以解决代码跨页时,章节标题,页眉等汉字不显示的问题
\lstset{language=C++, %用于设置语言为C++
    keywordstyle=\color{keywordcolor} \bfseries, %设置关键词
    identifierstyle=,
    basicstyle=\ttfamily, 
    commentstyle=\color{blue} \textit,
    stringstyle=\ttfamily, 
    showstringspaces=false,
    %frame=shadowbox, %边框
    captionpos=b
}
%%%

%\hypersetup{CJKbookmarks=true} %解决section不能使用中文的问题

\begin{document}
\begin{CJK}{UTF8}{gbsn}     %CJK:支持中文

\else
    
\begin{description}
    \item{\textbf{问题}}: Implement atoi to convert a string to an integer. \textit{(leetcode 8)}
	\item{\textbf{Note}}: The function first discards as many whitespace characters as necessary until the first non-whitespace character is found. Then, starting from this character, takes an optional initial plus or minus sign followed by as many numerical digits as possible, and interprets them as a numerical value.

The string can contain additional characters after those that form the integral number, which are ignored and have no effect on the behavior of this function.

If the first sequence of non-whitespace characters in str is not a valid integral number, or if no such sequence exists because either str is empty or it contains only whitespace characters, no conversion is performed.

If no valid conversion could be performed, a zero value is returned. If the correct value is out of the range of representable values, INT\_MAX (2147483647) or INT\_MIN (-2147483648) is returned.
    \item{\textbf{Care}} : \fbox{时间复杂度O(n), 空间复杂度O(1)}
    \\这个问题主要要注意两个方面,一个是空格和非法字符,一个是溢出,空格和非法字符都很好处理,溢出的处理方式就值得去研究,在C++中INT\_MAX的绝对值比INT\_MIN小1,所以负数个数比正数个数多一个,你可以都转化位负数来判断溢出. 当然,对付溢出还有一种偷懒的方法就是使用double类型取存储数据.
    \begin{lstlisting}
int atoi(const char *str){
	int sum = 0;  // 存负值.
	bool isMinus = false;
	const char* p = str;
	if(!p)	return 0;
	while(*p == ' ' && *p != '\0') p++; //空格
	if(*p == '\0')	return 0;
	if(*p == '-'){
		isMinus = true;
		p++;
	}else{
		if(*p == '+') p++;
	}
	while(*p != '\0'){
		int cur = *p++ - '0';
		if(!(cur >= 0 && cur <= 9))	break;  //非法字符
		if(isMinus && sum == INT_MIN/10 && cur > INT_MAX%10 + 1){
			return INT_MIN;
		}
		if(isMinus && sum < INT_MIN/10){
			return INT_MIN;
		}
		if(!isMinus && sum == -INT_MAX/10 && cur > INT_MAX%10){
			return INT_MAX;
		}
		if(!isMinus && sum < -INT_MAX/10){
			return INT_MAX;
		}
		sum = sum*10 - cur;
	}
	if(!isMinus)	return -sum;
	return sum;
}
    \end{lstlisting}
\end{description}

\fi

\ifx allfiles undefined
\end{CJK}
\end{document}
\fi

\subsection{Valid Palindrome}
\ifx allfiles undefined
\documentclass{article}
\usepackage{CJK}
\usepackage{verbatim}

%%%代码
\usepackage{color}
\usepackage{xcolor}
\definecolor{keywordcolor}{rgb}{0.8,0.1,0.5}
\usepackage{listings}
\lstset{breaklines}%这条命令可以让LaTeX自动将长的代码行换行排版
\lstset{extendedchars=false}%这一条命令可以解决代码跨页时,章节标题,页眉等汉字不显示的问题
\lstset{language=C++, %用于设置语言为C++
    keywordstyle=\color{keywordcolor} \bfseries, %设置关键词
    identifierstyle=,
    basicstyle=\ttfamily, 
    commentstyle=\color{blue} \textit,
    stringstyle=\ttfamily, 
    showstringspaces=false,
    %frame=shadowbox, %边框
    captionpos=b
}
%%%

%\hypersetup{CJKbookmarks=true} %解决section不能使用中文的问题

\begin{document}
\begin{CJK}{UTF8}{gbsn}     %CJK:支持中文

\else
    
\begin{description}
    \item{\textbf{问题}}: Given a string, determine if it is a palindrome, considering only alphanumeric characters and ignoring cases.\textit{(leetcode 125)}

	\item{\textbf{举例}}: "A man, a plan, a canal: Panama" is a palindrome. "race a car" is not a palindrome. 
    \item{\textbf{Care}} : \fbox{时间复杂度O(n) , 空间复杂度O(1)}
    \begin{lstlisting}
bool isAlpha(char c){
	if(c <= 'Z' && c >= 'A') return true;
	if(c <= 'z' && c >= 'a') return true;
	if(c <= '9' && c >= '0') return true;
	return false;
}

bool isPalindrome(string s) {
		int len = s.length();
		int start = 0, end = len - 1;
		while(start < end){
			while(start < end && !isAlpha(s[start]))	start++;
			if(start == end)	return true;
			while(end > start && !isAlpha(s[end]))	end--;
			if(s[start] != s[end] && abs(s[start] - s[end]) != abs('a' - 'A')) return false;
			start++;
			end--;
		}
		return true;
	}
    \end{lstlisting}
\end{description}

\fi

\ifx allfiles undefined
\end{CJK}
\end{document}
\fi

\subsection{LongestPalindrome}
\ifx allfiles undefined
\documentclass{article}
\usepackage{CJK}
\usepackage{verbatim}

%%%代码
\usepackage{color}
\usepackage{xcolor}
\definecolor{keywordcolor}{rgb}{0.8,0.1,0.5}
\usepackage{listings}
\lstset{breaklines}%这条命令可以让LaTeX自动将长的代码行换行排版
\lstset{extendedchars=false}%这一条命令可以解决代码跨页时,章节标题,页眉等汉字不显示的问题
\lstset{language=C++, %用于设置语言为C++
    keywordstyle=\color{keywordcolor} \bfseries, %设置关键词
    identifierstyle=,
    basicstyle=\ttfamily, 
    commentstyle=\color{blue} \textit,
    stringstyle=\ttfamily, 
    showstringspaces=false,
    %frame=shadowbox, %边框
    captionpos=b
}
%%%

%\hypersetup{CJKbookmarks=true} %解决section不能使用中文的问题

\begin{document}
\begin{CJK}{UTF8}{gbsn}     %CJK:支持中文

\else
    
\begin{description}
    \item{\textbf{问题}}: Longest Palindromic Substring. \textit{(leetcode 5)}
    \item{\textbf{Manacher}} : \fbox{时间复杂度O(n), 空间复杂度O(n)}
    \\这里介绍的是Manacher算法,关于此算法的思想和证明可以参考我们

	ThreeCobblers主页上\href{https://github.com/ThreeCobblers/Paladin/blob/master/blog/string/LongestPalindrome.md}{sosohu的博文}
    \begin{lstlisting}
string longestPalindrome(string const& s) {
        int n = s.length();
        if(n == 0)  return "";
        string str = "#";
        int count = 0;
        for(int i = 0; i < n; i++){
            str += s[i];
            str += '#';
        }
        int id = 0, mx = 0;
        vector<int> p(2*n+1, 0);
        p[0] = 1;
        for(int i = 1; i < 2*n + 1; i++){
            int j = 2*id - i;
            p[i] = mx > i? min(p[j], mx - i) : 1;
            while(i + p[i] < 2*n + 1 && i - p[i] >= 0){
                if(str[i + p[i]] != str[i - p[i]]) break;
                p[i]++;
            }
            if(i + p[i] > mx){
                mx = i+ p[i];
                id = i;
            }
        }
        int max = INT_MIN;
        int pos = 0;
        for(int i = 0; i < 2*n + 1; i++){
            if(max < p[i]){
                max = p[i];
                pos = i;
            }
        }
        int index, len;
        len = (max - 1);
        index = pos/2 - len/2;
        return s.substr(index, len);
    }
    \end{lstlisting}
\end{description}

\fi

\ifx allfiles undefined
\end{CJK}
\end{document}
\fi

\subsection{Anagrams}
\ifx allfiles undefined
\documentclass{article}
\usepackage{CJK}
\usepackage{verbatim}

%%%代码
\usepackage{color}
\usepackage{xcolor}
\definecolor{keywordcolor}{rgb}{0.8,0.1,0.5}
\usepackage{listings}
\lstset{breaklines}%这条命令可以让LaTeX自动将长的代码行换行排版
\lstset{extendedchars=false}%这一条命令可以解决代码跨页时,章节标题,页眉等汉字不显示的问题
\lstset{language=C++, %用于设置语言为C++
    keywordstyle=\color{keywordcolor} \bfseries, %设置关键词
    identifierstyle=,
    basicstyle=\ttfamily, 
    commentstyle=\color{blue} \textit,
    stringstyle=\ttfamily, 
    showstringspaces=false,
    %frame=shadowbox, %边框
    captionpos=b
}
%%%

%\hypersetup{CJKbookmarks=true} %解决section不能使用中文的问题

\begin{document}
\begin{CJK}{UTF8}{gbsn}     %CJK:支持中文

\else
    
\begin{description}
    \item{\textbf{问题}}: Given an array of strings, return all groups of strings that are anagrams.
	\item{\textbf{Note}}: All inputs will be in lower-case. 
	\textit{(leetcode 49)}
    \item{\textbf{sort}} : \fbox{时间复杂度O(n*m) , 空间复杂度O(n*m)}
    \\变位词是字符串的一种常见问题,很多时候都是先把一群互为变位词的词选出一个为它们的索引词(key),然后就可以把它们放在一起存储.这道题如果不是先这样,而是一个一个查找那么是非常低效的.
    \begin{lstlisting}
vector<string> anagrams(vector<string> &strs) {
	int size = strs.size();
	unordered_map<string, vector<string> > table;
	for(int i = 0; i < size; i++){
		string index = strs[i];
		sort(index.begin(), index.end());
		table[index].push_back(strs[i]);
	}
	vector<string> ret;
	for(unordered_map<string, vector<string> >::iterator iter = table.begin();
		iter != table.end(); iter++){
		if(iter->second.size() > 1){
			ret.insert(ret.end(), iter->second.begin(), iter->second.end());
		}
	}
return ret;
}
    \end{lstlisting}
\end{description}

\fi

\ifx allfiles undefined
\end{CJK}
\end{document}
\fi

\subsection{Valid Number}
\ifx allfiles undefined
\documentclass{article}
\usepackage{CJK}
\usepackage{verbatim}

%%%代码
\usepackage{color}
\usepackage{xcolor}
\definecolor{keywordcolor}{rgb}{0.8,0.1,0.5}
\usepackage{listings}
\lstset{breaklines}%这条命令可以让LaTeX自动将长的代码行换行排版
\lstset{extendedchars=false}%这一条命令可以解决代码跨页时,章节标题,页眉等汉字不显示的问题
\lstset{language=C++, %用于设置语言为C++
    keywordstyle=\color{keywordcolor} \bfseries, %设置关键词
    identifierstyle=,
    basicstyle=\ttfamily, 
    commentstyle=\color{blue} \textit,
    stringstyle=\ttfamily, 
    showstringspaces=false,
    %frame=shadowbox, %边框
    captionpos=b
}
%%%

%\hypersetup{CJKbookmarks=true} %解决section不能使用中文的问题

\begin{document}
\begin{CJK}{UTF8}{gbsn}     %CJK:支持中文

\else

    
\begin{description}
    \item{\textbf{问题}}: Validate if a given string is numeric.
	\item{\textbf{举例}} "0" is true " 0.1 " is true "abc" is false "1 a" is false "2e10" is true
    \item{\textbf{NFA}} : \fbox{时间复杂度O(n), 空间复杂度O(1)}
    \\这种问题最优美的解法就是先写出正则表达式,然后根据正则表达式画出NFA,然后根据NFA的状态转移写出代码.\\

	\textbf{正则表达式: $s^*(+|-)?((d^+.?)|(.d))d^*(e(+|-)?d^+)?$   s为空格,d为数字}

	画出状态转移图如下:\\ \\

$
\psmatrix[mnode=circle,colsep=1]
0 & 1 & 3 & 4 & 5 & 6 \\
  &   & 2 & 8 &   & 7 \\
\endpsmatrix
$
\psset{shortput=nab,arrows=->,labelsep=3pt, nodesep=3pt}
\small
\nccircle{1,1}{.3cm}
\nbput*{s}
\ncline{1,1}{1,2}
\nbput*{+/-}
\ncarc[arcangle=45]{1,1}{1,3}
\naput*{.}
\ncline{1,1}{2,3}
\nbput*{d}
\ncline{1,2}{1,3}^{.}
\ncline{1,2}{2,3}^[npos=.3]{d}
\ncline{1,3}{1,4}^{d}
\nccircle{1,4}{.3cm}
\nbput*{d}
\ncline{1,4}{1,5}^{e}
\ncline{1,4}{2,4}^{s}
\ncline{1,5}{1,6}^{+/-}
\ncline{1,5}{2,6}^[npos=.3]{d}
\ncline{1,6}{2,6}^[npos=.3]{d}
\nccircle{2,3}{.3cm}
\nbput*{d}
\ncline{2,3}{1,4}^{.}
\ncline{2,3}{1,4}^{d}
\ncline{2,3}{2,4}^{s}
\ncline{2,6}{2,4}^{s}
\nccircle{2,4}{.3cm}
\nbput*{s}

    \begin{lstlisting}
enum lexical{ valid, space, sign, number, dot, e };

int isType(const char c){
	switch(c){
		case ' ': return 1;
		case '+': ;
		case '-': return 2;
		case '.': return 4;
		case 'e': return 5;
		default: if(c <= '9' && c >= '0')	return 3;
				 else return 0;
	}
}

bool isNumber(const char *s) {
	if(!s)	return false;
	//状态转移矩阵, -1表示非法
	int map[9][6] = {
		-1, 0, 1, 2, 3, -1,  // 0状态的转移
		-1, -1, -1, 2, 3, -1,
		-1, 8, -1, 2, 4, 5,
		-1, -1, -1, 4, -1, -1,
		-1, 8, -1, 4, -1, 5,
		-1, -1, 6, 7, -1, -1,
		-1, -1, -1, 7, -1, -1,
		-1, 8, -1, 7, -1, -1,
		-1, 8, -1, -1, -1, -1 
	};
	int state = 0;
	while(*s){
		state = map[state][isType(*s)];
		if(state == -1)	return false;
		s++;
	}
	return state == 2 || state == 4 || state == 7 || state == 8;
}
    \end{lstlisting}
    知道NFA转移图后想到像上面这些编写代码也是蛮难的,这样编写的好处是可以灵活的改变NFA,不会对代码构成很大影响,值得学习.
\end{description}

\fi

\ifx allfiles undefined
\end{CJK}
\end{document}
\fi

\subsection{Regular Expression Matching}
\ifx allfiles undefined
\documentclass{article}
\usepackage{CJK}
\usepackage{verbatim}

%%%代码
\usepackage{color}
\usepackage{xcolor}
\definecolor{keywordcolor}{rgb}{0.8,0.1,0.5}
\usepackage{listings}
\lstset{breaklines}%这条命令可以让LaTeX自动将长的代码行换行排版
\lstset{extendedchars=false}%这一条命令可以解决代码跨页时,章节标题,页眉等汉字不显示的问题
\lstset{language=C++, %用于设置语言为C++
    keywordstyle=\color{keywordcolor} \bfseries, %设置关键词
    identifierstyle=,
    basicstyle=\ttfamily, 
    commentstyle=\color{blue} \textit,
    stringstyle=\ttfamily, 
    showstringspaces=false,
    %frame=shadowbox, %边框
    captionpos=b
}
%%%

%\hypersetup{CJKbookmarks=true} %解决section不能使用中文的问题

\begin{document}
\begin{CJK}{UTF8}{gbsn}     %CJK:支持中文

\else
    
\begin{description}
    \item{\textbf{问题}}:\\
	Implement regular expression matching with support for '.' and '*'.\\
	'.' Matches any single character.\\
	'*' Matches zero or more of the preceding element.\\
	The matching should cover the entire input string (not partial).\\
	\textit{(leetcode 10)}
	\item{\textbf{举例}}: \\
	isMatch("aa","a") → false\\
	isMatch("aa","aa") → true\\
	isMatch("aaa","aa") → false\\
	isMatch("aa", "a*") → true\\
	isMatch("aa", ".*") → true\\
	isMatch("ab", ".*") → true\\
	isMatch("aab", "c*a*b") → true\\
    \item{\textbf{DP}} : \fbox{时间复杂度O(n*m), 空间复杂度O(n*m)}
    \\本题使用动态规划求解,设isMatch[i][j]表示s[1...i]与p[1...j]是否匹配\\
	递推关系是如下:

$$ isMatch[i][j]=\left\{
\begin{array}{lcr}
1 $ {\qquad\qquad\qquad\qquad\qquad\qquad\qquad\qquad}$ { \qquad\qquad\qquad\qquad\qquad i=0,j=0 } \\
0 $ {\qquad\qquad\qquad\qquad\qquad\qquad\qquad\qquad}$ { \qquad\qquad\qquad\qquad\qquad i\ne0,j=0 } \\
{isMatch[i-1][j-1]\ \&\&\ (s_{i-1}\ ==\ p_{j-1}\ ||\ p_{j-1}\ ==\ '.')} $ {\qquad}$ {p_j\ne'*'} \\
{isMatch[i][j] = isMatch[i][j-2]\ ||\ isMatch[i-1][j]}$ {\quad} $ { p_j='*',p_{j-1}='.'} \\
{isMatch[i][j-1]} $ {\qquad\qquad\qquad\qquad\qquad\qquad\qquad\qquad} $ { p_j='*',p_{j-1}='*'}\\
{isMatch[i][j-2]\ ||\ (s_{i-1} == p_{j-1}\ \&\&\ isMatch[i-1][j])}$ {\qquad\qquad} $ {other}
\end{array}
\right.
$$
    \begin{lstlisting}
	bool isMatch(const char *s, const char *p) {
		int ls = strlen(s);
		int lp = strlen(p);
		vector<vector<bool> > isMatch(ls+1, vector<bool>(lp+1, false));
		isMatch[0][0] = true;
		int di, dj;
		for(int i = 0; i < ls + 1; i++){ // 起始坐标
			di = i - 1;
			for(int j = 1; j < lp + 1; j++){ // 起始坐标
				dj = j - 1;
				if(p[dj] != '*'){
					isMatch[i][j] = di != -1 && isMatch[i-1][j-1] && (s[di] == p[dj] || p[dj] == '.');
				}else{
					if(dj == 0) { isMatch[i][j] = false; continue; }
					if(p[dj-1] == '.'){
							isMatch[i][j] = isMatch[i][j-2] || (di != -1 && isMatch[i-1][j]);
					}else{
						if(p[dj-1] == '*')	isMatch[i][j] = isMatch[i][j-1];	
						else isMatch[i][j] = isMatch[i][j-2] || (di != -1 && s[di] == p[dj-1] && isMatch[i-1][j]);
					}
				}
			}
		}
		return isMatch[ls][lp];
	}
    \end{lstlisting}
    \textit{}
\end{description}

\fi

\ifx allfiles undefined
\end{CJK}
\end{document}
\fi

\subsection{Wildcard Matching}
\ifx allfiles undefined
\documentclass{article}
\usepackage{CJK}
\usepackage{verbatim}

%%%代码
\usepackage{color}
\usepackage{xcolor}
\definecolor{keywordcolor}{rgb}{0.8,0.1,0.5}
\usepackage{listings}
\lstset{breaklines}%这条命令可以让LaTeX自动将长的代码行换行排版
\lstset{extendedchars=false}%这一条命令可以解决代码跨页时,章节标题,页眉等汉字不显示的问题
\lstset{language=C++, %用于设置语言为C++
    keywordstyle=\color{keywordcolor} \bfseries, %设置关键词
    identifierstyle=,
    basicstyle=\ttfamily, 
    commentstyle=\color{blue} \textit,
    stringstyle=\ttfamily, 
    showstringspaces=false,
    %frame=shadowbox, %边框
    captionpos=b
}
%%%

%\hypersetup{CJKbookmarks=true} %解决section不能使用中文的问题

\begin{document}
\begin{CJK}{UTF8}{gbsn}     %CJK:支持中文

\else
    
\begin{description}
    \item{\textbf{问题}}:\\
	 Implement wildcard pattern matching with support for '?' and '*'. '?'\\
	 Matches any single character. '*' Matches any sequence of characters (including the empty sequence).\\
	 The matching should cover the entire input string (not partial). \textit{(leetcode 44)}
	\item{\textbf{举例}}: \\
	isMatch("aa","a") → false\\
	isMatch("aa","aa") → true\\
	isMatch("aaa","aa") → false\\
	isMatch("aa", "*") → true\\
	isMatch("aa", "a*") → true\\
	isMatch("ab", "?*") → true\\
	isMatch("aab", "c*a*b") → false\\
    \item{\textbf{迭代}} : \fbox{时间复杂度O(n*m) , 空间复杂度O(1)}
    \\这道题其实是可以套用Regular Expression Matching的解法,使用动态规划来计算,但是那样在leetcode上会超时,具体原因我还不清楚. 本题的解法其实也很明了,唯一需要注意的就是每次我们遇到*时候都会记下此次*的位置以及主串位置,然后一次一次的试探*是否要生成字母,试探不成功就回到刚才我们记下的状态然后试探下一个状态,但是,当我们遇到下一个*时候,就可以更新这个记录点了,因为都已经到这个*了,上个*走到这个*走的路肯定是对的,即使不太对,也可以通过这个*不断生成字母来弥补,这就是需要注意的地方,不是很好理解,需要仔细琢磨.
    \begin{lstlisting}
bool isMatch(const char *s, const char *p){
	int ls = strlen(s);
	int lp = strlen(p);
	const char *ps = s, *pp = p, *lasts = NULL, *lastp = NULL;
	if(!s || !p)	return false;
	while(*s){
		switch(*p){
			case '?':	s++; p++; break;
			case '*':	while(*(p+1) && *(p+1) == '*')	p++;
						lasts = s; lastp = p; p++; break;
			default:	if(*s == *p){ s++; p++;}
						else{
							if(lasts){
								s = ++lasts;
								p = lastp + 1;
							}else{
								return false;
							}
						}
		}
	}
	if(*p == '\0' && *s == '\0')	return true;
	if((*p) == '\0' && *(p-1) != '*')	return false;
	while(*p && *p == '*'){
		p++;
	}
	if(*p != '\0')	return false;
	return true;
}
    \end{lstlisting}
\end{description}

\fi

\ifx allfiles undefined
\end{CJK}
\end{document}
\fi


\fi

\ifx allfiles undefined
\end{CJK}
\end{document}
\fi

\newpage

\fi

\ifx allfiles undefined
\end{CJK}
\end{document}
\fi


%\chapter{数组}
%    
\subsection{Spiral Matrix}
    
\begin{description}
    \item{\textbf{问题}}:\\
Given a matrix of m x n elements (m rows, n columns), return all elements of the matrix in spiral order.\\
\textit{(leetcode 54)}
    \item{\textbf{举例}}:\\
Given the following matrix:\\
\\
$[$ \\
 $[ 1, 2, 3 ]$, \\
 $[ 4, 5, 6 ]$, \\
 $[ 7, 8, 9 ]$ \\
$]$ \\
You should return $[1,2,3,6,9,8,7,4,5]$.
    \item{\textbf{???}} : \fbox{时间复杂度O($n^2$), 空间复杂度O(1)}
    \\从外到内一环一环的处理,需要注意一些边界条件
    \begin{lstlisting}
vector<int> spiralOrder(vector<vector<int> > &matrix) {
	vector<int> result;
	int n = matrix.size();
	if(n == 0)	return result;
	int m = matrix[0].size();
	int magrin = 0;
	while(m - 1 - magrin >= magrin && n - 1 - magrin >= magrin){
		for(int j = magrin; j <= m - 1 - magrin; j++)
			result.push_back(matrix[magrin][j]);
		for(int i = magrin + 1; i < n - 1 - magrin; i++)
			result.push_back(matrix[i][m-1-magrin]);
		if(n - 1 - magrin != magrin)
			for(int j = m - 1 - magrin; j >= magrin; j-- )
				result.push_back(matrix[n-1-magrin][j]);
		if(m - 1 - magrin != magrin)
			for(int i = n - 1 - magrin - 1; i > magrin; i--)
				result.push_back(matrix[i][magrin]);
		magrin++;
	}
	return result;
}
    \end{lstlisting}
\end{description}

\subsection{Spiral Matrix II}
    
\begin{description}
    \item{\textbf{问题}}:\\
Given an integer n, generate a square matrix filled with elements from 1 to $n^2$ in spiral order.\\
\textit{(leetcode 59)}
    \item{\textbf{举例}}:\\
Given n = 3,\\
\\
You should return the following matrix:\\
$[$ \\
 $[ 1, 2, 3 ]$, \\
 $[ 8, 9, 4 ]$, \\
 $[ 7, 6, 5 ]$ \\
$]$
    \item{\textbf{???}} : \fbox{时间复杂度O($n^2$), 空间复杂度O(1)}
    \begin{lstlisting}
vector<vector<int> > generateMatrix(int n) {
	vector<vector<int> > matrix(n, vector<int>(n, 0));
	int pos = 1;
	int magrin = 0;
	while(n - 1 - magrin >= magrin && n - 1 - magrin >= magrin){
		for(int j = magrin; j <= n - 1 - magrin; j++)
			matrix[magrin][j] = pos++;
		for(int i = magrin + 1; i < n - 1 - magrin; i++)
			matrix[i][n-1-magrin] = pos++;
		if(n - 1 - magrin != magrin)
			for(int j = n - 1 - magrin; j >= magrin; j-- )
				matrix[n-1-magrin][j] = pos++;
		if(n - 1 - magrin != magrin)
			for(int i = n - 1 - magrin - 1; i > magrin; i--)
				matrix[i][magrin] = pos++;
		magrin++;
	}
	return matrix;
}
    \end{lstlisting}
\end{description}

\subsection{Set Matrix Zeroes}
    
\begin{description}
    \item{\textbf{问题}}:\\
Given a m x n matrix, if an element is 0, set its entire row and column to 0. Do it in place.\\
\textit{(leetcode 73)}
    \item{\textbf{Follow Up}}:\\
Did you use extra space? \\
A straight forward solution using O(mn) space is probably a bad idea. \\
A simple improvement uses O(m + n) space, but still not the best solution. \\
Could you devise a constant space solution?
    \item{\textbf{???}} : \fbox{时间复杂度O($n^2$), 空间复杂度O(1)}
    \\从上到下一行一行的处理,如果上一个存在0,可以先保留上一行的现场,然后根据上一行的原来值更新本行,然后处理上一行.唯一需要注意的就是,如果本行出现新0需要更新该0所在列的上面所有行.
    \begin{lstlisting}
void setZeroes(vector<vector<int> > &matrix) {
	int n = matrix.size();
	if(n == 0)	return;
	int m = matrix[0].size();
	bool lastZero = false;
	for(int i = 0; i < n; i++){
		bool thisZero = false;
		for(int j = 0; j < m; j++){
			if(matrix[i][j] == 0){
				int up = i - 1;
				while(up >= 0){
					matrix[up][j] = 0;
					up--;
				}
				thisZero = true;
			}
			if(i > 0 && matrix[i-1][j] == 0)
				matrix[i][j] = 0;
		}
		if(lastZero){
			for(int j = 0; j < m; j++)
				matrix[i-1][j] = 0;
		}
		lastZero = thisZero;
	}
	if(lastZero){
		for(int j = 0; j < m; j++)
			matrix[n-1][j] = 0;
	}
}
    \end{lstlisting}
\end{description}

\subsection{Pascal's Triangle}
    
\begin{description}
    \item{\textbf{问题}}:\\
Given numRows, generate the first numRows of Pascal's triangle. \\
\textit{(leetcode 118)}
    \item{\textbf{举例}}:\\
Given numRows = 5, \\
Return \\
\\
$[$ \\
     $[1]$, \\
    $[1,1]$, \\
   $[1,2,1]$, \\
  $[1,3,3,1]$, \\
 $[1,4,6,4,1]$ \\
$]$
    \item{\textbf{???}} : \fbox{时间复杂度O($n^2$), 空间复杂度O(1)}
    \begin{lstlisting}
vector<vector<int> > generate(int numRows) {
	vector<vector<int> > result;
	if(numRows == 0)	return result;
	result.push_back(vector<int>{1});
	for(int i = 1; i < numRows; i++){
		vector<int> cur;
		cur.push_back(1);
		for(int j = 0; j < result[i-1].size() - 1; j++)
			cur.push_back(result[i-1][j] + result[i-1][j+1]);
		cur.push_back(1);
		result.push_back(cur);
	}
	return result;
}
    \end{lstlisting}
\end{description}

\subsection{Pascal's Triangle II}
    
\begin{description}
    \item{\textbf{问题}}:\\
Given an index k, return the kth row of the Pascal's triangle. \\
\textit{(leetcode 119)}
    \item{\textbf{举例}}:\\
Given k = 3, \\
Return $[1,3,3,1]$.
    \item{\textbf{Note}}:\\
Could you optimize your algorithm to use only O(k) extra space?
    \item{\textbf{???}} : \fbox{时间复杂度O($k^2$), 空间复杂度O(k)}
    \\这个空间复杂度可以优化,因为每次我们只需要上一行就可以产生本行,所有之前的行可以不存储
    \begin{lstlisting}
	vector<int> getRow(int rowIndex) {
		rowIndex++;
		if(rowIndex <= 0)	return vector<int>();
		vector<int> result{1};
		for(int i = 1; i < rowIndex; i++){
			vector<int> cur;
			cur.push_back(1);
			for(int j = 0; j < result.size() - 1; j++)
				cur.push_back(result[j] + result[j+1]);
			cur.push_back(1);
			result.swap(cur);
		}
		return result;
	}
    \end{lstlisting}
\end{description}



\chapter{栈}
    
\subsection{Spiral Matrix}
    
\begin{description}
    \item{\textbf{问题}}:\\
Given a matrix of m x n elements (m rows, n columns), return all elements of the matrix in spiral order.\\
\textit{(leetcode 54)}
    \item{\textbf{举例}}:\\
Given the following matrix:\\
\\
$[$ \\
 $[ 1, 2, 3 ]$, \\
 $[ 4, 5, 6 ]$, \\
 $[ 7, 8, 9 ]$ \\
$]$ \\
You should return $[1,2,3,6,9,8,7,4,5]$.
    \item{\textbf{???}} : \fbox{时间复杂度O($n^2$), 空间复杂度O(1)}
    \\从外到内一环一环的处理,需要注意一些边界条件
    \begin{lstlisting}
vector<int> spiralOrder(vector<vector<int> > &matrix) {
	vector<int> result;
	int n = matrix.size();
	if(n == 0)	return result;
	int m = matrix[0].size();
	int magrin = 0;
	while(m - 1 - magrin >= magrin && n - 1 - magrin >= magrin){
		for(int j = magrin; j <= m - 1 - magrin; j++)
			result.push_back(matrix[magrin][j]);
		for(int i = magrin + 1; i < n - 1 - magrin; i++)
			result.push_back(matrix[i][m-1-magrin]);
		if(n - 1 - magrin != magrin)
			for(int j = m - 1 - magrin; j >= magrin; j-- )
				result.push_back(matrix[n-1-magrin][j]);
		if(m - 1 - magrin != magrin)
			for(int i = n - 1 - magrin - 1; i > magrin; i--)
				result.push_back(matrix[i][magrin]);
		magrin++;
	}
	return result;
}
    \end{lstlisting}
\end{description}

\subsection{Spiral Matrix II}
    
\begin{description}
    \item{\textbf{问题}}:\\
Given an integer n, generate a square matrix filled with elements from 1 to $n^2$ in spiral order.\\
\textit{(leetcode 59)}
    \item{\textbf{举例}}:\\
Given n = 3,\\
\\
You should return the following matrix:\\
$[$ \\
 $[ 1, 2, 3 ]$, \\
 $[ 8, 9, 4 ]$, \\
 $[ 7, 6, 5 ]$ \\
$]$
    \item{\textbf{???}} : \fbox{时间复杂度O($n^2$), 空间复杂度O(1)}
    \begin{lstlisting}
vector<vector<int> > generateMatrix(int n) {
	vector<vector<int> > matrix(n, vector<int>(n, 0));
	int pos = 1;
	int magrin = 0;
	while(n - 1 - magrin >= magrin && n - 1 - magrin >= magrin){
		for(int j = magrin; j <= n - 1 - magrin; j++)
			matrix[magrin][j] = pos++;
		for(int i = magrin + 1; i < n - 1 - magrin; i++)
			matrix[i][n-1-magrin] = pos++;
		if(n - 1 - magrin != magrin)
			for(int j = n - 1 - magrin; j >= magrin; j-- )
				matrix[n-1-magrin][j] = pos++;
		if(n - 1 - magrin != magrin)
			for(int i = n - 1 - magrin - 1; i > magrin; i--)
				matrix[i][magrin] = pos++;
		magrin++;
	}
	return matrix;
}
    \end{lstlisting}
\end{description}

\subsection{Set Matrix Zeroes}
    
\begin{description}
    \item{\textbf{问题}}:\\
Given a m x n matrix, if an element is 0, set its entire row and column to 0. Do it in place.\\
\textit{(leetcode 73)}
    \item{\textbf{Follow Up}}:\\
Did you use extra space? \\
A straight forward solution using O(mn) space is probably a bad idea. \\
A simple improvement uses O(m + n) space, but still not the best solution. \\
Could you devise a constant space solution?
    \item{\textbf{???}} : \fbox{时间复杂度O($n^2$), 空间复杂度O(1)}
    \\从上到下一行一行的处理,如果上一个存在0,可以先保留上一行的现场,然后根据上一行的原来值更新本行,然后处理上一行.唯一需要注意的就是,如果本行出现新0需要更新该0所在列的上面所有行.
    \begin{lstlisting}
void setZeroes(vector<vector<int> > &matrix) {
	int n = matrix.size();
	if(n == 0)	return;
	int m = matrix[0].size();
	bool lastZero = false;
	for(int i = 0; i < n; i++){
		bool thisZero = false;
		for(int j = 0; j < m; j++){
			if(matrix[i][j] == 0){
				int up = i - 1;
				while(up >= 0){
					matrix[up][j] = 0;
					up--;
				}
				thisZero = true;
			}
			if(i > 0 && matrix[i-1][j] == 0)
				matrix[i][j] = 0;
		}
		if(lastZero){
			for(int j = 0; j < m; j++)
				matrix[i-1][j] = 0;
		}
		lastZero = thisZero;
	}
	if(lastZero){
		for(int j = 0; j < m; j++)
			matrix[n-1][j] = 0;
	}
}
    \end{lstlisting}
\end{description}

\subsection{Pascal's Triangle}
    
\begin{description}
    \item{\textbf{问题}}:\\
Given numRows, generate the first numRows of Pascal's triangle. \\
\textit{(leetcode 118)}
    \item{\textbf{举例}}:\\
Given numRows = 5, \\
Return \\
\\
$[$ \\
     $[1]$, \\
    $[1,1]$, \\
   $[1,2,1]$, \\
  $[1,3,3,1]$, \\
 $[1,4,6,4,1]$ \\
$]$
    \item{\textbf{???}} : \fbox{时间复杂度O($n^2$), 空间复杂度O(1)}
    \begin{lstlisting}
vector<vector<int> > generate(int numRows) {
	vector<vector<int> > result;
	if(numRows == 0)	return result;
	result.push_back(vector<int>{1});
	for(int i = 1; i < numRows; i++){
		vector<int> cur;
		cur.push_back(1);
		for(int j = 0; j < result[i-1].size() - 1; j++)
			cur.push_back(result[i-1][j] + result[i-1][j+1]);
		cur.push_back(1);
		result.push_back(cur);
	}
	return result;
}
    \end{lstlisting}
\end{description}

\subsection{Pascal's Triangle II}
    
\begin{description}
    \item{\textbf{问题}}:\\
Given an index k, return the kth row of the Pascal's triangle. \\
\textit{(leetcode 119)}
    \item{\textbf{举例}}:\\
Given k = 3, \\
Return $[1,3,3,1]$.
    \item{\textbf{Note}}:\\
Could you optimize your algorithm to use only O(k) extra space?
    \item{\textbf{???}} : \fbox{时间复杂度O($k^2$), 空间复杂度O(k)}
    \\这个空间复杂度可以优化,因为每次我们只需要上一行就可以产生本行,所有之前的行可以不存储
    \begin{lstlisting}
	vector<int> getRow(int rowIndex) {
		rowIndex++;
		if(rowIndex <= 0)	return vector<int>();
		vector<int> result{1};
		for(int i = 1; i < rowIndex; i++){
			vector<int> cur;
			cur.push_back(1);
			for(int j = 0; j < result.size() - 1; j++)
				cur.push_back(result[j] + result[j+1]);
			cur.push_back(1);
			result.swap(cur);
		}
		return result;
	}
    \end{lstlisting}
\end{description}



\chapter{图}
    
\subsection{Spiral Matrix}
    
\begin{description}
    \item{\textbf{问题}}:\\
Given a matrix of m x n elements (m rows, n columns), return all elements of the matrix in spiral order.\\
\textit{(leetcode 54)}
    \item{\textbf{举例}}:\\
Given the following matrix:\\
\\
$[$ \\
 $[ 1, 2, 3 ]$, \\
 $[ 4, 5, 6 ]$, \\
 $[ 7, 8, 9 ]$ \\
$]$ \\
You should return $[1,2,3,6,9,8,7,4,5]$.
    \item{\textbf{???}} : \fbox{时间复杂度O($n^2$), 空间复杂度O(1)}
    \\从外到内一环一环的处理,需要注意一些边界条件
    \begin{lstlisting}
vector<int> spiralOrder(vector<vector<int> > &matrix) {
	vector<int> result;
	int n = matrix.size();
	if(n == 0)	return result;
	int m = matrix[0].size();
	int magrin = 0;
	while(m - 1 - magrin >= magrin && n - 1 - magrin >= magrin){
		for(int j = magrin; j <= m - 1 - magrin; j++)
			result.push_back(matrix[magrin][j]);
		for(int i = magrin + 1; i < n - 1 - magrin; i++)
			result.push_back(matrix[i][m-1-magrin]);
		if(n - 1 - magrin != magrin)
			for(int j = m - 1 - magrin; j >= magrin; j-- )
				result.push_back(matrix[n-1-magrin][j]);
		if(m - 1 - magrin != magrin)
			for(int i = n - 1 - magrin - 1; i > magrin; i--)
				result.push_back(matrix[i][magrin]);
		magrin++;
	}
	return result;
}
    \end{lstlisting}
\end{description}

\subsection{Spiral Matrix II}
    
\begin{description}
    \item{\textbf{问题}}:\\
Given an integer n, generate a square matrix filled with elements from 1 to $n^2$ in spiral order.\\
\textit{(leetcode 59)}
    \item{\textbf{举例}}:\\
Given n = 3,\\
\\
You should return the following matrix:\\
$[$ \\
 $[ 1, 2, 3 ]$, \\
 $[ 8, 9, 4 ]$, \\
 $[ 7, 6, 5 ]$ \\
$]$
    \item{\textbf{???}} : \fbox{时间复杂度O($n^2$), 空间复杂度O(1)}
    \begin{lstlisting}
vector<vector<int> > generateMatrix(int n) {
	vector<vector<int> > matrix(n, vector<int>(n, 0));
	int pos = 1;
	int magrin = 0;
	while(n - 1 - magrin >= magrin && n - 1 - magrin >= magrin){
		for(int j = magrin; j <= n - 1 - magrin; j++)
			matrix[magrin][j] = pos++;
		for(int i = magrin + 1; i < n - 1 - magrin; i++)
			matrix[i][n-1-magrin] = pos++;
		if(n - 1 - magrin != magrin)
			for(int j = n - 1 - magrin; j >= magrin; j-- )
				matrix[n-1-magrin][j] = pos++;
		if(n - 1 - magrin != magrin)
			for(int i = n - 1 - magrin - 1; i > magrin; i--)
				matrix[i][magrin] = pos++;
		magrin++;
	}
	return matrix;
}
    \end{lstlisting}
\end{description}

\subsection{Set Matrix Zeroes}
    
\begin{description}
    \item{\textbf{问题}}:\\
Given a m x n matrix, if an element is 0, set its entire row and column to 0. Do it in place.\\
\textit{(leetcode 73)}
    \item{\textbf{Follow Up}}:\\
Did you use extra space? \\
A straight forward solution using O(mn) space is probably a bad idea. \\
A simple improvement uses O(m + n) space, but still not the best solution. \\
Could you devise a constant space solution?
    \item{\textbf{???}} : \fbox{时间复杂度O($n^2$), 空间复杂度O(1)}
    \\从上到下一行一行的处理,如果上一个存在0,可以先保留上一行的现场,然后根据上一行的原来值更新本行,然后处理上一行.唯一需要注意的就是,如果本行出现新0需要更新该0所在列的上面所有行.
    \begin{lstlisting}
void setZeroes(vector<vector<int> > &matrix) {
	int n = matrix.size();
	if(n == 0)	return;
	int m = matrix[0].size();
	bool lastZero = false;
	for(int i = 0; i < n; i++){
		bool thisZero = false;
		for(int j = 0; j < m; j++){
			if(matrix[i][j] == 0){
				int up = i - 1;
				while(up >= 0){
					matrix[up][j] = 0;
					up--;
				}
				thisZero = true;
			}
			if(i > 0 && matrix[i-1][j] == 0)
				matrix[i][j] = 0;
		}
		if(lastZero){
			for(int j = 0; j < m; j++)
				matrix[i-1][j] = 0;
		}
		lastZero = thisZero;
	}
	if(lastZero){
		for(int j = 0; j < m; j++)
			matrix[n-1][j] = 0;
	}
}
    \end{lstlisting}
\end{description}

\subsection{Pascal's Triangle}
    
\begin{description}
    \item{\textbf{问题}}:\\
Given numRows, generate the first numRows of Pascal's triangle. \\
\textit{(leetcode 118)}
    \item{\textbf{举例}}:\\
Given numRows = 5, \\
Return \\
\\
$[$ \\
     $[1]$, \\
    $[1,1]$, \\
   $[1,2,1]$, \\
  $[1,3,3,1]$, \\
 $[1,4,6,4,1]$ \\
$]$
    \item{\textbf{???}} : \fbox{时间复杂度O($n^2$), 空间复杂度O(1)}
    \begin{lstlisting}
vector<vector<int> > generate(int numRows) {
	vector<vector<int> > result;
	if(numRows == 0)	return result;
	result.push_back(vector<int>{1});
	for(int i = 1; i < numRows; i++){
		vector<int> cur;
		cur.push_back(1);
		for(int j = 0; j < result[i-1].size() - 1; j++)
			cur.push_back(result[i-1][j] + result[i-1][j+1]);
		cur.push_back(1);
		result.push_back(cur);
	}
	return result;
}
    \end{lstlisting}
\end{description}

\subsection{Pascal's Triangle II}
    
\begin{description}
    \item{\textbf{问题}}:\\
Given an index k, return the kth row of the Pascal's triangle. \\
\textit{(leetcode 119)}
    \item{\textbf{举例}}:\\
Given k = 3, \\
Return $[1,3,3,1]$.
    \item{\textbf{Note}}:\\
Could you optimize your algorithm to use only O(k) extra space?
    \item{\textbf{???}} : \fbox{时间复杂度O($k^2$), 空间复杂度O(k)}
    \\这个空间复杂度可以优化,因为每次我们只需要上一行就可以产生本行,所有之前的行可以不存储
    \begin{lstlisting}
	vector<int> getRow(int rowIndex) {
		rowIndex++;
		if(rowIndex <= 0)	return vector<int>();
		vector<int> result{1};
		for(int i = 1; i < rowIndex; i++){
			vector<int> cur;
			cur.push_back(1);
			for(int j = 0; j < result.size() - 1; j++)
				cur.push_back(result[j] + result[j+1]);
			cur.push_back(1);
			result.swap(cur);
		}
		return result;
	}
    \end{lstlisting}
\end{description}



%\chapter{哈希}
%    
\subsection{Spiral Matrix}
    
\begin{description}
    \item{\textbf{问题}}:\\
Given a matrix of m x n elements (m rows, n columns), return all elements of the matrix in spiral order.\\
\textit{(leetcode 54)}
    \item{\textbf{举例}}:\\
Given the following matrix:\\
\\
$[$ \\
 $[ 1, 2, 3 ]$, \\
 $[ 4, 5, 6 ]$, \\
 $[ 7, 8, 9 ]$ \\
$]$ \\
You should return $[1,2,3,6,9,8,7,4,5]$.
    \item{\textbf{???}} : \fbox{时间复杂度O($n^2$), 空间复杂度O(1)}
    \\从外到内一环一环的处理,需要注意一些边界条件
    \begin{lstlisting}
vector<int> spiralOrder(vector<vector<int> > &matrix) {
	vector<int> result;
	int n = matrix.size();
	if(n == 0)	return result;
	int m = matrix[0].size();
	int magrin = 0;
	while(m - 1 - magrin >= magrin && n - 1 - magrin >= magrin){
		for(int j = magrin; j <= m - 1 - magrin; j++)
			result.push_back(matrix[magrin][j]);
		for(int i = magrin + 1; i < n - 1 - magrin; i++)
			result.push_back(matrix[i][m-1-magrin]);
		if(n - 1 - magrin != magrin)
			for(int j = m - 1 - magrin; j >= magrin; j-- )
				result.push_back(matrix[n-1-magrin][j]);
		if(m - 1 - magrin != magrin)
			for(int i = n - 1 - magrin - 1; i > magrin; i--)
				result.push_back(matrix[i][magrin]);
		magrin++;
	}
	return result;
}
    \end{lstlisting}
\end{description}

\subsection{Spiral Matrix II}
    
\begin{description}
    \item{\textbf{问题}}:\\
Given an integer n, generate a square matrix filled with elements from 1 to $n^2$ in spiral order.\\
\textit{(leetcode 59)}
    \item{\textbf{举例}}:\\
Given n = 3,\\
\\
You should return the following matrix:\\
$[$ \\
 $[ 1, 2, 3 ]$, \\
 $[ 8, 9, 4 ]$, \\
 $[ 7, 6, 5 ]$ \\
$]$
    \item{\textbf{???}} : \fbox{时间复杂度O($n^2$), 空间复杂度O(1)}
    \begin{lstlisting}
vector<vector<int> > generateMatrix(int n) {
	vector<vector<int> > matrix(n, vector<int>(n, 0));
	int pos = 1;
	int magrin = 0;
	while(n - 1 - magrin >= magrin && n - 1 - magrin >= magrin){
		for(int j = magrin; j <= n - 1 - magrin; j++)
			matrix[magrin][j] = pos++;
		for(int i = magrin + 1; i < n - 1 - magrin; i++)
			matrix[i][n-1-magrin] = pos++;
		if(n - 1 - magrin != magrin)
			for(int j = n - 1 - magrin; j >= magrin; j-- )
				matrix[n-1-magrin][j] = pos++;
		if(n - 1 - magrin != magrin)
			for(int i = n - 1 - magrin - 1; i > magrin; i--)
				matrix[i][magrin] = pos++;
		magrin++;
	}
	return matrix;
}
    \end{lstlisting}
\end{description}

\subsection{Set Matrix Zeroes}
    
\begin{description}
    \item{\textbf{问题}}:\\
Given a m x n matrix, if an element is 0, set its entire row and column to 0. Do it in place.\\
\textit{(leetcode 73)}
    \item{\textbf{Follow Up}}:\\
Did you use extra space? \\
A straight forward solution using O(mn) space is probably a bad idea. \\
A simple improvement uses O(m + n) space, but still not the best solution. \\
Could you devise a constant space solution?
    \item{\textbf{???}} : \fbox{时间复杂度O($n^2$), 空间复杂度O(1)}
    \\从上到下一行一行的处理,如果上一个存在0,可以先保留上一行的现场,然后根据上一行的原来值更新本行,然后处理上一行.唯一需要注意的就是,如果本行出现新0需要更新该0所在列的上面所有行.
    \begin{lstlisting}
void setZeroes(vector<vector<int> > &matrix) {
	int n = matrix.size();
	if(n == 0)	return;
	int m = matrix[0].size();
	bool lastZero = false;
	for(int i = 0; i < n; i++){
		bool thisZero = false;
		for(int j = 0; j < m; j++){
			if(matrix[i][j] == 0){
				int up = i - 1;
				while(up >= 0){
					matrix[up][j] = 0;
					up--;
				}
				thisZero = true;
			}
			if(i > 0 && matrix[i-1][j] == 0)
				matrix[i][j] = 0;
		}
		if(lastZero){
			for(int j = 0; j < m; j++)
				matrix[i-1][j] = 0;
		}
		lastZero = thisZero;
	}
	if(lastZero){
		for(int j = 0; j < m; j++)
			matrix[n-1][j] = 0;
	}
}
    \end{lstlisting}
\end{description}

\subsection{Pascal's Triangle}
    
\begin{description}
    \item{\textbf{问题}}:\\
Given numRows, generate the first numRows of Pascal's triangle. \\
\textit{(leetcode 118)}
    \item{\textbf{举例}}:\\
Given numRows = 5, \\
Return \\
\\
$[$ \\
     $[1]$, \\
    $[1,1]$, \\
   $[1,2,1]$, \\
  $[1,3,3,1]$, \\
 $[1,4,6,4,1]$ \\
$]$
    \item{\textbf{???}} : \fbox{时间复杂度O($n^2$), 空间复杂度O(1)}
    \begin{lstlisting}
vector<vector<int> > generate(int numRows) {
	vector<vector<int> > result;
	if(numRows == 0)	return result;
	result.push_back(vector<int>{1});
	for(int i = 1; i < numRows; i++){
		vector<int> cur;
		cur.push_back(1);
		for(int j = 0; j < result[i-1].size() - 1; j++)
			cur.push_back(result[i-1][j] + result[i-1][j+1]);
		cur.push_back(1);
		result.push_back(cur);
	}
	return result;
}
    \end{lstlisting}
\end{description}

\subsection{Pascal's Triangle II}
    
\begin{description}
    \item{\textbf{问题}}:\\
Given an index k, return the kth row of the Pascal's triangle. \\
\textit{(leetcode 119)}
    \item{\textbf{举例}}:\\
Given k = 3, \\
Return $[1,3,3,1]$.
    \item{\textbf{Note}}:\\
Could you optimize your algorithm to use only O(k) extra space?
    \item{\textbf{???}} : \fbox{时间复杂度O($k^2$), 空间复杂度O(k)}
    \\这个空间复杂度可以优化,因为每次我们只需要上一行就可以产生本行,所有之前的行可以不存储
    \begin{lstlisting}
	vector<int> getRow(int rowIndex) {
		rowIndex++;
		if(rowIndex <= 0)	return vector<int>();
		vector<int> result{1};
		for(int i = 1; i < rowIndex; i++){
			vector<int> cur;
			cur.push_back(1);
			for(int j = 0; j < result.size() - 1; j++)
				cur.push_back(result[j] + result[j+1]);
			cur.push_back(1);
			result.swap(cur);
		}
		return result;
	}
    \end{lstlisting}
\end{description}



\chapter{其他数据结构}
    
\subsection{Spiral Matrix}
    
\begin{description}
    \item{\textbf{问题}}:\\
Given a matrix of m x n elements (m rows, n columns), return all elements of the matrix in spiral order.\\
\textit{(leetcode 54)}
    \item{\textbf{举例}}:\\
Given the following matrix:\\
\\
$[$ \\
 $[ 1, 2, 3 ]$, \\
 $[ 4, 5, 6 ]$, \\
 $[ 7, 8, 9 ]$ \\
$]$ \\
You should return $[1,2,3,6,9,8,7,4,5]$.
    \item{\textbf{???}} : \fbox{时间复杂度O($n^2$), 空间复杂度O(1)}
    \\从外到内一环一环的处理,需要注意一些边界条件
    \begin{lstlisting}
vector<int> spiralOrder(vector<vector<int> > &matrix) {
	vector<int> result;
	int n = matrix.size();
	if(n == 0)	return result;
	int m = matrix[0].size();
	int magrin = 0;
	while(m - 1 - magrin >= magrin && n - 1 - magrin >= magrin){
		for(int j = magrin; j <= m - 1 - magrin; j++)
			result.push_back(matrix[magrin][j]);
		for(int i = magrin + 1; i < n - 1 - magrin; i++)
			result.push_back(matrix[i][m-1-magrin]);
		if(n - 1 - magrin != magrin)
			for(int j = m - 1 - magrin; j >= magrin; j-- )
				result.push_back(matrix[n-1-magrin][j]);
		if(m - 1 - magrin != magrin)
			for(int i = n - 1 - magrin - 1; i > magrin; i--)
				result.push_back(matrix[i][magrin]);
		magrin++;
	}
	return result;
}
    \end{lstlisting}
\end{description}

\subsection{Spiral Matrix II}
    
\begin{description}
    \item{\textbf{问题}}:\\
Given an integer n, generate a square matrix filled with elements from 1 to $n^2$ in spiral order.\\
\textit{(leetcode 59)}
    \item{\textbf{举例}}:\\
Given n = 3,\\
\\
You should return the following matrix:\\
$[$ \\
 $[ 1, 2, 3 ]$, \\
 $[ 8, 9, 4 ]$, \\
 $[ 7, 6, 5 ]$ \\
$]$
    \item{\textbf{???}} : \fbox{时间复杂度O($n^2$), 空间复杂度O(1)}
    \begin{lstlisting}
vector<vector<int> > generateMatrix(int n) {
	vector<vector<int> > matrix(n, vector<int>(n, 0));
	int pos = 1;
	int magrin = 0;
	while(n - 1 - magrin >= magrin && n - 1 - magrin >= magrin){
		for(int j = magrin; j <= n - 1 - magrin; j++)
			matrix[magrin][j] = pos++;
		for(int i = magrin + 1; i < n - 1 - magrin; i++)
			matrix[i][n-1-magrin] = pos++;
		if(n - 1 - magrin != magrin)
			for(int j = n - 1 - magrin; j >= magrin; j-- )
				matrix[n-1-magrin][j] = pos++;
		if(n - 1 - magrin != magrin)
			for(int i = n - 1 - magrin - 1; i > magrin; i--)
				matrix[i][magrin] = pos++;
		magrin++;
	}
	return matrix;
}
    \end{lstlisting}
\end{description}

\subsection{Set Matrix Zeroes}
    
\begin{description}
    \item{\textbf{问题}}:\\
Given a m x n matrix, if an element is 0, set its entire row and column to 0. Do it in place.\\
\textit{(leetcode 73)}
    \item{\textbf{Follow Up}}:\\
Did you use extra space? \\
A straight forward solution using O(mn) space is probably a bad idea. \\
A simple improvement uses O(m + n) space, but still not the best solution. \\
Could you devise a constant space solution?
    \item{\textbf{???}} : \fbox{时间复杂度O($n^2$), 空间复杂度O(1)}
    \\从上到下一行一行的处理,如果上一个存在0,可以先保留上一行的现场,然后根据上一行的原来值更新本行,然后处理上一行.唯一需要注意的就是,如果本行出现新0需要更新该0所在列的上面所有行.
    \begin{lstlisting}
void setZeroes(vector<vector<int> > &matrix) {
	int n = matrix.size();
	if(n == 0)	return;
	int m = matrix[0].size();
	bool lastZero = false;
	for(int i = 0; i < n; i++){
		bool thisZero = false;
		for(int j = 0; j < m; j++){
			if(matrix[i][j] == 0){
				int up = i - 1;
				while(up >= 0){
					matrix[up][j] = 0;
					up--;
				}
				thisZero = true;
			}
			if(i > 0 && matrix[i-1][j] == 0)
				matrix[i][j] = 0;
		}
		if(lastZero){
			for(int j = 0; j < m; j++)
				matrix[i-1][j] = 0;
		}
		lastZero = thisZero;
	}
	if(lastZero){
		for(int j = 0; j < m; j++)
			matrix[n-1][j] = 0;
	}
}
    \end{lstlisting}
\end{description}

\subsection{Pascal's Triangle}
    
\begin{description}
    \item{\textbf{问题}}:\\
Given numRows, generate the first numRows of Pascal's triangle. \\
\textit{(leetcode 118)}
    \item{\textbf{举例}}:\\
Given numRows = 5, \\
Return \\
\\
$[$ \\
     $[1]$, \\
    $[1,1]$, \\
   $[1,2,1]$, \\
  $[1,3,3,1]$, \\
 $[1,4,6,4,1]$ \\
$]$
    \item{\textbf{???}} : \fbox{时间复杂度O($n^2$), 空间复杂度O(1)}
    \begin{lstlisting}
vector<vector<int> > generate(int numRows) {
	vector<vector<int> > result;
	if(numRows == 0)	return result;
	result.push_back(vector<int>{1});
	for(int i = 1; i < numRows; i++){
		vector<int> cur;
		cur.push_back(1);
		for(int j = 0; j < result[i-1].size() - 1; j++)
			cur.push_back(result[i-1][j] + result[i-1][j+1]);
		cur.push_back(1);
		result.push_back(cur);
	}
	return result;
}
    \end{lstlisting}
\end{description}

\subsection{Pascal's Triangle II}
    
\begin{description}
    \item{\textbf{问题}}:\\
Given an index k, return the kth row of the Pascal's triangle. \\
\textit{(leetcode 119)}
    \item{\textbf{举例}}:\\
Given k = 3, \\
Return $[1,3,3,1]$.
    \item{\textbf{Note}}:\\
Could you optimize your algorithm to use only O(k) extra space?
    \item{\textbf{???}} : \fbox{时间复杂度O($k^2$), 空间复杂度O(k)}
    \\这个空间复杂度可以优化,因为每次我们只需要上一行就可以产生本行,所有之前的行可以不存储
    \begin{lstlisting}
	vector<int> getRow(int rowIndex) {
		rowIndex++;
		if(rowIndex <= 0)	return vector<int>();
		vector<int> result{1};
		for(int i = 1; i < rowIndex; i++){
			vector<int> cur;
			cur.push_back(1);
			for(int j = 0; j < result.size() - 1; j++)
				cur.push_back(result[j] + result[j+1]);
			cur.push_back(1);
			result.swap(cur);
		}
		return result;
	}
    \end{lstlisting}
\end{description}



\chapter{排序}
    
\subsection{Spiral Matrix}
    
\begin{description}
    \item{\textbf{问题}}:\\
Given a matrix of m x n elements (m rows, n columns), return all elements of the matrix in spiral order.\\
\textit{(leetcode 54)}
    \item{\textbf{举例}}:\\
Given the following matrix:\\
\\
$[$ \\
 $[ 1, 2, 3 ]$, \\
 $[ 4, 5, 6 ]$, \\
 $[ 7, 8, 9 ]$ \\
$]$ \\
You should return $[1,2,3,6,9,8,7,4,5]$.
    \item{\textbf{???}} : \fbox{时间复杂度O($n^2$), 空间复杂度O(1)}
    \\从外到内一环一环的处理,需要注意一些边界条件
    \begin{lstlisting}
vector<int> spiralOrder(vector<vector<int> > &matrix) {
	vector<int> result;
	int n = matrix.size();
	if(n == 0)	return result;
	int m = matrix[0].size();
	int magrin = 0;
	while(m - 1 - magrin >= magrin && n - 1 - magrin >= magrin){
		for(int j = magrin; j <= m - 1 - magrin; j++)
			result.push_back(matrix[magrin][j]);
		for(int i = magrin + 1; i < n - 1 - magrin; i++)
			result.push_back(matrix[i][m-1-magrin]);
		if(n - 1 - magrin != magrin)
			for(int j = m - 1 - magrin; j >= magrin; j-- )
				result.push_back(matrix[n-1-magrin][j]);
		if(m - 1 - magrin != magrin)
			for(int i = n - 1 - magrin - 1; i > magrin; i--)
				result.push_back(matrix[i][magrin]);
		magrin++;
	}
	return result;
}
    \end{lstlisting}
\end{description}

\subsection{Spiral Matrix II}
    
\begin{description}
    \item{\textbf{问题}}:\\
Given an integer n, generate a square matrix filled with elements from 1 to $n^2$ in spiral order.\\
\textit{(leetcode 59)}
    \item{\textbf{举例}}:\\
Given n = 3,\\
\\
You should return the following matrix:\\
$[$ \\
 $[ 1, 2, 3 ]$, \\
 $[ 8, 9, 4 ]$, \\
 $[ 7, 6, 5 ]$ \\
$]$
    \item{\textbf{???}} : \fbox{时间复杂度O($n^2$), 空间复杂度O(1)}
    \begin{lstlisting}
vector<vector<int> > generateMatrix(int n) {
	vector<vector<int> > matrix(n, vector<int>(n, 0));
	int pos = 1;
	int magrin = 0;
	while(n - 1 - magrin >= magrin && n - 1 - magrin >= magrin){
		for(int j = magrin; j <= n - 1 - magrin; j++)
			matrix[magrin][j] = pos++;
		for(int i = magrin + 1; i < n - 1 - magrin; i++)
			matrix[i][n-1-magrin] = pos++;
		if(n - 1 - magrin != magrin)
			for(int j = n - 1 - magrin; j >= magrin; j-- )
				matrix[n-1-magrin][j] = pos++;
		if(n - 1 - magrin != magrin)
			for(int i = n - 1 - magrin - 1; i > magrin; i--)
				matrix[i][magrin] = pos++;
		magrin++;
	}
	return matrix;
}
    \end{lstlisting}
\end{description}

\subsection{Set Matrix Zeroes}
    
\begin{description}
    \item{\textbf{问题}}:\\
Given a m x n matrix, if an element is 0, set its entire row and column to 0. Do it in place.\\
\textit{(leetcode 73)}
    \item{\textbf{Follow Up}}:\\
Did you use extra space? \\
A straight forward solution using O(mn) space is probably a bad idea. \\
A simple improvement uses O(m + n) space, but still not the best solution. \\
Could you devise a constant space solution?
    \item{\textbf{???}} : \fbox{时间复杂度O($n^2$), 空间复杂度O(1)}
    \\从上到下一行一行的处理,如果上一个存在0,可以先保留上一行的现场,然后根据上一行的原来值更新本行,然后处理上一行.唯一需要注意的就是,如果本行出现新0需要更新该0所在列的上面所有行.
    \begin{lstlisting}
void setZeroes(vector<vector<int> > &matrix) {
	int n = matrix.size();
	if(n == 0)	return;
	int m = matrix[0].size();
	bool lastZero = false;
	for(int i = 0; i < n; i++){
		bool thisZero = false;
		for(int j = 0; j < m; j++){
			if(matrix[i][j] == 0){
				int up = i - 1;
				while(up >= 0){
					matrix[up][j] = 0;
					up--;
				}
				thisZero = true;
			}
			if(i > 0 && matrix[i-1][j] == 0)
				matrix[i][j] = 0;
		}
		if(lastZero){
			for(int j = 0; j < m; j++)
				matrix[i-1][j] = 0;
		}
		lastZero = thisZero;
	}
	if(lastZero){
		for(int j = 0; j < m; j++)
			matrix[n-1][j] = 0;
	}
}
    \end{lstlisting}
\end{description}

\subsection{Pascal's Triangle}
    
\begin{description}
    \item{\textbf{问题}}:\\
Given numRows, generate the first numRows of Pascal's triangle. \\
\textit{(leetcode 118)}
    \item{\textbf{举例}}:\\
Given numRows = 5, \\
Return \\
\\
$[$ \\
     $[1]$, \\
    $[1,1]$, \\
   $[1,2,1]$, \\
  $[1,3,3,1]$, \\
 $[1,4,6,4,1]$ \\
$]$
    \item{\textbf{???}} : \fbox{时间复杂度O($n^2$), 空间复杂度O(1)}
    \begin{lstlisting}
vector<vector<int> > generate(int numRows) {
	vector<vector<int> > result;
	if(numRows == 0)	return result;
	result.push_back(vector<int>{1});
	for(int i = 1; i < numRows; i++){
		vector<int> cur;
		cur.push_back(1);
		for(int j = 0; j < result[i-1].size() - 1; j++)
			cur.push_back(result[i-1][j] + result[i-1][j+1]);
		cur.push_back(1);
		result.push_back(cur);
	}
	return result;
}
    \end{lstlisting}
\end{description}

\subsection{Pascal's Triangle II}
    
\begin{description}
    \item{\textbf{问题}}:\\
Given an index k, return the kth row of the Pascal's triangle. \\
\textit{(leetcode 119)}
    \item{\textbf{举例}}:\\
Given k = 3, \\
Return $[1,3,3,1]$.
    \item{\textbf{Note}}:\\
Could you optimize your algorithm to use only O(k) extra space?
    \item{\textbf{???}} : \fbox{时间复杂度O($k^2$), 空间复杂度O(k)}
    \\这个空间复杂度可以优化,因为每次我们只需要上一行就可以产生本行,所有之前的行可以不存储
    \begin{lstlisting}
	vector<int> getRow(int rowIndex) {
		rowIndex++;
		if(rowIndex <= 0)	return vector<int>();
		vector<int> result{1};
		for(int i = 1; i < rowIndex; i++){
			vector<int> cur;
			cur.push_back(1);
			for(int j = 0; j < result.size() - 1; j++)
				cur.push_back(result[j] + result[j+1]);
			cur.push_back(1);
			result.swap(cur);
		}
		return result;
	}
    \end{lstlisting}
\end{description}



\chapter{二分查找}
    
\subsection{Spiral Matrix}
    
\begin{description}
    \item{\textbf{问题}}:\\
Given a matrix of m x n elements (m rows, n columns), return all elements of the matrix in spiral order.\\
\textit{(leetcode 54)}
    \item{\textbf{举例}}:\\
Given the following matrix:\\
\\
$[$ \\
 $[ 1, 2, 3 ]$, \\
 $[ 4, 5, 6 ]$, \\
 $[ 7, 8, 9 ]$ \\
$]$ \\
You should return $[1,2,3,6,9,8,7,4,5]$.
    \item{\textbf{???}} : \fbox{时间复杂度O($n^2$), 空间复杂度O(1)}
    \\从外到内一环一环的处理,需要注意一些边界条件
    \begin{lstlisting}
vector<int> spiralOrder(vector<vector<int> > &matrix) {
	vector<int> result;
	int n = matrix.size();
	if(n == 0)	return result;
	int m = matrix[0].size();
	int magrin = 0;
	while(m - 1 - magrin >= magrin && n - 1 - magrin >= magrin){
		for(int j = magrin; j <= m - 1 - magrin; j++)
			result.push_back(matrix[magrin][j]);
		for(int i = magrin + 1; i < n - 1 - magrin; i++)
			result.push_back(matrix[i][m-1-magrin]);
		if(n - 1 - magrin != magrin)
			for(int j = m - 1 - magrin; j >= magrin; j-- )
				result.push_back(matrix[n-1-magrin][j]);
		if(m - 1 - magrin != magrin)
			for(int i = n - 1 - magrin - 1; i > magrin; i--)
				result.push_back(matrix[i][magrin]);
		magrin++;
	}
	return result;
}
    \end{lstlisting}
\end{description}

\subsection{Spiral Matrix II}
    
\begin{description}
    \item{\textbf{问题}}:\\
Given an integer n, generate a square matrix filled with elements from 1 to $n^2$ in spiral order.\\
\textit{(leetcode 59)}
    \item{\textbf{举例}}:\\
Given n = 3,\\
\\
You should return the following matrix:\\
$[$ \\
 $[ 1, 2, 3 ]$, \\
 $[ 8, 9, 4 ]$, \\
 $[ 7, 6, 5 ]$ \\
$]$
    \item{\textbf{???}} : \fbox{时间复杂度O($n^2$), 空间复杂度O(1)}
    \begin{lstlisting}
vector<vector<int> > generateMatrix(int n) {
	vector<vector<int> > matrix(n, vector<int>(n, 0));
	int pos = 1;
	int magrin = 0;
	while(n - 1 - magrin >= magrin && n - 1 - magrin >= magrin){
		for(int j = magrin; j <= n - 1 - magrin; j++)
			matrix[magrin][j] = pos++;
		for(int i = magrin + 1; i < n - 1 - magrin; i++)
			matrix[i][n-1-magrin] = pos++;
		if(n - 1 - magrin != magrin)
			for(int j = n - 1 - magrin; j >= magrin; j-- )
				matrix[n-1-magrin][j] = pos++;
		if(n - 1 - magrin != magrin)
			for(int i = n - 1 - magrin - 1; i > magrin; i--)
				matrix[i][magrin] = pos++;
		magrin++;
	}
	return matrix;
}
    \end{lstlisting}
\end{description}

\subsection{Set Matrix Zeroes}
    
\begin{description}
    \item{\textbf{问题}}:\\
Given a m x n matrix, if an element is 0, set its entire row and column to 0. Do it in place.\\
\textit{(leetcode 73)}
    \item{\textbf{Follow Up}}:\\
Did you use extra space? \\
A straight forward solution using O(mn) space is probably a bad idea. \\
A simple improvement uses O(m + n) space, but still not the best solution. \\
Could you devise a constant space solution?
    \item{\textbf{???}} : \fbox{时间复杂度O($n^2$), 空间复杂度O(1)}
    \\从上到下一行一行的处理,如果上一个存在0,可以先保留上一行的现场,然后根据上一行的原来值更新本行,然后处理上一行.唯一需要注意的就是,如果本行出现新0需要更新该0所在列的上面所有行.
    \begin{lstlisting}
void setZeroes(vector<vector<int> > &matrix) {
	int n = matrix.size();
	if(n == 0)	return;
	int m = matrix[0].size();
	bool lastZero = false;
	for(int i = 0; i < n; i++){
		bool thisZero = false;
		for(int j = 0; j < m; j++){
			if(matrix[i][j] == 0){
				int up = i - 1;
				while(up >= 0){
					matrix[up][j] = 0;
					up--;
				}
				thisZero = true;
			}
			if(i > 0 && matrix[i-1][j] == 0)
				matrix[i][j] = 0;
		}
		if(lastZero){
			for(int j = 0; j < m; j++)
				matrix[i-1][j] = 0;
		}
		lastZero = thisZero;
	}
	if(lastZero){
		for(int j = 0; j < m; j++)
			matrix[n-1][j] = 0;
	}
}
    \end{lstlisting}
\end{description}

\subsection{Pascal's Triangle}
    
\begin{description}
    \item{\textbf{问题}}:\\
Given numRows, generate the first numRows of Pascal's triangle. \\
\textit{(leetcode 118)}
    \item{\textbf{举例}}:\\
Given numRows = 5, \\
Return \\
\\
$[$ \\
     $[1]$, \\
    $[1,1]$, \\
   $[1,2,1]$, \\
  $[1,3,3,1]$, \\
 $[1,4,6,4,1]$ \\
$]$
    \item{\textbf{???}} : \fbox{时间复杂度O($n^2$), 空间复杂度O(1)}
    \begin{lstlisting}
vector<vector<int> > generate(int numRows) {
	vector<vector<int> > result;
	if(numRows == 0)	return result;
	result.push_back(vector<int>{1});
	for(int i = 1; i < numRows; i++){
		vector<int> cur;
		cur.push_back(1);
		for(int j = 0; j < result[i-1].size() - 1; j++)
			cur.push_back(result[i-1][j] + result[i-1][j+1]);
		cur.push_back(1);
		result.push_back(cur);
	}
	return result;
}
    \end{lstlisting}
\end{description}

\subsection{Pascal's Triangle II}
    
\begin{description}
    \item{\textbf{问题}}:\\
Given an index k, return the kth row of the Pascal's triangle. \\
\textit{(leetcode 119)}
    \item{\textbf{举例}}:\\
Given k = 3, \\
Return $[1,3,3,1]$.
    \item{\textbf{Note}}:\\
Could you optimize your algorithm to use only O(k) extra space?
    \item{\textbf{???}} : \fbox{时间复杂度O($k^2$), 空间复杂度O(k)}
    \\这个空间复杂度可以优化,因为每次我们只需要上一行就可以产生本行,所有之前的行可以不存储
    \begin{lstlisting}
	vector<int> getRow(int rowIndex) {
		rowIndex++;
		if(rowIndex <= 0)	return vector<int>();
		vector<int> result{1};
		for(int i = 1; i < rowIndex; i++){
			vector<int> cur;
			cur.push_back(1);
			for(int j = 0; j < result.size() - 1; j++)
				cur.push_back(result[j] + result[j+1]);
			cur.push_back(1);
			result.swap(cur);
		}
		return result;
	}
    \end{lstlisting}
\end{description}



\chapter{分治}
    
\subsection{Spiral Matrix}
    
\begin{description}
    \item{\textbf{问题}}:\\
Given a matrix of m x n elements (m rows, n columns), return all elements of the matrix in spiral order.\\
\textit{(leetcode 54)}
    \item{\textbf{举例}}:\\
Given the following matrix:\\
\\
$[$ \\
 $[ 1, 2, 3 ]$, \\
 $[ 4, 5, 6 ]$, \\
 $[ 7, 8, 9 ]$ \\
$]$ \\
You should return $[1,2,3,6,9,8,7,4,5]$.
    \item{\textbf{???}} : \fbox{时间复杂度O($n^2$), 空间复杂度O(1)}
    \\从外到内一环一环的处理,需要注意一些边界条件
    \begin{lstlisting}
vector<int> spiralOrder(vector<vector<int> > &matrix) {
	vector<int> result;
	int n = matrix.size();
	if(n == 0)	return result;
	int m = matrix[0].size();
	int magrin = 0;
	while(m - 1 - magrin >= magrin && n - 1 - magrin >= magrin){
		for(int j = magrin; j <= m - 1 - magrin; j++)
			result.push_back(matrix[magrin][j]);
		for(int i = magrin + 1; i < n - 1 - magrin; i++)
			result.push_back(matrix[i][m-1-magrin]);
		if(n - 1 - magrin != magrin)
			for(int j = m - 1 - magrin; j >= magrin; j-- )
				result.push_back(matrix[n-1-magrin][j]);
		if(m - 1 - magrin != magrin)
			for(int i = n - 1 - magrin - 1; i > magrin; i--)
				result.push_back(matrix[i][magrin]);
		magrin++;
	}
	return result;
}
    \end{lstlisting}
\end{description}

\subsection{Spiral Matrix II}
    
\begin{description}
    \item{\textbf{问题}}:\\
Given an integer n, generate a square matrix filled with elements from 1 to $n^2$ in spiral order.\\
\textit{(leetcode 59)}
    \item{\textbf{举例}}:\\
Given n = 3,\\
\\
You should return the following matrix:\\
$[$ \\
 $[ 1, 2, 3 ]$, \\
 $[ 8, 9, 4 ]$, \\
 $[ 7, 6, 5 ]$ \\
$]$
    \item{\textbf{???}} : \fbox{时间复杂度O($n^2$), 空间复杂度O(1)}
    \begin{lstlisting}
vector<vector<int> > generateMatrix(int n) {
	vector<vector<int> > matrix(n, vector<int>(n, 0));
	int pos = 1;
	int magrin = 0;
	while(n - 1 - magrin >= magrin && n - 1 - magrin >= magrin){
		for(int j = magrin; j <= n - 1 - magrin; j++)
			matrix[magrin][j] = pos++;
		for(int i = magrin + 1; i < n - 1 - magrin; i++)
			matrix[i][n-1-magrin] = pos++;
		if(n - 1 - magrin != magrin)
			for(int j = n - 1 - magrin; j >= magrin; j-- )
				matrix[n-1-magrin][j] = pos++;
		if(n - 1 - magrin != magrin)
			for(int i = n - 1 - magrin - 1; i > magrin; i--)
				matrix[i][magrin] = pos++;
		magrin++;
	}
	return matrix;
}
    \end{lstlisting}
\end{description}

\subsection{Set Matrix Zeroes}
    
\begin{description}
    \item{\textbf{问题}}:\\
Given a m x n matrix, if an element is 0, set its entire row and column to 0. Do it in place.\\
\textit{(leetcode 73)}
    \item{\textbf{Follow Up}}:\\
Did you use extra space? \\
A straight forward solution using O(mn) space is probably a bad idea. \\
A simple improvement uses O(m + n) space, but still not the best solution. \\
Could you devise a constant space solution?
    \item{\textbf{???}} : \fbox{时间复杂度O($n^2$), 空间复杂度O(1)}
    \\从上到下一行一行的处理,如果上一个存在0,可以先保留上一行的现场,然后根据上一行的原来值更新本行,然后处理上一行.唯一需要注意的就是,如果本行出现新0需要更新该0所在列的上面所有行.
    \begin{lstlisting}
void setZeroes(vector<vector<int> > &matrix) {
	int n = matrix.size();
	if(n == 0)	return;
	int m = matrix[0].size();
	bool lastZero = false;
	for(int i = 0; i < n; i++){
		bool thisZero = false;
		for(int j = 0; j < m; j++){
			if(matrix[i][j] == 0){
				int up = i - 1;
				while(up >= 0){
					matrix[up][j] = 0;
					up--;
				}
				thisZero = true;
			}
			if(i > 0 && matrix[i-1][j] == 0)
				matrix[i][j] = 0;
		}
		if(lastZero){
			for(int j = 0; j < m; j++)
				matrix[i-1][j] = 0;
		}
		lastZero = thisZero;
	}
	if(lastZero){
		for(int j = 0; j < m; j++)
			matrix[n-1][j] = 0;
	}
}
    \end{lstlisting}
\end{description}

\subsection{Pascal's Triangle}
    
\begin{description}
    \item{\textbf{问题}}:\\
Given numRows, generate the first numRows of Pascal's triangle. \\
\textit{(leetcode 118)}
    \item{\textbf{举例}}:\\
Given numRows = 5, \\
Return \\
\\
$[$ \\
     $[1]$, \\
    $[1,1]$, \\
   $[1,2,1]$, \\
  $[1,3,3,1]$, \\
 $[1,4,6,4,1]$ \\
$]$
    \item{\textbf{???}} : \fbox{时间复杂度O($n^2$), 空间复杂度O(1)}
    \begin{lstlisting}
vector<vector<int> > generate(int numRows) {
	vector<vector<int> > result;
	if(numRows == 0)	return result;
	result.push_back(vector<int>{1});
	for(int i = 1; i < numRows; i++){
		vector<int> cur;
		cur.push_back(1);
		for(int j = 0; j < result[i-1].size() - 1; j++)
			cur.push_back(result[i-1][j] + result[i-1][j+1]);
		cur.push_back(1);
		result.push_back(cur);
	}
	return result;
}
    \end{lstlisting}
\end{description}

\subsection{Pascal's Triangle II}
    
\begin{description}
    \item{\textbf{问题}}:\\
Given an index k, return the kth row of the Pascal's triangle. \\
\textit{(leetcode 119)}
    \item{\textbf{举例}}:\\
Given k = 3, \\
Return $[1,3,3,1]$.
    \item{\textbf{Note}}:\\
Could you optimize your algorithm to use only O(k) extra space?
    \item{\textbf{???}} : \fbox{时间复杂度O($k^2$), 空间复杂度O(k)}
    \\这个空间复杂度可以优化,因为每次我们只需要上一行就可以产生本行,所有之前的行可以不存储
    \begin{lstlisting}
	vector<int> getRow(int rowIndex) {
		rowIndex++;
		if(rowIndex <= 0)	return vector<int>();
		vector<int> result{1};
		for(int i = 1; i < rowIndex; i++){
			vector<int> cur;
			cur.push_back(1);
			for(int j = 0; j < result.size() - 1; j++)
				cur.push_back(result[j] + result[j+1]);
			cur.push_back(1);
			result.swap(cur);
		}
		return result;
	}
    \end{lstlisting}
\end{description}



\chapter{搜索}
    
\subsection{Spiral Matrix}
    
\begin{description}
    \item{\textbf{问题}}:\\
Given a matrix of m x n elements (m rows, n columns), return all elements of the matrix in spiral order.\\
\textit{(leetcode 54)}
    \item{\textbf{举例}}:\\
Given the following matrix:\\
\\
$[$ \\
 $[ 1, 2, 3 ]$, \\
 $[ 4, 5, 6 ]$, \\
 $[ 7, 8, 9 ]$ \\
$]$ \\
You should return $[1,2,3,6,9,8,7,4,5]$.
    \item{\textbf{???}} : \fbox{时间复杂度O($n^2$), 空间复杂度O(1)}
    \\从外到内一环一环的处理,需要注意一些边界条件
    \begin{lstlisting}
vector<int> spiralOrder(vector<vector<int> > &matrix) {
	vector<int> result;
	int n = matrix.size();
	if(n == 0)	return result;
	int m = matrix[0].size();
	int magrin = 0;
	while(m - 1 - magrin >= magrin && n - 1 - magrin >= magrin){
		for(int j = magrin; j <= m - 1 - magrin; j++)
			result.push_back(matrix[magrin][j]);
		for(int i = magrin + 1; i < n - 1 - magrin; i++)
			result.push_back(matrix[i][m-1-magrin]);
		if(n - 1 - magrin != magrin)
			for(int j = m - 1 - magrin; j >= magrin; j-- )
				result.push_back(matrix[n-1-magrin][j]);
		if(m - 1 - magrin != magrin)
			for(int i = n - 1 - magrin - 1; i > magrin; i--)
				result.push_back(matrix[i][magrin]);
		magrin++;
	}
	return result;
}
    \end{lstlisting}
\end{description}

\subsection{Spiral Matrix II}
    
\begin{description}
    \item{\textbf{问题}}:\\
Given an integer n, generate a square matrix filled with elements from 1 to $n^2$ in spiral order.\\
\textit{(leetcode 59)}
    \item{\textbf{举例}}:\\
Given n = 3,\\
\\
You should return the following matrix:\\
$[$ \\
 $[ 1, 2, 3 ]$, \\
 $[ 8, 9, 4 ]$, \\
 $[ 7, 6, 5 ]$ \\
$]$
    \item{\textbf{???}} : \fbox{时间复杂度O($n^2$), 空间复杂度O(1)}
    \begin{lstlisting}
vector<vector<int> > generateMatrix(int n) {
	vector<vector<int> > matrix(n, vector<int>(n, 0));
	int pos = 1;
	int magrin = 0;
	while(n - 1 - magrin >= magrin && n - 1 - magrin >= magrin){
		for(int j = magrin; j <= n - 1 - magrin; j++)
			matrix[magrin][j] = pos++;
		for(int i = magrin + 1; i < n - 1 - magrin; i++)
			matrix[i][n-1-magrin] = pos++;
		if(n - 1 - magrin != magrin)
			for(int j = n - 1 - magrin; j >= magrin; j-- )
				matrix[n-1-magrin][j] = pos++;
		if(n - 1 - magrin != magrin)
			for(int i = n - 1 - magrin - 1; i > magrin; i--)
				matrix[i][magrin] = pos++;
		magrin++;
	}
	return matrix;
}
    \end{lstlisting}
\end{description}

\subsection{Set Matrix Zeroes}
    
\begin{description}
    \item{\textbf{问题}}:\\
Given a m x n matrix, if an element is 0, set its entire row and column to 0. Do it in place.\\
\textit{(leetcode 73)}
    \item{\textbf{Follow Up}}:\\
Did you use extra space? \\
A straight forward solution using O(mn) space is probably a bad idea. \\
A simple improvement uses O(m + n) space, but still not the best solution. \\
Could you devise a constant space solution?
    \item{\textbf{???}} : \fbox{时间复杂度O($n^2$), 空间复杂度O(1)}
    \\从上到下一行一行的处理,如果上一个存在0,可以先保留上一行的现场,然后根据上一行的原来值更新本行,然后处理上一行.唯一需要注意的就是,如果本行出现新0需要更新该0所在列的上面所有行.
    \begin{lstlisting}
void setZeroes(vector<vector<int> > &matrix) {
	int n = matrix.size();
	if(n == 0)	return;
	int m = matrix[0].size();
	bool lastZero = false;
	for(int i = 0; i < n; i++){
		bool thisZero = false;
		for(int j = 0; j < m; j++){
			if(matrix[i][j] == 0){
				int up = i - 1;
				while(up >= 0){
					matrix[up][j] = 0;
					up--;
				}
				thisZero = true;
			}
			if(i > 0 && matrix[i-1][j] == 0)
				matrix[i][j] = 0;
		}
		if(lastZero){
			for(int j = 0; j < m; j++)
				matrix[i-1][j] = 0;
		}
		lastZero = thisZero;
	}
	if(lastZero){
		for(int j = 0; j < m; j++)
			matrix[n-1][j] = 0;
	}
}
    \end{lstlisting}
\end{description}

\subsection{Pascal's Triangle}
    
\begin{description}
    \item{\textbf{问题}}:\\
Given numRows, generate the first numRows of Pascal's triangle. \\
\textit{(leetcode 118)}
    \item{\textbf{举例}}:\\
Given numRows = 5, \\
Return \\
\\
$[$ \\
     $[1]$, \\
    $[1,1]$, \\
   $[1,2,1]$, \\
  $[1,3,3,1]$, \\
 $[1,4,6,4,1]$ \\
$]$
    \item{\textbf{???}} : \fbox{时间复杂度O($n^2$), 空间复杂度O(1)}
    \begin{lstlisting}
vector<vector<int> > generate(int numRows) {
	vector<vector<int> > result;
	if(numRows == 0)	return result;
	result.push_back(vector<int>{1});
	for(int i = 1; i < numRows; i++){
		vector<int> cur;
		cur.push_back(1);
		for(int j = 0; j < result[i-1].size() - 1; j++)
			cur.push_back(result[i-1][j] + result[i-1][j+1]);
		cur.push_back(1);
		result.push_back(cur);
	}
	return result;
}
    \end{lstlisting}
\end{description}

\subsection{Pascal's Triangle II}
    
\begin{description}
    \item{\textbf{问题}}:\\
Given an index k, return the kth row of the Pascal's triangle. \\
\textit{(leetcode 119)}
    \item{\textbf{举例}}:\\
Given k = 3, \\
Return $[1,3,3,1]$.
    \item{\textbf{Note}}:\\
Could you optimize your algorithm to use only O(k) extra space?
    \item{\textbf{???}} : \fbox{时间复杂度O($k^2$), 空间复杂度O(k)}
    \\这个空间复杂度可以优化,因为每次我们只需要上一行就可以产生本行,所有之前的行可以不存储
    \begin{lstlisting}
	vector<int> getRow(int rowIndex) {
		rowIndex++;
		if(rowIndex <= 0)	return vector<int>();
		vector<int> result{1};
		for(int i = 1; i < rowIndex; i++){
			vector<int> cur;
			cur.push_back(1);
			for(int j = 0; j < result.size() - 1; j++)
				cur.push_back(result[j] + result[j+1]);
			cur.push_back(1);
			result.swap(cur);
		}
		return result;
	}
    \end{lstlisting}
\end{description}



\chapter{回溯}
    
\subsection{Spiral Matrix}
    
\begin{description}
    \item{\textbf{问题}}:\\
Given a matrix of m x n elements (m rows, n columns), return all elements of the matrix in spiral order.\\
\textit{(leetcode 54)}
    \item{\textbf{举例}}:\\
Given the following matrix:\\
\\
$[$ \\
 $[ 1, 2, 3 ]$, \\
 $[ 4, 5, 6 ]$, \\
 $[ 7, 8, 9 ]$ \\
$]$ \\
You should return $[1,2,3,6,9,8,7,4,5]$.
    \item{\textbf{???}} : \fbox{时间复杂度O($n^2$), 空间复杂度O(1)}
    \\从外到内一环一环的处理,需要注意一些边界条件
    \begin{lstlisting}
vector<int> spiralOrder(vector<vector<int> > &matrix) {
	vector<int> result;
	int n = matrix.size();
	if(n == 0)	return result;
	int m = matrix[0].size();
	int magrin = 0;
	while(m - 1 - magrin >= magrin && n - 1 - magrin >= magrin){
		for(int j = magrin; j <= m - 1 - magrin; j++)
			result.push_back(matrix[magrin][j]);
		for(int i = magrin + 1; i < n - 1 - magrin; i++)
			result.push_back(matrix[i][m-1-magrin]);
		if(n - 1 - magrin != magrin)
			for(int j = m - 1 - magrin; j >= magrin; j-- )
				result.push_back(matrix[n-1-magrin][j]);
		if(m - 1 - magrin != magrin)
			for(int i = n - 1 - magrin - 1; i > magrin; i--)
				result.push_back(matrix[i][magrin]);
		magrin++;
	}
	return result;
}
    \end{lstlisting}
\end{description}

\subsection{Spiral Matrix II}
    
\begin{description}
    \item{\textbf{问题}}:\\
Given an integer n, generate a square matrix filled with elements from 1 to $n^2$ in spiral order.\\
\textit{(leetcode 59)}
    \item{\textbf{举例}}:\\
Given n = 3,\\
\\
You should return the following matrix:\\
$[$ \\
 $[ 1, 2, 3 ]$, \\
 $[ 8, 9, 4 ]$, \\
 $[ 7, 6, 5 ]$ \\
$]$
    \item{\textbf{???}} : \fbox{时间复杂度O($n^2$), 空间复杂度O(1)}
    \begin{lstlisting}
vector<vector<int> > generateMatrix(int n) {
	vector<vector<int> > matrix(n, vector<int>(n, 0));
	int pos = 1;
	int magrin = 0;
	while(n - 1 - magrin >= magrin && n - 1 - magrin >= magrin){
		for(int j = magrin; j <= n - 1 - magrin; j++)
			matrix[magrin][j] = pos++;
		for(int i = magrin + 1; i < n - 1 - magrin; i++)
			matrix[i][n-1-magrin] = pos++;
		if(n - 1 - magrin != magrin)
			for(int j = n - 1 - magrin; j >= magrin; j-- )
				matrix[n-1-magrin][j] = pos++;
		if(n - 1 - magrin != magrin)
			for(int i = n - 1 - magrin - 1; i > magrin; i--)
				matrix[i][magrin] = pos++;
		magrin++;
	}
	return matrix;
}
    \end{lstlisting}
\end{description}

\subsection{Set Matrix Zeroes}
    
\begin{description}
    \item{\textbf{问题}}:\\
Given a m x n matrix, if an element is 0, set its entire row and column to 0. Do it in place.\\
\textit{(leetcode 73)}
    \item{\textbf{Follow Up}}:\\
Did you use extra space? \\
A straight forward solution using O(mn) space is probably a bad idea. \\
A simple improvement uses O(m + n) space, but still not the best solution. \\
Could you devise a constant space solution?
    \item{\textbf{???}} : \fbox{时间复杂度O($n^2$), 空间复杂度O(1)}
    \\从上到下一行一行的处理,如果上一个存在0,可以先保留上一行的现场,然后根据上一行的原来值更新本行,然后处理上一行.唯一需要注意的就是,如果本行出现新0需要更新该0所在列的上面所有行.
    \begin{lstlisting}
void setZeroes(vector<vector<int> > &matrix) {
	int n = matrix.size();
	if(n == 0)	return;
	int m = matrix[0].size();
	bool lastZero = false;
	for(int i = 0; i < n; i++){
		bool thisZero = false;
		for(int j = 0; j < m; j++){
			if(matrix[i][j] == 0){
				int up = i - 1;
				while(up >= 0){
					matrix[up][j] = 0;
					up--;
				}
				thisZero = true;
			}
			if(i > 0 && matrix[i-1][j] == 0)
				matrix[i][j] = 0;
		}
		if(lastZero){
			for(int j = 0; j < m; j++)
				matrix[i-1][j] = 0;
		}
		lastZero = thisZero;
	}
	if(lastZero){
		for(int j = 0; j < m; j++)
			matrix[n-1][j] = 0;
	}
}
    \end{lstlisting}
\end{description}

\subsection{Pascal's Triangle}
    
\begin{description}
    \item{\textbf{问题}}:\\
Given numRows, generate the first numRows of Pascal's triangle. \\
\textit{(leetcode 118)}
    \item{\textbf{举例}}:\\
Given numRows = 5, \\
Return \\
\\
$[$ \\
     $[1]$, \\
    $[1,1]$, \\
   $[1,2,1]$, \\
  $[1,3,3,1]$, \\
 $[1,4,6,4,1]$ \\
$]$
    \item{\textbf{???}} : \fbox{时间复杂度O($n^2$), 空间复杂度O(1)}
    \begin{lstlisting}
vector<vector<int> > generate(int numRows) {
	vector<vector<int> > result;
	if(numRows == 0)	return result;
	result.push_back(vector<int>{1});
	for(int i = 1; i < numRows; i++){
		vector<int> cur;
		cur.push_back(1);
		for(int j = 0; j < result[i-1].size() - 1; j++)
			cur.push_back(result[i-1][j] + result[i-1][j+1]);
		cur.push_back(1);
		result.push_back(cur);
	}
	return result;
}
    \end{lstlisting}
\end{description}

\subsection{Pascal's Triangle II}
    
\begin{description}
    \item{\textbf{问题}}:\\
Given an index k, return the kth row of the Pascal's triangle. \\
\textit{(leetcode 119)}
    \item{\textbf{举例}}:\\
Given k = 3, \\
Return $[1,3,3,1]$.
    \item{\textbf{Note}}:\\
Could you optimize your algorithm to use only O(k) extra space?
    \item{\textbf{???}} : \fbox{时间复杂度O($k^2$), 空间复杂度O(k)}
    \\这个空间复杂度可以优化,因为每次我们只需要上一行就可以产生本行,所有之前的行可以不存储
    \begin{lstlisting}
	vector<int> getRow(int rowIndex) {
		rowIndex++;
		if(rowIndex <= 0)	return vector<int>();
		vector<int> result{1};
		for(int i = 1; i < rowIndex; i++){
			vector<int> cur;
			cur.push_back(1);
			for(int j = 0; j < result.size() - 1; j++)
				cur.push_back(result[j] + result[j+1]);
			cur.push_back(1);
			result.swap(cur);
		}
		return result;
	}
    \end{lstlisting}
\end{description}



\chapter{贪心}
    
\subsection{Spiral Matrix}
    
\begin{description}
    \item{\textbf{问题}}:\\
Given a matrix of m x n elements (m rows, n columns), return all elements of the matrix in spiral order.\\
\textit{(leetcode 54)}
    \item{\textbf{举例}}:\\
Given the following matrix:\\
\\
$[$ \\
 $[ 1, 2, 3 ]$, \\
 $[ 4, 5, 6 ]$, \\
 $[ 7, 8, 9 ]$ \\
$]$ \\
You should return $[1,2,3,6,9,8,7,4,5]$.
    \item{\textbf{???}} : \fbox{时间复杂度O($n^2$), 空间复杂度O(1)}
    \\从外到内一环一环的处理,需要注意一些边界条件
    \begin{lstlisting}
vector<int> spiralOrder(vector<vector<int> > &matrix) {
	vector<int> result;
	int n = matrix.size();
	if(n == 0)	return result;
	int m = matrix[0].size();
	int magrin = 0;
	while(m - 1 - magrin >= magrin && n - 1 - magrin >= magrin){
		for(int j = magrin; j <= m - 1 - magrin; j++)
			result.push_back(matrix[magrin][j]);
		for(int i = magrin + 1; i < n - 1 - magrin; i++)
			result.push_back(matrix[i][m-1-magrin]);
		if(n - 1 - magrin != magrin)
			for(int j = m - 1 - magrin; j >= magrin; j-- )
				result.push_back(matrix[n-1-magrin][j]);
		if(m - 1 - magrin != magrin)
			for(int i = n - 1 - magrin - 1; i > magrin; i--)
				result.push_back(matrix[i][magrin]);
		magrin++;
	}
	return result;
}
    \end{lstlisting}
\end{description}

\subsection{Spiral Matrix II}
    
\begin{description}
    \item{\textbf{问题}}:\\
Given an integer n, generate a square matrix filled with elements from 1 to $n^2$ in spiral order.\\
\textit{(leetcode 59)}
    \item{\textbf{举例}}:\\
Given n = 3,\\
\\
You should return the following matrix:\\
$[$ \\
 $[ 1, 2, 3 ]$, \\
 $[ 8, 9, 4 ]$, \\
 $[ 7, 6, 5 ]$ \\
$]$
    \item{\textbf{???}} : \fbox{时间复杂度O($n^2$), 空间复杂度O(1)}
    \begin{lstlisting}
vector<vector<int> > generateMatrix(int n) {
	vector<vector<int> > matrix(n, vector<int>(n, 0));
	int pos = 1;
	int magrin = 0;
	while(n - 1 - magrin >= magrin && n - 1 - magrin >= magrin){
		for(int j = magrin; j <= n - 1 - magrin; j++)
			matrix[magrin][j] = pos++;
		for(int i = magrin + 1; i < n - 1 - magrin; i++)
			matrix[i][n-1-magrin] = pos++;
		if(n - 1 - magrin != magrin)
			for(int j = n - 1 - magrin; j >= magrin; j-- )
				matrix[n-1-magrin][j] = pos++;
		if(n - 1 - magrin != magrin)
			for(int i = n - 1 - magrin - 1; i > magrin; i--)
				matrix[i][magrin] = pos++;
		magrin++;
	}
	return matrix;
}
    \end{lstlisting}
\end{description}

\subsection{Set Matrix Zeroes}
    
\begin{description}
    \item{\textbf{问题}}:\\
Given a m x n matrix, if an element is 0, set its entire row and column to 0. Do it in place.\\
\textit{(leetcode 73)}
    \item{\textbf{Follow Up}}:\\
Did you use extra space? \\
A straight forward solution using O(mn) space is probably a bad idea. \\
A simple improvement uses O(m + n) space, but still not the best solution. \\
Could you devise a constant space solution?
    \item{\textbf{???}} : \fbox{时间复杂度O($n^2$), 空间复杂度O(1)}
    \\从上到下一行一行的处理,如果上一个存在0,可以先保留上一行的现场,然后根据上一行的原来值更新本行,然后处理上一行.唯一需要注意的就是,如果本行出现新0需要更新该0所在列的上面所有行.
    \begin{lstlisting}
void setZeroes(vector<vector<int> > &matrix) {
	int n = matrix.size();
	if(n == 0)	return;
	int m = matrix[0].size();
	bool lastZero = false;
	for(int i = 0; i < n; i++){
		bool thisZero = false;
		for(int j = 0; j < m; j++){
			if(matrix[i][j] == 0){
				int up = i - 1;
				while(up >= 0){
					matrix[up][j] = 0;
					up--;
				}
				thisZero = true;
			}
			if(i > 0 && matrix[i-1][j] == 0)
				matrix[i][j] = 0;
		}
		if(lastZero){
			for(int j = 0; j < m; j++)
				matrix[i-1][j] = 0;
		}
		lastZero = thisZero;
	}
	if(lastZero){
		for(int j = 0; j < m; j++)
			matrix[n-1][j] = 0;
	}
}
    \end{lstlisting}
\end{description}

\subsection{Pascal's Triangle}
    
\begin{description}
    \item{\textbf{问题}}:\\
Given numRows, generate the first numRows of Pascal's triangle. \\
\textit{(leetcode 118)}
    \item{\textbf{举例}}:\\
Given numRows = 5, \\
Return \\
\\
$[$ \\
     $[1]$, \\
    $[1,1]$, \\
   $[1,2,1]$, \\
  $[1,3,3,1]$, \\
 $[1,4,6,4,1]$ \\
$]$
    \item{\textbf{???}} : \fbox{时间复杂度O($n^2$), 空间复杂度O(1)}
    \begin{lstlisting}
vector<vector<int> > generate(int numRows) {
	vector<vector<int> > result;
	if(numRows == 0)	return result;
	result.push_back(vector<int>{1});
	for(int i = 1; i < numRows; i++){
		vector<int> cur;
		cur.push_back(1);
		for(int j = 0; j < result[i-1].size() - 1; j++)
			cur.push_back(result[i-1][j] + result[i-1][j+1]);
		cur.push_back(1);
		result.push_back(cur);
	}
	return result;
}
    \end{lstlisting}
\end{description}

\subsection{Pascal's Triangle II}
    
\begin{description}
    \item{\textbf{问题}}:\\
Given an index k, return the kth row of the Pascal's triangle. \\
\textit{(leetcode 119)}
    \item{\textbf{举例}}:\\
Given k = 3, \\
Return $[1,3,3,1]$.
    \item{\textbf{Note}}:\\
Could you optimize your algorithm to use only O(k) extra space?
    \item{\textbf{???}} : \fbox{时间复杂度O($k^2$), 空间复杂度O(k)}
    \\这个空间复杂度可以优化,因为每次我们只需要上一行就可以产生本行,所有之前的行可以不存储
    \begin{lstlisting}
	vector<int> getRow(int rowIndex) {
		rowIndex++;
		if(rowIndex <= 0)	return vector<int>();
		vector<int> result{1};
		for(int i = 1; i < rowIndex; i++){
			vector<int> cur;
			cur.push_back(1);
			for(int j = 0; j < result.size() - 1; j++)
				cur.push_back(result[j] + result[j+1]);
			cur.push_back(1);
			result.swap(cur);
		}
		return result;
	}
    \end{lstlisting}
\end{description}



\chapter{动态规划}
    
\subsection{Spiral Matrix}
    
\begin{description}
    \item{\textbf{问题}}:\\
Given a matrix of m x n elements (m rows, n columns), return all elements of the matrix in spiral order.\\
\textit{(leetcode 54)}
    \item{\textbf{举例}}:\\
Given the following matrix:\\
\\
$[$ \\
 $[ 1, 2, 3 ]$, \\
 $[ 4, 5, 6 ]$, \\
 $[ 7, 8, 9 ]$ \\
$]$ \\
You should return $[1,2,3,6,9,8,7,4,5]$.
    \item{\textbf{???}} : \fbox{时间复杂度O($n^2$), 空间复杂度O(1)}
    \\从外到内一环一环的处理,需要注意一些边界条件
    \begin{lstlisting}
vector<int> spiralOrder(vector<vector<int> > &matrix) {
	vector<int> result;
	int n = matrix.size();
	if(n == 0)	return result;
	int m = matrix[0].size();
	int magrin = 0;
	while(m - 1 - magrin >= magrin && n - 1 - magrin >= magrin){
		for(int j = magrin; j <= m - 1 - magrin; j++)
			result.push_back(matrix[magrin][j]);
		for(int i = magrin + 1; i < n - 1 - magrin; i++)
			result.push_back(matrix[i][m-1-magrin]);
		if(n - 1 - magrin != magrin)
			for(int j = m - 1 - magrin; j >= magrin; j-- )
				result.push_back(matrix[n-1-magrin][j]);
		if(m - 1 - magrin != magrin)
			for(int i = n - 1 - magrin - 1; i > magrin; i--)
				result.push_back(matrix[i][magrin]);
		magrin++;
	}
	return result;
}
    \end{lstlisting}
\end{description}

\subsection{Spiral Matrix II}
    
\begin{description}
    \item{\textbf{问题}}:\\
Given an integer n, generate a square matrix filled with elements from 1 to $n^2$ in spiral order.\\
\textit{(leetcode 59)}
    \item{\textbf{举例}}:\\
Given n = 3,\\
\\
You should return the following matrix:\\
$[$ \\
 $[ 1, 2, 3 ]$, \\
 $[ 8, 9, 4 ]$, \\
 $[ 7, 6, 5 ]$ \\
$]$
    \item{\textbf{???}} : \fbox{时间复杂度O($n^2$), 空间复杂度O(1)}
    \begin{lstlisting}
vector<vector<int> > generateMatrix(int n) {
	vector<vector<int> > matrix(n, vector<int>(n, 0));
	int pos = 1;
	int magrin = 0;
	while(n - 1 - magrin >= magrin && n - 1 - magrin >= magrin){
		for(int j = magrin; j <= n - 1 - magrin; j++)
			matrix[magrin][j] = pos++;
		for(int i = magrin + 1; i < n - 1 - magrin; i++)
			matrix[i][n-1-magrin] = pos++;
		if(n - 1 - magrin != magrin)
			for(int j = n - 1 - magrin; j >= magrin; j-- )
				matrix[n-1-magrin][j] = pos++;
		if(n - 1 - magrin != magrin)
			for(int i = n - 1 - magrin - 1; i > magrin; i--)
				matrix[i][magrin] = pos++;
		magrin++;
	}
	return matrix;
}
    \end{lstlisting}
\end{description}

\subsection{Set Matrix Zeroes}
    
\begin{description}
    \item{\textbf{问题}}:\\
Given a m x n matrix, if an element is 0, set its entire row and column to 0. Do it in place.\\
\textit{(leetcode 73)}
    \item{\textbf{Follow Up}}:\\
Did you use extra space? \\
A straight forward solution using O(mn) space is probably a bad idea. \\
A simple improvement uses O(m + n) space, but still not the best solution. \\
Could you devise a constant space solution?
    \item{\textbf{???}} : \fbox{时间复杂度O($n^2$), 空间复杂度O(1)}
    \\从上到下一行一行的处理,如果上一个存在0,可以先保留上一行的现场,然后根据上一行的原来值更新本行,然后处理上一行.唯一需要注意的就是,如果本行出现新0需要更新该0所在列的上面所有行.
    \begin{lstlisting}
void setZeroes(vector<vector<int> > &matrix) {
	int n = matrix.size();
	if(n == 0)	return;
	int m = matrix[0].size();
	bool lastZero = false;
	for(int i = 0; i < n; i++){
		bool thisZero = false;
		for(int j = 0; j < m; j++){
			if(matrix[i][j] == 0){
				int up = i - 1;
				while(up >= 0){
					matrix[up][j] = 0;
					up--;
				}
				thisZero = true;
			}
			if(i > 0 && matrix[i-1][j] == 0)
				matrix[i][j] = 0;
		}
		if(lastZero){
			for(int j = 0; j < m; j++)
				matrix[i-1][j] = 0;
		}
		lastZero = thisZero;
	}
	if(lastZero){
		for(int j = 0; j < m; j++)
			matrix[n-1][j] = 0;
	}
}
    \end{lstlisting}
\end{description}

\subsection{Pascal's Triangle}
    
\begin{description}
    \item{\textbf{问题}}:\\
Given numRows, generate the first numRows of Pascal's triangle. \\
\textit{(leetcode 118)}
    \item{\textbf{举例}}:\\
Given numRows = 5, \\
Return \\
\\
$[$ \\
     $[1]$, \\
    $[1,1]$, \\
   $[1,2,1]$, \\
  $[1,3,3,1]$, \\
 $[1,4,6,4,1]$ \\
$]$
    \item{\textbf{???}} : \fbox{时间复杂度O($n^2$), 空间复杂度O(1)}
    \begin{lstlisting}
vector<vector<int> > generate(int numRows) {
	vector<vector<int> > result;
	if(numRows == 0)	return result;
	result.push_back(vector<int>{1});
	for(int i = 1; i < numRows; i++){
		vector<int> cur;
		cur.push_back(1);
		for(int j = 0; j < result[i-1].size() - 1; j++)
			cur.push_back(result[i-1][j] + result[i-1][j+1]);
		cur.push_back(1);
		result.push_back(cur);
	}
	return result;
}
    \end{lstlisting}
\end{description}

\subsection{Pascal's Triangle II}
    
\begin{description}
    \item{\textbf{问题}}:\\
Given an index k, return the kth row of the Pascal's triangle. \\
\textit{(leetcode 119)}
    \item{\textbf{举例}}:\\
Given k = 3, \\
Return $[1,3,3,1]$.
    \item{\textbf{Note}}:\\
Could you optimize your algorithm to use only O(k) extra space?
    \item{\textbf{???}} : \fbox{时间复杂度O($k^2$), 空间复杂度O(k)}
    \\这个空间复杂度可以优化,因为每次我们只需要上一行就可以产生本行,所有之前的行可以不存储
    \begin{lstlisting}
	vector<int> getRow(int rowIndex) {
		rowIndex++;
		if(rowIndex <= 0)	return vector<int>();
		vector<int> result{1};
		for(int i = 1; i < rowIndex; i++){
			vector<int> cur;
			cur.push_back(1);
			for(int j = 0; j < result.size() - 1; j++)
				cur.push_back(result[j] + result[j+1]);
			cur.push_back(1);
			result.swap(cur);
		}
		return result;
	}
    \end{lstlisting}
\end{description}



\chapter{位操作}
    
\subsection{Spiral Matrix}
    
\begin{description}
    \item{\textbf{问题}}:\\
Given a matrix of m x n elements (m rows, n columns), return all elements of the matrix in spiral order.\\
\textit{(leetcode 54)}
    \item{\textbf{举例}}:\\
Given the following matrix:\\
\\
$[$ \\
 $[ 1, 2, 3 ]$, \\
 $[ 4, 5, 6 ]$, \\
 $[ 7, 8, 9 ]$ \\
$]$ \\
You should return $[1,2,3,6,9,8,7,4,5]$.
    \item{\textbf{???}} : \fbox{时间复杂度O($n^2$), 空间复杂度O(1)}
    \\从外到内一环一环的处理,需要注意一些边界条件
    \begin{lstlisting}
vector<int> spiralOrder(vector<vector<int> > &matrix) {
	vector<int> result;
	int n = matrix.size();
	if(n == 0)	return result;
	int m = matrix[0].size();
	int magrin = 0;
	while(m - 1 - magrin >= magrin && n - 1 - magrin >= magrin){
		for(int j = magrin; j <= m - 1 - magrin; j++)
			result.push_back(matrix[magrin][j]);
		for(int i = magrin + 1; i < n - 1 - magrin; i++)
			result.push_back(matrix[i][m-1-magrin]);
		if(n - 1 - magrin != magrin)
			for(int j = m - 1 - magrin; j >= magrin; j-- )
				result.push_back(matrix[n-1-magrin][j]);
		if(m - 1 - magrin != magrin)
			for(int i = n - 1 - magrin - 1; i > magrin; i--)
				result.push_back(matrix[i][magrin]);
		magrin++;
	}
	return result;
}
    \end{lstlisting}
\end{description}

\subsection{Spiral Matrix II}
    
\begin{description}
    \item{\textbf{问题}}:\\
Given an integer n, generate a square matrix filled with elements from 1 to $n^2$ in spiral order.\\
\textit{(leetcode 59)}
    \item{\textbf{举例}}:\\
Given n = 3,\\
\\
You should return the following matrix:\\
$[$ \\
 $[ 1, 2, 3 ]$, \\
 $[ 8, 9, 4 ]$, \\
 $[ 7, 6, 5 ]$ \\
$]$
    \item{\textbf{???}} : \fbox{时间复杂度O($n^2$), 空间复杂度O(1)}
    \begin{lstlisting}
vector<vector<int> > generateMatrix(int n) {
	vector<vector<int> > matrix(n, vector<int>(n, 0));
	int pos = 1;
	int magrin = 0;
	while(n - 1 - magrin >= magrin && n - 1 - magrin >= magrin){
		for(int j = magrin; j <= n - 1 - magrin; j++)
			matrix[magrin][j] = pos++;
		for(int i = magrin + 1; i < n - 1 - magrin; i++)
			matrix[i][n-1-magrin] = pos++;
		if(n - 1 - magrin != magrin)
			for(int j = n - 1 - magrin; j >= magrin; j-- )
				matrix[n-1-magrin][j] = pos++;
		if(n - 1 - magrin != magrin)
			for(int i = n - 1 - magrin - 1; i > magrin; i--)
				matrix[i][magrin] = pos++;
		magrin++;
	}
	return matrix;
}
    \end{lstlisting}
\end{description}

\subsection{Set Matrix Zeroes}
    
\begin{description}
    \item{\textbf{问题}}:\\
Given a m x n matrix, if an element is 0, set its entire row and column to 0. Do it in place.\\
\textit{(leetcode 73)}
    \item{\textbf{Follow Up}}:\\
Did you use extra space? \\
A straight forward solution using O(mn) space is probably a bad idea. \\
A simple improvement uses O(m + n) space, but still not the best solution. \\
Could you devise a constant space solution?
    \item{\textbf{???}} : \fbox{时间复杂度O($n^2$), 空间复杂度O(1)}
    \\从上到下一行一行的处理,如果上一个存在0,可以先保留上一行的现场,然后根据上一行的原来值更新本行,然后处理上一行.唯一需要注意的就是,如果本行出现新0需要更新该0所在列的上面所有行.
    \begin{lstlisting}
void setZeroes(vector<vector<int> > &matrix) {
	int n = matrix.size();
	if(n == 0)	return;
	int m = matrix[0].size();
	bool lastZero = false;
	for(int i = 0; i < n; i++){
		bool thisZero = false;
		for(int j = 0; j < m; j++){
			if(matrix[i][j] == 0){
				int up = i - 1;
				while(up >= 0){
					matrix[up][j] = 0;
					up--;
				}
				thisZero = true;
			}
			if(i > 0 && matrix[i-1][j] == 0)
				matrix[i][j] = 0;
		}
		if(lastZero){
			for(int j = 0; j < m; j++)
				matrix[i-1][j] = 0;
		}
		lastZero = thisZero;
	}
	if(lastZero){
		for(int j = 0; j < m; j++)
			matrix[n-1][j] = 0;
	}
}
    \end{lstlisting}
\end{description}

\subsection{Pascal's Triangle}
    
\begin{description}
    \item{\textbf{问题}}:\\
Given numRows, generate the first numRows of Pascal's triangle. \\
\textit{(leetcode 118)}
    \item{\textbf{举例}}:\\
Given numRows = 5, \\
Return \\
\\
$[$ \\
     $[1]$, \\
    $[1,1]$, \\
   $[1,2,1]$, \\
  $[1,3,3,1]$, \\
 $[1,4,6,4,1]$ \\
$]$
    \item{\textbf{???}} : \fbox{时间复杂度O($n^2$), 空间复杂度O(1)}
    \begin{lstlisting}
vector<vector<int> > generate(int numRows) {
	vector<vector<int> > result;
	if(numRows == 0)	return result;
	result.push_back(vector<int>{1});
	for(int i = 1; i < numRows; i++){
		vector<int> cur;
		cur.push_back(1);
		for(int j = 0; j < result[i-1].size() - 1; j++)
			cur.push_back(result[i-1][j] + result[i-1][j+1]);
		cur.push_back(1);
		result.push_back(cur);
	}
	return result;
}
    \end{lstlisting}
\end{description}

\subsection{Pascal's Triangle II}
    
\begin{description}
    \item{\textbf{问题}}:\\
Given an index k, return the kth row of the Pascal's triangle. \\
\textit{(leetcode 119)}
    \item{\textbf{举例}}:\\
Given k = 3, \\
Return $[1,3,3,1]$.
    \item{\textbf{Note}}:\\
Could you optimize your algorithm to use only O(k) extra space?
    \item{\textbf{???}} : \fbox{时间复杂度O($k^2$), 空间复杂度O(k)}
    \\这个空间复杂度可以优化,因为每次我们只需要上一行就可以产生本行,所有之前的行可以不存储
    \begin{lstlisting}
	vector<int> getRow(int rowIndex) {
		rowIndex++;
		if(rowIndex <= 0)	return vector<int>();
		vector<int> result{1};
		for(int i = 1; i < rowIndex; i++){
			vector<int> cur;
			cur.push_back(1);
			for(int j = 0; j < result.size() - 1; j++)
				cur.push_back(result[j] + result[j+1]);
			cur.push_back(1);
			result.swap(cur);
		}
		return result;
	}
    \end{lstlisting}
\end{description}



\chapter{数学}
    
\subsection{Spiral Matrix}
    
\begin{description}
    \item{\textbf{问题}}:\\
Given a matrix of m x n elements (m rows, n columns), return all elements of the matrix in spiral order.\\
\textit{(leetcode 54)}
    \item{\textbf{举例}}:\\
Given the following matrix:\\
\\
$[$ \\
 $[ 1, 2, 3 ]$, \\
 $[ 4, 5, 6 ]$, \\
 $[ 7, 8, 9 ]$ \\
$]$ \\
You should return $[1,2,3,6,9,8,7,4,5]$.
    \item{\textbf{???}} : \fbox{时间复杂度O($n^2$), 空间复杂度O(1)}
    \\从外到内一环一环的处理,需要注意一些边界条件
    \begin{lstlisting}
vector<int> spiralOrder(vector<vector<int> > &matrix) {
	vector<int> result;
	int n = matrix.size();
	if(n == 0)	return result;
	int m = matrix[0].size();
	int magrin = 0;
	while(m - 1 - magrin >= magrin && n - 1 - magrin >= magrin){
		for(int j = magrin; j <= m - 1 - magrin; j++)
			result.push_back(matrix[magrin][j]);
		for(int i = magrin + 1; i < n - 1 - magrin; i++)
			result.push_back(matrix[i][m-1-magrin]);
		if(n - 1 - magrin != magrin)
			for(int j = m - 1 - magrin; j >= magrin; j-- )
				result.push_back(matrix[n-1-magrin][j]);
		if(m - 1 - magrin != magrin)
			for(int i = n - 1 - magrin - 1; i > magrin; i--)
				result.push_back(matrix[i][magrin]);
		magrin++;
	}
	return result;
}
    \end{lstlisting}
\end{description}

\subsection{Spiral Matrix II}
    
\begin{description}
    \item{\textbf{问题}}:\\
Given an integer n, generate a square matrix filled with elements from 1 to $n^2$ in spiral order.\\
\textit{(leetcode 59)}
    \item{\textbf{举例}}:\\
Given n = 3,\\
\\
You should return the following matrix:\\
$[$ \\
 $[ 1, 2, 3 ]$, \\
 $[ 8, 9, 4 ]$, \\
 $[ 7, 6, 5 ]$ \\
$]$
    \item{\textbf{???}} : \fbox{时间复杂度O($n^2$), 空间复杂度O(1)}
    \begin{lstlisting}
vector<vector<int> > generateMatrix(int n) {
	vector<vector<int> > matrix(n, vector<int>(n, 0));
	int pos = 1;
	int magrin = 0;
	while(n - 1 - magrin >= magrin && n - 1 - magrin >= magrin){
		for(int j = magrin; j <= n - 1 - magrin; j++)
			matrix[magrin][j] = pos++;
		for(int i = magrin + 1; i < n - 1 - magrin; i++)
			matrix[i][n-1-magrin] = pos++;
		if(n - 1 - magrin != magrin)
			for(int j = n - 1 - magrin; j >= magrin; j-- )
				matrix[n-1-magrin][j] = pos++;
		if(n - 1 - magrin != magrin)
			for(int i = n - 1 - magrin - 1; i > magrin; i--)
				matrix[i][magrin] = pos++;
		magrin++;
	}
	return matrix;
}
    \end{lstlisting}
\end{description}

\subsection{Set Matrix Zeroes}
    
\begin{description}
    \item{\textbf{问题}}:\\
Given a m x n matrix, if an element is 0, set its entire row and column to 0. Do it in place.\\
\textit{(leetcode 73)}
    \item{\textbf{Follow Up}}:\\
Did you use extra space? \\
A straight forward solution using O(mn) space is probably a bad idea. \\
A simple improvement uses O(m + n) space, but still not the best solution. \\
Could you devise a constant space solution?
    \item{\textbf{???}} : \fbox{时间复杂度O($n^2$), 空间复杂度O(1)}
    \\从上到下一行一行的处理,如果上一个存在0,可以先保留上一行的现场,然后根据上一行的原来值更新本行,然后处理上一行.唯一需要注意的就是,如果本行出现新0需要更新该0所在列的上面所有行.
    \begin{lstlisting}
void setZeroes(vector<vector<int> > &matrix) {
	int n = matrix.size();
	if(n == 0)	return;
	int m = matrix[0].size();
	bool lastZero = false;
	for(int i = 0; i < n; i++){
		bool thisZero = false;
		for(int j = 0; j < m; j++){
			if(matrix[i][j] == 0){
				int up = i - 1;
				while(up >= 0){
					matrix[up][j] = 0;
					up--;
				}
				thisZero = true;
			}
			if(i > 0 && matrix[i-1][j] == 0)
				matrix[i][j] = 0;
		}
		if(lastZero){
			for(int j = 0; j < m; j++)
				matrix[i-1][j] = 0;
		}
		lastZero = thisZero;
	}
	if(lastZero){
		for(int j = 0; j < m; j++)
			matrix[n-1][j] = 0;
	}
}
    \end{lstlisting}
\end{description}

\subsection{Pascal's Triangle}
    
\begin{description}
    \item{\textbf{问题}}:\\
Given numRows, generate the first numRows of Pascal's triangle. \\
\textit{(leetcode 118)}
    \item{\textbf{举例}}:\\
Given numRows = 5, \\
Return \\
\\
$[$ \\
     $[1]$, \\
    $[1,1]$, \\
   $[1,2,1]$, \\
  $[1,3,3,1]$, \\
 $[1,4,6,4,1]$ \\
$]$
    \item{\textbf{???}} : \fbox{时间复杂度O($n^2$), 空间复杂度O(1)}
    \begin{lstlisting}
vector<vector<int> > generate(int numRows) {
	vector<vector<int> > result;
	if(numRows == 0)	return result;
	result.push_back(vector<int>{1});
	for(int i = 1; i < numRows; i++){
		vector<int> cur;
		cur.push_back(1);
		for(int j = 0; j < result[i-1].size() - 1; j++)
			cur.push_back(result[i-1][j] + result[i-1][j+1]);
		cur.push_back(1);
		result.push_back(cur);
	}
	return result;
}
    \end{lstlisting}
\end{description}

\subsection{Pascal's Triangle II}
    
\begin{description}
    \item{\textbf{问题}}:\\
Given an index k, return the kth row of the Pascal's triangle. \\
\textit{(leetcode 119)}
    \item{\textbf{举例}}:\\
Given k = 3, \\
Return $[1,3,3,1]$.
    \item{\textbf{Note}}:\\
Could you optimize your algorithm to use only O(k) extra space?
    \item{\textbf{???}} : \fbox{时间复杂度O($k^2$), 空间复杂度O(k)}
    \\这个空间复杂度可以优化,因为每次我们只需要上一行就可以产生本行,所有之前的行可以不存储
    \begin{lstlisting}
	vector<int> getRow(int rowIndex) {
		rowIndex++;
		if(rowIndex <= 0)	return vector<int>();
		vector<int> result{1};
		for(int i = 1; i < rowIndex; i++){
			vector<int> cur;
			cur.push_back(1);
			for(int j = 0; j < result.size() - 1; j++)
				cur.push_back(result[j] + result[j+1]);
			cur.push_back(1);
			result.swap(cur);
		}
		return result;
	}
    \end{lstlisting}
\end{description}



\end{CJK}
\end{document}
