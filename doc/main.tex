\documentclass[oneside]{book}
%\usepackage{CJK}
\usepackage{CJKutf8}
\usepackage{graphics,graphicx}
\usepackage{pstricks,pst-node,pst-tree}
\usepackage{verbatim}
\usepackage{amsmath}
\usepackage{titlesec}
\usepackage{fancyhdr} %页眉页脚布局
\usepackage{tocloft} % 目录相关


%%%代码
\usepackage{color}
\usepackage{xcolor}
\definecolor{keywordcolor}{rgb}{0.8,0.1,0.5}
\usepackage{listings}
\lstset{extendedchars=false}%这一条命令可以解决代码跨页时,章节标题,页眉等汉字不显示的问题
\definecolor{darkgreen}{rgb}{0.0, 0.2, 0.13}
\definecolor{grey}{rgb}{0.55, 0.57, 0.67}
\definecolor{darkgray}{rgb}{0.66, 0.66, 0.66}
\lstset{language=C++, %用于设置语言为C++
    numbers=left,
    numberstyle=\ttfamily\scriptsize,
    backgroundcolor=\color{darkgray}, frame=single,framesep=3pt,framexleftmargin=8mm,%frameround=fttt,
    basicstyle=\ttfamily\small,
    keywordstyle=\ttfamily\bf\color{blue},
    ndkeywordstyle=\ttfamily\bf\color{brown},
    commentstyle=\color{darkgreen},
    identifierstyle=\ttfamily\color{black}\bfseries,
    stringstyle=\color{pink}\ttfamily,showstringspaces=false,
    breaklines=true,
	tabsize=4,
    escapeinside=``
}

%%%

%\hypersetup{CJKbookmarks=true} %解决section不能使用中文的问题

\usepackage[top=1in, bottom=1in, left=1.25in, right=1.25in]{geometry} %页边距

%最好保证 hyperref 是最后加载的宏包
\usepackage[pdftex,bookmarksnumbered,colorlinks,linkcolor=black,anchorcolor=blue,citecolor=green,unicode=true]{hyperref}

\begin{document}
\begin{CJK}{UTF8}{gbsn}     %CJK:支持中文

%%文章中章节等转化为中文
\renewcommand{\contentsname}{目录}
\renewcommand{\figurename}{图}
\renewcommand{\tablename}{表}
\titleformat{\chapter}{\centering\Huge\bfseries}{第\,\thechapter\,章}{1em}{}
%%

%%目录中章节
\renewcommand{\cftchapfont}{\bfseries}
\renewcommand{\cftchappagefont}{\bfseries}
\renewcommand{\cftchappresnum}{第}
\renewcommand{\cftchapaftersnum}{章:}
\renewcommand{\cftchapnumwidth}{4em}      %  add 'chapter' word before number
%%

%%布局
\pagestyle{fancy}
\renewcommand{\chaptermark}[1]{\markboth{\small 第\,\thechapter\,章\quad #1}{}}
\renewcommand{\sectionmark}[1]{\markright{\small\thesection\quad #1}{}}
\fancyhf{}
\fancyhead[ER]{\leftmark}
\fancyhead[OL]{\rightmark}
\fancyhead[EL,OR]{$\cdot$\ \thepage\ $\cdot$}
\renewcommand{\headrulewidth}{0.4pt}
%%

\title{编程辑略}
\author{ \\ \\ \\ \\ \\ \\ sosohu \\ \\ Email: ustc.sosohu@gmail.com \\ Blog: \href{https://sosohu.github.io}{sosohu.github.io}}
\date{}

\maketitle

%\newpage

\chapter*{前言}
    
\subsection{Spiral Matrix}
    
\begin{description}
    \item{\textbf{问题}}:\\
Given a matrix of m x n elements (m rows, n columns), return all elements of the matrix in spiral order.\\
\textit{(leetcode 54)}
    \item{\textbf{举例}}:\\
Given the following matrix:\\
\\
$[$ \\
 $[ 1, 2, 3 ]$, \\
 $[ 4, 5, 6 ]$, \\
 $[ 7, 8, 9 ]$ \\
$]$ \\
You should return $[1,2,3,6,9,8,7,4,5]$.
    \item{\textbf{???}} : \fbox{时间复杂度O($n^2$), 空间复杂度O(1)}
    \\从外到内一环一环的处理,需要注意一些边界条件
    \begin{lstlisting}
vector<int> spiralOrder(vector<vector<int> > &matrix) {
	vector<int> result;
	int n = matrix.size();
	if(n == 0)	return result;
	int m = matrix[0].size();
	int magrin = 0;
	while(m - 1 - magrin >= magrin && n - 1 - magrin >= magrin){
		for(int j = magrin; j <= m - 1 - magrin; j++)
			result.push_back(matrix[magrin][j]);
		for(int i = magrin + 1; i < n - 1 - magrin; i++)
			result.push_back(matrix[i][m-1-magrin]);
		if(n - 1 - magrin != magrin)
			for(int j = m - 1 - magrin; j >= magrin; j-- )
				result.push_back(matrix[n-1-magrin][j]);
		if(m - 1 - magrin != magrin)
			for(int i = n - 1 - magrin - 1; i > magrin; i--)
				result.push_back(matrix[i][magrin]);
		magrin++;
	}
	return result;
}
    \end{lstlisting}
\end{description}

\subsection{Spiral Matrix II}
    
\begin{description}
    \item{\textbf{问题}}:\\
Given an integer n, generate a square matrix filled with elements from 1 to $n^2$ in spiral order.\\
\textit{(leetcode 59)}
    \item{\textbf{举例}}:\\
Given n = 3,\\
\\
You should return the following matrix:\\
$[$ \\
 $[ 1, 2, 3 ]$, \\
 $[ 8, 9, 4 ]$, \\
 $[ 7, 6, 5 ]$ \\
$]$
    \item{\textbf{???}} : \fbox{时间复杂度O($n^2$), 空间复杂度O(1)}
    \begin{lstlisting}
vector<vector<int> > generateMatrix(int n) {
	vector<vector<int> > matrix(n, vector<int>(n, 0));
	int pos = 1;
	int magrin = 0;
	while(n - 1 - magrin >= magrin && n - 1 - magrin >= magrin){
		for(int j = magrin; j <= n - 1 - magrin; j++)
			matrix[magrin][j] = pos++;
		for(int i = magrin + 1; i < n - 1 - magrin; i++)
			matrix[i][n-1-magrin] = pos++;
		if(n - 1 - magrin != magrin)
			for(int j = n - 1 - magrin; j >= magrin; j-- )
				matrix[n-1-magrin][j] = pos++;
		if(n - 1 - magrin != magrin)
			for(int i = n - 1 - magrin - 1; i > magrin; i--)
				matrix[i][magrin] = pos++;
		magrin++;
	}
	return matrix;
}
    \end{lstlisting}
\end{description}

\subsection{Set Matrix Zeroes}
    
\begin{description}
    \item{\textbf{问题}}:\\
Given a m x n matrix, if an element is 0, set its entire row and column to 0. Do it in place.\\
\textit{(leetcode 73)}
    \item{\textbf{Follow Up}}:\\
Did you use extra space? \\
A straight forward solution using O(mn) space is probably a bad idea. \\
A simple improvement uses O(m + n) space, but still not the best solution. \\
Could you devise a constant space solution?
    \item{\textbf{???}} : \fbox{时间复杂度O($n^2$), 空间复杂度O(1)}
    \\从上到下一行一行的处理,如果上一个存在0,可以先保留上一行的现场,然后根据上一行的原来值更新本行,然后处理上一行.唯一需要注意的就是,如果本行出现新0需要更新该0所在列的上面所有行.
    \begin{lstlisting}
void setZeroes(vector<vector<int> > &matrix) {
	int n = matrix.size();
	if(n == 0)	return;
	int m = matrix[0].size();
	bool lastZero = false;
	for(int i = 0; i < n; i++){
		bool thisZero = false;
		for(int j = 0; j < m; j++){
			if(matrix[i][j] == 0){
				int up = i - 1;
				while(up >= 0){
					matrix[up][j] = 0;
					up--;
				}
				thisZero = true;
			}
			if(i > 0 && matrix[i-1][j] == 0)
				matrix[i][j] = 0;
		}
		if(lastZero){
			for(int j = 0; j < m; j++)
				matrix[i-1][j] = 0;
		}
		lastZero = thisZero;
	}
	if(lastZero){
		for(int j = 0; j < m; j++)
			matrix[n-1][j] = 0;
	}
}
    \end{lstlisting}
\end{description}

\subsection{Pascal's Triangle}
    
\begin{description}
    \item{\textbf{问题}}:\\
Given numRows, generate the first numRows of Pascal's triangle. \\
\textit{(leetcode 118)}
    \item{\textbf{举例}}:\\
Given numRows = 5, \\
Return \\
\\
$[$ \\
     $[1]$, \\
    $[1,1]$, \\
   $[1,2,1]$, \\
  $[1,3,3,1]$, \\
 $[1,4,6,4,1]$ \\
$]$
    \item{\textbf{???}} : \fbox{时间复杂度O($n^2$), 空间复杂度O(1)}
    \begin{lstlisting}
vector<vector<int> > generate(int numRows) {
	vector<vector<int> > result;
	if(numRows == 0)	return result;
	result.push_back(vector<int>{1});
	for(int i = 1; i < numRows; i++){
		vector<int> cur;
		cur.push_back(1);
		for(int j = 0; j < result[i-1].size() - 1; j++)
			cur.push_back(result[i-1][j] + result[i-1][j+1]);
		cur.push_back(1);
		result.push_back(cur);
	}
	return result;
}
    \end{lstlisting}
\end{description}

\subsection{Pascal's Triangle II}
    
\begin{description}
    \item{\textbf{问题}}:\\
Given an index k, return the kth row of the Pascal's triangle. \\
\textit{(leetcode 119)}
    \item{\textbf{举例}}:\\
Given k = 3, \\
Return $[1,3,3,1]$.
    \item{\textbf{Note}}:\\
Could you optimize your algorithm to use only O(k) extra space?
    \item{\textbf{???}} : \fbox{时间复杂度O($k^2$), 空间复杂度O(k)}
    \\这个空间复杂度可以优化,因为每次我们只需要上一行就可以产生本行,所有之前的行可以不存储
    \begin{lstlisting}
	vector<int> getRow(int rowIndex) {
		rowIndex++;
		if(rowIndex <= 0)	return vector<int>();
		vector<int> result{1};
		for(int i = 1; i < rowIndex; i++){
			vector<int> cur;
			cur.push_back(1);
			for(int j = 0; j < result.size() - 1; j++)
				cur.push_back(result[j] + result[j+1]);
			cur.push_back(1);
			result.swap(cur);
		}
		return result;
	}
    \end{lstlisting}
\end{description}



\tableofcontents

\mainmatter

\chapter{链表}
    
\subsection{Spiral Matrix}
    
\begin{description}
    \item{\textbf{问题}}:\\
Given a matrix of m x n elements (m rows, n columns), return all elements of the matrix in spiral order.\\
\textit{(leetcode 54)}
    \item{\textbf{举例}}:\\
Given the following matrix:\\
\\
$[$ \\
 $[ 1, 2, 3 ]$, \\
 $[ 4, 5, 6 ]$, \\
 $[ 7, 8, 9 ]$ \\
$]$ \\
You should return $[1,2,3,6,9,8,7,4,5]$.
    \item{\textbf{???}} : \fbox{时间复杂度O($n^2$), 空间复杂度O(1)}
    \\从外到内一环一环的处理,需要注意一些边界条件
    \begin{lstlisting}
vector<int> spiralOrder(vector<vector<int> > &matrix) {
	vector<int> result;
	int n = matrix.size();
	if(n == 0)	return result;
	int m = matrix[0].size();
	int magrin = 0;
	while(m - 1 - magrin >= magrin && n - 1 - magrin >= magrin){
		for(int j = magrin; j <= m - 1 - magrin; j++)
			result.push_back(matrix[magrin][j]);
		for(int i = magrin + 1; i < n - 1 - magrin; i++)
			result.push_back(matrix[i][m-1-magrin]);
		if(n - 1 - magrin != magrin)
			for(int j = m - 1 - magrin; j >= magrin; j-- )
				result.push_back(matrix[n-1-magrin][j]);
		if(m - 1 - magrin != magrin)
			for(int i = n - 1 - magrin - 1; i > magrin; i--)
				result.push_back(matrix[i][magrin]);
		magrin++;
	}
	return result;
}
    \end{lstlisting}
\end{description}

\subsection{Spiral Matrix II}
    
\begin{description}
    \item{\textbf{问题}}:\\
Given an integer n, generate a square matrix filled with elements from 1 to $n^2$ in spiral order.\\
\textit{(leetcode 59)}
    \item{\textbf{举例}}:\\
Given n = 3,\\
\\
You should return the following matrix:\\
$[$ \\
 $[ 1, 2, 3 ]$, \\
 $[ 8, 9, 4 ]$, \\
 $[ 7, 6, 5 ]$ \\
$]$
    \item{\textbf{???}} : \fbox{时间复杂度O($n^2$), 空间复杂度O(1)}
    \begin{lstlisting}
vector<vector<int> > generateMatrix(int n) {
	vector<vector<int> > matrix(n, vector<int>(n, 0));
	int pos = 1;
	int magrin = 0;
	while(n - 1 - magrin >= magrin && n - 1 - magrin >= magrin){
		for(int j = magrin; j <= n - 1 - magrin; j++)
			matrix[magrin][j] = pos++;
		for(int i = magrin + 1; i < n - 1 - magrin; i++)
			matrix[i][n-1-magrin] = pos++;
		if(n - 1 - magrin != magrin)
			for(int j = n - 1 - magrin; j >= magrin; j-- )
				matrix[n-1-magrin][j] = pos++;
		if(n - 1 - magrin != magrin)
			for(int i = n - 1 - magrin - 1; i > magrin; i--)
				matrix[i][magrin] = pos++;
		magrin++;
	}
	return matrix;
}
    \end{lstlisting}
\end{description}

\subsection{Set Matrix Zeroes}
    
\begin{description}
    \item{\textbf{问题}}:\\
Given a m x n matrix, if an element is 0, set its entire row and column to 0. Do it in place.\\
\textit{(leetcode 73)}
    \item{\textbf{Follow Up}}:\\
Did you use extra space? \\
A straight forward solution using O(mn) space is probably a bad idea. \\
A simple improvement uses O(m + n) space, but still not the best solution. \\
Could you devise a constant space solution?
    \item{\textbf{???}} : \fbox{时间复杂度O($n^2$), 空间复杂度O(1)}
    \\从上到下一行一行的处理,如果上一个存在0,可以先保留上一行的现场,然后根据上一行的原来值更新本行,然后处理上一行.唯一需要注意的就是,如果本行出现新0需要更新该0所在列的上面所有行.
    \begin{lstlisting}
void setZeroes(vector<vector<int> > &matrix) {
	int n = matrix.size();
	if(n == 0)	return;
	int m = matrix[0].size();
	bool lastZero = false;
	for(int i = 0; i < n; i++){
		bool thisZero = false;
		for(int j = 0; j < m; j++){
			if(matrix[i][j] == 0){
				int up = i - 1;
				while(up >= 0){
					matrix[up][j] = 0;
					up--;
				}
				thisZero = true;
			}
			if(i > 0 && matrix[i-1][j] == 0)
				matrix[i][j] = 0;
		}
		if(lastZero){
			for(int j = 0; j < m; j++)
				matrix[i-1][j] = 0;
		}
		lastZero = thisZero;
	}
	if(lastZero){
		for(int j = 0; j < m; j++)
			matrix[n-1][j] = 0;
	}
}
    \end{lstlisting}
\end{description}

\subsection{Pascal's Triangle}
    
\begin{description}
    \item{\textbf{问题}}:\\
Given numRows, generate the first numRows of Pascal's triangle. \\
\textit{(leetcode 118)}
    \item{\textbf{举例}}:\\
Given numRows = 5, \\
Return \\
\\
$[$ \\
     $[1]$, \\
    $[1,1]$, \\
   $[1,2,1]$, \\
  $[1,3,3,1]$, \\
 $[1,4,6,4,1]$ \\
$]$
    \item{\textbf{???}} : \fbox{时间复杂度O($n^2$), 空间复杂度O(1)}
    \begin{lstlisting}
vector<vector<int> > generate(int numRows) {
	vector<vector<int> > result;
	if(numRows == 0)	return result;
	result.push_back(vector<int>{1});
	for(int i = 1; i < numRows; i++){
		vector<int> cur;
		cur.push_back(1);
		for(int j = 0; j < result[i-1].size() - 1; j++)
			cur.push_back(result[i-1][j] + result[i-1][j+1]);
		cur.push_back(1);
		result.push_back(cur);
	}
	return result;
}
    \end{lstlisting}
\end{description}

\subsection{Pascal's Triangle II}
    
\begin{description}
    \item{\textbf{问题}}:\\
Given an index k, return the kth row of the Pascal's triangle. \\
\textit{(leetcode 119)}
    \item{\textbf{举例}}:\\
Given k = 3, \\
Return $[1,3,3,1]$.
    \item{\textbf{Note}}:\\
Could you optimize your algorithm to use only O(k) extra space?
    \item{\textbf{???}} : \fbox{时间复杂度O($k^2$), 空间复杂度O(k)}
    \\这个空间复杂度可以优化,因为每次我们只需要上一行就可以产生本行,所有之前的行可以不存储
    \begin{lstlisting}
	vector<int> getRow(int rowIndex) {
		rowIndex++;
		if(rowIndex <= 0)	return vector<int>();
		vector<int> result{1};
		for(int i = 1; i < rowIndex; i++){
			vector<int> cur;
			cur.push_back(1);
			for(int j = 0; j < result.size() - 1; j++)
				cur.push_back(result[j] + result[j+1]);
			cur.push_back(1);
			result.swap(cur);
		}
		return result;
	}
    \end{lstlisting}
\end{description}



\chapter{树}
    
\subsection{Spiral Matrix}
    
\begin{description}
    \item{\textbf{问题}}:\\
Given a matrix of m x n elements (m rows, n columns), return all elements of the matrix in spiral order.\\
\textit{(leetcode 54)}
    \item{\textbf{举例}}:\\
Given the following matrix:\\
\\
$[$ \\
 $[ 1, 2, 3 ]$, \\
 $[ 4, 5, 6 ]$, \\
 $[ 7, 8, 9 ]$ \\
$]$ \\
You should return $[1,2,3,6,9,8,7,4,5]$.
    \item{\textbf{???}} : \fbox{时间复杂度O($n^2$), 空间复杂度O(1)}
    \\从外到内一环一环的处理,需要注意一些边界条件
    \begin{lstlisting}
vector<int> spiralOrder(vector<vector<int> > &matrix) {
	vector<int> result;
	int n = matrix.size();
	if(n == 0)	return result;
	int m = matrix[0].size();
	int magrin = 0;
	while(m - 1 - magrin >= magrin && n - 1 - magrin >= magrin){
		for(int j = magrin; j <= m - 1 - magrin; j++)
			result.push_back(matrix[magrin][j]);
		for(int i = magrin + 1; i < n - 1 - magrin; i++)
			result.push_back(matrix[i][m-1-magrin]);
		if(n - 1 - magrin != magrin)
			for(int j = m - 1 - magrin; j >= magrin; j-- )
				result.push_back(matrix[n-1-magrin][j]);
		if(m - 1 - magrin != magrin)
			for(int i = n - 1 - magrin - 1; i > magrin; i--)
				result.push_back(matrix[i][magrin]);
		magrin++;
	}
	return result;
}
    \end{lstlisting}
\end{description}

\subsection{Spiral Matrix II}
    
\begin{description}
    \item{\textbf{问题}}:\\
Given an integer n, generate a square matrix filled with elements from 1 to $n^2$ in spiral order.\\
\textit{(leetcode 59)}
    \item{\textbf{举例}}:\\
Given n = 3,\\
\\
You should return the following matrix:\\
$[$ \\
 $[ 1, 2, 3 ]$, \\
 $[ 8, 9, 4 ]$, \\
 $[ 7, 6, 5 ]$ \\
$]$
    \item{\textbf{???}} : \fbox{时间复杂度O($n^2$), 空间复杂度O(1)}
    \begin{lstlisting}
vector<vector<int> > generateMatrix(int n) {
	vector<vector<int> > matrix(n, vector<int>(n, 0));
	int pos = 1;
	int magrin = 0;
	while(n - 1 - magrin >= magrin && n - 1 - magrin >= magrin){
		for(int j = magrin; j <= n - 1 - magrin; j++)
			matrix[magrin][j] = pos++;
		for(int i = magrin + 1; i < n - 1 - magrin; i++)
			matrix[i][n-1-magrin] = pos++;
		if(n - 1 - magrin != magrin)
			for(int j = n - 1 - magrin; j >= magrin; j-- )
				matrix[n-1-magrin][j] = pos++;
		if(n - 1 - magrin != magrin)
			for(int i = n - 1 - magrin - 1; i > magrin; i--)
				matrix[i][magrin] = pos++;
		magrin++;
	}
	return matrix;
}
    \end{lstlisting}
\end{description}

\subsection{Set Matrix Zeroes}
    
\begin{description}
    \item{\textbf{问题}}:\\
Given a m x n matrix, if an element is 0, set its entire row and column to 0. Do it in place.\\
\textit{(leetcode 73)}
    \item{\textbf{Follow Up}}:\\
Did you use extra space? \\
A straight forward solution using O(mn) space is probably a bad idea. \\
A simple improvement uses O(m + n) space, but still not the best solution. \\
Could you devise a constant space solution?
    \item{\textbf{???}} : \fbox{时间复杂度O($n^2$), 空间复杂度O(1)}
    \\从上到下一行一行的处理,如果上一个存在0,可以先保留上一行的现场,然后根据上一行的原来值更新本行,然后处理上一行.唯一需要注意的就是,如果本行出现新0需要更新该0所在列的上面所有行.
    \begin{lstlisting}
void setZeroes(vector<vector<int> > &matrix) {
	int n = matrix.size();
	if(n == 0)	return;
	int m = matrix[0].size();
	bool lastZero = false;
	for(int i = 0; i < n; i++){
		bool thisZero = false;
		for(int j = 0; j < m; j++){
			if(matrix[i][j] == 0){
				int up = i - 1;
				while(up >= 0){
					matrix[up][j] = 0;
					up--;
				}
				thisZero = true;
			}
			if(i > 0 && matrix[i-1][j] == 0)
				matrix[i][j] = 0;
		}
		if(lastZero){
			for(int j = 0; j < m; j++)
				matrix[i-1][j] = 0;
		}
		lastZero = thisZero;
	}
	if(lastZero){
		for(int j = 0; j < m; j++)
			matrix[n-1][j] = 0;
	}
}
    \end{lstlisting}
\end{description}

\subsection{Pascal's Triangle}
    
\begin{description}
    \item{\textbf{问题}}:\\
Given numRows, generate the first numRows of Pascal's triangle. \\
\textit{(leetcode 118)}
    \item{\textbf{举例}}:\\
Given numRows = 5, \\
Return \\
\\
$[$ \\
     $[1]$, \\
    $[1,1]$, \\
   $[1,2,1]$, \\
  $[1,3,3,1]$, \\
 $[1,4,6,4,1]$ \\
$]$
    \item{\textbf{???}} : \fbox{时间复杂度O($n^2$), 空间复杂度O(1)}
    \begin{lstlisting}
vector<vector<int> > generate(int numRows) {
	vector<vector<int> > result;
	if(numRows == 0)	return result;
	result.push_back(vector<int>{1});
	for(int i = 1; i < numRows; i++){
		vector<int> cur;
		cur.push_back(1);
		for(int j = 0; j < result[i-1].size() - 1; j++)
			cur.push_back(result[i-1][j] + result[i-1][j+1]);
		cur.push_back(1);
		result.push_back(cur);
	}
	return result;
}
    \end{lstlisting}
\end{description}

\subsection{Pascal's Triangle II}
    
\begin{description}
    \item{\textbf{问题}}:\\
Given an index k, return the kth row of the Pascal's triangle. \\
\textit{(leetcode 119)}
    \item{\textbf{举例}}:\\
Given k = 3, \\
Return $[1,3,3,1]$.
    \item{\textbf{Note}}:\\
Could you optimize your algorithm to use only O(k) extra space?
    \item{\textbf{???}} : \fbox{时间复杂度O($k^2$), 空间复杂度O(k)}
    \\这个空间复杂度可以优化,因为每次我们只需要上一行就可以产生本行,所有之前的行可以不存储
    \begin{lstlisting}
	vector<int> getRow(int rowIndex) {
		rowIndex++;
		if(rowIndex <= 0)	return vector<int>();
		vector<int> result{1};
		for(int i = 1; i < rowIndex; i++){
			vector<int> cur;
			cur.push_back(1);
			for(int j = 0; j < result.size() - 1; j++)
				cur.push_back(result[j] + result[j+1]);
			cur.push_back(1);
			result.swap(cur);
		}
		return result;
	}
    \end{lstlisting}
\end{description}



\chapter{字符串}
    
\subsection{Spiral Matrix}
    
\begin{description}
    \item{\textbf{问题}}:\\
Given a matrix of m x n elements (m rows, n columns), return all elements of the matrix in spiral order.\\
\textit{(leetcode 54)}
    \item{\textbf{举例}}:\\
Given the following matrix:\\
\\
$[$ \\
 $[ 1, 2, 3 ]$, \\
 $[ 4, 5, 6 ]$, \\
 $[ 7, 8, 9 ]$ \\
$]$ \\
You should return $[1,2,3,6,9,8,7,4,5]$.
    \item{\textbf{???}} : \fbox{时间复杂度O($n^2$), 空间复杂度O(1)}
    \\从外到内一环一环的处理,需要注意一些边界条件
    \begin{lstlisting}
vector<int> spiralOrder(vector<vector<int> > &matrix) {
	vector<int> result;
	int n = matrix.size();
	if(n == 0)	return result;
	int m = matrix[0].size();
	int magrin = 0;
	while(m - 1 - magrin >= magrin && n - 1 - magrin >= magrin){
		for(int j = magrin; j <= m - 1 - magrin; j++)
			result.push_back(matrix[magrin][j]);
		for(int i = magrin + 1; i < n - 1 - magrin; i++)
			result.push_back(matrix[i][m-1-magrin]);
		if(n - 1 - magrin != magrin)
			for(int j = m - 1 - magrin; j >= magrin; j-- )
				result.push_back(matrix[n-1-magrin][j]);
		if(m - 1 - magrin != magrin)
			for(int i = n - 1 - magrin - 1; i > magrin; i--)
				result.push_back(matrix[i][magrin]);
		magrin++;
	}
	return result;
}
    \end{lstlisting}
\end{description}

\subsection{Spiral Matrix II}
    
\begin{description}
    \item{\textbf{问题}}:\\
Given an integer n, generate a square matrix filled with elements from 1 to $n^2$ in spiral order.\\
\textit{(leetcode 59)}
    \item{\textbf{举例}}:\\
Given n = 3,\\
\\
You should return the following matrix:\\
$[$ \\
 $[ 1, 2, 3 ]$, \\
 $[ 8, 9, 4 ]$, \\
 $[ 7, 6, 5 ]$ \\
$]$
    \item{\textbf{???}} : \fbox{时间复杂度O($n^2$), 空间复杂度O(1)}
    \begin{lstlisting}
vector<vector<int> > generateMatrix(int n) {
	vector<vector<int> > matrix(n, vector<int>(n, 0));
	int pos = 1;
	int magrin = 0;
	while(n - 1 - magrin >= magrin && n - 1 - magrin >= magrin){
		for(int j = magrin; j <= n - 1 - magrin; j++)
			matrix[magrin][j] = pos++;
		for(int i = magrin + 1; i < n - 1 - magrin; i++)
			matrix[i][n-1-magrin] = pos++;
		if(n - 1 - magrin != magrin)
			for(int j = n - 1 - magrin; j >= magrin; j-- )
				matrix[n-1-magrin][j] = pos++;
		if(n - 1 - magrin != magrin)
			for(int i = n - 1 - magrin - 1; i > magrin; i--)
				matrix[i][magrin] = pos++;
		magrin++;
	}
	return matrix;
}
    \end{lstlisting}
\end{description}

\subsection{Set Matrix Zeroes}
    
\begin{description}
    \item{\textbf{问题}}:\\
Given a m x n matrix, if an element is 0, set its entire row and column to 0. Do it in place.\\
\textit{(leetcode 73)}
    \item{\textbf{Follow Up}}:\\
Did you use extra space? \\
A straight forward solution using O(mn) space is probably a bad idea. \\
A simple improvement uses O(m + n) space, but still not the best solution. \\
Could you devise a constant space solution?
    \item{\textbf{???}} : \fbox{时间复杂度O($n^2$), 空间复杂度O(1)}
    \\从上到下一行一行的处理,如果上一个存在0,可以先保留上一行的现场,然后根据上一行的原来值更新本行,然后处理上一行.唯一需要注意的就是,如果本行出现新0需要更新该0所在列的上面所有行.
    \begin{lstlisting}
void setZeroes(vector<vector<int> > &matrix) {
	int n = matrix.size();
	if(n == 0)	return;
	int m = matrix[0].size();
	bool lastZero = false;
	for(int i = 0; i < n; i++){
		bool thisZero = false;
		for(int j = 0; j < m; j++){
			if(matrix[i][j] == 0){
				int up = i - 1;
				while(up >= 0){
					matrix[up][j] = 0;
					up--;
				}
				thisZero = true;
			}
			if(i > 0 && matrix[i-1][j] == 0)
				matrix[i][j] = 0;
		}
		if(lastZero){
			for(int j = 0; j < m; j++)
				matrix[i-1][j] = 0;
		}
		lastZero = thisZero;
	}
	if(lastZero){
		for(int j = 0; j < m; j++)
			matrix[n-1][j] = 0;
	}
}
    \end{lstlisting}
\end{description}

\subsection{Pascal's Triangle}
    
\begin{description}
    \item{\textbf{问题}}:\\
Given numRows, generate the first numRows of Pascal's triangle. \\
\textit{(leetcode 118)}
    \item{\textbf{举例}}:\\
Given numRows = 5, \\
Return \\
\\
$[$ \\
     $[1]$, \\
    $[1,1]$, \\
   $[1,2,1]$, \\
  $[1,3,3,1]$, \\
 $[1,4,6,4,1]$ \\
$]$
    \item{\textbf{???}} : \fbox{时间复杂度O($n^2$), 空间复杂度O(1)}
    \begin{lstlisting}
vector<vector<int> > generate(int numRows) {
	vector<vector<int> > result;
	if(numRows == 0)	return result;
	result.push_back(vector<int>{1});
	for(int i = 1; i < numRows; i++){
		vector<int> cur;
		cur.push_back(1);
		for(int j = 0; j < result[i-1].size() - 1; j++)
			cur.push_back(result[i-1][j] + result[i-1][j+1]);
		cur.push_back(1);
		result.push_back(cur);
	}
	return result;
}
    \end{lstlisting}
\end{description}

\subsection{Pascal's Triangle II}
    
\begin{description}
    \item{\textbf{问题}}:\\
Given an index k, return the kth row of the Pascal's triangle. \\
\textit{(leetcode 119)}
    \item{\textbf{举例}}:\\
Given k = 3, \\
Return $[1,3,3,1]$.
    \item{\textbf{Note}}:\\
Could you optimize your algorithm to use only O(k) extra space?
    \item{\textbf{???}} : \fbox{时间复杂度O($k^2$), 空间复杂度O(k)}
    \\这个空间复杂度可以优化,因为每次我们只需要上一行就可以产生本行,所有之前的行可以不存储
    \begin{lstlisting}
	vector<int> getRow(int rowIndex) {
		rowIndex++;
		if(rowIndex <= 0)	return vector<int>();
		vector<int> result{1};
		for(int i = 1; i < rowIndex; i++){
			vector<int> cur;
			cur.push_back(1);
			for(int j = 0; j < result.size() - 1; j++)
				cur.push_back(result[j] + result[j+1]);
			cur.push_back(1);
			result.swap(cur);
		}
		return result;
	}
    \end{lstlisting}
\end{description}



\chapter{数组}
    
\subsection{Spiral Matrix}
    
\begin{description}
    \item{\textbf{问题}}:\\
Given a matrix of m x n elements (m rows, n columns), return all elements of the matrix in spiral order.\\
\textit{(leetcode 54)}
    \item{\textbf{举例}}:\\
Given the following matrix:\\
\\
$[$ \\
 $[ 1, 2, 3 ]$, \\
 $[ 4, 5, 6 ]$, \\
 $[ 7, 8, 9 ]$ \\
$]$ \\
You should return $[1,2,3,6,9,8,7,4,5]$.
    \item{\textbf{???}} : \fbox{时间复杂度O($n^2$), 空间复杂度O(1)}
    \\从外到内一环一环的处理,需要注意一些边界条件
    \begin{lstlisting}
vector<int> spiralOrder(vector<vector<int> > &matrix) {
	vector<int> result;
	int n = matrix.size();
	if(n == 0)	return result;
	int m = matrix[0].size();
	int magrin = 0;
	while(m - 1 - magrin >= magrin && n - 1 - magrin >= magrin){
		for(int j = magrin; j <= m - 1 - magrin; j++)
			result.push_back(matrix[magrin][j]);
		for(int i = magrin + 1; i < n - 1 - magrin; i++)
			result.push_back(matrix[i][m-1-magrin]);
		if(n - 1 - magrin != magrin)
			for(int j = m - 1 - magrin; j >= magrin; j-- )
				result.push_back(matrix[n-1-magrin][j]);
		if(m - 1 - magrin != magrin)
			for(int i = n - 1 - magrin - 1; i > magrin; i--)
				result.push_back(matrix[i][magrin]);
		magrin++;
	}
	return result;
}
    \end{lstlisting}
\end{description}

\subsection{Spiral Matrix II}
    
\begin{description}
    \item{\textbf{问题}}:\\
Given an integer n, generate a square matrix filled with elements from 1 to $n^2$ in spiral order.\\
\textit{(leetcode 59)}
    \item{\textbf{举例}}:\\
Given n = 3,\\
\\
You should return the following matrix:\\
$[$ \\
 $[ 1, 2, 3 ]$, \\
 $[ 8, 9, 4 ]$, \\
 $[ 7, 6, 5 ]$ \\
$]$
    \item{\textbf{???}} : \fbox{时间复杂度O($n^2$), 空间复杂度O(1)}
    \begin{lstlisting}
vector<vector<int> > generateMatrix(int n) {
	vector<vector<int> > matrix(n, vector<int>(n, 0));
	int pos = 1;
	int magrin = 0;
	while(n - 1 - magrin >= magrin && n - 1 - magrin >= magrin){
		for(int j = magrin; j <= n - 1 - magrin; j++)
			matrix[magrin][j] = pos++;
		for(int i = magrin + 1; i < n - 1 - magrin; i++)
			matrix[i][n-1-magrin] = pos++;
		if(n - 1 - magrin != magrin)
			for(int j = n - 1 - magrin; j >= magrin; j-- )
				matrix[n-1-magrin][j] = pos++;
		if(n - 1 - magrin != magrin)
			for(int i = n - 1 - magrin - 1; i > magrin; i--)
				matrix[i][magrin] = pos++;
		magrin++;
	}
	return matrix;
}
    \end{lstlisting}
\end{description}

\subsection{Set Matrix Zeroes}
    
\begin{description}
    \item{\textbf{问题}}:\\
Given a m x n matrix, if an element is 0, set its entire row and column to 0. Do it in place.\\
\textit{(leetcode 73)}
    \item{\textbf{Follow Up}}:\\
Did you use extra space? \\
A straight forward solution using O(mn) space is probably a bad idea. \\
A simple improvement uses O(m + n) space, but still not the best solution. \\
Could you devise a constant space solution?
    \item{\textbf{???}} : \fbox{时间复杂度O($n^2$), 空间复杂度O(1)}
    \\从上到下一行一行的处理,如果上一个存在0,可以先保留上一行的现场,然后根据上一行的原来值更新本行,然后处理上一行.唯一需要注意的就是,如果本行出现新0需要更新该0所在列的上面所有行.
    \begin{lstlisting}
void setZeroes(vector<vector<int> > &matrix) {
	int n = matrix.size();
	if(n == 0)	return;
	int m = matrix[0].size();
	bool lastZero = false;
	for(int i = 0; i < n; i++){
		bool thisZero = false;
		for(int j = 0; j < m; j++){
			if(matrix[i][j] == 0){
				int up = i - 1;
				while(up >= 0){
					matrix[up][j] = 0;
					up--;
				}
				thisZero = true;
			}
			if(i > 0 && matrix[i-1][j] == 0)
				matrix[i][j] = 0;
		}
		if(lastZero){
			for(int j = 0; j < m; j++)
				matrix[i-1][j] = 0;
		}
		lastZero = thisZero;
	}
	if(lastZero){
		for(int j = 0; j < m; j++)
			matrix[n-1][j] = 0;
	}
}
    \end{lstlisting}
\end{description}

\subsection{Pascal's Triangle}
    
\begin{description}
    \item{\textbf{问题}}:\\
Given numRows, generate the first numRows of Pascal's triangle. \\
\textit{(leetcode 118)}
    \item{\textbf{举例}}:\\
Given numRows = 5, \\
Return \\
\\
$[$ \\
     $[1]$, \\
    $[1,1]$, \\
   $[1,2,1]$, \\
  $[1,3,3,1]$, \\
 $[1,4,6,4,1]$ \\
$]$
    \item{\textbf{???}} : \fbox{时间复杂度O($n^2$), 空间复杂度O(1)}
    \begin{lstlisting}
vector<vector<int> > generate(int numRows) {
	vector<vector<int> > result;
	if(numRows == 0)	return result;
	result.push_back(vector<int>{1});
	for(int i = 1; i < numRows; i++){
		vector<int> cur;
		cur.push_back(1);
		for(int j = 0; j < result[i-1].size() - 1; j++)
			cur.push_back(result[i-1][j] + result[i-1][j+1]);
		cur.push_back(1);
		result.push_back(cur);
	}
	return result;
}
    \end{lstlisting}
\end{description}

\subsection{Pascal's Triangle II}
    
\begin{description}
    \item{\textbf{问题}}:\\
Given an index k, return the kth row of the Pascal's triangle. \\
\textit{(leetcode 119)}
    \item{\textbf{举例}}:\\
Given k = 3, \\
Return $[1,3,3,1]$.
    \item{\textbf{Note}}:\\
Could you optimize your algorithm to use only O(k) extra space?
    \item{\textbf{???}} : \fbox{时间复杂度O($k^2$), 空间复杂度O(k)}
    \\这个空间复杂度可以优化,因为每次我们只需要上一行就可以产生本行,所有之前的行可以不存储
    \begin{lstlisting}
	vector<int> getRow(int rowIndex) {
		rowIndex++;
		if(rowIndex <= 0)	return vector<int>();
		vector<int> result{1};
		for(int i = 1; i < rowIndex; i++){
			vector<int> cur;
			cur.push_back(1);
			for(int j = 0; j < result.size() - 1; j++)
				cur.push_back(result[j] + result[j+1]);
			cur.push_back(1);
			result.swap(cur);
		}
		return result;
	}
    \end{lstlisting}
\end{description}



\chapter{栈和队列}
    
\subsection{Spiral Matrix}
    
\begin{description}
    \item{\textbf{问题}}:\\
Given a matrix of m x n elements (m rows, n columns), return all elements of the matrix in spiral order.\\
\textit{(leetcode 54)}
    \item{\textbf{举例}}:\\
Given the following matrix:\\
\\
$[$ \\
 $[ 1, 2, 3 ]$, \\
 $[ 4, 5, 6 ]$, \\
 $[ 7, 8, 9 ]$ \\
$]$ \\
You should return $[1,2,3,6,9,8,7,4,5]$.
    \item{\textbf{???}} : \fbox{时间复杂度O($n^2$), 空间复杂度O(1)}
    \\从外到内一环一环的处理,需要注意一些边界条件
    \begin{lstlisting}
vector<int> spiralOrder(vector<vector<int> > &matrix) {
	vector<int> result;
	int n = matrix.size();
	if(n == 0)	return result;
	int m = matrix[0].size();
	int magrin = 0;
	while(m - 1 - magrin >= magrin && n - 1 - magrin >= magrin){
		for(int j = magrin; j <= m - 1 - magrin; j++)
			result.push_back(matrix[magrin][j]);
		for(int i = magrin + 1; i < n - 1 - magrin; i++)
			result.push_back(matrix[i][m-1-magrin]);
		if(n - 1 - magrin != magrin)
			for(int j = m - 1 - magrin; j >= magrin; j-- )
				result.push_back(matrix[n-1-magrin][j]);
		if(m - 1 - magrin != magrin)
			for(int i = n - 1 - magrin - 1; i > magrin; i--)
				result.push_back(matrix[i][magrin]);
		magrin++;
	}
	return result;
}
    \end{lstlisting}
\end{description}

\subsection{Spiral Matrix II}
    
\begin{description}
    \item{\textbf{问题}}:\\
Given an integer n, generate a square matrix filled with elements from 1 to $n^2$ in spiral order.\\
\textit{(leetcode 59)}
    \item{\textbf{举例}}:\\
Given n = 3,\\
\\
You should return the following matrix:\\
$[$ \\
 $[ 1, 2, 3 ]$, \\
 $[ 8, 9, 4 ]$, \\
 $[ 7, 6, 5 ]$ \\
$]$
    \item{\textbf{???}} : \fbox{时间复杂度O($n^2$), 空间复杂度O(1)}
    \begin{lstlisting}
vector<vector<int> > generateMatrix(int n) {
	vector<vector<int> > matrix(n, vector<int>(n, 0));
	int pos = 1;
	int magrin = 0;
	while(n - 1 - magrin >= magrin && n - 1 - magrin >= magrin){
		for(int j = magrin; j <= n - 1 - magrin; j++)
			matrix[magrin][j] = pos++;
		for(int i = magrin + 1; i < n - 1 - magrin; i++)
			matrix[i][n-1-magrin] = pos++;
		if(n - 1 - magrin != magrin)
			for(int j = n - 1 - magrin; j >= magrin; j-- )
				matrix[n-1-magrin][j] = pos++;
		if(n - 1 - magrin != magrin)
			for(int i = n - 1 - magrin - 1; i > magrin; i--)
				matrix[i][magrin] = pos++;
		magrin++;
	}
	return matrix;
}
    \end{lstlisting}
\end{description}

\subsection{Set Matrix Zeroes}
    
\begin{description}
    \item{\textbf{问题}}:\\
Given a m x n matrix, if an element is 0, set its entire row and column to 0. Do it in place.\\
\textit{(leetcode 73)}
    \item{\textbf{Follow Up}}:\\
Did you use extra space? \\
A straight forward solution using O(mn) space is probably a bad idea. \\
A simple improvement uses O(m + n) space, but still not the best solution. \\
Could you devise a constant space solution?
    \item{\textbf{???}} : \fbox{时间复杂度O($n^2$), 空间复杂度O(1)}
    \\从上到下一行一行的处理,如果上一个存在0,可以先保留上一行的现场,然后根据上一行的原来值更新本行,然后处理上一行.唯一需要注意的就是,如果本行出现新0需要更新该0所在列的上面所有行.
    \begin{lstlisting}
void setZeroes(vector<vector<int> > &matrix) {
	int n = matrix.size();
	if(n == 0)	return;
	int m = matrix[0].size();
	bool lastZero = false;
	for(int i = 0; i < n; i++){
		bool thisZero = false;
		for(int j = 0; j < m; j++){
			if(matrix[i][j] == 0){
				int up = i - 1;
				while(up >= 0){
					matrix[up][j] = 0;
					up--;
				}
				thisZero = true;
			}
			if(i > 0 && matrix[i-1][j] == 0)
				matrix[i][j] = 0;
		}
		if(lastZero){
			for(int j = 0; j < m; j++)
				matrix[i-1][j] = 0;
		}
		lastZero = thisZero;
	}
	if(lastZero){
		for(int j = 0; j < m; j++)
			matrix[n-1][j] = 0;
	}
}
    \end{lstlisting}
\end{description}

\subsection{Pascal's Triangle}
    
\begin{description}
    \item{\textbf{问题}}:\\
Given numRows, generate the first numRows of Pascal's triangle. \\
\textit{(leetcode 118)}
    \item{\textbf{举例}}:\\
Given numRows = 5, \\
Return \\
\\
$[$ \\
     $[1]$, \\
    $[1,1]$, \\
   $[1,2,1]$, \\
  $[1,3,3,1]$, \\
 $[1,4,6,4,1]$ \\
$]$
    \item{\textbf{???}} : \fbox{时间复杂度O($n^2$), 空间复杂度O(1)}
    \begin{lstlisting}
vector<vector<int> > generate(int numRows) {
	vector<vector<int> > result;
	if(numRows == 0)	return result;
	result.push_back(vector<int>{1});
	for(int i = 1; i < numRows; i++){
		vector<int> cur;
		cur.push_back(1);
		for(int j = 0; j < result[i-1].size() - 1; j++)
			cur.push_back(result[i-1][j] + result[i-1][j+1]);
		cur.push_back(1);
		result.push_back(cur);
	}
	return result;
}
    \end{lstlisting}
\end{description}

\subsection{Pascal's Triangle II}
    
\begin{description}
    \item{\textbf{问题}}:\\
Given an index k, return the kth row of the Pascal's triangle. \\
\textit{(leetcode 119)}
    \item{\textbf{举例}}:\\
Given k = 3, \\
Return $[1,3,3,1]$.
    \item{\textbf{Note}}:\\
Could you optimize your algorithm to use only O(k) extra space?
    \item{\textbf{???}} : \fbox{时间复杂度O($k^2$), 空间复杂度O(k)}
    \\这个空间复杂度可以优化,因为每次我们只需要上一行就可以产生本行,所有之前的行可以不存储
    \begin{lstlisting}
	vector<int> getRow(int rowIndex) {
		rowIndex++;
		if(rowIndex <= 0)	return vector<int>();
		vector<int> result{1};
		for(int i = 1; i < rowIndex; i++){
			vector<int> cur;
			cur.push_back(1);
			for(int j = 0; j < result.size() - 1; j++)
				cur.push_back(result[j] + result[j+1]);
			cur.push_back(1);
			result.swap(cur);
		}
		return result;
	}
    \end{lstlisting}
\end{description}



\chapter{图}
    
\subsection{Spiral Matrix}
    
\begin{description}
    \item{\textbf{问题}}:\\
Given a matrix of m x n elements (m rows, n columns), return all elements of the matrix in spiral order.\\
\textit{(leetcode 54)}
    \item{\textbf{举例}}:\\
Given the following matrix:\\
\\
$[$ \\
 $[ 1, 2, 3 ]$, \\
 $[ 4, 5, 6 ]$, \\
 $[ 7, 8, 9 ]$ \\
$]$ \\
You should return $[1,2,3,6,9,8,7,4,5]$.
    \item{\textbf{???}} : \fbox{时间复杂度O($n^2$), 空间复杂度O(1)}
    \\从外到内一环一环的处理,需要注意一些边界条件
    \begin{lstlisting}
vector<int> spiralOrder(vector<vector<int> > &matrix) {
	vector<int> result;
	int n = matrix.size();
	if(n == 0)	return result;
	int m = matrix[0].size();
	int magrin = 0;
	while(m - 1 - magrin >= magrin && n - 1 - magrin >= magrin){
		for(int j = magrin; j <= m - 1 - magrin; j++)
			result.push_back(matrix[magrin][j]);
		for(int i = magrin + 1; i < n - 1 - magrin; i++)
			result.push_back(matrix[i][m-1-magrin]);
		if(n - 1 - magrin != magrin)
			for(int j = m - 1 - magrin; j >= magrin; j-- )
				result.push_back(matrix[n-1-magrin][j]);
		if(m - 1 - magrin != magrin)
			for(int i = n - 1 - magrin - 1; i > magrin; i--)
				result.push_back(matrix[i][magrin]);
		magrin++;
	}
	return result;
}
    \end{lstlisting}
\end{description}

\subsection{Spiral Matrix II}
    
\begin{description}
    \item{\textbf{问题}}:\\
Given an integer n, generate a square matrix filled with elements from 1 to $n^2$ in spiral order.\\
\textit{(leetcode 59)}
    \item{\textbf{举例}}:\\
Given n = 3,\\
\\
You should return the following matrix:\\
$[$ \\
 $[ 1, 2, 3 ]$, \\
 $[ 8, 9, 4 ]$, \\
 $[ 7, 6, 5 ]$ \\
$]$
    \item{\textbf{???}} : \fbox{时间复杂度O($n^2$), 空间复杂度O(1)}
    \begin{lstlisting}
vector<vector<int> > generateMatrix(int n) {
	vector<vector<int> > matrix(n, vector<int>(n, 0));
	int pos = 1;
	int magrin = 0;
	while(n - 1 - magrin >= magrin && n - 1 - magrin >= magrin){
		for(int j = magrin; j <= n - 1 - magrin; j++)
			matrix[magrin][j] = pos++;
		for(int i = magrin + 1; i < n - 1 - magrin; i++)
			matrix[i][n-1-magrin] = pos++;
		if(n - 1 - magrin != magrin)
			for(int j = n - 1 - magrin; j >= magrin; j-- )
				matrix[n-1-magrin][j] = pos++;
		if(n - 1 - magrin != magrin)
			for(int i = n - 1 - magrin - 1; i > magrin; i--)
				matrix[i][magrin] = pos++;
		magrin++;
	}
	return matrix;
}
    \end{lstlisting}
\end{description}

\subsection{Set Matrix Zeroes}
    
\begin{description}
    \item{\textbf{问题}}:\\
Given a m x n matrix, if an element is 0, set its entire row and column to 0. Do it in place.\\
\textit{(leetcode 73)}
    \item{\textbf{Follow Up}}:\\
Did you use extra space? \\
A straight forward solution using O(mn) space is probably a bad idea. \\
A simple improvement uses O(m + n) space, but still not the best solution. \\
Could you devise a constant space solution?
    \item{\textbf{???}} : \fbox{时间复杂度O($n^2$), 空间复杂度O(1)}
    \\从上到下一行一行的处理,如果上一个存在0,可以先保留上一行的现场,然后根据上一行的原来值更新本行,然后处理上一行.唯一需要注意的就是,如果本行出现新0需要更新该0所在列的上面所有行.
    \begin{lstlisting}
void setZeroes(vector<vector<int> > &matrix) {
	int n = matrix.size();
	if(n == 0)	return;
	int m = matrix[0].size();
	bool lastZero = false;
	for(int i = 0; i < n; i++){
		bool thisZero = false;
		for(int j = 0; j < m; j++){
			if(matrix[i][j] == 0){
				int up = i - 1;
				while(up >= 0){
					matrix[up][j] = 0;
					up--;
				}
				thisZero = true;
			}
			if(i > 0 && matrix[i-1][j] == 0)
				matrix[i][j] = 0;
		}
		if(lastZero){
			for(int j = 0; j < m; j++)
				matrix[i-1][j] = 0;
		}
		lastZero = thisZero;
	}
	if(lastZero){
		for(int j = 0; j < m; j++)
			matrix[n-1][j] = 0;
	}
}
    \end{lstlisting}
\end{description}

\subsection{Pascal's Triangle}
    
\begin{description}
    \item{\textbf{问题}}:\\
Given numRows, generate the first numRows of Pascal's triangle. \\
\textit{(leetcode 118)}
    \item{\textbf{举例}}:\\
Given numRows = 5, \\
Return \\
\\
$[$ \\
     $[1]$, \\
    $[1,1]$, \\
   $[1,2,1]$, \\
  $[1,3,3,1]$, \\
 $[1,4,6,4,1]$ \\
$]$
    \item{\textbf{???}} : \fbox{时间复杂度O($n^2$), 空间复杂度O(1)}
    \begin{lstlisting}
vector<vector<int> > generate(int numRows) {
	vector<vector<int> > result;
	if(numRows == 0)	return result;
	result.push_back(vector<int>{1});
	for(int i = 1; i < numRows; i++){
		vector<int> cur;
		cur.push_back(1);
		for(int j = 0; j < result[i-1].size() - 1; j++)
			cur.push_back(result[i-1][j] + result[i-1][j+1]);
		cur.push_back(1);
		result.push_back(cur);
	}
	return result;
}
    \end{lstlisting}
\end{description}

\subsection{Pascal's Triangle II}
    
\begin{description}
    \item{\textbf{问题}}:\\
Given an index k, return the kth row of the Pascal's triangle. \\
\textit{(leetcode 119)}
    \item{\textbf{举例}}:\\
Given k = 3, \\
Return $[1,3,3,1]$.
    \item{\textbf{Note}}:\\
Could you optimize your algorithm to use only O(k) extra space?
    \item{\textbf{???}} : \fbox{时间复杂度O($k^2$), 空间复杂度O(k)}
    \\这个空间复杂度可以优化,因为每次我们只需要上一行就可以产生本行,所有之前的行可以不存储
    \begin{lstlisting}
	vector<int> getRow(int rowIndex) {
		rowIndex++;
		if(rowIndex <= 0)	return vector<int>();
		vector<int> result{1};
		for(int i = 1; i < rowIndex; i++){
			vector<int> cur;
			cur.push_back(1);
			for(int j = 0; j < result.size() - 1; j++)
				cur.push_back(result[j] + result[j+1]);
			cur.push_back(1);
			result.swap(cur);
		}
		return result;
	}
    \end{lstlisting}
\end{description}



\chapter{哈希}
    
\subsection{Spiral Matrix}
    
\begin{description}
    \item{\textbf{问题}}:\\
Given a matrix of m x n elements (m rows, n columns), return all elements of the matrix in spiral order.\\
\textit{(leetcode 54)}
    \item{\textbf{举例}}:\\
Given the following matrix:\\
\\
$[$ \\
 $[ 1, 2, 3 ]$, \\
 $[ 4, 5, 6 ]$, \\
 $[ 7, 8, 9 ]$ \\
$]$ \\
You should return $[1,2,3,6,9,8,7,4,5]$.
    \item{\textbf{???}} : \fbox{时间复杂度O($n^2$), 空间复杂度O(1)}
    \\从外到内一环一环的处理,需要注意一些边界条件
    \begin{lstlisting}
vector<int> spiralOrder(vector<vector<int> > &matrix) {
	vector<int> result;
	int n = matrix.size();
	if(n == 0)	return result;
	int m = matrix[0].size();
	int magrin = 0;
	while(m - 1 - magrin >= magrin && n - 1 - magrin >= magrin){
		for(int j = magrin; j <= m - 1 - magrin; j++)
			result.push_back(matrix[magrin][j]);
		for(int i = magrin + 1; i < n - 1 - magrin; i++)
			result.push_back(matrix[i][m-1-magrin]);
		if(n - 1 - magrin != magrin)
			for(int j = m - 1 - magrin; j >= magrin; j-- )
				result.push_back(matrix[n-1-magrin][j]);
		if(m - 1 - magrin != magrin)
			for(int i = n - 1 - magrin - 1; i > magrin; i--)
				result.push_back(matrix[i][magrin]);
		magrin++;
	}
	return result;
}
    \end{lstlisting}
\end{description}

\subsection{Spiral Matrix II}
    
\begin{description}
    \item{\textbf{问题}}:\\
Given an integer n, generate a square matrix filled with elements from 1 to $n^2$ in spiral order.\\
\textit{(leetcode 59)}
    \item{\textbf{举例}}:\\
Given n = 3,\\
\\
You should return the following matrix:\\
$[$ \\
 $[ 1, 2, 3 ]$, \\
 $[ 8, 9, 4 ]$, \\
 $[ 7, 6, 5 ]$ \\
$]$
    \item{\textbf{???}} : \fbox{时间复杂度O($n^2$), 空间复杂度O(1)}
    \begin{lstlisting}
vector<vector<int> > generateMatrix(int n) {
	vector<vector<int> > matrix(n, vector<int>(n, 0));
	int pos = 1;
	int magrin = 0;
	while(n - 1 - magrin >= magrin && n - 1 - magrin >= magrin){
		for(int j = magrin; j <= n - 1 - magrin; j++)
			matrix[magrin][j] = pos++;
		for(int i = magrin + 1; i < n - 1 - magrin; i++)
			matrix[i][n-1-magrin] = pos++;
		if(n - 1 - magrin != magrin)
			for(int j = n - 1 - magrin; j >= magrin; j-- )
				matrix[n-1-magrin][j] = pos++;
		if(n - 1 - magrin != magrin)
			for(int i = n - 1 - magrin - 1; i > magrin; i--)
				matrix[i][magrin] = pos++;
		magrin++;
	}
	return matrix;
}
    \end{lstlisting}
\end{description}

\subsection{Set Matrix Zeroes}
    
\begin{description}
    \item{\textbf{问题}}:\\
Given a m x n matrix, if an element is 0, set its entire row and column to 0. Do it in place.\\
\textit{(leetcode 73)}
    \item{\textbf{Follow Up}}:\\
Did you use extra space? \\
A straight forward solution using O(mn) space is probably a bad idea. \\
A simple improvement uses O(m + n) space, but still not the best solution. \\
Could you devise a constant space solution?
    \item{\textbf{???}} : \fbox{时间复杂度O($n^2$), 空间复杂度O(1)}
    \\从上到下一行一行的处理,如果上一个存在0,可以先保留上一行的现场,然后根据上一行的原来值更新本行,然后处理上一行.唯一需要注意的就是,如果本行出现新0需要更新该0所在列的上面所有行.
    \begin{lstlisting}
void setZeroes(vector<vector<int> > &matrix) {
	int n = matrix.size();
	if(n == 0)	return;
	int m = matrix[0].size();
	bool lastZero = false;
	for(int i = 0; i < n; i++){
		bool thisZero = false;
		for(int j = 0; j < m; j++){
			if(matrix[i][j] == 0){
				int up = i - 1;
				while(up >= 0){
					matrix[up][j] = 0;
					up--;
				}
				thisZero = true;
			}
			if(i > 0 && matrix[i-1][j] == 0)
				matrix[i][j] = 0;
		}
		if(lastZero){
			for(int j = 0; j < m; j++)
				matrix[i-1][j] = 0;
		}
		lastZero = thisZero;
	}
	if(lastZero){
		for(int j = 0; j < m; j++)
			matrix[n-1][j] = 0;
	}
}
    \end{lstlisting}
\end{description}

\subsection{Pascal's Triangle}
    
\begin{description}
    \item{\textbf{问题}}:\\
Given numRows, generate the first numRows of Pascal's triangle. \\
\textit{(leetcode 118)}
    \item{\textbf{举例}}:\\
Given numRows = 5, \\
Return \\
\\
$[$ \\
     $[1]$, \\
    $[1,1]$, \\
   $[1,2,1]$, \\
  $[1,3,3,1]$, \\
 $[1,4,6,4,1]$ \\
$]$
    \item{\textbf{???}} : \fbox{时间复杂度O($n^2$), 空间复杂度O(1)}
    \begin{lstlisting}
vector<vector<int> > generate(int numRows) {
	vector<vector<int> > result;
	if(numRows == 0)	return result;
	result.push_back(vector<int>{1});
	for(int i = 1; i < numRows; i++){
		vector<int> cur;
		cur.push_back(1);
		for(int j = 0; j < result[i-1].size() - 1; j++)
			cur.push_back(result[i-1][j] + result[i-1][j+1]);
		cur.push_back(1);
		result.push_back(cur);
	}
	return result;
}
    \end{lstlisting}
\end{description}

\subsection{Pascal's Triangle II}
    
\begin{description}
    \item{\textbf{问题}}:\\
Given an index k, return the kth row of the Pascal's triangle. \\
\textit{(leetcode 119)}
    \item{\textbf{举例}}:\\
Given k = 3, \\
Return $[1,3,3,1]$.
    \item{\textbf{Note}}:\\
Could you optimize your algorithm to use only O(k) extra space?
    \item{\textbf{???}} : \fbox{时间复杂度O($k^2$), 空间复杂度O(k)}
    \\这个空间复杂度可以优化,因为每次我们只需要上一行就可以产生本行,所有之前的行可以不存储
    \begin{lstlisting}
	vector<int> getRow(int rowIndex) {
		rowIndex++;
		if(rowIndex <= 0)	return vector<int>();
		vector<int> result{1};
		for(int i = 1; i < rowIndex; i++){
			vector<int> cur;
			cur.push_back(1);
			for(int j = 0; j < result.size() - 1; j++)
				cur.push_back(result[j] + result[j+1]);
			cur.push_back(1);
			result.swap(cur);
		}
		return result;
	}
    \end{lstlisting}
\end{description}



\chapter{其他数据结构}
    
\subsection{Spiral Matrix}
    
\begin{description}
    \item{\textbf{问题}}:\\
Given a matrix of m x n elements (m rows, n columns), return all elements of the matrix in spiral order.\\
\textit{(leetcode 54)}
    \item{\textbf{举例}}:\\
Given the following matrix:\\
\\
$[$ \\
 $[ 1, 2, 3 ]$, \\
 $[ 4, 5, 6 ]$, \\
 $[ 7, 8, 9 ]$ \\
$]$ \\
You should return $[1,2,3,6,9,8,7,4,5]$.
    \item{\textbf{???}} : \fbox{时间复杂度O($n^2$), 空间复杂度O(1)}
    \\从外到内一环一环的处理,需要注意一些边界条件
    \begin{lstlisting}
vector<int> spiralOrder(vector<vector<int> > &matrix) {
	vector<int> result;
	int n = matrix.size();
	if(n == 0)	return result;
	int m = matrix[0].size();
	int magrin = 0;
	while(m - 1 - magrin >= magrin && n - 1 - magrin >= magrin){
		for(int j = magrin; j <= m - 1 - magrin; j++)
			result.push_back(matrix[magrin][j]);
		for(int i = magrin + 1; i < n - 1 - magrin; i++)
			result.push_back(matrix[i][m-1-magrin]);
		if(n - 1 - magrin != magrin)
			for(int j = m - 1 - magrin; j >= magrin; j-- )
				result.push_back(matrix[n-1-magrin][j]);
		if(m - 1 - magrin != magrin)
			for(int i = n - 1 - magrin - 1; i > magrin; i--)
				result.push_back(matrix[i][magrin]);
		magrin++;
	}
	return result;
}
    \end{lstlisting}
\end{description}

\subsection{Spiral Matrix II}
    
\begin{description}
    \item{\textbf{问题}}:\\
Given an integer n, generate a square matrix filled with elements from 1 to $n^2$ in spiral order.\\
\textit{(leetcode 59)}
    \item{\textbf{举例}}:\\
Given n = 3,\\
\\
You should return the following matrix:\\
$[$ \\
 $[ 1, 2, 3 ]$, \\
 $[ 8, 9, 4 ]$, \\
 $[ 7, 6, 5 ]$ \\
$]$
    \item{\textbf{???}} : \fbox{时间复杂度O($n^2$), 空间复杂度O(1)}
    \begin{lstlisting}
vector<vector<int> > generateMatrix(int n) {
	vector<vector<int> > matrix(n, vector<int>(n, 0));
	int pos = 1;
	int magrin = 0;
	while(n - 1 - magrin >= magrin && n - 1 - magrin >= magrin){
		for(int j = magrin; j <= n - 1 - magrin; j++)
			matrix[magrin][j] = pos++;
		for(int i = magrin + 1; i < n - 1 - magrin; i++)
			matrix[i][n-1-magrin] = pos++;
		if(n - 1 - magrin != magrin)
			for(int j = n - 1 - magrin; j >= magrin; j-- )
				matrix[n-1-magrin][j] = pos++;
		if(n - 1 - magrin != magrin)
			for(int i = n - 1 - magrin - 1; i > magrin; i--)
				matrix[i][magrin] = pos++;
		magrin++;
	}
	return matrix;
}
    \end{lstlisting}
\end{description}

\subsection{Set Matrix Zeroes}
    
\begin{description}
    \item{\textbf{问题}}:\\
Given a m x n matrix, if an element is 0, set its entire row and column to 0. Do it in place.\\
\textit{(leetcode 73)}
    \item{\textbf{Follow Up}}:\\
Did you use extra space? \\
A straight forward solution using O(mn) space is probably a bad idea. \\
A simple improvement uses O(m + n) space, but still not the best solution. \\
Could you devise a constant space solution?
    \item{\textbf{???}} : \fbox{时间复杂度O($n^2$), 空间复杂度O(1)}
    \\从上到下一行一行的处理,如果上一个存在0,可以先保留上一行的现场,然后根据上一行的原来值更新本行,然后处理上一行.唯一需要注意的就是,如果本行出现新0需要更新该0所在列的上面所有行.
    \begin{lstlisting}
void setZeroes(vector<vector<int> > &matrix) {
	int n = matrix.size();
	if(n == 0)	return;
	int m = matrix[0].size();
	bool lastZero = false;
	for(int i = 0; i < n; i++){
		bool thisZero = false;
		for(int j = 0; j < m; j++){
			if(matrix[i][j] == 0){
				int up = i - 1;
				while(up >= 0){
					matrix[up][j] = 0;
					up--;
				}
				thisZero = true;
			}
			if(i > 0 && matrix[i-1][j] == 0)
				matrix[i][j] = 0;
		}
		if(lastZero){
			for(int j = 0; j < m; j++)
				matrix[i-1][j] = 0;
		}
		lastZero = thisZero;
	}
	if(lastZero){
		for(int j = 0; j < m; j++)
			matrix[n-1][j] = 0;
	}
}
    \end{lstlisting}
\end{description}

\subsection{Pascal's Triangle}
    
\begin{description}
    \item{\textbf{问题}}:\\
Given numRows, generate the first numRows of Pascal's triangle. \\
\textit{(leetcode 118)}
    \item{\textbf{举例}}:\\
Given numRows = 5, \\
Return \\
\\
$[$ \\
     $[1]$, \\
    $[1,1]$, \\
   $[1,2,1]$, \\
  $[1,3,3,1]$, \\
 $[1,4,6,4,1]$ \\
$]$
    \item{\textbf{???}} : \fbox{时间复杂度O($n^2$), 空间复杂度O(1)}
    \begin{lstlisting}
vector<vector<int> > generate(int numRows) {
	vector<vector<int> > result;
	if(numRows == 0)	return result;
	result.push_back(vector<int>{1});
	for(int i = 1; i < numRows; i++){
		vector<int> cur;
		cur.push_back(1);
		for(int j = 0; j < result[i-1].size() - 1; j++)
			cur.push_back(result[i-1][j] + result[i-1][j+1]);
		cur.push_back(1);
		result.push_back(cur);
	}
	return result;
}
    \end{lstlisting}
\end{description}

\subsection{Pascal's Triangle II}
    
\begin{description}
    \item{\textbf{问题}}:\\
Given an index k, return the kth row of the Pascal's triangle. \\
\textit{(leetcode 119)}
    \item{\textbf{举例}}:\\
Given k = 3, \\
Return $[1,3,3,1]$.
    \item{\textbf{Note}}:\\
Could you optimize your algorithm to use only O(k) extra space?
    \item{\textbf{???}} : \fbox{时间复杂度O($k^2$), 空间复杂度O(k)}
    \\这个空间复杂度可以优化,因为每次我们只需要上一行就可以产生本行,所有之前的行可以不存储
    \begin{lstlisting}
	vector<int> getRow(int rowIndex) {
		rowIndex++;
		if(rowIndex <= 0)	return vector<int>();
		vector<int> result{1};
		for(int i = 1; i < rowIndex; i++){
			vector<int> cur;
			cur.push_back(1);
			for(int j = 0; j < result.size() - 1; j++)
				cur.push_back(result[j] + result[j+1]);
			cur.push_back(1);
			result.swap(cur);
		}
		return result;
	}
    \end{lstlisting}
\end{description}



\chapter{排序}
    
\subsection{Spiral Matrix}
    
\begin{description}
    \item{\textbf{问题}}:\\
Given a matrix of m x n elements (m rows, n columns), return all elements of the matrix in spiral order.\\
\textit{(leetcode 54)}
    \item{\textbf{举例}}:\\
Given the following matrix:\\
\\
$[$ \\
 $[ 1, 2, 3 ]$, \\
 $[ 4, 5, 6 ]$, \\
 $[ 7, 8, 9 ]$ \\
$]$ \\
You should return $[1,2,3,6,9,8,7,4,5]$.
    \item{\textbf{???}} : \fbox{时间复杂度O($n^2$), 空间复杂度O(1)}
    \\从外到内一环一环的处理,需要注意一些边界条件
    \begin{lstlisting}
vector<int> spiralOrder(vector<vector<int> > &matrix) {
	vector<int> result;
	int n = matrix.size();
	if(n == 0)	return result;
	int m = matrix[0].size();
	int magrin = 0;
	while(m - 1 - magrin >= magrin && n - 1 - magrin >= magrin){
		for(int j = magrin; j <= m - 1 - magrin; j++)
			result.push_back(matrix[magrin][j]);
		for(int i = magrin + 1; i < n - 1 - magrin; i++)
			result.push_back(matrix[i][m-1-magrin]);
		if(n - 1 - magrin != magrin)
			for(int j = m - 1 - magrin; j >= magrin; j-- )
				result.push_back(matrix[n-1-magrin][j]);
		if(m - 1 - magrin != magrin)
			for(int i = n - 1 - magrin - 1; i > magrin; i--)
				result.push_back(matrix[i][magrin]);
		magrin++;
	}
	return result;
}
    \end{lstlisting}
\end{description}

\subsection{Spiral Matrix II}
    
\begin{description}
    \item{\textbf{问题}}:\\
Given an integer n, generate a square matrix filled with elements from 1 to $n^2$ in spiral order.\\
\textit{(leetcode 59)}
    \item{\textbf{举例}}:\\
Given n = 3,\\
\\
You should return the following matrix:\\
$[$ \\
 $[ 1, 2, 3 ]$, \\
 $[ 8, 9, 4 ]$, \\
 $[ 7, 6, 5 ]$ \\
$]$
    \item{\textbf{???}} : \fbox{时间复杂度O($n^2$), 空间复杂度O(1)}
    \begin{lstlisting}
vector<vector<int> > generateMatrix(int n) {
	vector<vector<int> > matrix(n, vector<int>(n, 0));
	int pos = 1;
	int magrin = 0;
	while(n - 1 - magrin >= magrin && n - 1 - magrin >= magrin){
		for(int j = magrin; j <= n - 1 - magrin; j++)
			matrix[magrin][j] = pos++;
		for(int i = magrin + 1; i < n - 1 - magrin; i++)
			matrix[i][n-1-magrin] = pos++;
		if(n - 1 - magrin != magrin)
			for(int j = n - 1 - magrin; j >= magrin; j-- )
				matrix[n-1-magrin][j] = pos++;
		if(n - 1 - magrin != magrin)
			for(int i = n - 1 - magrin - 1; i > magrin; i--)
				matrix[i][magrin] = pos++;
		magrin++;
	}
	return matrix;
}
    \end{lstlisting}
\end{description}

\subsection{Set Matrix Zeroes}
    
\begin{description}
    \item{\textbf{问题}}:\\
Given a m x n matrix, if an element is 0, set its entire row and column to 0. Do it in place.\\
\textit{(leetcode 73)}
    \item{\textbf{Follow Up}}:\\
Did you use extra space? \\
A straight forward solution using O(mn) space is probably a bad idea. \\
A simple improvement uses O(m + n) space, but still not the best solution. \\
Could you devise a constant space solution?
    \item{\textbf{???}} : \fbox{时间复杂度O($n^2$), 空间复杂度O(1)}
    \\从上到下一行一行的处理,如果上一个存在0,可以先保留上一行的现场,然后根据上一行的原来值更新本行,然后处理上一行.唯一需要注意的就是,如果本行出现新0需要更新该0所在列的上面所有行.
    \begin{lstlisting}
void setZeroes(vector<vector<int> > &matrix) {
	int n = matrix.size();
	if(n == 0)	return;
	int m = matrix[0].size();
	bool lastZero = false;
	for(int i = 0; i < n; i++){
		bool thisZero = false;
		for(int j = 0; j < m; j++){
			if(matrix[i][j] == 0){
				int up = i - 1;
				while(up >= 0){
					matrix[up][j] = 0;
					up--;
				}
				thisZero = true;
			}
			if(i > 0 && matrix[i-1][j] == 0)
				matrix[i][j] = 0;
		}
		if(lastZero){
			for(int j = 0; j < m; j++)
				matrix[i-1][j] = 0;
		}
		lastZero = thisZero;
	}
	if(lastZero){
		for(int j = 0; j < m; j++)
			matrix[n-1][j] = 0;
	}
}
    \end{lstlisting}
\end{description}

\subsection{Pascal's Triangle}
    
\begin{description}
    \item{\textbf{问题}}:\\
Given numRows, generate the first numRows of Pascal's triangle. \\
\textit{(leetcode 118)}
    \item{\textbf{举例}}:\\
Given numRows = 5, \\
Return \\
\\
$[$ \\
     $[1]$, \\
    $[1,1]$, \\
   $[1,2,1]$, \\
  $[1,3,3,1]$, \\
 $[1,4,6,4,1]$ \\
$]$
    \item{\textbf{???}} : \fbox{时间复杂度O($n^2$), 空间复杂度O(1)}
    \begin{lstlisting}
vector<vector<int> > generate(int numRows) {
	vector<vector<int> > result;
	if(numRows == 0)	return result;
	result.push_back(vector<int>{1});
	for(int i = 1; i < numRows; i++){
		vector<int> cur;
		cur.push_back(1);
		for(int j = 0; j < result[i-1].size() - 1; j++)
			cur.push_back(result[i-1][j] + result[i-1][j+1]);
		cur.push_back(1);
		result.push_back(cur);
	}
	return result;
}
    \end{lstlisting}
\end{description}

\subsection{Pascal's Triangle II}
    
\begin{description}
    \item{\textbf{问题}}:\\
Given an index k, return the kth row of the Pascal's triangle. \\
\textit{(leetcode 119)}
    \item{\textbf{举例}}:\\
Given k = 3, \\
Return $[1,3,3,1]$.
    \item{\textbf{Note}}:\\
Could you optimize your algorithm to use only O(k) extra space?
    \item{\textbf{???}} : \fbox{时间复杂度O($k^2$), 空间复杂度O(k)}
    \\这个空间复杂度可以优化,因为每次我们只需要上一行就可以产生本行,所有之前的行可以不存储
    \begin{lstlisting}
	vector<int> getRow(int rowIndex) {
		rowIndex++;
		if(rowIndex <= 0)	return vector<int>();
		vector<int> result{1};
		for(int i = 1; i < rowIndex; i++){
			vector<int> cur;
			cur.push_back(1);
			for(int j = 0; j < result.size() - 1; j++)
				cur.push_back(result[j] + result[j+1]);
			cur.push_back(1);
			result.swap(cur);
		}
		return result;
	}
    \end{lstlisting}
\end{description}



\chapter{二分查找}
    
\subsection{Spiral Matrix}
    
\begin{description}
    \item{\textbf{问题}}:\\
Given a matrix of m x n elements (m rows, n columns), return all elements of the matrix in spiral order.\\
\textit{(leetcode 54)}
    \item{\textbf{举例}}:\\
Given the following matrix:\\
\\
$[$ \\
 $[ 1, 2, 3 ]$, \\
 $[ 4, 5, 6 ]$, \\
 $[ 7, 8, 9 ]$ \\
$]$ \\
You should return $[1,2,3,6,9,8,7,4,5]$.
    \item{\textbf{???}} : \fbox{时间复杂度O($n^2$), 空间复杂度O(1)}
    \\从外到内一环一环的处理,需要注意一些边界条件
    \begin{lstlisting}
vector<int> spiralOrder(vector<vector<int> > &matrix) {
	vector<int> result;
	int n = matrix.size();
	if(n == 0)	return result;
	int m = matrix[0].size();
	int magrin = 0;
	while(m - 1 - magrin >= magrin && n - 1 - magrin >= magrin){
		for(int j = magrin; j <= m - 1 - magrin; j++)
			result.push_back(matrix[magrin][j]);
		for(int i = magrin + 1; i < n - 1 - magrin; i++)
			result.push_back(matrix[i][m-1-magrin]);
		if(n - 1 - magrin != magrin)
			for(int j = m - 1 - magrin; j >= magrin; j-- )
				result.push_back(matrix[n-1-magrin][j]);
		if(m - 1 - magrin != magrin)
			for(int i = n - 1 - magrin - 1; i > magrin; i--)
				result.push_back(matrix[i][magrin]);
		magrin++;
	}
	return result;
}
    \end{lstlisting}
\end{description}

\subsection{Spiral Matrix II}
    
\begin{description}
    \item{\textbf{问题}}:\\
Given an integer n, generate a square matrix filled with elements from 1 to $n^2$ in spiral order.\\
\textit{(leetcode 59)}
    \item{\textbf{举例}}:\\
Given n = 3,\\
\\
You should return the following matrix:\\
$[$ \\
 $[ 1, 2, 3 ]$, \\
 $[ 8, 9, 4 ]$, \\
 $[ 7, 6, 5 ]$ \\
$]$
    \item{\textbf{???}} : \fbox{时间复杂度O($n^2$), 空间复杂度O(1)}
    \begin{lstlisting}
vector<vector<int> > generateMatrix(int n) {
	vector<vector<int> > matrix(n, vector<int>(n, 0));
	int pos = 1;
	int magrin = 0;
	while(n - 1 - magrin >= magrin && n - 1 - magrin >= magrin){
		for(int j = magrin; j <= n - 1 - magrin; j++)
			matrix[magrin][j] = pos++;
		for(int i = magrin + 1; i < n - 1 - magrin; i++)
			matrix[i][n-1-magrin] = pos++;
		if(n - 1 - magrin != magrin)
			for(int j = n - 1 - magrin; j >= magrin; j-- )
				matrix[n-1-magrin][j] = pos++;
		if(n - 1 - magrin != magrin)
			for(int i = n - 1 - magrin - 1; i > magrin; i--)
				matrix[i][magrin] = pos++;
		magrin++;
	}
	return matrix;
}
    \end{lstlisting}
\end{description}

\subsection{Set Matrix Zeroes}
    
\begin{description}
    \item{\textbf{问题}}:\\
Given a m x n matrix, if an element is 0, set its entire row and column to 0. Do it in place.\\
\textit{(leetcode 73)}
    \item{\textbf{Follow Up}}:\\
Did you use extra space? \\
A straight forward solution using O(mn) space is probably a bad idea. \\
A simple improvement uses O(m + n) space, but still not the best solution. \\
Could you devise a constant space solution?
    \item{\textbf{???}} : \fbox{时间复杂度O($n^2$), 空间复杂度O(1)}
    \\从上到下一行一行的处理,如果上一个存在0,可以先保留上一行的现场,然后根据上一行的原来值更新本行,然后处理上一行.唯一需要注意的就是,如果本行出现新0需要更新该0所在列的上面所有行.
    \begin{lstlisting}
void setZeroes(vector<vector<int> > &matrix) {
	int n = matrix.size();
	if(n == 0)	return;
	int m = matrix[0].size();
	bool lastZero = false;
	for(int i = 0; i < n; i++){
		bool thisZero = false;
		for(int j = 0; j < m; j++){
			if(matrix[i][j] == 0){
				int up = i - 1;
				while(up >= 0){
					matrix[up][j] = 0;
					up--;
				}
				thisZero = true;
			}
			if(i > 0 && matrix[i-1][j] == 0)
				matrix[i][j] = 0;
		}
		if(lastZero){
			for(int j = 0; j < m; j++)
				matrix[i-1][j] = 0;
		}
		lastZero = thisZero;
	}
	if(lastZero){
		for(int j = 0; j < m; j++)
			matrix[n-1][j] = 0;
	}
}
    \end{lstlisting}
\end{description}

\subsection{Pascal's Triangle}
    
\begin{description}
    \item{\textbf{问题}}:\\
Given numRows, generate the first numRows of Pascal's triangle. \\
\textit{(leetcode 118)}
    \item{\textbf{举例}}:\\
Given numRows = 5, \\
Return \\
\\
$[$ \\
     $[1]$, \\
    $[1,1]$, \\
   $[1,2,1]$, \\
  $[1,3,3,1]$, \\
 $[1,4,6,4,1]$ \\
$]$
    \item{\textbf{???}} : \fbox{时间复杂度O($n^2$), 空间复杂度O(1)}
    \begin{lstlisting}
vector<vector<int> > generate(int numRows) {
	vector<vector<int> > result;
	if(numRows == 0)	return result;
	result.push_back(vector<int>{1});
	for(int i = 1; i < numRows; i++){
		vector<int> cur;
		cur.push_back(1);
		for(int j = 0; j < result[i-1].size() - 1; j++)
			cur.push_back(result[i-1][j] + result[i-1][j+1]);
		cur.push_back(1);
		result.push_back(cur);
	}
	return result;
}
    \end{lstlisting}
\end{description}

\subsection{Pascal's Triangle II}
    
\begin{description}
    \item{\textbf{问题}}:\\
Given an index k, return the kth row of the Pascal's triangle. \\
\textit{(leetcode 119)}
    \item{\textbf{举例}}:\\
Given k = 3, \\
Return $[1,3,3,1]$.
    \item{\textbf{Note}}:\\
Could you optimize your algorithm to use only O(k) extra space?
    \item{\textbf{???}} : \fbox{时间复杂度O($k^2$), 空间复杂度O(k)}
    \\这个空间复杂度可以优化,因为每次我们只需要上一行就可以产生本行,所有之前的行可以不存储
    \begin{lstlisting}
	vector<int> getRow(int rowIndex) {
		rowIndex++;
		if(rowIndex <= 0)	return vector<int>();
		vector<int> result{1};
		for(int i = 1; i < rowIndex; i++){
			vector<int> cur;
			cur.push_back(1);
			for(int j = 0; j < result.size() - 1; j++)
				cur.push_back(result[j] + result[j+1]);
			cur.push_back(1);
			result.swap(cur);
		}
		return result;
	}
    \end{lstlisting}
\end{description}



\chapter{分治}
    
\subsection{Spiral Matrix}
    
\begin{description}
    \item{\textbf{问题}}:\\
Given a matrix of m x n elements (m rows, n columns), return all elements of the matrix in spiral order.\\
\textit{(leetcode 54)}
    \item{\textbf{举例}}:\\
Given the following matrix:\\
\\
$[$ \\
 $[ 1, 2, 3 ]$, \\
 $[ 4, 5, 6 ]$, \\
 $[ 7, 8, 9 ]$ \\
$]$ \\
You should return $[1,2,3,6,9,8,7,4,5]$.
    \item{\textbf{???}} : \fbox{时间复杂度O($n^2$), 空间复杂度O(1)}
    \\从外到内一环一环的处理,需要注意一些边界条件
    \begin{lstlisting}
vector<int> spiralOrder(vector<vector<int> > &matrix) {
	vector<int> result;
	int n = matrix.size();
	if(n == 0)	return result;
	int m = matrix[0].size();
	int magrin = 0;
	while(m - 1 - magrin >= magrin && n - 1 - magrin >= magrin){
		for(int j = magrin; j <= m - 1 - magrin; j++)
			result.push_back(matrix[magrin][j]);
		for(int i = magrin + 1; i < n - 1 - magrin; i++)
			result.push_back(matrix[i][m-1-magrin]);
		if(n - 1 - magrin != magrin)
			for(int j = m - 1 - magrin; j >= magrin; j-- )
				result.push_back(matrix[n-1-magrin][j]);
		if(m - 1 - magrin != magrin)
			for(int i = n - 1 - magrin - 1; i > magrin; i--)
				result.push_back(matrix[i][magrin]);
		magrin++;
	}
	return result;
}
    \end{lstlisting}
\end{description}

\subsection{Spiral Matrix II}
    
\begin{description}
    \item{\textbf{问题}}:\\
Given an integer n, generate a square matrix filled with elements from 1 to $n^2$ in spiral order.\\
\textit{(leetcode 59)}
    \item{\textbf{举例}}:\\
Given n = 3,\\
\\
You should return the following matrix:\\
$[$ \\
 $[ 1, 2, 3 ]$, \\
 $[ 8, 9, 4 ]$, \\
 $[ 7, 6, 5 ]$ \\
$]$
    \item{\textbf{???}} : \fbox{时间复杂度O($n^2$), 空间复杂度O(1)}
    \begin{lstlisting}
vector<vector<int> > generateMatrix(int n) {
	vector<vector<int> > matrix(n, vector<int>(n, 0));
	int pos = 1;
	int magrin = 0;
	while(n - 1 - magrin >= magrin && n - 1 - magrin >= magrin){
		for(int j = magrin; j <= n - 1 - magrin; j++)
			matrix[magrin][j] = pos++;
		for(int i = magrin + 1; i < n - 1 - magrin; i++)
			matrix[i][n-1-magrin] = pos++;
		if(n - 1 - magrin != magrin)
			for(int j = n - 1 - magrin; j >= magrin; j-- )
				matrix[n-1-magrin][j] = pos++;
		if(n - 1 - magrin != magrin)
			for(int i = n - 1 - magrin - 1; i > magrin; i--)
				matrix[i][magrin] = pos++;
		magrin++;
	}
	return matrix;
}
    \end{lstlisting}
\end{description}

\subsection{Set Matrix Zeroes}
    
\begin{description}
    \item{\textbf{问题}}:\\
Given a m x n matrix, if an element is 0, set its entire row and column to 0. Do it in place.\\
\textit{(leetcode 73)}
    \item{\textbf{Follow Up}}:\\
Did you use extra space? \\
A straight forward solution using O(mn) space is probably a bad idea. \\
A simple improvement uses O(m + n) space, but still not the best solution. \\
Could you devise a constant space solution?
    \item{\textbf{???}} : \fbox{时间复杂度O($n^2$), 空间复杂度O(1)}
    \\从上到下一行一行的处理,如果上一个存在0,可以先保留上一行的现场,然后根据上一行的原来值更新本行,然后处理上一行.唯一需要注意的就是,如果本行出现新0需要更新该0所在列的上面所有行.
    \begin{lstlisting}
void setZeroes(vector<vector<int> > &matrix) {
	int n = matrix.size();
	if(n == 0)	return;
	int m = matrix[0].size();
	bool lastZero = false;
	for(int i = 0; i < n; i++){
		bool thisZero = false;
		for(int j = 0; j < m; j++){
			if(matrix[i][j] == 0){
				int up = i - 1;
				while(up >= 0){
					matrix[up][j] = 0;
					up--;
				}
				thisZero = true;
			}
			if(i > 0 && matrix[i-1][j] == 0)
				matrix[i][j] = 0;
		}
		if(lastZero){
			for(int j = 0; j < m; j++)
				matrix[i-1][j] = 0;
		}
		lastZero = thisZero;
	}
	if(lastZero){
		for(int j = 0; j < m; j++)
			matrix[n-1][j] = 0;
	}
}
    \end{lstlisting}
\end{description}

\subsection{Pascal's Triangle}
    
\begin{description}
    \item{\textbf{问题}}:\\
Given numRows, generate the first numRows of Pascal's triangle. \\
\textit{(leetcode 118)}
    \item{\textbf{举例}}:\\
Given numRows = 5, \\
Return \\
\\
$[$ \\
     $[1]$, \\
    $[1,1]$, \\
   $[1,2,1]$, \\
  $[1,3,3,1]$, \\
 $[1,4,6,4,1]$ \\
$]$
    \item{\textbf{???}} : \fbox{时间复杂度O($n^2$), 空间复杂度O(1)}
    \begin{lstlisting}
vector<vector<int> > generate(int numRows) {
	vector<vector<int> > result;
	if(numRows == 0)	return result;
	result.push_back(vector<int>{1});
	for(int i = 1; i < numRows; i++){
		vector<int> cur;
		cur.push_back(1);
		for(int j = 0; j < result[i-1].size() - 1; j++)
			cur.push_back(result[i-1][j] + result[i-1][j+1]);
		cur.push_back(1);
		result.push_back(cur);
	}
	return result;
}
    \end{lstlisting}
\end{description}

\subsection{Pascal's Triangle II}
    
\begin{description}
    \item{\textbf{问题}}:\\
Given an index k, return the kth row of the Pascal's triangle. \\
\textit{(leetcode 119)}
    \item{\textbf{举例}}:\\
Given k = 3, \\
Return $[1,3,3,1]$.
    \item{\textbf{Note}}:\\
Could you optimize your algorithm to use only O(k) extra space?
    \item{\textbf{???}} : \fbox{时间复杂度O($k^2$), 空间复杂度O(k)}
    \\这个空间复杂度可以优化,因为每次我们只需要上一行就可以产生本行,所有之前的行可以不存储
    \begin{lstlisting}
	vector<int> getRow(int rowIndex) {
		rowIndex++;
		if(rowIndex <= 0)	return vector<int>();
		vector<int> result{1};
		for(int i = 1; i < rowIndex; i++){
			vector<int> cur;
			cur.push_back(1);
			for(int j = 0; j < result.size() - 1; j++)
				cur.push_back(result[j] + result[j+1]);
			cur.push_back(1);
			result.swap(cur);
		}
		return result;
	}
    \end{lstlisting}
\end{description}



\chapter{搜索}
    
\subsection{Spiral Matrix}
    
\begin{description}
    \item{\textbf{问题}}:\\
Given a matrix of m x n elements (m rows, n columns), return all elements of the matrix in spiral order.\\
\textit{(leetcode 54)}
    \item{\textbf{举例}}:\\
Given the following matrix:\\
\\
$[$ \\
 $[ 1, 2, 3 ]$, \\
 $[ 4, 5, 6 ]$, \\
 $[ 7, 8, 9 ]$ \\
$]$ \\
You should return $[1,2,3,6,9,8,7,4,5]$.
    \item{\textbf{???}} : \fbox{时间复杂度O($n^2$), 空间复杂度O(1)}
    \\从外到内一环一环的处理,需要注意一些边界条件
    \begin{lstlisting}
vector<int> spiralOrder(vector<vector<int> > &matrix) {
	vector<int> result;
	int n = matrix.size();
	if(n == 0)	return result;
	int m = matrix[0].size();
	int magrin = 0;
	while(m - 1 - magrin >= magrin && n - 1 - magrin >= magrin){
		for(int j = magrin; j <= m - 1 - magrin; j++)
			result.push_back(matrix[magrin][j]);
		for(int i = magrin + 1; i < n - 1 - magrin; i++)
			result.push_back(matrix[i][m-1-magrin]);
		if(n - 1 - magrin != magrin)
			for(int j = m - 1 - magrin; j >= magrin; j-- )
				result.push_back(matrix[n-1-magrin][j]);
		if(m - 1 - magrin != magrin)
			for(int i = n - 1 - magrin - 1; i > magrin; i--)
				result.push_back(matrix[i][magrin]);
		magrin++;
	}
	return result;
}
    \end{lstlisting}
\end{description}

\subsection{Spiral Matrix II}
    
\begin{description}
    \item{\textbf{问题}}:\\
Given an integer n, generate a square matrix filled with elements from 1 to $n^2$ in spiral order.\\
\textit{(leetcode 59)}
    \item{\textbf{举例}}:\\
Given n = 3,\\
\\
You should return the following matrix:\\
$[$ \\
 $[ 1, 2, 3 ]$, \\
 $[ 8, 9, 4 ]$, \\
 $[ 7, 6, 5 ]$ \\
$]$
    \item{\textbf{???}} : \fbox{时间复杂度O($n^2$), 空间复杂度O(1)}
    \begin{lstlisting}
vector<vector<int> > generateMatrix(int n) {
	vector<vector<int> > matrix(n, vector<int>(n, 0));
	int pos = 1;
	int magrin = 0;
	while(n - 1 - magrin >= magrin && n - 1 - magrin >= magrin){
		for(int j = magrin; j <= n - 1 - magrin; j++)
			matrix[magrin][j] = pos++;
		for(int i = magrin + 1; i < n - 1 - magrin; i++)
			matrix[i][n-1-magrin] = pos++;
		if(n - 1 - magrin != magrin)
			for(int j = n - 1 - magrin; j >= magrin; j-- )
				matrix[n-1-magrin][j] = pos++;
		if(n - 1 - magrin != magrin)
			for(int i = n - 1 - magrin - 1; i > magrin; i--)
				matrix[i][magrin] = pos++;
		magrin++;
	}
	return matrix;
}
    \end{lstlisting}
\end{description}

\subsection{Set Matrix Zeroes}
    
\begin{description}
    \item{\textbf{问题}}:\\
Given a m x n matrix, if an element is 0, set its entire row and column to 0. Do it in place.\\
\textit{(leetcode 73)}
    \item{\textbf{Follow Up}}:\\
Did you use extra space? \\
A straight forward solution using O(mn) space is probably a bad idea. \\
A simple improvement uses O(m + n) space, but still not the best solution. \\
Could you devise a constant space solution?
    \item{\textbf{???}} : \fbox{时间复杂度O($n^2$), 空间复杂度O(1)}
    \\从上到下一行一行的处理,如果上一个存在0,可以先保留上一行的现场,然后根据上一行的原来值更新本行,然后处理上一行.唯一需要注意的就是,如果本行出现新0需要更新该0所在列的上面所有行.
    \begin{lstlisting}
void setZeroes(vector<vector<int> > &matrix) {
	int n = matrix.size();
	if(n == 0)	return;
	int m = matrix[0].size();
	bool lastZero = false;
	for(int i = 0; i < n; i++){
		bool thisZero = false;
		for(int j = 0; j < m; j++){
			if(matrix[i][j] == 0){
				int up = i - 1;
				while(up >= 0){
					matrix[up][j] = 0;
					up--;
				}
				thisZero = true;
			}
			if(i > 0 && matrix[i-1][j] == 0)
				matrix[i][j] = 0;
		}
		if(lastZero){
			for(int j = 0; j < m; j++)
				matrix[i-1][j] = 0;
		}
		lastZero = thisZero;
	}
	if(lastZero){
		for(int j = 0; j < m; j++)
			matrix[n-1][j] = 0;
	}
}
    \end{lstlisting}
\end{description}

\subsection{Pascal's Triangle}
    
\begin{description}
    \item{\textbf{问题}}:\\
Given numRows, generate the first numRows of Pascal's triangle. \\
\textit{(leetcode 118)}
    \item{\textbf{举例}}:\\
Given numRows = 5, \\
Return \\
\\
$[$ \\
     $[1]$, \\
    $[1,1]$, \\
   $[1,2,1]$, \\
  $[1,3,3,1]$, \\
 $[1,4,6,4,1]$ \\
$]$
    \item{\textbf{???}} : \fbox{时间复杂度O($n^2$), 空间复杂度O(1)}
    \begin{lstlisting}
vector<vector<int> > generate(int numRows) {
	vector<vector<int> > result;
	if(numRows == 0)	return result;
	result.push_back(vector<int>{1});
	for(int i = 1; i < numRows; i++){
		vector<int> cur;
		cur.push_back(1);
		for(int j = 0; j < result[i-1].size() - 1; j++)
			cur.push_back(result[i-1][j] + result[i-1][j+1]);
		cur.push_back(1);
		result.push_back(cur);
	}
	return result;
}
    \end{lstlisting}
\end{description}

\subsection{Pascal's Triangle II}
    
\begin{description}
    \item{\textbf{问题}}:\\
Given an index k, return the kth row of the Pascal's triangle. \\
\textit{(leetcode 119)}
    \item{\textbf{举例}}:\\
Given k = 3, \\
Return $[1,3,3,1]$.
    \item{\textbf{Note}}:\\
Could you optimize your algorithm to use only O(k) extra space?
    \item{\textbf{???}} : \fbox{时间复杂度O($k^2$), 空间复杂度O(k)}
    \\这个空间复杂度可以优化,因为每次我们只需要上一行就可以产生本行,所有之前的行可以不存储
    \begin{lstlisting}
	vector<int> getRow(int rowIndex) {
		rowIndex++;
		if(rowIndex <= 0)	return vector<int>();
		vector<int> result{1};
		for(int i = 1; i < rowIndex; i++){
			vector<int> cur;
			cur.push_back(1);
			for(int j = 0; j < result.size() - 1; j++)
				cur.push_back(result[j] + result[j+1]);
			cur.push_back(1);
			result.swap(cur);
		}
		return result;
	}
    \end{lstlisting}
\end{description}



\chapter{回溯}
    
\subsection{Spiral Matrix}
    
\begin{description}
    \item{\textbf{问题}}:\\
Given a matrix of m x n elements (m rows, n columns), return all elements of the matrix in spiral order.\\
\textit{(leetcode 54)}
    \item{\textbf{举例}}:\\
Given the following matrix:\\
\\
$[$ \\
 $[ 1, 2, 3 ]$, \\
 $[ 4, 5, 6 ]$, \\
 $[ 7, 8, 9 ]$ \\
$]$ \\
You should return $[1,2,3,6,9,8,7,4,5]$.
    \item{\textbf{???}} : \fbox{时间复杂度O($n^2$), 空间复杂度O(1)}
    \\从外到内一环一环的处理,需要注意一些边界条件
    \begin{lstlisting}
vector<int> spiralOrder(vector<vector<int> > &matrix) {
	vector<int> result;
	int n = matrix.size();
	if(n == 0)	return result;
	int m = matrix[0].size();
	int magrin = 0;
	while(m - 1 - magrin >= magrin && n - 1 - magrin >= magrin){
		for(int j = magrin; j <= m - 1 - magrin; j++)
			result.push_back(matrix[magrin][j]);
		for(int i = magrin + 1; i < n - 1 - magrin; i++)
			result.push_back(matrix[i][m-1-magrin]);
		if(n - 1 - magrin != magrin)
			for(int j = m - 1 - magrin; j >= magrin; j-- )
				result.push_back(matrix[n-1-magrin][j]);
		if(m - 1 - magrin != magrin)
			for(int i = n - 1 - magrin - 1; i > magrin; i--)
				result.push_back(matrix[i][magrin]);
		magrin++;
	}
	return result;
}
    \end{lstlisting}
\end{description}

\subsection{Spiral Matrix II}
    
\begin{description}
    \item{\textbf{问题}}:\\
Given an integer n, generate a square matrix filled with elements from 1 to $n^2$ in spiral order.\\
\textit{(leetcode 59)}
    \item{\textbf{举例}}:\\
Given n = 3,\\
\\
You should return the following matrix:\\
$[$ \\
 $[ 1, 2, 3 ]$, \\
 $[ 8, 9, 4 ]$, \\
 $[ 7, 6, 5 ]$ \\
$]$
    \item{\textbf{???}} : \fbox{时间复杂度O($n^2$), 空间复杂度O(1)}
    \begin{lstlisting}
vector<vector<int> > generateMatrix(int n) {
	vector<vector<int> > matrix(n, vector<int>(n, 0));
	int pos = 1;
	int magrin = 0;
	while(n - 1 - magrin >= magrin && n - 1 - magrin >= magrin){
		for(int j = magrin; j <= n - 1 - magrin; j++)
			matrix[magrin][j] = pos++;
		for(int i = magrin + 1; i < n - 1 - magrin; i++)
			matrix[i][n-1-magrin] = pos++;
		if(n - 1 - magrin != magrin)
			for(int j = n - 1 - magrin; j >= magrin; j-- )
				matrix[n-1-magrin][j] = pos++;
		if(n - 1 - magrin != magrin)
			for(int i = n - 1 - magrin - 1; i > magrin; i--)
				matrix[i][magrin] = pos++;
		magrin++;
	}
	return matrix;
}
    \end{lstlisting}
\end{description}

\subsection{Set Matrix Zeroes}
    
\begin{description}
    \item{\textbf{问题}}:\\
Given a m x n matrix, if an element is 0, set its entire row and column to 0. Do it in place.\\
\textit{(leetcode 73)}
    \item{\textbf{Follow Up}}:\\
Did you use extra space? \\
A straight forward solution using O(mn) space is probably a bad idea. \\
A simple improvement uses O(m + n) space, but still not the best solution. \\
Could you devise a constant space solution?
    \item{\textbf{???}} : \fbox{时间复杂度O($n^2$), 空间复杂度O(1)}
    \\从上到下一行一行的处理,如果上一个存在0,可以先保留上一行的现场,然后根据上一行的原来值更新本行,然后处理上一行.唯一需要注意的就是,如果本行出现新0需要更新该0所在列的上面所有行.
    \begin{lstlisting}
void setZeroes(vector<vector<int> > &matrix) {
	int n = matrix.size();
	if(n == 0)	return;
	int m = matrix[0].size();
	bool lastZero = false;
	for(int i = 0; i < n; i++){
		bool thisZero = false;
		for(int j = 0; j < m; j++){
			if(matrix[i][j] == 0){
				int up = i - 1;
				while(up >= 0){
					matrix[up][j] = 0;
					up--;
				}
				thisZero = true;
			}
			if(i > 0 && matrix[i-1][j] == 0)
				matrix[i][j] = 0;
		}
		if(lastZero){
			for(int j = 0; j < m; j++)
				matrix[i-1][j] = 0;
		}
		lastZero = thisZero;
	}
	if(lastZero){
		for(int j = 0; j < m; j++)
			matrix[n-1][j] = 0;
	}
}
    \end{lstlisting}
\end{description}

\subsection{Pascal's Triangle}
    
\begin{description}
    \item{\textbf{问题}}:\\
Given numRows, generate the first numRows of Pascal's triangle. \\
\textit{(leetcode 118)}
    \item{\textbf{举例}}:\\
Given numRows = 5, \\
Return \\
\\
$[$ \\
     $[1]$, \\
    $[1,1]$, \\
   $[1,2,1]$, \\
  $[1,3,3,1]$, \\
 $[1,4,6,4,1]$ \\
$]$
    \item{\textbf{???}} : \fbox{时间复杂度O($n^2$), 空间复杂度O(1)}
    \begin{lstlisting}
vector<vector<int> > generate(int numRows) {
	vector<vector<int> > result;
	if(numRows == 0)	return result;
	result.push_back(vector<int>{1});
	for(int i = 1; i < numRows; i++){
		vector<int> cur;
		cur.push_back(1);
		for(int j = 0; j < result[i-1].size() - 1; j++)
			cur.push_back(result[i-1][j] + result[i-1][j+1]);
		cur.push_back(1);
		result.push_back(cur);
	}
	return result;
}
    \end{lstlisting}
\end{description}

\subsection{Pascal's Triangle II}
    
\begin{description}
    \item{\textbf{问题}}:\\
Given an index k, return the kth row of the Pascal's triangle. \\
\textit{(leetcode 119)}
    \item{\textbf{举例}}:\\
Given k = 3, \\
Return $[1,3,3,1]$.
    \item{\textbf{Note}}:\\
Could you optimize your algorithm to use only O(k) extra space?
    \item{\textbf{???}} : \fbox{时间复杂度O($k^2$), 空间复杂度O(k)}
    \\这个空间复杂度可以优化,因为每次我们只需要上一行就可以产生本行,所有之前的行可以不存储
    \begin{lstlisting}
	vector<int> getRow(int rowIndex) {
		rowIndex++;
		if(rowIndex <= 0)	return vector<int>();
		vector<int> result{1};
		for(int i = 1; i < rowIndex; i++){
			vector<int> cur;
			cur.push_back(1);
			for(int j = 0; j < result.size() - 1; j++)
				cur.push_back(result[j] + result[j+1]);
			cur.push_back(1);
			result.swap(cur);
		}
		return result;
	}
    \end{lstlisting}
\end{description}



\chapter{贪心}
    
\subsection{Spiral Matrix}
    
\begin{description}
    \item{\textbf{问题}}:\\
Given a matrix of m x n elements (m rows, n columns), return all elements of the matrix in spiral order.\\
\textit{(leetcode 54)}
    \item{\textbf{举例}}:\\
Given the following matrix:\\
\\
$[$ \\
 $[ 1, 2, 3 ]$, \\
 $[ 4, 5, 6 ]$, \\
 $[ 7, 8, 9 ]$ \\
$]$ \\
You should return $[1,2,3,6,9,8,7,4,5]$.
    \item{\textbf{???}} : \fbox{时间复杂度O($n^2$), 空间复杂度O(1)}
    \\从外到内一环一环的处理,需要注意一些边界条件
    \begin{lstlisting}
vector<int> spiralOrder(vector<vector<int> > &matrix) {
	vector<int> result;
	int n = matrix.size();
	if(n == 0)	return result;
	int m = matrix[0].size();
	int magrin = 0;
	while(m - 1 - magrin >= magrin && n - 1 - magrin >= magrin){
		for(int j = magrin; j <= m - 1 - magrin; j++)
			result.push_back(matrix[magrin][j]);
		for(int i = magrin + 1; i < n - 1 - magrin; i++)
			result.push_back(matrix[i][m-1-magrin]);
		if(n - 1 - magrin != magrin)
			for(int j = m - 1 - magrin; j >= magrin; j-- )
				result.push_back(matrix[n-1-magrin][j]);
		if(m - 1 - magrin != magrin)
			for(int i = n - 1 - magrin - 1; i > magrin; i--)
				result.push_back(matrix[i][magrin]);
		magrin++;
	}
	return result;
}
    \end{lstlisting}
\end{description}

\subsection{Spiral Matrix II}
    
\begin{description}
    \item{\textbf{问题}}:\\
Given an integer n, generate a square matrix filled with elements from 1 to $n^2$ in spiral order.\\
\textit{(leetcode 59)}
    \item{\textbf{举例}}:\\
Given n = 3,\\
\\
You should return the following matrix:\\
$[$ \\
 $[ 1, 2, 3 ]$, \\
 $[ 8, 9, 4 ]$, \\
 $[ 7, 6, 5 ]$ \\
$]$
    \item{\textbf{???}} : \fbox{时间复杂度O($n^2$), 空间复杂度O(1)}
    \begin{lstlisting}
vector<vector<int> > generateMatrix(int n) {
	vector<vector<int> > matrix(n, vector<int>(n, 0));
	int pos = 1;
	int magrin = 0;
	while(n - 1 - magrin >= magrin && n - 1 - magrin >= magrin){
		for(int j = magrin; j <= n - 1 - magrin; j++)
			matrix[magrin][j] = pos++;
		for(int i = magrin + 1; i < n - 1 - magrin; i++)
			matrix[i][n-1-magrin] = pos++;
		if(n - 1 - magrin != magrin)
			for(int j = n - 1 - magrin; j >= magrin; j-- )
				matrix[n-1-magrin][j] = pos++;
		if(n - 1 - magrin != magrin)
			for(int i = n - 1 - magrin - 1; i > magrin; i--)
				matrix[i][magrin] = pos++;
		magrin++;
	}
	return matrix;
}
    \end{lstlisting}
\end{description}

\subsection{Set Matrix Zeroes}
    
\begin{description}
    \item{\textbf{问题}}:\\
Given a m x n matrix, if an element is 0, set its entire row and column to 0. Do it in place.\\
\textit{(leetcode 73)}
    \item{\textbf{Follow Up}}:\\
Did you use extra space? \\
A straight forward solution using O(mn) space is probably a bad idea. \\
A simple improvement uses O(m + n) space, but still not the best solution. \\
Could you devise a constant space solution?
    \item{\textbf{???}} : \fbox{时间复杂度O($n^2$), 空间复杂度O(1)}
    \\从上到下一行一行的处理,如果上一个存在0,可以先保留上一行的现场,然后根据上一行的原来值更新本行,然后处理上一行.唯一需要注意的就是,如果本行出现新0需要更新该0所在列的上面所有行.
    \begin{lstlisting}
void setZeroes(vector<vector<int> > &matrix) {
	int n = matrix.size();
	if(n == 0)	return;
	int m = matrix[0].size();
	bool lastZero = false;
	for(int i = 0; i < n; i++){
		bool thisZero = false;
		for(int j = 0; j < m; j++){
			if(matrix[i][j] == 0){
				int up = i - 1;
				while(up >= 0){
					matrix[up][j] = 0;
					up--;
				}
				thisZero = true;
			}
			if(i > 0 && matrix[i-1][j] == 0)
				matrix[i][j] = 0;
		}
		if(lastZero){
			for(int j = 0; j < m; j++)
				matrix[i-1][j] = 0;
		}
		lastZero = thisZero;
	}
	if(lastZero){
		for(int j = 0; j < m; j++)
			matrix[n-1][j] = 0;
	}
}
    \end{lstlisting}
\end{description}

\subsection{Pascal's Triangle}
    
\begin{description}
    \item{\textbf{问题}}:\\
Given numRows, generate the first numRows of Pascal's triangle. \\
\textit{(leetcode 118)}
    \item{\textbf{举例}}:\\
Given numRows = 5, \\
Return \\
\\
$[$ \\
     $[1]$, \\
    $[1,1]$, \\
   $[1,2,1]$, \\
  $[1,3,3,1]$, \\
 $[1,4,6,4,1]$ \\
$]$
    \item{\textbf{???}} : \fbox{时间复杂度O($n^2$), 空间复杂度O(1)}
    \begin{lstlisting}
vector<vector<int> > generate(int numRows) {
	vector<vector<int> > result;
	if(numRows == 0)	return result;
	result.push_back(vector<int>{1});
	for(int i = 1; i < numRows; i++){
		vector<int> cur;
		cur.push_back(1);
		for(int j = 0; j < result[i-1].size() - 1; j++)
			cur.push_back(result[i-1][j] + result[i-1][j+1]);
		cur.push_back(1);
		result.push_back(cur);
	}
	return result;
}
    \end{lstlisting}
\end{description}

\subsection{Pascal's Triangle II}
    
\begin{description}
    \item{\textbf{问题}}:\\
Given an index k, return the kth row of the Pascal's triangle. \\
\textit{(leetcode 119)}
    \item{\textbf{举例}}:\\
Given k = 3, \\
Return $[1,3,3,1]$.
    \item{\textbf{Note}}:\\
Could you optimize your algorithm to use only O(k) extra space?
    \item{\textbf{???}} : \fbox{时间复杂度O($k^2$), 空间复杂度O(k)}
    \\这个空间复杂度可以优化,因为每次我们只需要上一行就可以产生本行,所有之前的行可以不存储
    \begin{lstlisting}
	vector<int> getRow(int rowIndex) {
		rowIndex++;
		if(rowIndex <= 0)	return vector<int>();
		vector<int> result{1};
		for(int i = 1; i < rowIndex; i++){
			vector<int> cur;
			cur.push_back(1);
			for(int j = 0; j < result.size() - 1; j++)
				cur.push_back(result[j] + result[j+1]);
			cur.push_back(1);
			result.swap(cur);
		}
		return result;
	}
    \end{lstlisting}
\end{description}



\chapter{动态规划}
    
\subsection{Spiral Matrix}
    
\begin{description}
    \item{\textbf{问题}}:\\
Given a matrix of m x n elements (m rows, n columns), return all elements of the matrix in spiral order.\\
\textit{(leetcode 54)}
    \item{\textbf{举例}}:\\
Given the following matrix:\\
\\
$[$ \\
 $[ 1, 2, 3 ]$, \\
 $[ 4, 5, 6 ]$, \\
 $[ 7, 8, 9 ]$ \\
$]$ \\
You should return $[1,2,3,6,9,8,7,4,5]$.
    \item{\textbf{???}} : \fbox{时间复杂度O($n^2$), 空间复杂度O(1)}
    \\从外到内一环一环的处理,需要注意一些边界条件
    \begin{lstlisting}
vector<int> spiralOrder(vector<vector<int> > &matrix) {
	vector<int> result;
	int n = matrix.size();
	if(n == 0)	return result;
	int m = matrix[0].size();
	int magrin = 0;
	while(m - 1 - magrin >= magrin && n - 1 - magrin >= magrin){
		for(int j = magrin; j <= m - 1 - magrin; j++)
			result.push_back(matrix[magrin][j]);
		for(int i = magrin + 1; i < n - 1 - magrin; i++)
			result.push_back(matrix[i][m-1-magrin]);
		if(n - 1 - magrin != magrin)
			for(int j = m - 1 - magrin; j >= magrin; j-- )
				result.push_back(matrix[n-1-magrin][j]);
		if(m - 1 - magrin != magrin)
			for(int i = n - 1 - magrin - 1; i > magrin; i--)
				result.push_back(matrix[i][magrin]);
		magrin++;
	}
	return result;
}
    \end{lstlisting}
\end{description}

\subsection{Spiral Matrix II}
    
\begin{description}
    \item{\textbf{问题}}:\\
Given an integer n, generate a square matrix filled with elements from 1 to $n^2$ in spiral order.\\
\textit{(leetcode 59)}
    \item{\textbf{举例}}:\\
Given n = 3,\\
\\
You should return the following matrix:\\
$[$ \\
 $[ 1, 2, 3 ]$, \\
 $[ 8, 9, 4 ]$, \\
 $[ 7, 6, 5 ]$ \\
$]$
    \item{\textbf{???}} : \fbox{时间复杂度O($n^2$), 空间复杂度O(1)}
    \begin{lstlisting}
vector<vector<int> > generateMatrix(int n) {
	vector<vector<int> > matrix(n, vector<int>(n, 0));
	int pos = 1;
	int magrin = 0;
	while(n - 1 - magrin >= magrin && n - 1 - magrin >= magrin){
		for(int j = magrin; j <= n - 1 - magrin; j++)
			matrix[magrin][j] = pos++;
		for(int i = magrin + 1; i < n - 1 - magrin; i++)
			matrix[i][n-1-magrin] = pos++;
		if(n - 1 - magrin != magrin)
			for(int j = n - 1 - magrin; j >= magrin; j-- )
				matrix[n-1-magrin][j] = pos++;
		if(n - 1 - magrin != magrin)
			for(int i = n - 1 - magrin - 1; i > magrin; i--)
				matrix[i][magrin] = pos++;
		magrin++;
	}
	return matrix;
}
    \end{lstlisting}
\end{description}

\subsection{Set Matrix Zeroes}
    
\begin{description}
    \item{\textbf{问题}}:\\
Given a m x n matrix, if an element is 0, set its entire row and column to 0. Do it in place.\\
\textit{(leetcode 73)}
    \item{\textbf{Follow Up}}:\\
Did you use extra space? \\
A straight forward solution using O(mn) space is probably a bad idea. \\
A simple improvement uses O(m + n) space, but still not the best solution. \\
Could you devise a constant space solution?
    \item{\textbf{???}} : \fbox{时间复杂度O($n^2$), 空间复杂度O(1)}
    \\从上到下一行一行的处理,如果上一个存在0,可以先保留上一行的现场,然后根据上一行的原来值更新本行,然后处理上一行.唯一需要注意的就是,如果本行出现新0需要更新该0所在列的上面所有行.
    \begin{lstlisting}
void setZeroes(vector<vector<int> > &matrix) {
	int n = matrix.size();
	if(n == 0)	return;
	int m = matrix[0].size();
	bool lastZero = false;
	for(int i = 0; i < n; i++){
		bool thisZero = false;
		for(int j = 0; j < m; j++){
			if(matrix[i][j] == 0){
				int up = i - 1;
				while(up >= 0){
					matrix[up][j] = 0;
					up--;
				}
				thisZero = true;
			}
			if(i > 0 && matrix[i-1][j] == 0)
				matrix[i][j] = 0;
		}
		if(lastZero){
			for(int j = 0; j < m; j++)
				matrix[i-1][j] = 0;
		}
		lastZero = thisZero;
	}
	if(lastZero){
		for(int j = 0; j < m; j++)
			matrix[n-1][j] = 0;
	}
}
    \end{lstlisting}
\end{description}

\subsection{Pascal's Triangle}
    
\begin{description}
    \item{\textbf{问题}}:\\
Given numRows, generate the first numRows of Pascal's triangle. \\
\textit{(leetcode 118)}
    \item{\textbf{举例}}:\\
Given numRows = 5, \\
Return \\
\\
$[$ \\
     $[1]$, \\
    $[1,1]$, \\
   $[1,2,1]$, \\
  $[1,3,3,1]$, \\
 $[1,4,6,4,1]$ \\
$]$
    \item{\textbf{???}} : \fbox{时间复杂度O($n^2$), 空间复杂度O(1)}
    \begin{lstlisting}
vector<vector<int> > generate(int numRows) {
	vector<vector<int> > result;
	if(numRows == 0)	return result;
	result.push_back(vector<int>{1});
	for(int i = 1; i < numRows; i++){
		vector<int> cur;
		cur.push_back(1);
		for(int j = 0; j < result[i-1].size() - 1; j++)
			cur.push_back(result[i-1][j] + result[i-1][j+1]);
		cur.push_back(1);
		result.push_back(cur);
	}
	return result;
}
    \end{lstlisting}
\end{description}

\subsection{Pascal's Triangle II}
    
\begin{description}
    \item{\textbf{问题}}:\\
Given an index k, return the kth row of the Pascal's triangle. \\
\textit{(leetcode 119)}
    \item{\textbf{举例}}:\\
Given k = 3, \\
Return $[1,3,3,1]$.
    \item{\textbf{Note}}:\\
Could you optimize your algorithm to use only O(k) extra space?
    \item{\textbf{???}} : \fbox{时间复杂度O($k^2$), 空间复杂度O(k)}
    \\这个空间复杂度可以优化,因为每次我们只需要上一行就可以产生本行,所有之前的行可以不存储
    \begin{lstlisting}
	vector<int> getRow(int rowIndex) {
		rowIndex++;
		if(rowIndex <= 0)	return vector<int>();
		vector<int> result{1};
		for(int i = 1; i < rowIndex; i++){
			vector<int> cur;
			cur.push_back(1);
			for(int j = 0; j < result.size() - 1; j++)
				cur.push_back(result[j] + result[j+1]);
			cur.push_back(1);
			result.swap(cur);
		}
		return result;
	}
    \end{lstlisting}
\end{description}



\chapter{位操作}
    
\subsection{Spiral Matrix}
    
\begin{description}
    \item{\textbf{问题}}:\\
Given a matrix of m x n elements (m rows, n columns), return all elements of the matrix in spiral order.\\
\textit{(leetcode 54)}
    \item{\textbf{举例}}:\\
Given the following matrix:\\
\\
$[$ \\
 $[ 1, 2, 3 ]$, \\
 $[ 4, 5, 6 ]$, \\
 $[ 7, 8, 9 ]$ \\
$]$ \\
You should return $[1,2,3,6,9,8,7,4,5]$.
    \item{\textbf{???}} : \fbox{时间复杂度O($n^2$), 空间复杂度O(1)}
    \\从外到内一环一环的处理,需要注意一些边界条件
    \begin{lstlisting}
vector<int> spiralOrder(vector<vector<int> > &matrix) {
	vector<int> result;
	int n = matrix.size();
	if(n == 0)	return result;
	int m = matrix[0].size();
	int magrin = 0;
	while(m - 1 - magrin >= magrin && n - 1 - magrin >= magrin){
		for(int j = magrin; j <= m - 1 - magrin; j++)
			result.push_back(matrix[magrin][j]);
		for(int i = magrin + 1; i < n - 1 - magrin; i++)
			result.push_back(matrix[i][m-1-magrin]);
		if(n - 1 - magrin != magrin)
			for(int j = m - 1 - magrin; j >= magrin; j-- )
				result.push_back(matrix[n-1-magrin][j]);
		if(m - 1 - magrin != magrin)
			for(int i = n - 1 - magrin - 1; i > magrin; i--)
				result.push_back(matrix[i][magrin]);
		magrin++;
	}
	return result;
}
    \end{lstlisting}
\end{description}

\subsection{Spiral Matrix II}
    
\begin{description}
    \item{\textbf{问题}}:\\
Given an integer n, generate a square matrix filled with elements from 1 to $n^2$ in spiral order.\\
\textit{(leetcode 59)}
    \item{\textbf{举例}}:\\
Given n = 3,\\
\\
You should return the following matrix:\\
$[$ \\
 $[ 1, 2, 3 ]$, \\
 $[ 8, 9, 4 ]$, \\
 $[ 7, 6, 5 ]$ \\
$]$
    \item{\textbf{???}} : \fbox{时间复杂度O($n^2$), 空间复杂度O(1)}
    \begin{lstlisting}
vector<vector<int> > generateMatrix(int n) {
	vector<vector<int> > matrix(n, vector<int>(n, 0));
	int pos = 1;
	int magrin = 0;
	while(n - 1 - magrin >= magrin && n - 1 - magrin >= magrin){
		for(int j = magrin; j <= n - 1 - magrin; j++)
			matrix[magrin][j] = pos++;
		for(int i = magrin + 1; i < n - 1 - magrin; i++)
			matrix[i][n-1-magrin] = pos++;
		if(n - 1 - magrin != magrin)
			for(int j = n - 1 - magrin; j >= magrin; j-- )
				matrix[n-1-magrin][j] = pos++;
		if(n - 1 - magrin != magrin)
			for(int i = n - 1 - magrin - 1; i > magrin; i--)
				matrix[i][magrin] = pos++;
		magrin++;
	}
	return matrix;
}
    \end{lstlisting}
\end{description}

\subsection{Set Matrix Zeroes}
    
\begin{description}
    \item{\textbf{问题}}:\\
Given a m x n matrix, if an element is 0, set its entire row and column to 0. Do it in place.\\
\textit{(leetcode 73)}
    \item{\textbf{Follow Up}}:\\
Did you use extra space? \\
A straight forward solution using O(mn) space is probably a bad idea. \\
A simple improvement uses O(m + n) space, but still not the best solution. \\
Could you devise a constant space solution?
    \item{\textbf{???}} : \fbox{时间复杂度O($n^2$), 空间复杂度O(1)}
    \\从上到下一行一行的处理,如果上一个存在0,可以先保留上一行的现场,然后根据上一行的原来值更新本行,然后处理上一行.唯一需要注意的就是,如果本行出现新0需要更新该0所在列的上面所有行.
    \begin{lstlisting}
void setZeroes(vector<vector<int> > &matrix) {
	int n = matrix.size();
	if(n == 0)	return;
	int m = matrix[0].size();
	bool lastZero = false;
	for(int i = 0; i < n; i++){
		bool thisZero = false;
		for(int j = 0; j < m; j++){
			if(matrix[i][j] == 0){
				int up = i - 1;
				while(up >= 0){
					matrix[up][j] = 0;
					up--;
				}
				thisZero = true;
			}
			if(i > 0 && matrix[i-1][j] == 0)
				matrix[i][j] = 0;
		}
		if(lastZero){
			for(int j = 0; j < m; j++)
				matrix[i-1][j] = 0;
		}
		lastZero = thisZero;
	}
	if(lastZero){
		for(int j = 0; j < m; j++)
			matrix[n-1][j] = 0;
	}
}
    \end{lstlisting}
\end{description}

\subsection{Pascal's Triangle}
    
\begin{description}
    \item{\textbf{问题}}:\\
Given numRows, generate the first numRows of Pascal's triangle. \\
\textit{(leetcode 118)}
    \item{\textbf{举例}}:\\
Given numRows = 5, \\
Return \\
\\
$[$ \\
     $[1]$, \\
    $[1,1]$, \\
   $[1,2,1]$, \\
  $[1,3,3,1]$, \\
 $[1,4,6,4,1]$ \\
$]$
    \item{\textbf{???}} : \fbox{时间复杂度O($n^2$), 空间复杂度O(1)}
    \begin{lstlisting}
vector<vector<int> > generate(int numRows) {
	vector<vector<int> > result;
	if(numRows == 0)	return result;
	result.push_back(vector<int>{1});
	for(int i = 1; i < numRows; i++){
		vector<int> cur;
		cur.push_back(1);
		for(int j = 0; j < result[i-1].size() - 1; j++)
			cur.push_back(result[i-1][j] + result[i-1][j+1]);
		cur.push_back(1);
		result.push_back(cur);
	}
	return result;
}
    \end{lstlisting}
\end{description}

\subsection{Pascal's Triangle II}
    
\begin{description}
    \item{\textbf{问题}}:\\
Given an index k, return the kth row of the Pascal's triangle. \\
\textit{(leetcode 119)}
    \item{\textbf{举例}}:\\
Given k = 3, \\
Return $[1,3,3,1]$.
    \item{\textbf{Note}}:\\
Could you optimize your algorithm to use only O(k) extra space?
    \item{\textbf{???}} : \fbox{时间复杂度O($k^2$), 空间复杂度O(k)}
    \\这个空间复杂度可以优化,因为每次我们只需要上一行就可以产生本行,所有之前的行可以不存储
    \begin{lstlisting}
	vector<int> getRow(int rowIndex) {
		rowIndex++;
		if(rowIndex <= 0)	return vector<int>();
		vector<int> result{1};
		for(int i = 1; i < rowIndex; i++){
			vector<int> cur;
			cur.push_back(1);
			for(int j = 0; j < result.size() - 1; j++)
				cur.push_back(result[j] + result[j+1]);
			cur.push_back(1);
			result.swap(cur);
		}
		return result;
	}
    \end{lstlisting}
\end{description}



\chapter{数学}
    
\subsection{Spiral Matrix}
    
\begin{description}
    \item{\textbf{问题}}:\\
Given a matrix of m x n elements (m rows, n columns), return all elements of the matrix in spiral order.\\
\textit{(leetcode 54)}
    \item{\textbf{举例}}:\\
Given the following matrix:\\
\\
$[$ \\
 $[ 1, 2, 3 ]$, \\
 $[ 4, 5, 6 ]$, \\
 $[ 7, 8, 9 ]$ \\
$]$ \\
You should return $[1,2,3,6,9,8,7,4,5]$.
    \item{\textbf{???}} : \fbox{时间复杂度O($n^2$), 空间复杂度O(1)}
    \\从外到内一环一环的处理,需要注意一些边界条件
    \begin{lstlisting}
vector<int> spiralOrder(vector<vector<int> > &matrix) {
	vector<int> result;
	int n = matrix.size();
	if(n == 0)	return result;
	int m = matrix[0].size();
	int magrin = 0;
	while(m - 1 - magrin >= magrin && n - 1 - magrin >= magrin){
		for(int j = magrin; j <= m - 1 - magrin; j++)
			result.push_back(matrix[magrin][j]);
		for(int i = magrin + 1; i < n - 1 - magrin; i++)
			result.push_back(matrix[i][m-1-magrin]);
		if(n - 1 - magrin != magrin)
			for(int j = m - 1 - magrin; j >= magrin; j-- )
				result.push_back(matrix[n-1-magrin][j]);
		if(m - 1 - magrin != magrin)
			for(int i = n - 1 - magrin - 1; i > magrin; i--)
				result.push_back(matrix[i][magrin]);
		magrin++;
	}
	return result;
}
    \end{lstlisting}
\end{description}

\subsection{Spiral Matrix II}
    
\begin{description}
    \item{\textbf{问题}}:\\
Given an integer n, generate a square matrix filled with elements from 1 to $n^2$ in spiral order.\\
\textit{(leetcode 59)}
    \item{\textbf{举例}}:\\
Given n = 3,\\
\\
You should return the following matrix:\\
$[$ \\
 $[ 1, 2, 3 ]$, \\
 $[ 8, 9, 4 ]$, \\
 $[ 7, 6, 5 ]$ \\
$]$
    \item{\textbf{???}} : \fbox{时间复杂度O($n^2$), 空间复杂度O(1)}
    \begin{lstlisting}
vector<vector<int> > generateMatrix(int n) {
	vector<vector<int> > matrix(n, vector<int>(n, 0));
	int pos = 1;
	int magrin = 0;
	while(n - 1 - magrin >= magrin && n - 1 - magrin >= magrin){
		for(int j = magrin; j <= n - 1 - magrin; j++)
			matrix[magrin][j] = pos++;
		for(int i = magrin + 1; i < n - 1 - magrin; i++)
			matrix[i][n-1-magrin] = pos++;
		if(n - 1 - magrin != magrin)
			for(int j = n - 1 - magrin; j >= magrin; j-- )
				matrix[n-1-magrin][j] = pos++;
		if(n - 1 - magrin != magrin)
			for(int i = n - 1 - magrin - 1; i > magrin; i--)
				matrix[i][magrin] = pos++;
		magrin++;
	}
	return matrix;
}
    \end{lstlisting}
\end{description}

\subsection{Set Matrix Zeroes}
    
\begin{description}
    \item{\textbf{问题}}:\\
Given a m x n matrix, if an element is 0, set its entire row and column to 0. Do it in place.\\
\textit{(leetcode 73)}
    \item{\textbf{Follow Up}}:\\
Did you use extra space? \\
A straight forward solution using O(mn) space is probably a bad idea. \\
A simple improvement uses O(m + n) space, but still not the best solution. \\
Could you devise a constant space solution?
    \item{\textbf{???}} : \fbox{时间复杂度O($n^2$), 空间复杂度O(1)}
    \\从上到下一行一行的处理,如果上一个存在0,可以先保留上一行的现场,然后根据上一行的原来值更新本行,然后处理上一行.唯一需要注意的就是,如果本行出现新0需要更新该0所在列的上面所有行.
    \begin{lstlisting}
void setZeroes(vector<vector<int> > &matrix) {
	int n = matrix.size();
	if(n == 0)	return;
	int m = matrix[0].size();
	bool lastZero = false;
	for(int i = 0; i < n; i++){
		bool thisZero = false;
		for(int j = 0; j < m; j++){
			if(matrix[i][j] == 0){
				int up = i - 1;
				while(up >= 0){
					matrix[up][j] = 0;
					up--;
				}
				thisZero = true;
			}
			if(i > 0 && matrix[i-1][j] == 0)
				matrix[i][j] = 0;
		}
		if(lastZero){
			for(int j = 0; j < m; j++)
				matrix[i-1][j] = 0;
		}
		lastZero = thisZero;
	}
	if(lastZero){
		for(int j = 0; j < m; j++)
			matrix[n-1][j] = 0;
	}
}
    \end{lstlisting}
\end{description}

\subsection{Pascal's Triangle}
    
\begin{description}
    \item{\textbf{问题}}:\\
Given numRows, generate the first numRows of Pascal's triangle. \\
\textit{(leetcode 118)}
    \item{\textbf{举例}}:\\
Given numRows = 5, \\
Return \\
\\
$[$ \\
     $[1]$, \\
    $[1,1]$, \\
   $[1,2,1]$, \\
  $[1,3,3,1]$, \\
 $[1,4,6,4,1]$ \\
$]$
    \item{\textbf{???}} : \fbox{时间复杂度O($n^2$), 空间复杂度O(1)}
    \begin{lstlisting}
vector<vector<int> > generate(int numRows) {
	vector<vector<int> > result;
	if(numRows == 0)	return result;
	result.push_back(vector<int>{1});
	for(int i = 1; i < numRows; i++){
		vector<int> cur;
		cur.push_back(1);
		for(int j = 0; j < result[i-1].size() - 1; j++)
			cur.push_back(result[i-1][j] + result[i-1][j+1]);
		cur.push_back(1);
		result.push_back(cur);
	}
	return result;
}
    \end{lstlisting}
\end{description}

\subsection{Pascal's Triangle II}
    
\begin{description}
    \item{\textbf{问题}}:\\
Given an index k, return the kth row of the Pascal's triangle. \\
\textit{(leetcode 119)}
    \item{\textbf{举例}}:\\
Given k = 3, \\
Return $[1,3,3,1]$.
    \item{\textbf{Note}}:\\
Could you optimize your algorithm to use only O(k) extra space?
    \item{\textbf{???}} : \fbox{时间复杂度O($k^2$), 空间复杂度O(k)}
    \\这个空间复杂度可以优化,因为每次我们只需要上一行就可以产生本行,所有之前的行可以不存储
    \begin{lstlisting}
	vector<int> getRow(int rowIndex) {
		rowIndex++;
		if(rowIndex <= 0)	return vector<int>();
		vector<int> result{1};
		for(int i = 1; i < rowIndex; i++){
			vector<int> cur;
			cur.push_back(1);
			for(int j = 0; j < result.size() - 1; j++)
				cur.push_back(result[j] + result[j+1]);
			cur.push_back(1);
			result.swap(cur);
		}
		return result;
	}
    \end{lstlisting}
\end{description}



\end{CJK}
\end{document}
