\ifx allfiles undefined
\documentclass{article}
\usepackage{CJK}
\usepackage{verbatim}

%%%代码
\usepackage{color}
\usepackage{xcolor}
\definecolor{keywordcolor}{rgb}{0.8,0.1,0.5}
\usepackage{listings}
\lstset{breaklines}%这条命令可以让LaTeX自动将长的代码行换行排版
\lstset{extendedchars=false}%这一条命令可以解决代码跨页时,章节标题,页眉等汉字不显示的问题
\lstset{language=C++, %用于设置语言为C++
    keywordstyle=\color{keywordcolor} \bfseries, %设置关键词
    identifierstyle=,
    basicstyle=\ttfamily, 
    commentstyle=\color{blue} \textit,
    stringstyle=\ttfamily, 
    showstringspaces=false,
    %frame=shadowbox, %边框
    captionpos=b
}
%%%

%\hypersetup{CJKbookmarks=true} %解决section不能使用中文的问题

\begin{document}
\begin{CJK}{UTF8}{gbsn}     %CJK:支持中文

\else
    
\begin{description}
    \item{\textbf{问题}}:\\
There are N children standing in a line. Each child is assigned a rating value.
You are giving candies to these children subjected to the following requirements:\\
Each child must have at least one candy.\\
Children with a higher rating get more candies than their neighbors.\\
What is the minimum candies you must give? \textit{(leetcode 135)}
    \item{\textbf{贪心}} : \fbox{时间复杂度O(n), 空间复杂度O(n)}
	\\我们对每个位置设两个变量left[i]和right[i]分别表示位于i左边/右边连续的小于A[i]的数目. 然后我们可以使用递推公式left[i] = A[i] $>$ A[i-1]?  left[i-1]+1 : 0来求left数组,对应right[i] = A[i] $>$ A[i+1]? right[i+1]+1 : 0来求right数组.\\
    \\这题老实说感觉应该算动态规划,不过贪心也算是一种特殊的动态规划了,而且left和right数组的求值都是直接通过最优子问题求出,不是多个选出的,从这种角度来说应该算是贪心算法.
    \begin{lstlisting}
int candy(vector<int> &ratings) {
	int size = ratings.size();
	vector<int> left(size, 0), right(size, 0);
	for(int i = 1; i < size; i++)
		if(ratings[i] > ratings[i-1]) left[i] = left[i-1] + 1;
		else	left[i] = 0;
	for(int i = size - 2; i >= 0; i--)
		if(ratings[i] > ratings[i+1]) right[i] = right[i+1] + 1;
		else	right[i] = 0;
	int sum = 0;
	for(int i = 0; i < size; i++)
		sum += max(left[i], right[i]) + 1;
	return sum;
}
    \end{lstlisting}
    \textit{left[i]和right[i]是一种常用的方法}
\end{description}

\fi

\ifx allfiles undefined
\end{CJK}
\end{document}
\fi
