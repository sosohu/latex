\ifx allfiles undefined
\documentclass{article}
\usepackage{CJK}
\usepackage{verbatim}

%%%代码
\usepackage{color}
\usepackage{xcolor}
\definecolor{keywordcolor}{rgb}{0.8,0.1,0.5}
\usepackage{listings}
\lstset{breaklines}%这条命令可以让LaTeX自动将长的代码行换行排版
\lstset{extendedchars=false}%这一条命令可以解决代码跨页时,章节标题,页眉等汉字不显示的问题
\lstset{language=C++, %用于设置语言为C++
    keywordstyle=\color{keywordcolor} \bfseries, %设置关键词
    identifierstyle=,
    basicstyle=\ttfamily, 
    commentstyle=\color{blue} \textit,
    stringstyle=\ttfamily, 
    showstringspaces=false,
    %frame=shadowbox, %边框
    captionpos=b
}
%%%

%\hypersetup{CJKbookmarks=true} %解决section不能使用中文的问题

\begin{document}
\begin{CJK}{UTF8}{gbsn}     %CJK:支持中文

\else
    
\begin{description}
    \item{\textbf{问题}}: \\
Given an array of non-negative integers, you are initially positioned at the first index of the array.Each element in the array represents your maximum jump length at that position. Determine if you are able to reach the last index.\textit{(leetcode 55)}

    \item{\textbf{举例}}: \\
A = [2,3,1,1,4], return true.\\
A = [3,2,1,0,4], return false.

    \item{\textbf{贪心}} : \fbox{时间复杂度O(n), 空间复杂度O(1)}
    \\以A = [2,3,1,1,4]举例,从后往前看,其实能到A[4]的有A[1],A[2],A[3],我们知道从A[1]能到A[4]那么必然能从A[1]到A[2]或者A[3]再到A[4],所以我们直接贪心的选取从A[3]到的A[4]. 以这种策略,我们每次只需要选择离目的位置最近的那个跳板就可以了.
    \begin{lstlisting}
bool canJump(int A[], int n) {
	int pos = n - 1;
	while(pos > 0){ 
		int j = pos;
		while(--j >= 0)
			//最近的那个跳板
			if(A[j] + j >= pos) break;
		pos = j;
	}
	return pos == 0;
}   
    \end{lstlisting}
\end{description}

\fi

\ifx allfiles undefined
\end{CJK}
\end{document}
\fi
