\ifx allfiles undefined
\documentclass{article}
\usepackage{CJK}

%%%代码
\usepackage{color}
\usepackage{xcolor}
\definecolor{keywordcolor}{rgb}{0.8,0.1,0.5}
\usepackage{listings}
\lstset{breaklines}%这条命令可以让LaTeX自动将长的代码行换行排版
\lstset{extendedchars=false}%这一条命令可以解决代码跨页时,章节标题,页眉等汉字不显示的问题
\lstset{language=C++, %用于设置语言为C++
    keywordstyle=\color{keywordcolor} \bfseries, %设置关键词
    identifierstyle=,
    basicstyle=\ttfamily, 
    commentstyle=\color{blue} \textit,
    stringstyle=\ttfamily, 
    showstringspaces=false,
    %frame=shadowbox, %边框
    captionpos=b
}
%%%

%\hypersetup{CJKbookmarks=true} %解决section不能使用中文的问题

\begin{document}
\begin{CJK}{UTF8}{gbsn}     %CJK:支持中文

\else
    
\qquad贪心法,又称贪心算法,是一种在每一步选择中都采取在当前状态下最好或最优(即最有利)的选择,从而希望导致结果是最好或最优的算法。[1]比如在旅行推销员问题中,如果旅行员每次都选择最近的城市,那这就是一种贪心算法。\\
\\
\qquad贪心算法在有最优子结构的问题中尤为有效。最优子结构的意思是局部最优解能决定全局最优解。简单地说,问题能够分解成子问题来解决,子问题的最优解能递推到最终问题的最优解。\\
\\
\qquad贪心算法与动态规划的不同在于它每对每个子问题的解决方案都做出选择,不能回退。动态规划则会保存以前的运算结果,并根据以前的结果对当前进行选择,有回退功能。\\
\\
\qquad贪心法可以解决一些最优化问题,如:求图中的最小生成树,求哈夫曼编码,Dijkstra算法……对于其他问题,贪心法一般不能得到我们所要求的答案。一旦一个问题可以通过贪心法来解决,那么贪心法一般是解决这个问题的最好办法。由于贪心法的高效性以及其所求得的答案比较接近最优结果,贪心法也可以用作辅助算法或者直接解决一些要求结果不特别精确的问题。

\fi

\ifx allfiles undefined
\end{CJK}
\end{document}
\fi
