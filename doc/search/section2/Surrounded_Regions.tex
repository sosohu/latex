    
\begin{description}
    \item{\textbf{问题}}:\\
Given a 2D board containing 'X' and 'O', capture all regions surrounded by 'X'.\\
\\
A region is captured by flipping all 'O's into 'X's in that surrounded region.\\
\textit{(leetcode 130)}
    \item{\textbf{举例}}:\\
X X X X\\
X O O X\\
X X O X\\
X O X X\\
After running your function, the board should be:\\
\\
X X X X\\
X X X X\\
X X X X\\
X O X X\\
    \item{\textbf{BFS}} : \fbox{时间复杂度O(n*m), 空间复杂度O(1)}
    \\这题可以用BFS/DFS进行搜索,但是这里有一个trick: 我们知道O要想存活必须能够扩展到边界,否则就肯定被围,所以我们与其随便找个'O'点开始BFS不如从边界某个'O'开始BFS,因为你随便一点BFS时候在你搜索过程中你还不清楚到底这块会不会被围;但是你从边界的'O'开始BFS那么只要与它连通的肯定是活的,就可以通过这个trcik节省空间开销.
    \begin{lstlisting}
void visit(vector<vector<char> >& board, queue<pair<int, int> > &search, 
			int n, int m, int i, int j){
	if(i < 0 || i > n-1 || j < 0 || j > m-1 || board[i][j] != 'O')	return;
	search.push(make_pair(i, j));
	board[i][j] = '-';
}

void bfs(vector<vector<char> > &board, int n, int m, int i, int j){
	if(board[i][j] != 'O')	return;
	queue<pair<int,int> > search;
	pair<int, int> cur;
	search.push(make_pair(i, j));
	board[i][j] = '-';
	while(!search.empty()){
		cur = search.front();
		search.pop();
		i = cur.first;
		j = cur.second;
		visit(board, search, n, m, i-1, j);
		visit(board, search, n, m, i+1, j);
		visit(board, search, n, m, i, j-1);
		visit(board, search, n, m, i, j+1);
}
}

void solve(vector<vector<char> > &board) {
	int n = board.size();
	if(n == 0) return;
	int m = board[0].size();
	//从四个边开始搜,那么找到的必然是活的
	for(int i = 0; i < n; i++){
		bfs(board, n, m, i, 0);
		bfs(board, n, m, i, m-1);
	}
	for(int j = 0; j < m; j++){
		bfs(board, n, m, 0, j);
		bfs(board, n, m, n-1, j);
	}
	for(int i = 0; i < n; i++)
		for(int j = 0; j < m; j++){
			if(board[i][j] == 'O') board[i][j] = 'X';
			if(board[i][j] == '-') board[i][j] = 'O';
		}
}
    \end{lstlisting}
\end{description}
