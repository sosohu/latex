\ifx allfiles undefined
\documentclass{article}
\usepackage{CJK}
\usepackage{verbatim}

%%%代码
\usepackage{color}
\usepackage{xcolor}
\definecolor{keywordcolor}{rgb}{0.8,0.1,0.5}
\usepackage{listings}
\lstset{breaklines}%这条命令可以让LaTeX自动将长的代码行换行排版
\lstset{extendedchars=false}%这一条命令可以解决代码跨页时,章节标题,页眉等汉字不显示的问题
\lstset{language=C++, %用于设置语言为C++
    keywordstyle=\color{keywordcolor} \bfseries, %设置关键词
    identifierstyle=,
    basicstyle=\ttfamily, 
    commentstyle=\color{blue} \textit,
    stringstyle=\ttfamily, 
    showstringspaces=false,
    %frame=shadowbox, %边框
    captionpos=b
}
%%%

%\hypersetup{CJKbookmarks=true} %解决section不能使用中文的问题

\begin{document}
\begin{CJK}{UTF8}{gbsn}     %CJK:支持中文

\else
    
%XXX 问题
\begin{description}
    \item{\textbf{问题}}: Given a binary tree, return the inorder traversal of its nodes' values. \textit{(leetcode 94)}
    \item{\textbf{递归}} : \fbox{时间复杂度O(n) , 空间复杂度O($lgn$)}
    \\
    \begin{lstlisting}
void BiTree::InOrderTraversal(TreeNode* root){
    if(!root)    return;
    if(root->left)
        InOrderTraversal(root->left);
    cout<<root->val<<endl;
    if(root->right)
        InOrderTraversal(root->right);
    \end{lstlisting}
    \item{\textbf{栈式迭代}} : \fbox{时间复杂度O(n) , 空间复杂度O($lgn$)}
    \\这里需要说一下的是,数据结构那本书上写了两种栈式迭代方法,这是其中之一,使用两重循环的那个
    \begin{lstlisting}
void BiTree::InOrderTraversal(TreeNode* root){
    vector<int> data;
    if(!root)    return data;
    stack<TreeNode*> s;
    TreeNode *pos = root;
    while(!s.empty() || pos){
        while(pos){
            s.push(pos);
            pos = pos->left;
        }
        pos = s.top();
        s.pop();
        std::cout<<pos->val<<std::endl;
        pos = pos->right;  //这个非常重要
    }
}
    \end{lstlisting}
    \item{\textbf{栈式迭代}} : \fbox{时间复杂度O(n) , 空间复杂度O($lgn$)}
    \\这里需要说一下的是,数据结构那本书上写了两种栈式迭代方法,这是其中之二,使用一重循环,实际上是一样的
    \begin{lstlisting}
void BiTree::InOrderTraversal(TreeNode* root){
    vector<int> data;
    if(!root)    return data;
    stack<TreeNode*> s;
    TreeNode *pos = root;
    while(!s.empty() || pos){
        if(pos){
            s.push(pos);
            pos = pos->left;
        }else{
            pos = s.top();
            s.pop();
            std::cout<<pos->val<<std::endl;
            pos = pos->right;  //这个非常重要
        }
    }
}
    \end{lstlisting}
    \item{\textbf{Mirror迭代}} : \fbox{时间复杂度O(n) , 空间复杂度O(1)}
    \\这里Mirror方法和前序的Mirror方法基本一样,唯一的区别就是打印当前值的时机
    \begin{lstlisting}
void BiTree::InOrderTraversal(TreeNode* root){
    if(!root)    return;
    TreeNode *curr = root, *next;
    while(curr){
        next = curr->left;
        if(!next){
            cout<<curr->val<<endl;
            curr = curr->right;
            continue;
        }
        while(next->right && next->right != curr){
            next = next->right;
        }
        if(next->right == curr){
            next->right = NULL;
            cout<<curr->val<<endl;
            curr = curr->right;
        }else{
            next->right = curr;
            curr = curr->left;
        }
    }
}
    \end{lstlisting}
\end{description}

\fi

\ifx allfiles undefined
\end{CJK}
\end{document}
\fi
