\ifx allfiles undefined
\documentclass{article}
\usepackage{CJK}
\usepackage{verbatim}

%%%代码
\usepackage{color}
\usepackage{xcolor}
\definecolor{keywordcolor}{rgb}{0.8,0.1,0.5}
\usepackage{listings}
\lstset{breaklines}%这条命令可以让LaTeX自动将长的代码行换行排版
\lstset{extendedchars=false}%这一条命令可以解决代码跨页时,章节标题,页眉等汉字不显示的问题
\lstset{language=C++, %用于设置语言为C++
    keywordstyle=\color{keywordcolor} \bfseries, %设置关键词
    identifierstyle=,
    basicstyle=\ttfamily, 
    commentstyle=\color{blue} \textit,
    stringstyle=\ttfamily, 
    showstringspaces=false,
    %frame=shadowbox, %边框
    captionpos=b
}
%%%

%\hypersetup{CJKbookmarks=true} %解决section不能使用中文的问题

\begin{document}
\begin{CJK}{UTF8}{gbsn}     %CJK:支持中文

\else
    
\begin{description}
    \item{\textbf{问题}}: Given a binary tree, determine if it is a valid binary search tree (BST). \textit{(leetcode 98)}
    \\判断一个二叉树是否是合法的BST,我们可以想到BST树的中序序列是非减序列,于是我们可以使用中序遍历这颗二叉树,在遍历的过程中查看是否有反常的数据.
    \\当然,根据上面说的三种中序遍历的方法,这里同样有三种解法.
    \item{\textbf{递归}} : \fbox{时间复杂度O(n) , 空间复杂度O($lgn$)}
    \begin{lstlisting}
bool dfs(TreeNode *root, int& up){
    if(!root)    return true;
    if(root->left){
        bool left =  dfs(root->left, up);
        if(!left) return false;
    }
    if(root->val <= up && (MIN || up != (-1)<<31)){ 
        return false;
    }    
    if(root->val == (-1)<<31)
        MIN = true;
    up = root->val;
    if(root->right){
        bool right = dfs(root->right, up);
        if(!right)    return false;
    }
    return true;
}

bool isValidBST(TreeNode *root) {
    if(!root)    return true;
    int up = (-1)<<31;
    MIN = false;
    return dfs(root, up);
}
    \end{lstlisting}
    \textit{这里可以看到一些边界条件的判断,显得有点复杂,其实就是简单的中序遍历}
    \item{\textbf{栈式迭代}} : \fbox{时间复杂度O(n) , 空间复杂度O($lgn$)}
    \begin{lstlisting}
bool isValidBST(TreeNode *root) {
    if(!root)    return false;
    stack<TreeNode*> s;
    TreeNode *p = root;
    s.push(root);
    while(p->left){
        s.push(p->left);
        p = p->left;
    }
    int last = p->val;
    s.pop();
    if(p->right){
        s.push(p->right);
        p = p->right;
        while(p->left){
            s.push(p->left);
            p = p->left;
        }
    }
    while(!s.empty()){
        p = s.top();
        s.pop();
        if(last >= p->val) return false;
        last = p->val;
        if(p->right){
            s.push(p->right);
            p = p->right;
            while(p->left){
                s.push(p->left);
                p = p->left;
            }
        }
    }
    return true;
}
    \end{lstlisting}
    \item{\textbf{Mirror迭代}} : \fbox{时间复杂度O(n) , 空间复杂度O(1)}
    \\这里使用Mirror建立线索然后进行中序遍历,在中序遍历的同时进行判断
    \begin{lstlisting}
bool isValidBST(TreeNode *root) {
    if(!root)    return true;
    TreeNode *curr = root, *next;
    int last = INT_MIN;
    bool isFirst = true;
    bool ret = true;
    while(curr){
        if(!curr->left){
            if(!isFirst && curr->val <= last){
                ret = false;
            }
            if(isFirst)
                isFirst = false;
            last = curr->val;
            curr = curr->right;
            continue;
        }
        next = curr->left;
        while(next->right){
            if(next->right == curr)    break;
            next = next->right;
        }
        if(next->right == curr){
            next->right = NULL;
            if(!isFirst && curr->val <= last){
                ret = false;
            }
            if(isFirst)
                isFirst = false;
            last = curr->val;
            curr = curr->right;
        }else{
            next->right = curr;
            curr = curr->left;
        }
    }
    return ret;
}
    \end{lstlisting}
    \textit{有了Mirror算法,是不是你已经爱上它了,再也不用栈这么麻烦了,不过有一点需要注意的是一旦你使用Mirror算法,那么必须保证把整个树全遍历一遍,不能中途退出,因为那样树的结构被改变了}
\end{description}

\fi

\ifx allfiles undefined
\end{CJK}
\end{document}
\fi
