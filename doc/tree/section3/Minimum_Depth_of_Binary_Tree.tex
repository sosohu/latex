\ifx allfiles undefined
\documentclass{article}
\usepackage{CJK}
\usepackage{verbatim}

%%%代码
\usepackage{color}
\usepackage{xcolor}
\definecolor{keywordcolor}{rgb}{0.8,0.1,0.5}
\usepackage{listings}
\lstset{breaklines}%这条命令可以让LaTeX自动将长的代码行换行排版
\lstset{extendedchars=false}%这一条命令可以解决代码跨页时,章节标题,页眉等汉字不显示的问题
\lstset{language=C++, %用于设置语言为C++
    keywordstyle=\color{keywordcolor} \bfseries, %设置关键词
    identifierstyle=,
    basicstyle=\ttfamily, 
    commentstyle=\color{blue} \textit,
    stringstyle=\ttfamily, 
    showstringspaces=false,
    %frame=shadowbox, %边框
    captionpos=b
}
%%%

%\hypersetup{CJKbookmarks=true} %解决section不能使用中文的问题

\begin{document}
\begin{CJK}{UTF8}{gbsn}     %CJK:支持中文

\else
    
\begin{description}
    \item{\textbf{问题}}: Given a binary tree, find its minimum depth. The minimum depth is the number of nodes along the shortest path from the root node down to the nearest leaf node. \textit{(leetcode 111)}
    \item{\textbf{递归}} : \fbox{时间复杂度O(n), 空间复杂度O($lgn$)}
    \\自下而上的递归,非常的简单
    \begin{lstlisting}
int minDepth(TreeNode *root) {
    if(!root)    return 0;
    if(!root->left && !root->right)    return 1;
    if(!root->left)
        return minDepth(root->right) + 1;
    if(!root->right)
        return minDepth(root->left) + 1;
    return min(minDepth(root->left), minDepth(root->right)) + 1;
}
    \end{lstlisting}
    \item{\textbf{DFS}} : \fbox{时间复杂度O(n) , 空间复杂度O($lgn$)}
    \\这也是递归,但是是一种自上而下的递归,可以进行剪枝而不必把整个树都访问一遍
    \begin{lstlisting}
void dfs(TreeNode *root, int &result, int depth){
    if(result < depth + 1) return;
    if(!root->left && !root->right){
        result = depth + 1;
        return;
    }
    if(root->left)
        dfs(root->left, result, depth + 1);
    if(root->right)
        dfs(root->right, result, depth + 1);
}

    int minDepth(TreeNode *root) {
    if(!root)    return 0;
    int result = INT_MAX;
    dfs(root, result, 0);
    return result;
}
    \end{lstlisting}
    \item{\textbf{迭代}} : \fbox{时间复杂度O(n) , 空间复杂度O($lgn$)}
    \\同样我们也可以剪枝
    \begin{lstlisting}
int minDepth(TreeNode *root) {
    if(!root)    return 0;
    stack<pair<TreeNode*, int> > s;
    s.push(make_pair(root, 1));
    int     result = -((1<<31) + 1);
    TreeNode *node;
    int depth;
    while(!s.empty()){
        node = s.top().first;
        depth = s.top().second;
        s.pop();
        if(result < depth)    continue;
        if(!node->left && !node->right)
            result = depth;
    
        if(node->left )
            s.push(make_pair(node->left, depth + 1));
        if(node->right)
            s.push(make_pair(node->right, depth + 1));
    }
    return result;
}
    \end{lstlisting}
\end{description}

\fi

\ifx allfiles undefined
\end{CJK}
\end{document}
\fi
