\ifx allfiles undefined
\documentclass{article}
\usepackage{CJK}
\usepackage{verbatim}

%%%代码
\usepackage{color}
\usepackage{xcolor}
\definecolor{keywordcolor}{rgb}{0.8,0.1,0.5}
\usepackage{listings}
\lstset{breaklines}%这条命令可以让LaTeX自动将长的代码行换行排版
\lstset{extendedchars=false}%这一条命令可以解决代码跨页时,章节标题,页眉等汉字不显示的问题
\lstset{language=C++, %用于设置语言为C++
    keywordstyle=\color{keywordcolor} \bfseries, %设置关键词
    identifierstyle=,
    basicstyle=\ttfamily, 
    commentstyle=\color{blue} \textit,
    stringstyle=\ttfamily, 
    showstringspaces=false,
    %frame=shadowbox, %边框
    captionpos=b
}
%%%

%\hypersetup{CJKbookmarks=true} %解决section不能使用中文的问题

\begin{document}
\begin{CJK}{UTF8}{gbsn}     %CJK:支持中文

\else
    
\begin{description}
    \item{\textbf{问题}}: Given a list, rotate the list.
    \item{\textbf{头插法}} : \fbox{时间复杂度O(n) , 空间复杂度O(1)}
    \\新建一个头节点,然后依次将后面节点插入到头节点之后,便形成链表的倒序.
    \begin{lstlisting}
ListNode *ReverseList(ListNode *pHead){
	if(!pHead || !pHead->next)	return pHead;
	ListNode * newhead = new ListNode();
	ListNode *pos = pHead, *tmp;
	newhead->next = pHead;
	while(pos->next){
		tmp = pos->next;
		pos->next = tmp->next;
		tmp->next = newhead->next;
		newhead->next = tmp;
	}
	return newhead->next;
}
    \end{lstlisting}
    \item{\textbf{就地反转}} : \fbox{时间复杂度O(n) , 空间复杂度O(1)}
	\\所谓就地反转就是每次把i指向i+1直接转为i指向i+1即可.
	\begin{lstlisting}
ListNode *ReverseList(ListNode *pHead){
	if(!pHead || !pHead->next)	return pHead;
	ListNode *pre = pHead, *cur = pHead->next, *next = NULL;
	pHead->next = NULL;
	while(cur){
		next = cur->next;
		cur->next = pre;
		pre = cur;
		cur = next;
	}
	return pre;
}
	\end{lstlisting}
    \item{\textbf{递归}} : \fbox{时间复杂度O(n) , 空间复杂度O(1)}
	\\链表其实就是一个二叉树,所以可以套用二叉树的递归做法.
	\begin{lstlisting}
ListNode *ReverseList(ListNode *pHead){
	if(!pHead || !pHead->next)	return pHead;
	// 返回left为left段倒转后的首节点
	ListNode *left = ReverseList(pHead->next); 
	// pHead->next 一开始是left段的首节点,倒转后就是末节点
	pHead->next->next = pHead; 
	pHead->next = NULL;
	return left;
}
	\end{lstlisting}
\end{description}

\fi

\ifx allfiles undefined
\end{CJK}
\end{document}
\fi
