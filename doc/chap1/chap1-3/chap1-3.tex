\ifx allfiles undefined
\documentclass{article}
\usepackage{CJK}

%%%代码
\usepackage{color}
\usepackage{xcolor}
\definecolor{keywordcolor}{rgb}{0.8,0.1,0.5}
\usepackage{listings}
\lstset{breaklines}%这条命令可以让LaTeX自动将长的代码行换行排版
\lstset{extendedchars=false}%这一条命令可以解决代码跨页时,章节标题,页眉等汉字不显示的问题
\lstset{language=C++, %用于设置语言为C++
    keywordstyle=\color{keywordcolor} \bfseries, %设置关键词
    identifierstyle=,
    basicstyle=\ttfamily, 
    commentstyle=\color{blue} \textit,
    stringstyle=\ttfamily, 
    showstringspaces=false,
    %frame=shadowbox, %边框
    captionpos=b
}
%%%

%\hypersetup{CJKbookmarks=true} %解决section不能使用中文的问题

\begin{document}
\begin{CJK}{UTF8}{gbsn}     %CJK:支持中文

\else
    
\qquad基于链表的问题,大多需要前面所说的基本操作和基本问题的组合,另外,对于链表这个容器的认识也要很深刻才行,链表容器的优势在于能够快速的插入和删除指定节点,而且节省空间,但是它的问题就是不能随机访问,可以说和vector是互补的.
\subsection{Add Two Numbers}
\ifx allfiles undefined
\documentclass{article}
\usepackage{CJK}
\usepackage{verbatim}

%%%代码
\usepackage{color}
\usepackage{xcolor}
\definecolor{keywordcolor}{rgb}{0.8,0.1,0.5}
\usepackage{listings}
\lstset{breaklines}%这条命令可以让LaTeX自动将长的代码行换行排版
\lstset{extendedchars=false}%这一条命令可以解决代码跨页时,章节标题,页眉等汉字不显示的问题
\lstset{language=C++, %用于设置语言为C++
    keywordstyle=\color{keywordcolor} \bfseries, %设置关键词
    identifierstyle=,
    basicstyle=\ttfamily, 
    commentstyle=\color{blue} \textit,
    stringstyle=\ttfamily, 
    showstringspaces=false,
    %frame=shadowbox, %边框
    captionpos=b
}
%%%

%\hypersetup{CJKbookmarks=true} %解决section不能使用中文的问题

\begin{document}
\begin{CJK}{UTF8}{gbsn}     %CJK:支持中文

\else
    
\begin{description}
    \item{\textbf{问题}}: You are given two linked lists representing two non-negative numbers. The digits are stored in reverse order and each of their nodes contain a single digit. Add the two numbers and return it as a linked list. \textit{(leetcode 2)}
	\item{\textbf{举例}}: Input: (2 $\rightarrow$ 4 $\rightarrow$ 3) + (5 $\rightarrow$ 6 $\rightarrow$ 4), Output: 7 $\rightarrow$ 0 $\rightarrow$ 8
    \item{\textbf{Care}} : \fbox{时间复杂度O(n+m), 空间复杂度O(1)}
    \\这道题主要是要注意进位和一些边界条件判断的简化,这里有很多边界条件,怎么把很多if语句归纳到一起使代码更简洁.
    \begin{lstlisting}
ListNode *addTwoNumbers(ListNode *l1, ListNode *l2) {
	ListNode head(0), *ret = &head;
	int cin = 0, val;
	while(l1 || l2 || cin){
		val = (l1? l1->val : 0) + (l2? l2->val : 0) + cin;
		ret = ret->next = new ListNode(val % 10);
		cin = val / 10;
		if(l1)	l1 = l1->next;
		if(l2)	l2 = l2->next;
	}
	return head.next;
}
    \end{lstlisting}
\end{description}

\fi

\ifx allfiles undefined
\end{CJK}
\end{document}
\fi

\subsection{Remove Nth Node From End of List}
\ifx allfiles undefined
\documentclass{article}
\usepackage{CJK}
\usepackage{verbatim}

%%%代码
\usepackage{color}
\usepackage{xcolor}
\definecolor{keywordcolor}{rgb}{0.8,0.1,0.5}
\usepackage{listings}
\lstset{breaklines}%这条命令可以让LaTeX自动将长的代码行换行排版
\lstset{extendedchars=false}%这一条命令可以解决代码跨页时,章节标题,页眉等汉字不显示的问题
\lstset{language=C++, %用于设置语言为C++
    keywordstyle=\color{keywordcolor} \bfseries, %设置关键词
    identifierstyle=,
    basicstyle=\ttfamily, 
    commentstyle=\color{blue} \textit,
    stringstyle=\ttfamily, 
    showstringspaces=false,
    %frame=shadowbox, %边框
    captionpos=b
}
%%%

%\hypersetup{CJKbookmarks=true} %解决section不能使用中文的问题

\begin{document}
\begin{CJK}{UTF8}{gbsn}     %CJK:支持中文

\else
    
\begin{description}
    \item{\textbf{问题}}: Given a linked list, remove the nth node from the end of list and return its head. \textit{(leetcode 19)}
    \item{\textbf{举例}} : Given linked list: $1\rightarrow2\rightarrow3\rightarrow4\rightarrow5$, and n = 2. After removing the second node from the end, the linked list becomes $1\rightarrow2\rightarrow3\rightarrow5$.
    \item{\textbf{Note}} : Given n will always be valid.
    \item{\textbf{两个指针}} : \fbox{时间复杂度O(n), 空间复杂度O(1)}
    \\两个指针相距n,然后一起后移.
    \begin{lstlisting}
ListNode *removeNthFromEnd(ListNode *head, int n) {
	if(!head || n < 1)	return head;
	ListNode newhead(0);
	newhead.next = head;
	ListNode *last = &newhead, *first = &newhead;
	while(n-- > 0 && last->next){
		last = last->next;
	}
	while(last->next){
		first = first->next;
		last = last->next;
	}
	first->next = first->next->next;
	return newhead.next;
}
    \end{lstlisting}
\end{description}

\fi

\ifx allfiles undefined
\end{CJK}
\end{document}
\fi

\subsection{Merge k Sorted Lists}
\ifx allfiles undefined
\documentclass{article}
\usepackage{CJK}
\usepackage{verbatim}

%%%代码
\usepackage{color}
\usepackage{xcolor}
\definecolor{keywordcolor}{rgb}{0.8,0.1,0.5}
\usepackage{listings}
\lstset{breaklines}%这条命令可以让LaTeX自动将长的代码行换行排版
\lstset{extendedchars=false}%这一条命令可以解决代码跨页时,章节标题,页眉等汉字不显示的问题
\lstset{language=C++, %用于设置语言为C++
    keywordstyle=\color{keywordcolor} \bfseries, %设置关键词
    identifierstyle=,
    basicstyle=\ttfamily, 
    commentstyle=\color{blue} \textit,
    stringstyle=\ttfamily, 
    showstringspaces=false,
    %frame=shadowbox, %边框
    captionpos=b
}
%%%

%\hypersetup{CJKbookmarks=true} %解决section不能使用中文的问题

\begin{document}
\begin{CJK}{UTF8}{gbsn}     %CJK:支持中文

\else
    
\begin{description}
    \item{\textbf{问题}}: Merge k sorted linked lists and return it as one sorted list. Analyze and describe its complexity. \textit{(leetcode 23)}
    \item{\textbf{}} : \fbox{时间复杂度 , 空间复杂度O}
    \\这道题其实是Merge two list的延伸,我们在Merge两个链表时候,每次都是比较一下取较小的入新链表,当有k个链表的时候,我们也是取这k个节点最小的那个,然后用它的next替换它,再进行下次选取.我们想一下,最naive的做法是,每次这k个都比较一遍,那么这个每取一个的时间复杂度就是O(k),整个的时间复杂度就是O(k*k*n). 其实这里面的过程有一个信息被我们遗漏了,或者说是没有被利用好,那就是当我们取出这k个节点最小那个时候,其实除了最小的那个其他的节点之间大小关系也有比较过,比如是节点s和t,然后新替入的那个进来之后,再来一次取最小值,这时候其实s和t的大小关系我们是已知的,但是我们没有利用,而是有可能还会取比较一次! 为了避免这个重复比较,我们想到使用最小堆的方法: 把这个k个节点放入最小堆,然后取堆顶点,再加入新节点,整个操作只需要O(lgk)的时间复杂度,所以整个的时间复杂度降为O(k*nlgk)
    \begin{lstlisting}
ListNode *mergeKLists(vector<ListNode*> &lists) {
	multimap<int, int> data;
	ListNode ret(0) , *pos = &ret;
	for(int i = 0; i < lists.size(); i++){
		if(lists[i])
			data.insert(make_pair(lists[i]->val, i));
	}
	while(!data.empty()){
		int index = data.begin()->second;
		data.erase(data.begin());
		pos = pos->next = lists[index];
		lists[index] = lists[index]->next;
		if(lists[index])
			data.insert(make_pair(lists[index]->val, index));
	}
	return ret.next;
}
    \end{lstlisting}
	上面没有使用堆,但是使用的是tree map,它其实可以用来模拟堆操作,整个的时间复杂度和用堆是一样的.
\end{description}

\fi

\ifx allfiles undefined
\end{CJK}
\end{document}
\fi

\subsection{Swap Nodes in Pairs}
\ifx allfiles undefined
\documentclass{article}
\usepackage{CJK}
\usepackage{verbatim}

%%%代码
\usepackage{color}
\usepackage{xcolor}
\definecolor{keywordcolor}{rgb}{0.8,0.1,0.5}
\usepackage{listings}
\lstset{breaklines}%这条命令可以让LaTeX自动将长的代码行换行排版
\lstset{extendedchars=false}%这一条命令可以解决代码跨页时,章节标题,页眉等汉字不显示的问题
\lstset{language=C++, %用于设置语言为C++
    keywordstyle=\color{keywordcolor} \bfseries, %设置关键词
    identifierstyle=,
    basicstyle=\ttfamily, 
    commentstyle=\color{blue} \textit,
    stringstyle=\ttfamily, 
    showstringspaces=false,
    %frame=shadowbox, %边框
    captionpos=b
}
%%%

%\hypersetup{CJKbookmarks=true} %解决section不能使用中文的问题

\begin{document}
\begin{CJK}{UTF8}{gbsn}     %CJK:支持中文

\else
    
\begin{description}
    \item{\textbf{问题}}: Given a linked list, swap every two adjacent nodes and return its head. \textit{(leetcode 24)}
    \item{\textbf{举例}} : Given $1\rightarrow2\rightarrow3\rightarrow4$, you should return the list as $2\rightarrow1\rightarrow4\rightarrow3$.
    \item{\textbf{Note}} : Your algorithm should use only constant space. You may not modify the values in the list, only nodes itself can be changed.
    \item{\textbf{两个指针}} : \fbox{时间复杂度O(n), 空间复杂度O(1)}
    \\这题主要是考查细心程度,还有对边界条件的考虑.
    \begin{lstlisting}
ListNode *swapPairs(ListNode *head) {
	if(!head || !head->next)	return head;
	ListNode newhead(0), *first = &newhead, *second = &newhead;
	newhead.next = head;
	while(first->next && first->next->next){
		second = first->next;
		first->next = second->next;
		second->next = first->next->next;
		first->next->next = second;
		first = second;
	}
	return newhead.next;
}
    \end{lstlisting}
\end{description}

\fi

\ifx allfiles undefined
\end{CJK}
\end{document}
\fi

\subsection{Reverse Nodes in k-Group}
\ifx allfiles undefined
\documentclass{article}
\usepackage{CJK}
\usepackage{verbatim}

%%%代码
\usepackage{color}
\usepackage{xcolor}
\definecolor{keywordcolor}{rgb}{0.8,0.1,0.5}
\usepackage{listings}
\lstset{breaklines}%这条命令可以让LaTeX自动将长的代码行换行排版
\lstset{extendedchars=false}%这一条命令可以解决代码跨页时,章节标题,页眉等汉字不显示的问题
\lstset{language=C++, %用于设置语言为C++
    keywordstyle=\color{keywordcolor} \bfseries, %设置关键词
    identifierstyle=,
    basicstyle=\ttfamily, 
    commentstyle=\color{blue} \textit,
    stringstyle=\ttfamily, 
    showstringspaces=false,
    %frame=shadowbox, %边框
    captionpos=b
}
%%%

%\hypersetup{CJKbookmarks=true} %解决section不能使用中文的问题

\begin{document}
\begin{CJK}{UTF8}{gbsn}     %CJK:支持中文

\else
    
\begin{description}
    \item{\textbf{问题}}: Given a linked list, reverse the nodes of a linked list k at a time and return its modified list. \textit{(leetcode 25)}
    \item{\textbf{举例}} : Given this linked list: $1\rightarrow2\rightarrow3\rightarrow4\rightarrow5$ , For k = 2, you should return: $2\rightarrow1\rightarrow4\rightarrow3\rightarrow5$, For k = 3, you should return: $3\rightarrow2\rightarrow1\rightarrow4\rightarrow5$
    \item{\textbf{Note}} : If the number of nodes is not a multiple of k then left-out nodes in the end should remain as it is. You may not alter the values in the nodes, only nodes itself may be changed. Only constant memory is allowed.
    \item{\textbf{Care}} : \fbox{时间复杂度O(n), 空间复杂度O(1)}
    \\这题最保守的做法就是先算出链表的长度,然后一段一段的反转.
    \begin{lstlisting}
ListNode *reverseKGroup(ListNode *head, int k) {
	if(!head || k <= 1)	return head;
	ListNode newhead(0), *first = &newhead, *second = &newhead, *pos = head;
	newhead.next = head;
	int count = 0;
	while(pos){
		pos = pos->next;
		count++;
	}
	while(count >= k){
		second = first->next;
		for(int i = 0; i < k - 1; i++){
			pos = second->next;
			second->next = pos->next;
			pos->next = first->next;
			first->next = pos;
		}
		count = count - k;
		first = second;
	}
	return newhead.next;
}
    \end{lstlisting}
\end{description}

\fi

\ifx allfiles undefined
\end{CJK}
\end{document}
\fi

\subsection{Rotate List}
\ifx allfiles undefined
\documentclass{article}
\usepackage{CJK}
\usepackage{verbatim}

%%%代码
\usepackage{color}
\usepackage{xcolor}
\definecolor{keywordcolor}{rgb}{0.8,0.1,0.5}
\usepackage{listings}
\lstset{breaklines}%这条命令可以让LaTeX自动将长的代码行换行排版
\lstset{extendedchars=false}%这一条命令可以解决代码跨页时,章节标题,页眉等汉字不显示的问题
\lstset{language=C++, %用于设置语言为C++
    keywordstyle=\color{keywordcolor} \bfseries, %设置关键词
    identifierstyle=,
    basicstyle=\ttfamily, 
    commentstyle=\color{blue} \textit,
    stringstyle=\ttfamily, 
    showstringspaces=false,
    %frame=shadowbox, %边框
    captionpos=b
}
%%%

%\hypersetup{CJKbookmarks=true} %解决section不能使用中文的问题

\begin{document}
\begin{CJK}{UTF8}{gbsn}     %CJK:支持中文

\else
    
\begin{description}
    \item{\textbf{问题}}: Given a list, rotate the list to the right by k places, where k is non-negative. \textit{(leetcode 61)}
    \item{\textbf{示例}} : Given $1\rightarrow2\rightarrow3\rightarrow4\rightarrow5\rightarrow NULL$ and k = 2, return $4\rightarrow5\rightarrow1\rightarrow2\rightarrow3\rightarrow NULL$.
    \item{\textbf{Care}} : \fbox{时间复杂度O(n), 空间复杂度O(1)}
    \\首尾连接一下轻松搞定链表旋转
    \begin{lstlisting}
ListNode *rotateRight(ListNode *head, int k) {
	if(!head || k < 1)	return head;
	ListNode *pos = head, *pre = head;
	int count = 0;
	for(; count < k && pos; count++){
		pos = pos->next;
	}
	if(!pos)	return rotateRight(head, k%count);
	while(pos->next){
		pos = pos->next;
		pre = pre->next;
	}
	pos->next = head;
	head = pre->next;
	pre->next = NULL;
	return head;
}
    \end{lstlisting}
\end{description}

\fi

\ifx allfiles undefined
\end{CJK}
\end{document}
\fi

\subsection{Reorder List}
\ifx allfiles undefined
\documentclass{article}
\usepackage{CJK}
\usepackage{verbatim}

%%%代码
\usepackage{color}
\usepackage{xcolor}
\definecolor{keywordcolor}{rgb}{0.8,0.1,0.5}
\usepackage{listings}
\lstset{breaklines}%这条命令可以让LaTeX自动将长的代码行换行排版
\lstset{extendedchars=false}%这一条命令可以解决代码跨页时,章节标题,页眉等汉字不显示的问题
\lstset{language=C++, %用于设置语言为C++
    keywordstyle=\color{keywordcolor} \bfseries, %设置关键词
    identifierstyle=,
    basicstyle=\ttfamily, 
    commentstyle=\color{blue} \textit,
    stringstyle=\ttfamily, 
    showstringspaces=false,
    %frame=shadowbox, %边框
    captionpos=b
}
%%%

%\hypersetup{CJKbookmarks=true} %解决section不能使用中文的问题

\begin{document}
\begin{CJK}{UTF8}{gbsn}     %CJK:支持中文

\else
    
\begin{description}
    \item{\textbf{问题}}: Given a singly linked list L: $L_0 \rightarrow L_1 \rightarrow … \rightarrow L_{n-1} \rightarrow L_n$, reorder it to: $L_0 \rightarrow L_n \rightarrow L_1 \rightarrow L_{n-1} \rightarrow L_2 \rightarrow L_{n-2} \rightarrow...$\textit{(leetcode 143)}
    \item{\textbf{反转}} : \fbox{时间复杂度O(n) , 空间复杂度O(1)}
    \\先获得链表的中间节点,再对后半部分反转,然后间隔插入到前半部分.
    \begin{lstlisting}
void reorderList(ListNode *head) {
	if(!head || !head->next)	return;
	ListNode *slow = head, *fast = head->next, *left = head, *right, *tmp;
	while(fast && fast->next){
		slow = slow->next;
		fast = fast->next->next;
	}
	right = slow->next;
	while(right->next){
		tmp = right->next;
		right->next = tmp->next;
		tmp->next = slow->next;
		slow->next = tmp;
	}
	right = slow->next;
	slow->next = NULL;
	while(left->next){
		tmp = right->next;
		right->next = left->next;
		left->next = right;
		right = tmp;
		left = left->next->next;
	}
	left->next = right;
}
    \end{lstlisting}
\end{description}

\fi

\ifx allfiles undefined
\end{CJK}
\end{document}
\fi

\subsection{Remove Duplicates from Sorted List}
\ifx allfiles undefined
\documentclass{article}
\usepackage{CJK}
\usepackage{verbatim}

%%%代码
\usepackage{color}
\usepackage{xcolor}
\definecolor{keywordcolor}{rgb}{0.8,0.1,0.5}
\usepackage{listings}
\lstset{breaklines}%这条命令可以让LaTeX自动将长的代码行换行排版
\lstset{extendedchars=false}%这一条命令可以解决代码跨页时,章节标题,页眉等汉字不显示的问题
\lstset{language=C++, %用于设置语言为C++
    keywordstyle=\color{keywordcolor} \bfseries, %设置关键词
    identifierstyle=,
    basicstyle=\ttfamily, 
    commentstyle=\color{blue} \textit,
    stringstyle=\ttfamily, 
    showstringspaces=false,
    %frame=shadowbox, %边框
    captionpos=b
}
%%%

%\hypersetup{CJKbookmarks=true} %解决section不能使用中文的问题

\begin{document}
\begin{CJK}{UTF8}{gbsn}     %CJK:支持中文

\else
    
\begin{description}
    \item{\textbf{问题}}: Given a sorted linked list, delete all duplicates such that each element appear only once. \textit{(leetcode 83)}
    \item{\textbf{示例}} : Given $1\rightarrow1\rightarrow2$, return $1\rightarrow2$. Given $1\rightarrow1\rightarrow2\rightarrow3\rightarrow3$, return $1\rightarrow2\rightarrow3$.
    \item{\textbf{两个指针}} : \fbox{时间复杂度O(n), 空间复杂度O(1)}
    \\遇到一样的就删除,不一样就转移.
    \begin{lstlisting}
ListNode *deleteDuplicates(ListNode *head) {
	if(!head || !head->next)	return head;
	ListNode *pre = head , *next = head->next;
	while(next){
		if(pre->val == next->val){
			pre->next = next->next;
		}else{
			pre = pre->next;
		}
		next = pre->next;
	}
	return head;
}
    \end{lstlisting}
\end{description}

\fi

\ifx allfiles undefined
\end{CJK}
\end{document}
\fi

\subsection{Remove Duplicates from Sorted List II}
\ifx allfiles undefined
\documentclass{article}
\usepackage{CJK}
\usepackage{verbatim}

%%%代码
\usepackage{color}
\usepackage{xcolor}
\definecolor{keywordcolor}{rgb}{0.8,0.1,0.5}
\usepackage{listings}
\lstset{breaklines}%这条命令可以让LaTeX自动将长的代码行换行排版
\lstset{extendedchars=false}%这一条命令可以解决代码跨页时,章节标题,页眉等汉字不显示的问题
\lstset{language=C++, %用于设置语言为C++
    keywordstyle=\color{keywordcolor} \bfseries, %设置关键词
    identifierstyle=,
    basicstyle=\ttfamily, 
    commentstyle=\color{blue} \textit,
    stringstyle=\ttfamily, 
    showstringspaces=false,
    %frame=shadowbox, %边框
    captionpos=b
}
%%%

%\hypersetup{CJKbookmarks=true} %解决section不能使用中文的问题

\begin{document}
\begin{CJK}{UTF8}{gbsn}     %CJK:支持中文

\else
    
\begin{description}
    \item{\textbf{问题}}: Given a sorted linked list, delete all nodes that have duplicate numbers, leaving only distinct numbers from the original list. \textit{(leetcode 82)}
    \item{\textbf{示例}} : Given $1\rightarrow2\rightarrow3\rightarrow3\rightarrow4\rightarrow4\rightarrow5$, return $1\rightarrow2\rightarrow5$. Given $1\rightarrow1\rightarrow1\rightarrow2\rightarrow3$, return $2\rightarrow3$.
    \item{\textbf{两个指针}} : \fbox{时间复杂度O(n) , 空间复杂度O(1)}
    \\需要注意的就是如果最后几个是相同的也要删除.
    \begin{lstlisting}
ListNode *deleteDuplicates(ListNode *head) {
	if(!head || !head->next)	return head;
	ListNode newhead(0);
	ListNode *pre = &newhead, *next = head->next;
	newhead.next = head;
	while(next){
		if(pre->next->val == next->val){
			next = next->next;
		}else{
			if(pre->next->next == next){
				pre = pre->next;
			}else{
				pre->next = next;
			}
			next = next->next;
		}
	}
	if(pre->next->next != next)
		pre->next = next;
	return newhead.next;
}
    \end{lstlisting}
    \textit{}
\end{description}

\fi

\ifx allfiles undefined
\end{CJK}
\end{document}
\fi

\subsection{Partition List}
\ifx allfiles undefined
\documentclass{article}
\usepackage{CJK}
\usepackage{verbatim}

%%%代码
\usepackage{color}
\usepackage{xcolor}
\definecolor{keywordcolor}{rgb}{0.8,0.1,0.5}
\usepackage{listings}
\lstset{breaklines}%这条命令可以让LaTeX自动将长的代码行换行排版
\lstset{extendedchars=false}%这一条命令可以解决代码跨页时,章节标题,页眉等汉字不显示的问题
\lstset{language=C++, %用于设置语言为C++
    keywordstyle=\color{keywordcolor} \bfseries, %设置关键词
    identifierstyle=,
    basicstyle=\ttfamily, 
    commentstyle=\color{blue} \textit,
    stringstyle=\ttfamily, 
    showstringspaces=false,
    %frame=shadowbox, %边框
    captionpos=b
}
%%%

%\hypersetup{CJKbookmarks=true} %解决section不能使用中文的问题

\begin{document}
\begin{CJK}{UTF8}{gbsn}     %CJK:支持中文

\else
    
\begin{description}
    \item{\textbf{问题}}:Given a linked list and a value x, partition it such that all nodes less than x come before nodes greater than or equal to x. \textit{(leetcode 86)}
    \item{\textbf{举例}} : Given $1\rightarrow4\rightarrow3\rightarrow2\rightarrow5\rightarrow2$ and x = 3, return $1\rightarrow2\rightarrow2\rightarrow4\rightarrow3\rightarrow5$.
    \item{\textbf{Care}} : \fbox{时间复杂度O(n), 空间复杂度O(1)}
    \\分别用两个链表指针维护小于x和大于x的,最后再粘结到一起就行了.
    \begin{lstlisting}
ListNode *partition(ListNode *head, int x) {
	if(!head || !head->next)	return head;
	ListNode	less(0), greater(0), *pos = head;
	ListNode	*lpos = &less, *gpos = &greater;
	while(pos){
		if(pos->val < x){
			lpos = lpos->next = pos;
		}else{
			gpos = gpos->next = pos;
		}
		pos = pos->next;
	}
	gpos->next = NULL;
	lpos->next = greater.next;
	return less.next;
}
    \end{lstlisting}
\end{description}

\fi

\ifx allfiles undefined
\end{CJK}
\end{document}
\fi

\subsection{Reverse Linked List II}
\ifx allfiles undefined
\documentclass{article}
\usepackage{CJK}
\usepackage{verbatim}

%%%代码
\usepackage{color}
\usepackage{xcolor}
\definecolor{keywordcolor}{rgb}{0.8,0.1,0.5}
\usepackage{listings}
\lstset{breaklines}%这条命令可以让LaTeX自动将长的代码行换行排版
\lstset{extendedchars=false}%这一条命令可以解决代码跨页时,章节标题,页眉等汉字不显示的问题
\lstset{language=C++, %用于设置语言为C++
    keywordstyle=\color{keywordcolor} \bfseries, %设置关键词
    identifierstyle=,
    basicstyle=\ttfamily, 
    commentstyle=\color{blue} \textit,
    stringstyle=\ttfamily, 
    showstringspaces=false,
    %frame=shadowbox, %边框
    captionpos=b
}
%%%

%\hypersetup{CJKbookmarks=true} %解决section不能使用中文的问题

\begin{document}
\begin{CJK}{UTF8}{gbsn}     %CJK:支持中文

\else
    
\begin{description}
    \item{\textbf{问题}}: Reverse a linked list from position m to n. Do it in-place and in one-pass. \textit{(leetcode 92)}
	\item{\textbf{举例}} : Given $1\rightarrow2\rightarrow3\rightarrow4\rightarrow5\rightarrow NULL$, m = 2 and n = 4, return $1\rightarrow4\rightarrow3\rightarrow2\rightarrow5\rightarrow NULL$.
    \item{\textbf{Note}} : Given m, n satisfy the following condition: 1 ≤ m ≤ n ≤ length of list.
    \item{\textbf{Care}} : \fbox{时间复杂度O(n), 空间复杂度O(1)}
    \\这题主要是边界条件比较复杂,如何写出优雅的囊括边界条件检测的代码是个问题.
    \begin{lstlisting}
ListNode *reverseBetween(ListNode *head, int m, int n) {
	if(!head || !head->next || m >= n)	return head;
	ListNode newhead(0);
	newhead.next = head;
	ListNode *pos = &newhead, *next, *tmp;
	for(int i = 0; i < m - 1; i++){
		pos = pos->next;
	}
	next = pos->next;
	for(int i = 0; i < n - m; i++){
		tmp = next->next;
		next->next = tmp->next;
		tmp->next = pos->next;
		pos->next = tmp;
	}
	return newhead.next;
}
    \end{lstlisting}
	使用一个新的头节点是一个很好的习惯.
\end{description}

\fi

\ifx allfiles undefined
\end{CJK}
\end{document}
\fi

\subsection{Copy List with Random Pointer}
\ifx allfiles undefined
\documentclass{article}
\usepackage{CJK}
\usepackage{verbatim}

%%%代码
\usepackage{color}
\usepackage{xcolor}
\definecolor{keywordcolor}{rgb}{0.8,0.1,0.5}
\usepackage{listings}
\lstset{breaklines}%这条命令可以让LaTeX自动将长的代码行换行排版
\lstset{extendedchars=false}%这一条命令可以解决代码跨页时,章节标题,页眉等汉字不显示的问题
\lstset{language=C++, %用于设置语言为C++
    keywordstyle=\color{keywordcolor} \bfseries, %设置关键词
    identifierstyle=,
    basicstyle=\ttfamily, 
    commentstyle=\color{blue} \textit,
    stringstyle=\ttfamily, 
    showstringspaces=false,
    %frame=shadowbox, %边框
    captionpos=b
}
%%%

%\hypersetup{CJKbookmarks=true} %解决section不能使用中文的问题

\begin{document}
\begin{CJK}{UTF8}{gbsn}     %CJK:支持中文

\else
    
\begin{description}
    \item{\textbf{问题}}: A linked list is given such that each node contains an additional random pointer which could point to any node in the list or null. Return a deep copy of the list. \textit{(leetcode 138)}
    \item{\textbf{哈希表}} : \fbox{时间复杂度O(n), 空间复杂度O(n)}
    \\链表的缺点在于不能随机访问,但是如果和哈希表来联合使用那么就可以做到随机访问了.这道题在遍历链表时候,一边遍历一边建立新的链表节点,然后使用哈希表记录新老节点的映射关系,等到再一次遍历老链表的时候根据映射表就可以快速的定位到新链表对应位置,不必再从头开始找.
    \begin{lstlisting}
RandomListNode *copyRandomList(RandomListNode *head){
	if(head == NULL) return NULL;
	stack<RandomListNode*> to_be;
	unordered_map<RandomListNode*, RandomListNode*> have_new;
	RandomListNode* pos, *next, *rand;
	to_be.push(head);
	pos = head;
	RandomListNode* new_head, *new_pos, *new_next, *new_rand;
	new_head = new RandomListNode(head->label);
	new_pos = new_head;
	have_new[pos] = new_pos;
	while(!to_be.empty()){
		pos = to_be.top();
		to_be.pop();
		next = pos->next;
		rand = pos->random;
		new_pos = have_new[pos];
		if(next != NULL){
			if(have_new.count(next) == 1){
				new_pos->next = have_new[next];
			}
			else{
				new_next  = new RandomListNode(next->label);
				new_pos->next = new_next;
				have_new[next] = new_next;
				to_be.push(next);
			}
		}else{
			new_pos->next = NULL;
		}
		if(rand != NULL){
			if(have_new.count(rand) == 1){
				new_pos->random = have_new[rand];
			}
			else{
				new_rand = new RandomListNode(rand->label);
				new_pos->random = new_rand;
				have_new[rand] = new_rand;
				to_be.push(rand);
			}
		}else{
			new_pos->random = NULL;
		}
	}
	return new_head;
}
    \end{lstlisting}
    \item{\textbf{插入法}} : \fbox{时间复杂度O(n), 空间复杂度O(1)}
    \\插入法实质上也是在建立新老节点的对应关系,不像哈希表这种需要额外空间的对应,而是非常巧妙的在老链表的结构上建立映射关系,举重若轻.
    \begin{lstlisting}
RandomListNode *copyRandomList(RandomListNode *head){
	if(!head)	return NULL;
	RandomListNode *pos = head, *tmp, *ret;
	while(pos){
		tmp = new RandomListNode(pos->label);	
		tmp->next = pos->next;
		pos->next = tmp;
		pos = pos->next->next;
	}
	pos = head;
	while(pos){
		pos->next->random = pos->random? pos->random->next : NULL;
		pos = pos->next->next;
	}
	pos = head;
	ret = pos->next;
	while(pos){
		tmp = pos->next;
		pos->next = tmp->next;
		tmp->next = pos->next? pos->next->next : NULL;
		pos = pos->next;
	}
	return ret;
}
    \end{lstlisting}
	其实这道题你还可以这样看,这里链表的节点有个next指针和random指针,其实这就是一个图了(链表是特殊的树,树是特殊的图),所以套用图的DFS和BFS,你在处理过程中就有两种选择: 比如你建立老链表第一个节点的对应新节点后,你可以类似DFS选择直接深度优先的访问next节点next的next节点等等;同样你也可以类似BFS下次访问next节点和rand节点.所以你可以使用递归的方式实现.
\end{description}

\fi

\ifx allfiles undefined
\end{CJK}
\end{document}
\fi


\fi

\ifx allfiles undefined
\end{CJK}
\end{document}
\fi
