\ifx allfiles undefined
\documentclass{article}
\usepackage{CJK}
\usepackage{verbatim}

%%%代码
\usepackage{color}
\usepackage{xcolor}
\definecolor{keywordcolor}{rgb}{0.8,0.1,0.5}
\usepackage{listings}
\lstset{breaklines}%这条命令可以让LaTeX自动将长的代码行换行排版
\lstset{extendedchars=false}%这一条命令可以解决代码跨页时,章节标题,页眉等汉字不显示的问题
\lstset{language=C++, %用于设置语言为C++
    keywordstyle=\color{keywordcolor} \bfseries, %设置关键词
    identifierstyle=,
    basicstyle=\ttfamily, 
    commentstyle=\color{blue} \textit,
    stringstyle=\ttfamily, 
    showstringspaces=false,
    %frame=shadowbox, %边框
    captionpos=b
}
%%%

%\hypersetup{CJKbookmarks=true} %解决section不能使用中文的问题

\begin{document}
\begin{CJK}{UTF8}{gbsn}     %CJK:支持中文

\else
    
\begin{description}
    \item{\textbf{问题}}: Given two lists, return the first node where they meet, if it is not exsit, return Null.
    \item{\textbf{双指针}} : \fbox{时间复杂度(n+m), 空间复杂度O(1)}
    \\找到两个链表第一个相交点,首先遍历两个链表,得到两个链表的长度,比如说l1, l2.然后先移动较长的那个链表(比如l1>l2),移动l1-l2步,这样双方剩余的节点数就相等了.接着以前往后走,第一次相遇的点就是答案
    \begin{lstlisting}
ListNode *getFirstMeet(ListNode *pHead1, ListNode *pHead2){
	if(!pHead1 || !pHead2)	return NULL;
	ListNode *pos1 = pHead1, *pos2 = pHead2;
	int l1 = 0, l2 = 0;
	while(pos1->next){
		l1++;
		pos1 = pos1->next;
	}
	while(pos2->next){
		l2++;
		pos2 = pos2->next;
	}
	if(pos1 != pos2)	return NULL;
	pos1 = l1 > l2? pHead1 : pHead2;
	pos2 = l1 > l2? pHead2 : pHead1;
	int count = l1 > l2? l1 - l2 : l2 -l1;
	while(count-- > 0){
		pos1 = pos1->next;
	}
	while(pos1 && pos2){
		if(pos1 == pos2)	return pos1;
		pos1 = pos1->next;
		pos2 = pos2->next;
	}
}
    \end{lstlisting}
\end{description}

\fi

\ifx allfiles undefined
\end{CJK}
\end{document}
\fi
