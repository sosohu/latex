\ifx allfiles undefined
\documentclass{article}
\usepackage{CJK}
\usepackage{verbatim}

%%%代码
\usepackage{color}
\usepackage{xcolor}
\definecolor{keywordcolor}{rgb}{0.8,0.1,0.5}
\usepackage{listings}
\lstset{breaklines}%这条命令可以让LaTeX自动将长的代码行换行排版
\lstset{extendedchars=false}%这一条命令可以解决代码跨页时,章节标题,页眉等汉字不显示的问题
\lstset{language=C++, %用于设置语言为C++
    keywordstyle=\color{keywordcolor} \bfseries, %设置关键词
    identifierstyle=,
    basicstyle=\ttfamily, 
    commentstyle=\color{blue} \textit,
    stringstyle=\ttfamily, 
    showstringspaces=false,
    %frame=shadowbox, %边框
    captionpos=b
}
%%%

%\hypersetup{CJKbookmarks=true} %解决section不能使用中文的问题

\begin{document}
\begin{CJK}{UTF8}{gbsn}     %CJK:支持中文

\else
    
\begin{description}
    \item{\textbf{问题}}: Given a linked list, return the node where the cycle begins. If there is no cycle, return null. \textit{(leetcode 142)}
    \item{\textbf{快慢指针}} : \fbox{时间复杂度O(n) , 空间复杂度O(1)}
    \\
    \begin{lstlisting}
ListNode *GetFirstCircleNode(ListNode* pHead){
	if(!pHead || !pHead->next)	return pHead;
	ListNode *slow = pHead, *fast = pHead;
	// fast指针先进入环内
	while(fast && fast->next){
		slow = slow->next;
		fast = fast->next->next;
		if(slow == fast)
			break;
	}
	if(!fast || !fast->next)	
		return NULL; // 不存在环
	// slow指针再次从头开始, fast指针减速
	slow = pHead;
	while(slow != fast){
		slow = slow->next;
		fast = fast->next;
	}
	return slow; // 相遇点就是环的入口
}
    \end{lstlisting}
\end{description}

\fi

\ifx allfiles undefined
\end{CJK}
\end{document}
\fi
