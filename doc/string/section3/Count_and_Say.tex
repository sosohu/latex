\ifx allfiles undefined
\documentclass{article}
\usepackage{CJK}
\usepackage{verbatim}

%%%代码
\usepackage{color}
\usepackage{xcolor}
\definecolor{keywordcolor}{rgb}{0.8,0.1,0.5}
\usepackage{listings}
\lstset{breaklines}%这条命令可以让LaTeX自动将长的代码行换行排版
\lstset{extendedchars=false}%这一条命令可以解决代码跨页时,章节标题,页眉等汉字不显示的问题
\lstset{language=C++, %用于设置语言为C++
    keywordstyle=\color{keywordcolor} \bfseries, %设置关键词
    identifierstyle=,
    basicstyle=\ttfamily, 
    commentstyle=\color{blue} \textit,
    stringstyle=\ttfamily, 
    showstringspaces=false,
    %frame=shadowbox, %边框
    captionpos=b
}
%%%

%\hypersetup{CJKbookmarks=true} %解决section不能使用中文的问题

\begin{document}
\begin{CJK}{UTF8}{gbsn}     %CJK:支持中文

\else
    
\begin{description}
    \item{\textbf{问题}}:\\
	The count-and-say sequence is the sequence of integers beginning as follows:\\
	1, 11, 21, 1211, 111221, ...\\
	1 is read off as "one 1" or 11.\\
	11 is read off as "two 1s" or 21.\\
	21 is read off as "one 2, then one 1" or 1211.\\
	Given an integer n, generate the nth sequence.\textit{(leetcode 38)}
	\item{\textbf{Note}}The sequence of integers will be represented as a string. 
    \item{\textbf{递推}} : \fbox{时间复杂度O(n*m) , 空间复杂度O(m)}
    \\根据递推关系一步步的推.
    \begin{lstlisting}
string parse(string& str){
	string ret;
	int last = 0, len = str.length();
	for(int i = 0; i < len; i++){
		last = i;
		while(i < len - 1 && str[i] == str[i+1])	i++;
		ret.push_back(i - last + 1 + '0');	
		ret.push_back(str[last]);	
	}
	return ret;
}

string countAndSay(int n) {
	string last = "1", cur;
	if(n < 1)	return cur;
	while(n-- > 1){
		last = cur = parse(last);
	}
	return last;
}
    \end{lstlisting}
\end{description}

\fi

\ifx allfiles undefined
\end{CJK}
\end{document}
\fi
