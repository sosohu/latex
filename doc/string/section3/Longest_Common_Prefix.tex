\ifx allfiles undefined
\documentclass{article}
\usepackage{CJK}
\usepackage{verbatim}

%%%代码
\usepackage{color}
\usepackage{xcolor}
\definecolor{keywordcolor}{rgb}{0.8,0.1,0.5}
\usepackage{listings}
\lstset{breaklines}%这条命令可以让LaTeX自动将长的代码行换行排版
\lstset{extendedchars=false}%这一条命令可以解决代码跨页时,章节标题,页眉等汉字不显示的问题
\lstset{language=C++, %用于设置语言为C++
    keywordstyle=\color{keywordcolor} \bfseries, %设置关键词
    identifierstyle=,
    basicstyle=\ttfamily, 
    commentstyle=\color{blue} \textit,
    stringstyle=\ttfamily, 
    showstringspaces=false,
    %frame=shadowbox, %边框
    captionpos=b
}
%%%

%\hypersetup{CJKbookmarks=true} %解决section不能使用中文的问题

\begin{document}
\begin{CJK}{UTF8}{gbsn}     %CJK:支持中文

\else
    
\begin{description}
    \item{\textbf{问题}}: Write a function to find the longest common prefix string amongst an array of strings. \textit{(leetcode 14)}
    \item{\textbf{Care}} : \fbox{时间复杂度O(n*m) , 空间复杂度O(1)}
    \\这道题比较简单,需要注意的还是怎么写的简洁优雅.
    \begin{lstlisting}
string longestCommonPrefix(vector<string> &strs) {
	string ret;
	if(strs.size() == 0)	return ret;
	int len = INT_MAX;
	for(int i = 0; i < strs.size(); i++)
		if(len > strs[i].length())	len = strs[i].length();
	for(int j = 0; j < len; j++){
		for(int i = 1; i < strs.size(); i++)
			if(strs[i][j] != strs[i-1][j])	return ret;
		ret.push_back(strs[0][j]);
	}
	return ret;
}
    \end{lstlisting}
    \textit{}
\end{description}

\fi

\ifx allfiles undefined
\end{CJK}
\end{document}
\fi
