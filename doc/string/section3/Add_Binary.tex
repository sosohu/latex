\ifx allfiles undefined
\documentclass{article}
\usepackage{CJK}
\usepackage{verbatim}

%%%代码
\usepackage{color}
\usepackage{xcolor}
\definecolor{keywordcolor}{rgb}{0.8,0.1,0.5}
\usepackage{listings}
\lstset{breaklines}%这条命令可以让LaTeX自动将长的代码行换行排版
\lstset{extendedchars=false}%这一条命令可以解决代码跨页时,章节标题,页眉等汉字不显示的问题
\lstset{language=C++, %用于设置语言为C++
    keywordstyle=\color{keywordcolor} \bfseries, %设置关键词
    identifierstyle=,
    basicstyle=\ttfamily, 
    commentstyle=\color{blue} \textit,
    stringstyle=\ttfamily, 
    showstringspaces=false,
    %frame=shadowbox, %边框
    captionpos=b
}
%%%

%\hypersetup{CJKbookmarks=true} %解决section不能使用中文的问题

\begin{document}
\begin{CJK}{UTF8}{gbsn}     %CJK:支持中文

\else
    
\begin{description}
    \item{\textbf{问题}}: Given two binary strings, return their sum (also a binary string).\textit{(leetcode 67)}
	\item{\textbf{举例}}:\\
	a = "11"\\
	b = "1"\\
	Return "100".
    \item{\textbf{迭代}} : \fbox{时间复杂度O(n), 空间复杂度O(1)}
    \\和链表的那个完全类似,解法也是基本一致,优雅的代码.
    \begin{lstlisting}
string addBinary(string a, string b) {
	int la = a.length(), lb = b.length();
	int carry = 0, sum = 0;
	string ret;
	while(la || lb || carry){
		sum = carry + (la > 0? a[la-1] - '0' : 0) + (lb > 0? b[lb-1] - '0' : 0);
		ret.push_back(sum%2 + '0');
		carry = sum/2;
		if(la)	la--;
		if(lb)	lb--;
	}
	reverse(ret.begin(), ret.end());
	return ret;
}
    \end{lstlisting}
\end{description}

\fi

\ifx allfiles undefined
\end{CJK}
\end{document}
\fi
