\ifx allfiles undefined
\documentclass{article}
\usepackage{CJK}

%%%代码
\usepackage{color}
\usepackage{xcolor}
\definecolor{keywordcolor}{rgb}{0.8,0.1,0.5}
\usepackage{listings}
\lstset{breaklines}%这条命令可以让LaTeX自动将长的代码行换行排版
\lstset{extendedchars=false}%这一条命令可以解决代码跨页时,章节标题,页眉等汉字不显示的问题
\lstset{language=C++, %用于设置语言为C++
    keywordstyle=\color{keywordcolor} \bfseries, %设置关键词
    identifierstyle=,
    basicstyle=\ttfamily, 
    commentstyle=\color{blue} \textit,
    stringstyle=\ttfamily, 
    showstringspaces=false,
    %frame=shadowbox, %边框
    captionpos=b
}
%%%

%\hypersetup{CJKbookmarks=true} %解决section不能使用中文的问题

\begin{document}
\begin{CJK}{UTF8}{gbsn}     %CJK:支持中文

\else
    
\qquad字符串的库函数有很多,虽然大多数算法上都不是很难,但是需要考虑很多细节问题,所以需要研究一下.关于字符串库函数需要注意的问题大致可以总结为以下几点:

\begin{itemize}
\item 输入参数是什么类型,该不该是const修饰
\item 输入的指针是否为NULL
\item 字符串处理完毕后,新产生的字符串末尾是否要加$'\backslash0'$
\item 要不要返回值,该返回什么类型的(对于要返回值的是为了实现链式操作)
\end{itemize}

\textbf{还有一些我觉得可能需要考虑的:}

\begin{itemize}
\item 有没有地址重叠
\item src和dest地址一样
\item 需不需要优化做法
\end{itemize}

\subsection{strlen}
\ifx allfiles undefined
\documentclass{article}
\usepackage{CJK}
\usepackage{verbatim}

%%%代码
\usepackage{color}
\usepackage{xcolor}
\definecolor{keywordcolor}{rgb}{0.8,0.1,0.5}
\usepackage{listings}
\lstset{breaklines}%这条命令可以让LaTeX自动将长的代码行换行排版
\lstset{extendedchars=false}%这一条命令可以解决代码跨页时,章节标题,页眉等汉字不显示的问题
\lstset{language=C++, %用于设置语言为C++
    keywordstyle=\color{keywordcolor} \bfseries, %设置关键词
    identifierstyle=,
    basicstyle=\ttfamily, 
    commentstyle=\color{blue} \textit,
    stringstyle=\ttfamily, 
    showstringspaces=false,
    %frame=shadowbox, %边框
    captionpos=b
}
%%%

%\hypersetup{CJKbookmarks=true} %解决section不能使用中文的问题

\begin{document}
\begin{CJK}{UTF8}{gbsn}     %CJK:支持中文

\else
    
\begin{description}
    \item{\textbf{问题}}: 求出字符串的长度.
    \item{\textbf{Care}} : \fbox{时间复杂度O(n), 空间复杂度O(1)}
    \\考虑输入指针不为空,输入参数使用const char
    \begin{lstlisting}
int strlen(const char *str){
	int len = 0;
	assert(str);  //判断不为NULL
	while(*str++ != '\0'){
		len++;
	}
	return len;
}
    \end{lstlisting}
\end{description}

\fi

\ifx allfiles undefined
\end{CJK}
\end{document}
\fi

\subsection{strcat}
\ifx allfiles undefined
\documentclass{article}
\usepackage{CJK}
\usepackage{verbatim}

%%%代码
\usepackage{color}
\usepackage{xcolor}
\definecolor{keywordcolor}{rgb}{0.8,0.1,0.5}
\usepackage{listings}
\lstset{breaklines}%这条命令可以让LaTeX自动将长的代码行换行排版
\lstset{extendedchars=false}%这一条命令可以解决代码跨页时,章节标题,页眉等汉字不显示的问题
\lstset{language=C++, %用于设置语言为C++
    keywordstyle=\color{keywordcolor} \bfseries, %设置关键词
    identifierstyle=,
    basicstyle=\ttfamily, 
    commentstyle=\color{blue} \textit,
    stringstyle=\ttfamily, 
    showstringspaces=false,
    %frame=shadowbox, %边框
    captionpos=b
}
%%%

%\hypersetup{CJKbookmarks=true} %解决section不能使用中文的问题

\begin{document}
\begin{CJK}{UTF8}{gbsn}     %CJK:支持中文

\else
    
\begin{description}
    \item{\textbf{问题}}: 将一个字符串连接到另一个字符串后面
    \item{\textbf{Care}} : \fbox{时间复杂度O(n), 空间复杂度O(1)}
	\\输入参数const性和非NULL,尾赋$'\backslash0'$, 有返回值
    \begin{lstlisting}
char *strcat(char *strDest, const char *strScr){
	assert(strDest && strScr);
	if(!strScr)	return strDest;
	char* p = strDest;
	while(*p){
		p++;
	}
	while(*strScr){
		*p++ = *strScr++;
	}
	*p = '\0';
	return strDest;
}
    \end{lstlisting}
\end{description}

\fi

\ifx allfiles undefined
\end{CJK}
\end{document}
\fi

\subsection{strcmp}
\ifx allfiles undefined
\documentclass{article}
\usepackage{CJK}
\usepackage{verbatim}

%%%代码
\usepackage{color}
\usepackage{xcolor}
\definecolor{keywordcolor}{rgb}{0.8,0.1,0.5}
\usepackage{listings}
\lstset{breaklines}%这条命令可以让LaTeX自动将长的代码行换行排版
\lstset{extendedchars=false}%这一条命令可以解决代码跨页时,章节标题,页眉等汉字不显示的问题
\lstset{language=C++, %用于设置语言为C++
    keywordstyle=\color{keywordcolor} \bfseries, %设置关键词
    identifierstyle=,
    basicstyle=\ttfamily, 
    commentstyle=\color{blue} \textit,
    stringstyle=\ttfamily, 
    showstringspaces=false,
    %frame=shadowbox, %边框
    captionpos=b
}
%%%

%\hypersetup{CJKbookmarks=true} %解决section不能使用中文的问题

\begin{document}
\begin{CJK}{UTF8}{gbsn}     %CJK:支持中文

\else
    
\begin{description}
    \item{\textbf{问题}}: 比较两个字符串是否完全相同.
    \item{\textbf{Care}} : \fbox{时间复杂度O(n), 空间复杂度O(1)}
	\\输入参数const性和非NULL,注意如何写的简洁.
    \begin{lstlisting}
int strcmp(const char *str1,const char *str2){
	if(str1 == str2)	return 0;
	assert(str1 && str2);
	//这样写很简洁
	while(*str1 && *str2 && (*str1 == *str2)){
		str1++;
		str2++;
	}
	return (*str1) - (*str2);
}
    \end{lstlisting}
\end{description}

\fi

\ifx allfiles undefined
\end{CJK}
\end{document}
\fi

\subsection{strcpy}
\ifx allfiles undefined
\documentclass{article}
\usepackage{CJK}
\usepackage{verbatim}

%%%代码
\usepackage{color}
\usepackage{xcolor}
\definecolor{keywordcolor}{rgb}{0.8,0.1,0.5}
\usepackage{listings}
\lstset{breaklines}%这条命令可以让LaTeX自动将长的代码行换行排版
\lstset{extendedchars=false}%这一条命令可以解决代码跨页时,章节标题,页眉等汉字不显示的问题
\lstset{language=C++, %用于设置语言为C++
    keywordstyle=\color{keywordcolor} \bfseries, %设置关键词
    identifierstyle=,
    basicstyle=\ttfamily, 
    commentstyle=\color{blue} \textit,
    stringstyle=\ttfamily, 
    showstringspaces=false,
    %frame=shadowbox, %边框
    captionpos=b
}
%%%

%\hypersetup{CJKbookmarks=true} %解决section不能使用中文的问题

\begin{document}
\begin{CJK}{UTF8}{gbsn}     %CJK:支持中文

\else
    
\begin{description}
    \item{\textbf{问题}}: 将源字符串完全拷贝给目的字符串.
    \item{\textbf{Care}} : \fbox{时间复杂度O(n), 空间复杂度O(1)}
	\\考虑输入参数的const性,输入参数是否合法,返回值,最后位置为$'\backslash0'$
    \begin{lstlisting}
char *strcpy(char *dest,const char *src){
	if(dest == src)	return dest;
	assert(dest && src);  //输入参数不为NULL
	char* address = dest;
	int lens = strlen(src);
	int lend = strlen(dest);
	if(src + lens > dest){ // 从后往前拷贝
		dest[lens] = '\0';   //当lens < lend就很重要
		for(int i = lens - 1; i >= 0; i--){
			dest[i] = src[i];
		}
	}else{ //从前往后拷贝
		for(int i = 0; i < lens; i++){
			dest[i] = src[i];
		}
		dest[lens] = '\0';  //当lens < lend就很重要
	}
	return address; //注意此函数有返回值
}
    \end{lstlisting}
	\textbf{至于src和dest地址重叠问题,libc里面并没有考虑,我觉得考虑一下还是很好的.}
\end{description}

\fi

\ifx allfiles undefined
\end{CJK}
\end{document}
\fi

\subsection{strncpy}
\ifx allfiles undefined
\documentclass{article}
\usepackage{CJK}
\usepackage{verbatim}

%%%代码
\usepackage{color}
\usepackage{xcolor}
\definecolor{keywordcolor}{rgb}{0.8,0.1,0.5}
\usepackage{listings}
\lstset{breaklines}%这条命令可以让LaTeX自动将长的代码行换行排版
\lstset{extendedchars=false}%这一条命令可以解决代码跨页时,章节标题,页眉等汉字不显示的问题
\lstset{language=C++, %用于设置语言为C++
    keywordstyle=\color{keywordcolor} \bfseries, %设置关键词
    identifierstyle=,
    basicstyle=\ttfamily, 
    commentstyle=\color{blue} \textit,
    stringstyle=\ttfamily, 
    showstringspaces=false,
    %frame=shadowbox, %边框
    captionpos=b
}
%%%

%\hypersetup{CJKbookmarks=true} %解决section不能使用中文的问题

\begin{document}
\begin{CJK}{UTF8}{gbsn}     %CJK:支持中文

\else
    
\begin{description}
    \item{\textbf{问题}}: 将源字符串拷贝前n个字符给目的字符串.
    \item{\textbf{Care}} : \fbox{时间复杂度O(n), 空间复杂度O(1)}
	\\注意事项同strcpy
    \begin{lstlisting}
char *strncpy(char *dest,const char *src, int n){
	assert(dest && src);
	char *address = dest;
	int lens = strlen(src);
	int lend = strlen(dest);
	if(src + lens > dest && src + n > dest){//从后往前拷贝
		dest[min(lens, n)] = '\0';
		for(int i = min(lens, n) - 1; i >= 0; i--){
			dest[i] = src[i];
		}
	}else{ //从前往后拷贝
		for(int i = 0; i < lens && i < n; i++){
			dest[i] = src[i];
		}
		dest[min(lens, n)] = '\0';
	}
	return address;
}
    \end{lstlisting}
	\textbf{注意不能仅仅因为src == dest就放弃拷贝.}
\end{description}

\fi

\ifx allfiles undefined
\end{CJK}
\end{document}
\fi

\subsection{memcpy}
\ifx allfiles undefined
\documentclass{article}
\usepackage{CJK}
\usepackage{verbatim}

%%%代码
\usepackage{color}
\usepackage{xcolor}
\definecolor{keywordcolor}{rgb}{0.8,0.1,0.5}
\usepackage{listings}
\lstset{breaklines}%这条命令可以让LaTeX自动将长的代码行换行排版
\lstset{extendedchars=false}%这一条命令可以解决代码跨页时,章节标题,页眉等汉字不显示的问题
\lstset{language=C++, %用于设置语言为C++
    keywordstyle=\color{keywordcolor} \bfseries, %设置关键词
    identifierstyle=,
    basicstyle=\ttfamily, 
    commentstyle=\color{blue} \textit,
    stringstyle=\ttfamily, 
    showstringspaces=false,
    %frame=shadowbox, %边框
    captionpos=b
}
%%%

%\hypersetup{CJKbookmarks=true} %解决section不能使用中文的问题

\begin{document}
\begin{CJK}{UTF8}{gbsn}     %CJK:支持中文

\else
    
\begin{description}
    \item{\textbf{问题}}: 将源地址开始的连续n个字节大小空间拷贝给给目的地址.
    \item{\textbf{Care}} : \fbox{时间复杂度O(n), 空间复杂度O(1)}
	\\注意输入参数是void*和const void*, 输出参数是void*,要保证输出参数非NULL
    \begin{lstlisting}
void *memcpy(void *dest, const void *src, size_t count){
	if(dest == src)	return dest;
	assert(src && dest);
	if(!src || !dest)	return NULL;
	unsigned char *d = (unsigned char*)dest, *s = (unsigned char*)src;
	while(count--){
		*d++ = *s++;
	}
	//不需要设置'\0',因为是内存拷贝
	return dest;
}
    \end{lstlisting}
	\textbf{libc使用了page copy, unsigned long copy做到快速拷贝}
\end{description}

\fi

\ifx allfiles undefined
\end{CJK}
\end{document}
\fi


\fi

\ifx allfiles undefined
\end{CJK}
\end{document}
\fi
