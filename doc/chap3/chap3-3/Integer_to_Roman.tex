\ifx allfiles undefined
\documentclass{article}
\usepackage{CJK}
\usepackage{verbatim}

%%%代码
\usepackage{color}
\usepackage{xcolor}
\definecolor{keywordcolor}{rgb}{0.8,0.1,0.5}
\usepackage{listings}
\lstset{breaklines}%这条命令可以让LaTeX自动将长的代码行换行排版
\lstset{extendedchars=false}%这一条命令可以解决代码跨页时,章节标题,页眉等汉字不显示的问题
\lstset{language=C++, %用于设置语言为C++
    keywordstyle=\color{keywordcolor} \bfseries, %设置关键词
    identifierstyle=,
    basicstyle=\ttfamily, 
    commentstyle=\color{blue} \textit,
    stringstyle=\ttfamily, 
    showstringspaces=false,
    %frame=shadowbox, %边框
    captionpos=b
}
%%%

%\hypersetup{CJKbookmarks=true} %解决section不能使用中文的问题

\begin{document}
\begin{CJK}{UTF8}{gbsn}     %CJK:支持中文

\else
    
\begin{description}
    \item{\textbf{问题}}: Given an integer, convert it to a roman numeral. Input is guaranteed to be within the range from 1 to 3999. \textit{(leetcode 12)}
    \item{\textbf{Trick}} : \fbox{时间复杂度O(1), 空间复杂度O(1)}
    \\这道题其实比较简单,但是怎么写的优雅是一个很难的事,下面的代码就很trick,需要记住.
    \begin{lstlisting}
string 	intToRoman(int num) {
		string ret;
		int stand[] = {1000, 900, 500, 400, 100, 90, 50, 40, 10, 9, 5, 4, 1};
		string symbol[] = {"M", "CM", "D", "CD", "C", "XC", "L", "XL", "X", "IX", "V", "IV", "I"};
		for(int i = 0; num > 0; i++){
			int count = num / stand[i];
			num = num % stand[i];
			for(; count > 0; count--) ret += symbol[i];
		}
		return ret;
	}
    \end{lstlisting}
    \textit{写的实在是太简洁,太美,太富有穿透力了,应该当做教科书一样背下来.}
\end{description}

\fi

\ifx allfiles undefined
\end{CJK}
\end{document}
\fi
