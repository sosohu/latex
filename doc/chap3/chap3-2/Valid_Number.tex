\ifx allfiles undefined
\documentclass{article}
\usepackage{CJK}
\usepackage{verbatim}

%%%代码
\usepackage{color}
\usepackage{xcolor}
\definecolor{keywordcolor}{rgb}{0.8,0.1,0.5}
\usepackage{listings}
\lstset{breaklines}%这条命令可以让LaTeX自动将长的代码行换行排版
\lstset{extendedchars=false}%这一条命令可以解决代码跨页时,章节标题,页眉等汉字不显示的问题
\lstset{language=C++, %用于设置语言为C++
    keywordstyle=\color{keywordcolor} \bfseries, %设置关键词
    identifierstyle=,
    basicstyle=\ttfamily, 
    commentstyle=\color{blue} \textit,
    stringstyle=\ttfamily, 
    showstringspaces=false,
    %frame=shadowbox, %边框
    captionpos=b
}
%%%

%\hypersetup{CJKbookmarks=true} %解决section不能使用中文的问题

\begin{document}
\begin{CJK}{UTF8}{gbsn}     %CJK:支持中文

\else

    
\begin{description}
    \item{\textbf{问题}}: Validate if a given string is numeric.
	\item{\textbf{举例}} "0" is true " 0.1 " is true "abc" is false "1 a" is false "2e10" is true
    \item{\textbf{NFA}} : \fbox{时间复杂度O(n), 空间复杂度O(1)}
    \\这种问题最优美的解法就是先写出正则表达式,然后根据正则表达式画出NFA,然后根据NFA的状态转移写出代码.\\

	\textbf{正则表达式: $s^*(+|-)?((d^+.?)|(.d))d^*(e(+|-)?d^+)?$   s为空格,d为数字}

	画出状态转移图如下:\\ \\

$
\psmatrix[mnode=circle,colsep=1]
0 & 1 & 3 & 4 & 5 & 6 \\
  &   & 2 & 8 &   & 7 \\
\endpsmatrix
$
\psset{shortput=nab,arrows=->,labelsep=3pt, nodesep=3pt}
\small
\nccircle{1,1}{.3cm}
\nbput*{s}
\ncline{1,1}{1,2}
\nbput*{+/-}
\ncarc[arcangle=45]{1,1}{1,3}
\naput*{.}
\ncline{1,1}{2,3}
\nbput*{d}
\ncline{1,2}{1,3}^{.}
\ncline{1,2}{2,3}^[npos=.3]{d}
\ncline{1,3}{1,4}^{d}
\nccircle{1,4}{.3cm}
\nbput*{d}
\ncline{1,4}{1,5}^{e}
\ncline{1,4}{2,4}^{s}
\ncline{1,5}{1,6}^{+/-}
\ncline{1,5}{2,6}^[npos=.3]{d}
\ncline{1,6}{2,6}^[npos=.3]{d}
\nccircle{2,3}{.3cm}
\nbput*{d}
\ncline{2,3}{1,4}^{.}
\ncline{2,3}{1,4}^{d}
\ncline{2,3}{2,4}^{s}
\ncline{2,6}{2,4}^{s}
\nccircle{2,4}{.3cm}
\nbput*{s}

    \begin{lstlisting}
enum lexical{ valid, space, sign, number, dot, e };

int isType(const char c){
	switch(c){
		case ' ': return 1;
		case '+': ;
		case '-': return 2;
		case '.': return 4;
		case 'e': return 5;
		default: if(c <= '9' && c >= '0')	return 3;
				 else return 0;
	}
}

bool isNumber(const char *s) {
	if(!s)	return false;
	//状态转移矩阵, -1表示非法
	int map[9][6] = {
		-1, 0, 1, 2, 3, -1,  // 0状态的转移
		-1, -1, -1, 2, 3, -1,
		-1, 8, -1, 2, 4, 5,
		-1, -1, -1, 4, -1, -1,
		-1, 8, -1, 4, -1, 5,
		-1, -1, 6, 7, -1, -1,
		-1, -1, -1, 7, -1, -1,
		-1, 8, -1, 7, -1, -1,
		-1, 8, -1, -1, -1, -1 
	};
	int state = 0;
	while(*s){
		state = map[state][isType(*s)];
		if(state == -1)	return false;
		s++;
	}
	return state == 2 || state == 4 || state == 7 || state == 8;
}
    \end{lstlisting}
    知道NFA转移图后想到像上面这些编写代码也是蛮难的,这样编写的好处是可以灵活的改变NFA,不会对代码构成很大影响,值得学习.
\end{description}

\fi

\ifx allfiles undefined
\end{CJK}
\end{document}
\fi
