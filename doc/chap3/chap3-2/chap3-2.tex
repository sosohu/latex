\ifx allfiles undefined
\documentclass{article}
\usepackage{CJK}

%%%代码
\usepackage{color}
\usepackage{xcolor}
\definecolor{keywordcolor}{rgb}{0.8,0.1,0.5}
\usepackage{listings}
\lstset{breaklines}%这条命令可以让LaTeX自动将长的代码行换行排版
\lstset{extendedchars=false}%这一条命令可以解决代码跨页时,章节标题,页眉等汉字不显示的问题
\lstset{language=C++, %用于设置语言为C++
    keywordstyle=\color{keywordcolor} \bfseries, %设置关键词
    identifierstyle=,
    basicstyle=\ttfamily, 
    commentstyle=\color{blue} \textit,
    stringstyle=\ttfamily, 
    showstringspaces=false,
    %frame=shadowbox, %边框
    captionpos=b
}
%%%

%\hypersetup{CJKbookmarks=true} %解决section不能使用中文的问题

\begin{document}
\begin{CJK}{UTF8}{gbsn}     %CJK:支持中文

\else
    
\qquad字符串的经典问题有很多,虽然很多解法都属于算法范畴,诸如动态规划等,这里还是归结为字符串的问题.

\subsection{strStr}
\ifx allfiles undefined
\documentclass{article}
\usepackage{CJK}
\usepackage{verbatim}

%%%代码
\usepackage{color}
\usepackage{xcolor}
\definecolor{keywordcolor}{rgb}{0.8,0.1,0.5}
\usepackage{listings}
\lstset{breaklines}%这条命令可以让LaTeX自动将长的代码行换行排版
\lstset{extendedchars=false}%这一条命令可以解决代码跨页时,章节标题,页眉等汉字不显示的问题
\lstset{language=C++, %用于设置语言为C++
    keywordstyle=\color{keywordcolor} \bfseries, %设置关键词
    identifierstyle=,
    basicstyle=\ttfamily, 
    commentstyle=\color{blue} \textit,
    stringstyle=\ttfamily, 
    showstringspaces=false,
    %frame=shadowbox, %边框
    captionpos=b
}
%%%

%\hypersetup{CJKbookmarks=true} %解决section不能使用中文的问题

\begin{document}
\begin{CJK}{UTF8}{gbsn}     %CJK:支持中文

\else
    
\begin{description}
    \item{\textbf{问题}}: Implement strStr(). Returns the index of the first occurrence of needle in haystack, or -1 if needle is not part of haystack. \textit{(leetcode 28)}
    \item{\textbf{KMP}} : \fbox{时间复杂度O(n), 空间复杂度O(m)}
    \\strstr属于字符串的库函数,放在这里讲完全是因为它是字符串问题中最有名的之一,其解法多样,这里推荐KMP解法,关于此解法的介绍可以参考我们ThreeCobblers主页上\href{https://github.com/ThreeCobblers/Paladin/blob/master/blog/string/KMP.md}{zhuoyuan的博文}
    \begin{lstlisting}
void gen_next(const char *p) {
    next[0] = -1;
    int i = 0;
    int j = -1;
    int lp = strlen(p);
    while(i < lp)
        if(j == -1 || p[i] == p[j]) i++, j++, next[i] = j;
        else j = next[j];
}
int kmp(const char *s, const char *p) {
    gen_next(p);
    int ls = strlen(s);
    int lp = strlen(p);
    int i = -1;
    int j = -1;
    while(i < ls && j < lp)
        if(j == -1 || s[i] == p[j]) i++, j++;
        else j = next[j];
    if(j == lp) return i - lp;
    return -1;
}
    \end{lstlisting}
    \textit{这段代码水很深,写的非常老道,需要好好揣测才可以明白,关于求next数组问题可以看上面说的那篇博客的介绍}
\end{description}

\fi

\ifx allfiles undefined
\end{CJK}
\end{document}
\fi

\subsection{atoi}
\ifx allfiles undefined
\documentclass{article}
\usepackage{CJK}
\usepackage{verbatim}

%%%代码
\usepackage{color}
\usepackage{xcolor}
\definecolor{keywordcolor}{rgb}{0.8,0.1,0.5}
\usepackage{listings}
\lstset{breaklines}%这条命令可以让LaTeX自动将长的代码行换行排版
\lstset{extendedchars=false}%这一条命令可以解决代码跨页时,章节标题,页眉等汉字不显示的问题
\lstset{language=C++, %用于设置语言为C++
    keywordstyle=\color{keywordcolor} \bfseries, %设置关键词
    identifierstyle=,
    basicstyle=\ttfamily, 
    commentstyle=\color{blue} \textit,
    stringstyle=\ttfamily, 
    showstringspaces=false,
    %frame=shadowbox, %边框
    captionpos=b
}
%%%

%\hypersetup{CJKbookmarks=true} %解决section不能使用中文的问题

\begin{document}
\begin{CJK}{UTF8}{gbsn}     %CJK:支持中文

\else
    
\begin{description}
    \item{\textbf{问题}}: Implement atoi to convert a string to an integer. \textit{(leetcode 8)}
	\item{\textbf{Note}}: The function first discards as many whitespace characters as necessary until the first non-whitespace character is found. Then, starting from this character, takes an optional initial plus or minus sign followed by as many numerical digits as possible, and interprets them as a numerical value.

The string can contain additional characters after those that form the integral number, which are ignored and have no effect on the behavior of this function.

If the first sequence of non-whitespace characters in str is not a valid integral number, or if no such sequence exists because either str is empty or it contains only whitespace characters, no conversion is performed.

If no valid conversion could be performed, a zero value is returned. If the correct value is out of the range of representable values, INT\_MAX (2147483647) or INT\_MIN (-2147483648) is returned.
    \item{\textbf{Care}} : \fbox{时间复杂度O(n), 空间复杂度O(1)}
    \\这个问题主要要注意两个方面,一个是空格和非法字符,一个是溢出,空格和非法字符都很好处理,溢出的处理方式就值得去研究,在C++中INT\_MAX的绝对值比INT\_MIN小1,所以负数个数比正数个数多一个,你可以都转化位负数来判断溢出. 当然,对付溢出还有一种偷懒的方法就是使用double类型取存储数据.
    \begin{lstlisting}
int atoi(const char *str){
	int sum = 0;  // 存负值.
	bool isMinus = false;
	const char* p = str;
	if(!p)	return 0;
	while(*p == ' ' && *p != '\0') p++; //空格
	if(*p == '\0')	return 0;
	if(*p == '-'){
		isMinus = true;
		p++;
	}else{
		if(*p == '+') p++;
	}
	while(*p != '\0'){
		int cur = *p++ - '0';
		if(!(cur >= 0 && cur <= 9))	break;  //非法字符
		if(isMinus && sum == INT_MIN/10 && cur > INT_MAX%10 + 1){
			return INT_MIN;
		}
		if(isMinus && sum < INT_MIN/10){
			return INT_MIN;
		}
		if(!isMinus && sum == -INT_MAX/10 && cur > INT_MAX%10){
			return INT_MAX;
		}
		if(!isMinus && sum < -INT_MAX/10){
			return INT_MAX;
		}
		sum = sum*10 - cur;
	}
	if(!isMinus)	return -sum;
	return sum;
}
    \end{lstlisting}
\end{description}

\fi

\ifx allfiles undefined
\end{CJK}
\end{document}
\fi

\subsection{Valid Palindrome}
\ifx allfiles undefined
\documentclass{article}
\usepackage{CJK}
\usepackage{verbatim}

%%%代码
\usepackage{color}
\usepackage{xcolor}
\definecolor{keywordcolor}{rgb}{0.8,0.1,0.5}
\usepackage{listings}
\lstset{breaklines}%这条命令可以让LaTeX自动将长的代码行换行排版
\lstset{extendedchars=false}%这一条命令可以解决代码跨页时,章节标题,页眉等汉字不显示的问题
\lstset{language=C++, %用于设置语言为C++
    keywordstyle=\color{keywordcolor} \bfseries, %设置关键词
    identifierstyle=,
    basicstyle=\ttfamily, 
    commentstyle=\color{blue} \textit,
    stringstyle=\ttfamily, 
    showstringspaces=false,
    %frame=shadowbox, %边框
    captionpos=b
}
%%%

%\hypersetup{CJKbookmarks=true} %解决section不能使用中文的问题

\begin{document}
\begin{CJK}{UTF8}{gbsn}     %CJK:支持中文

\else
    
\begin{description}
    \item{\textbf{问题}}: Given a string, determine if it is a palindrome, considering only alphanumeric characters and ignoring cases.\textit{(leetcode 125)}

	\item{\textbf{举例}}: "A man, a plan, a canal: Panama" is a palindrome. "race a car" is not a palindrome. 
    \item{\textbf{Care}} : \fbox{时间复杂度O(n) , 空间复杂度O(1)}
    \begin{lstlisting}
bool isAlpha(char c){
	if(c <= 'Z' && c >= 'A') return true;
	if(c <= 'z' && c >= 'a') return true;
	if(c <= '9' && c >= '0') return true;
	return false;
}

bool isPalindrome(string s) {
		int len = s.length();
		int start = 0, end = len - 1;
		while(start < end){
			while(start < end && !isAlpha(s[start]))	start++;
			if(start == end)	return true;
			while(end > start && !isAlpha(s[end]))	end--;
			if(s[start] != s[end] && abs(s[start] - s[end]) != abs('a' - 'A')) return false;
			start++;
			end--;
		}
		return true;
	}
    \end{lstlisting}
\end{description}

\fi

\ifx allfiles undefined
\end{CJK}
\end{document}
\fi

\subsection{LongestPalindrome}
\ifx allfiles undefined
\documentclass{article}
\usepackage{CJK}
\usepackage{verbatim}

%%%代码
\usepackage{color}
\usepackage{xcolor}
\definecolor{keywordcolor}{rgb}{0.8,0.1,0.5}
\usepackage{listings}
\lstset{breaklines}%这条命令可以让LaTeX自动将长的代码行换行排版
\lstset{extendedchars=false}%这一条命令可以解决代码跨页时,章节标题,页眉等汉字不显示的问题
\lstset{language=C++, %用于设置语言为C++
    keywordstyle=\color{keywordcolor} \bfseries, %设置关键词
    identifierstyle=,
    basicstyle=\ttfamily, 
    commentstyle=\color{blue} \textit,
    stringstyle=\ttfamily, 
    showstringspaces=false,
    %frame=shadowbox, %边框
    captionpos=b
}
%%%

%\hypersetup{CJKbookmarks=true} %解决section不能使用中文的问题

\begin{document}
\begin{CJK}{UTF8}{gbsn}     %CJK:支持中文

\else
    
\begin{description}
    \item{\textbf{问题}}: Longest Palindromic Substring. \textit{(leetcode 5)}
    \item{\textbf{Manacher}} : \fbox{时间复杂度O(n), 空间复杂度O(n)}
    \\这里介绍的是Manacher算法,关于此算法的思想和证明可以参考我们

	ThreeCobblers主页上\href{https://github.com/ThreeCobblers/Paladin/blob/master/blog/string/LongestPalindrome.md}{sosohu的博文}
    \begin{lstlisting}
string longestPalindrome(string const& s) {
        int n = s.length();
        if(n == 0)  return "";
        string str = "#";
        int count = 0;
        for(int i = 0; i < n; i++){
            str += s[i];
            str += '#';
        }
        int id = 0, mx = 0;
        vector<int> p(2*n+1, 0);
        p[0] = 1;
        for(int i = 1; i < 2*n + 1; i++){
            int j = 2*id - i;
            p[i] = mx > i? min(p[j], mx - i) : 1;
            while(i + p[i] < 2*n + 1 && i - p[i] >= 0){
                if(str[i + p[i]] != str[i - p[i]]) break;
                p[i]++;
            }
            if(i + p[i] > mx){
                mx = i+ p[i];
                id = i;
            }
        }
        int max = INT_MIN;
        int pos = 0;
        for(int i = 0; i < 2*n + 1; i++){
            if(max < p[i]){
                max = p[i];
                pos = i;
            }
        }
        int index, len;
        len = (max - 1);
        index = pos/2 - len/2;
        return s.substr(index, len);
    }
    \end{lstlisting}
\end{description}

\fi

\ifx allfiles undefined
\end{CJK}
\end{document}
\fi

\subsection{Anagrams}
\ifx allfiles undefined
\documentclass{article}
\usepackage{CJK}
\usepackage{verbatim}

%%%代码
\usepackage{color}
\usepackage{xcolor}
\definecolor{keywordcolor}{rgb}{0.8,0.1,0.5}
\usepackage{listings}
\lstset{breaklines}%这条命令可以让LaTeX自动将长的代码行换行排版
\lstset{extendedchars=false}%这一条命令可以解决代码跨页时,章节标题,页眉等汉字不显示的问题
\lstset{language=C++, %用于设置语言为C++
    keywordstyle=\color{keywordcolor} \bfseries, %设置关键词
    identifierstyle=,
    basicstyle=\ttfamily, 
    commentstyle=\color{blue} \textit,
    stringstyle=\ttfamily, 
    showstringspaces=false,
    %frame=shadowbox, %边框
    captionpos=b
}
%%%

%\hypersetup{CJKbookmarks=true} %解决section不能使用中文的问题

\begin{document}
\begin{CJK}{UTF8}{gbsn}     %CJK:支持中文

\else
    
\begin{description}
    \item{\textbf{问题}}: Given an array of strings, return all groups of strings that are anagrams.
	\item{\textbf{Note}}: All inputs will be in lower-case. 
	\textit{(leetcode 49)}
    \item{\textbf{sort}} : \fbox{时间复杂度O(n*m) , 空间复杂度O(n*m)}
    \\变位词是字符串的一种常见问题,很多时候都是先把一群互为变位词的词选出一个为它们的索引词(key),然后就可以把它们放在一起存储.这道题如果不是先这样,而是一个一个查找那么是非常低效的.
    \begin{lstlisting}
vector<string> anagrams(vector<string> &strs) {
	int size = strs.size();
	unordered_map<string, vector<string> > table;
	for(int i = 0; i < size; i++){
		string index = strs[i];
		sort(index.begin(), index.end());
		table[index].push_back(strs[i]);
	}
	vector<string> ret;
	for(unordered_map<string, vector<string> >::iterator iter = table.begin();
		iter != table.end(); iter++){
		if(iter->second.size() > 1){
			ret.insert(ret.end(), iter->second.begin(), iter->second.end());
		}
	}
return ret;
}
    \end{lstlisting}
\end{description}

\fi

\ifx allfiles undefined
\end{CJK}
\end{document}
\fi

\subsection{Valid Number}
\ifx allfiles undefined
\documentclass{article}
\usepackage{CJK}
\usepackage{verbatim}

%%%代码
\usepackage{color}
\usepackage{xcolor}
\definecolor{keywordcolor}{rgb}{0.8,0.1,0.5}
\usepackage{listings}
\lstset{breaklines}%这条命令可以让LaTeX自动将长的代码行换行排版
\lstset{extendedchars=false}%这一条命令可以解决代码跨页时,章节标题,页眉等汉字不显示的问题
\lstset{language=C++, %用于设置语言为C++
    keywordstyle=\color{keywordcolor} \bfseries, %设置关键词
    identifierstyle=,
    basicstyle=\ttfamily, 
    commentstyle=\color{blue} \textit,
    stringstyle=\ttfamily, 
    showstringspaces=false,
    %frame=shadowbox, %边框
    captionpos=b
}
%%%

%\hypersetup{CJKbookmarks=true} %解决section不能使用中文的问题

\begin{document}
\begin{CJK}{UTF8}{gbsn}     %CJK:支持中文

\else

    
\begin{description}
    \item{\textbf{问题}}: Validate if a given string is numeric.
	\item{\textbf{举例}} "0" is true " 0.1 " is true "abc" is false "1 a" is false "2e10" is true
    \item{\textbf{NFA}} : \fbox{时间复杂度O(n), 空间复杂度O(1)}
    \\这种问题最优美的解法就是先写出正则表达式,然后根据正则表达式画出NFA,然后根据NFA的状态转移写出代码.\\

	\textbf{正则表达式: $s^*(+|-)?((d^+.?)|(.d))d^*(e(+|-)?d^+)?$   s为空格,d为数字}

	画出状态转移图如下:\\ \\

$
\psmatrix[mnode=circle,colsep=1]
0 & 1 & 3 & 4 & 5 & 6 \\
  &   & 2 & 8 &   & 7 \\
\endpsmatrix
$
\psset{shortput=nab,arrows=->,labelsep=3pt, nodesep=3pt}
\small
\nccircle{1,1}{.3cm}
\nbput*{s}
\ncline{1,1}{1,2}
\nbput*{+/-}
\ncarc[arcangle=45]{1,1}{1,3}
\naput*{.}
\ncline{1,1}{2,3}
\nbput*{d}
\ncline{1,2}{1,3}^{.}
\ncline{1,2}{2,3}^[npos=.3]{d}
\ncline{1,3}{1,4}^{d}
\nccircle{1,4}{.3cm}
\nbput*{d}
\ncline{1,4}{1,5}^{e}
\ncline{1,4}{2,4}^{s}
\ncline{1,5}{1,6}^{+/-}
\ncline{1,5}{2,6}^[npos=.3]{d}
\ncline{1,6}{2,6}^[npos=.3]{d}
\nccircle{2,3}{.3cm}
\nbput*{d}
\ncline{2,3}{1,4}^{.}
\ncline{2,3}{1,4}^{d}
\ncline{2,3}{2,4}^{s}
\ncline{2,6}{2,4}^{s}
\nccircle{2,4}{.3cm}
\nbput*{s}

    \begin{lstlisting}
enum lexical{ valid, space, sign, number, dot, e };

int isType(const char c){
	switch(c){
		case ' ': return 1;
		case '+': ;
		case '-': return 2;
		case '.': return 4;
		case 'e': return 5;
		default: if(c <= '9' && c >= '0')	return 3;
				 else return 0;
	}
}

bool isNumber(const char *s) {
	if(!s)	return false;
	//状态转移矩阵, -1表示非法
	int map[9][6] = {
		-1, 0, 1, 2, 3, -1,  // 0状态的转移
		-1, -1, -1, 2, 3, -1,
		-1, 8, -1, 2, 4, 5,
		-1, -1, -1, 4, -1, -1,
		-1, 8, -1, 4, -1, 5,
		-1, -1, 6, 7, -1, -1,
		-1, -1, -1, 7, -1, -1,
		-1, 8, -1, 7, -1, -1,
		-1, 8, -1, -1, -1, -1 
	};
	int state = 0;
	while(*s){
		state = map[state][isType(*s)];
		if(state == -1)	return false;
		s++;
	}
	return state == 2 || state == 4 || state == 7 || state == 8;
}
    \end{lstlisting}
    知道NFA转移图后想到像上面这些编写代码也是蛮难的,这样编写的好处是可以灵活的改变NFA,不会对代码构成很大影响,值得学习.
\end{description}

\fi

\ifx allfiles undefined
\end{CJK}
\end{document}
\fi

\subsection{Regular Expression Matching}
\ifx allfiles undefined
\documentclass{article}
\usepackage{CJK}
\usepackage{verbatim}

%%%代码
\usepackage{color}
\usepackage{xcolor}
\definecolor{keywordcolor}{rgb}{0.8,0.1,0.5}
\usepackage{listings}
\lstset{breaklines}%这条命令可以让LaTeX自动将长的代码行换行排版
\lstset{extendedchars=false}%这一条命令可以解决代码跨页时,章节标题,页眉等汉字不显示的问题
\lstset{language=C++, %用于设置语言为C++
    keywordstyle=\color{keywordcolor} \bfseries, %设置关键词
    identifierstyle=,
    basicstyle=\ttfamily, 
    commentstyle=\color{blue} \textit,
    stringstyle=\ttfamily, 
    showstringspaces=false,
    %frame=shadowbox, %边框
    captionpos=b
}
%%%

%\hypersetup{CJKbookmarks=true} %解决section不能使用中文的问题

\begin{document}
\begin{CJK}{UTF8}{gbsn}     %CJK:支持中文

\else
    
\begin{description}
    \item{\textbf{问题}}:\\
	Implement regular expression matching with support for '.' and '*'.\\
	'.' Matches any single character.\\
	'*' Matches zero or more of the preceding element.\\
	The matching should cover the entire input string (not partial).\\
	\textit{(leetcode 10)}
	\item{\textbf{举例}}: \\
	isMatch("aa","a") → false\\
	isMatch("aa","aa") → true\\
	isMatch("aaa","aa") → false\\
	isMatch("aa", "a*") → true\\
	isMatch("aa", ".*") → true\\
	isMatch("ab", ".*") → true\\
	isMatch("aab", "c*a*b") → true\\
    \item{\textbf{DP}} : \fbox{时间复杂度O(n*m), 空间复杂度O(n*m)}
    \\本题使用动态规划求解,设isMatch[i][j]表示s[1...i]与p[1...j]是否匹配\\
	递推关系是如下:

$$ isMatch[i][j]=\left\{
\begin{array}{lcr}
1 $ {\qquad\qquad\qquad\qquad\qquad\qquad\qquad\qquad}$ { \qquad\qquad\qquad\qquad\qquad i=0,j=0 } \\
0 $ {\qquad\qquad\qquad\qquad\qquad\qquad\qquad\qquad}$ { \qquad\qquad\qquad\qquad\qquad i\ne0,j=0 } \\
{isMatch[i-1][j-1]\ \&\&\ (s_{i-1}\ ==\ p_{j-1}\ ||\ p_{j-1}\ ==\ '.')} $ {\qquad}$ {p_j\ne'*'} \\
{isMatch[i][j] = isMatch[i][j-2]\ ||\ isMatch[i-1][j]}$ {\quad} $ { p_j='*',p_{j-1}='.'} \\
{isMatch[i][j-1]} $ {\qquad\qquad\qquad\qquad\qquad\qquad\qquad\qquad} $ { p_j='*',p_{j-1}='*'}\\
{isMatch[i][j-2]\ ||\ (s_{i-1} == p_{j-1}\ \&\&\ isMatch[i-1][j])}$ {\qquad\qquad} $ {other}
\end{array}
\right.
$$
    \begin{lstlisting}
	bool isMatch(const char *s, const char *p) {
		int ls = strlen(s);
		int lp = strlen(p);
		vector<vector<bool> > isMatch(ls+1, vector<bool>(lp+1, false));
		isMatch[0][0] = true;
		int di, dj;
		for(int i = 0; i < ls + 1; i++){ // 起始坐标
			di = i - 1;
			for(int j = 1; j < lp + 1; j++){ // 起始坐标
				dj = j - 1;
				if(p[dj] != '*'){
					isMatch[i][j] = di != -1 && isMatch[i-1][j-1] && (s[di] == p[dj] || p[dj] == '.');
				}else{
					if(dj == 0) { isMatch[i][j] = false; continue; }
					if(p[dj-1] == '.'){
							isMatch[i][j] = isMatch[i][j-2] || (di != -1 && isMatch[i-1][j]);
					}else{
						if(p[dj-1] == '*')	isMatch[i][j] = isMatch[i][j-1];	
						else isMatch[i][j] = isMatch[i][j-2] || (di != -1 && s[di] == p[dj-1] && isMatch[i-1][j]);
					}
				}
			}
		}
		return isMatch[ls][lp];
	}
    \end{lstlisting}
    \textit{}
\end{description}

\fi

\ifx allfiles undefined
\end{CJK}
\end{document}
\fi

\subsection{Wildcard Matching}
\ifx allfiles undefined
\documentclass{article}
\usepackage{CJK}
\usepackage{verbatim}

%%%代码
\usepackage{color}
\usepackage{xcolor}
\definecolor{keywordcolor}{rgb}{0.8,0.1,0.5}
\usepackage{listings}
\lstset{breaklines}%这条命令可以让LaTeX自动将长的代码行换行排版
\lstset{extendedchars=false}%这一条命令可以解决代码跨页时,章节标题,页眉等汉字不显示的问题
\lstset{language=C++, %用于设置语言为C++
    keywordstyle=\color{keywordcolor} \bfseries, %设置关键词
    identifierstyle=,
    basicstyle=\ttfamily, 
    commentstyle=\color{blue} \textit,
    stringstyle=\ttfamily, 
    showstringspaces=false,
    %frame=shadowbox, %边框
    captionpos=b
}
%%%

%\hypersetup{CJKbookmarks=true} %解决section不能使用中文的问题

\begin{document}
\begin{CJK}{UTF8}{gbsn}     %CJK:支持中文

\else
    
\begin{description}
    \item{\textbf{问题}}:\\
	 Implement wildcard pattern matching with support for '?' and '*'. '?'\\
	 Matches any single character. '*' Matches any sequence of characters (including the empty sequence).\\
	 The matching should cover the entire input string (not partial). \textit{(leetcode 44)}
	\item{\textbf{举例}}: \\
	isMatch("aa","a") → false\\
	isMatch("aa","aa") → true\\
	isMatch("aaa","aa") → false\\
	isMatch("aa", "*") → true\\
	isMatch("aa", "a*") → true\\
	isMatch("ab", "?*") → true\\
	isMatch("aab", "c*a*b") → false\\
    \item{\textbf{迭代}} : \fbox{时间复杂度O(n*m) , 空间复杂度O(1)}
    \\这道题其实是可以套用Regular Expression Matching的解法,使用动态规划来计算,但是那样在leetcode上会超时,具体原因我还不清楚. 本题的解法其实也很明了,唯一需要注意的就是每次我们遇到*时候都会记下此次*的位置以及主串位置,然后一次一次的试探*是否要生成字母,试探不成功就回到刚才我们记下的状态然后试探下一个状态,但是,当我们遇到下一个*时候,就可以更新这个记录点了,因为都已经到这个*了,上个*走到这个*走的路肯定是对的,即使不太对,也可以通过这个*不断生成字母来弥补,这就是需要注意的地方,不是很好理解,需要仔细琢磨.
    \begin{lstlisting}
bool isMatch(const char *s, const char *p){
	int ls = strlen(s);
	int lp = strlen(p);
	const char *ps = s, *pp = p, *lasts = NULL, *lastp = NULL;
	if(!s || !p)	return false;
	while(*s){
		switch(*p){
			case '?':	s++; p++; break;
			case '*':	while(*(p+1) && *(p+1) == '*')	p++;
						lasts = s; lastp = p; p++; break;
			default:	if(*s == *p){ s++; p++;}
						else{
							if(lasts){
								s = ++lasts;
								p = lastp + 1;
							}else{
								return false;
							}
						}
		}
	}
	if(*p == '\0' && *s == '\0')	return true;
	if((*p) == '\0' && *(p-1) != '*')	return false;
	while(*p && *p == '*'){
		p++;
	}
	if(*p != '\0')	return false;
	return true;
}
    \end{lstlisting}
\end{description}

\fi

\ifx allfiles undefined
\end{CJK}
\end{document}
\fi


\fi

\ifx allfiles undefined
\end{CJK}
\end{document}
\fi
