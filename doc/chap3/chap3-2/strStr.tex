\ifx allfiles undefined
\documentclass{article}
\usepackage{CJK}
\usepackage{verbatim}

%%%代码
\usepackage{color}
\usepackage{xcolor}
\definecolor{keywordcolor}{rgb}{0.8,0.1,0.5}
\usepackage{listings}
\lstset{breaklines}%这条命令可以让LaTeX自动将长的代码行换行排版
\lstset{extendedchars=false}%这一条命令可以解决代码跨页时,章节标题,页眉等汉字不显示的问题
\lstset{language=C++, %用于设置语言为C++
    keywordstyle=\color{keywordcolor} \bfseries, %设置关键词
    identifierstyle=,
    basicstyle=\ttfamily, 
    commentstyle=\color{blue} \textit,
    stringstyle=\ttfamily, 
    showstringspaces=false,
    %frame=shadowbox, %边框
    captionpos=b
}
%%%

%\hypersetup{CJKbookmarks=true} %解决section不能使用中文的问题

\begin{document}
\begin{CJK}{UTF8}{gbsn}     %CJK:支持中文

\else
    
\begin{description}
    \item{\textbf{问题}}: Implement strStr(). Returns the index of the first occurrence of needle in haystack, or -1 if needle is not part of haystack. \textit{(leetcode 28)}
    \item{\textbf{KMP}} : \fbox{时间复杂度O(n), 空间复杂度O(m)}
    \\strstr属于字符串的库函数,放在这里讲完全是因为它是字符串问题中最有名的之一,其解法多样,这里推荐KMP解法,关于此解法的介绍可以参考我们ThreeCobblers主页上\href{https://github.com/ThreeCobblers/Paladin/blob/master/blog/string/KMP.md}{zhuoyuan的博文}
    \begin{lstlisting}
void gen_next(const char *p) {
    next[0] = -1;
    int i = 0;
    int j = -1;
    int lp = strlen(p);
    while(i < lp)
        if(j == -1 || p[i] == p[j]) i++, j++, next[i] = j;
        else j = next[j];
}
int kmp(const char *s, const char *p) {
    gen_next(p);
    int ls = strlen(s);
    int lp = strlen(p);
    int i = -1;
    int j = -1;
    while(i < ls && j < lp)
        if(j == -1 || s[i] == p[j]) i++, j++;
        else j = next[j];
    if(j == lp) return i - lp;
    return -1;
}
    \end{lstlisting}
    \textit{这段代码水很深,写的非常老道,需要好好揣测才可以明白,关于求next数组问题可以看上面说的那篇博客的介绍}
\end{description}

\fi

\ifx allfiles undefined
\end{CJK}
\end{document}
\fi
