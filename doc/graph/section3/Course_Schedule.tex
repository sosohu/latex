    
\begin{description}
    \item{\textbf{问题}}:\\
There are a total of n courses you have to take, labeled from 0 to n - 1.\\
\\
Some courses may have prerequisites, for example to take course 0 you have to first take course 1, which is expressed as a pair: $[0,1]$\\
\\
Given the total number of courses and a list of prerequisite pairs, is it possible for you to finish all courses?\\
\textit{(leetcode 207)}
    \item{\textbf{举例}}:\\
2, $[[1,0]]$ \\
There are a total of 2 courses to take. To take course 1 you should have finished course 0. So it is possible.\\
\\
2, $[[1,0],[0,1]]$ \\
There are a total of 2 courses to take. To take course 1 you should have finished course 0, and to take course 0 you should also have finished course 1. So it is impossible.\\
    \item{\textbf{BFS}} : \fbox{时间复杂度O(E), 空间复杂度O(V)}
    \\这是一题典型的拓扑排序问题,可以使用BFS/DFS进行排序.
    \begin{lstlisting}
bool canFinish(int numCourses, vector<vector<int>>& prerequisites) {
	vector<vector<int> > g(numCourses, vector<int>());
	vector<bool> visited(numCourses, false);
	vector<int> pre_count(numCourses, 0);
	for(int i = 0; i < prerequisites.size(); i++){
		g[prerequisites[i][1]].push_back(prerequisites[i][0]);
		pre_count[prerequisites[i][0]]++;
	}
	queue<int> pre;
	for(int i = 0; i < numCourses; i++)
		if(!pre_count[i]){
			pre.push(i);
			visited[i] = true;
	}
	while(!pre.empty()){
		int pos = pre.front();
		pre.pop();
		for(int i = 0; i < g[pos].size(); i++){
			pre_count[g[pos][i]]--;
			if(pre_count[g[pos][i]] < 0)	return false;
			if(pre_count[g[pos][i]] == 0){
				pre.push(g[pos][i]);
				visited[g[pos][i]] = true;
			}
		}
	}
	for(int i = 0; i < numCourses; i++)
		if(!visited[i])	return false;
	return true;
}
    \end{lstlisting}
\end{description}
