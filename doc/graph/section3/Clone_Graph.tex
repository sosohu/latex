    
\begin{description}
    \item{\textbf{问题}}:\\
Clone an undirected graph. Each node in the graph contains a label and a list of its neighbors.\\
\\
OJ's undirected graph serialization:\\
Nodes are labeled uniquely.\\
\\
We use $ \sharp $ as a separator for each node, and , as a separator for node label and each neighbor of the node.\\
\textit{(leetcode 133)}
    \item{\textbf{举例}}:\\
As an example, consider the serialized graph ${0,1,2\sharp1,2\sharp2,2}$.\\
\\
The graph has a total of three nodes, and therefore contains three parts as separated by $\sharp$.\\
\\
First node is labeled as 0. Connect node 0 to both nodes 1 and 2.\\
Second node is labeled as 1. Connect node 1 to node 2.\\
Third node is labeled as 2. Connect node 2 to node 2 (itself), thus forming a self-cycle.\\
Visually, the graph looks like the following:\\
\includegraphics{graph/section3/clonegraph.eps}
    \item{\textbf{BFS+Hash}} : \fbox{时间复杂度O(E), 空间复杂度O(V)}
    \\采用BFS一个个的搜集图的顶点,然后新建顶点和原有顶点通过hash建立一一映射关系方便后来构造neighbors
    \begin{lstlisting}
UndirectedGraphNode *cloneGraph(UndirectedGraphNode *node) {
	if(!node)	return NULL;
	unordered_map<UndirectedGraphNode*, UndirectedGraphNode*>	table;
	unordered_set<UndirectedGraphNode*> visited;
	queue<UndirectedGraphNode*> q;
	q.push(node);
	UndirectedGraphNode *new_pos, *pos;
	while(!q.empty()){
		pos = q.front();
		q.pop();
		new_pos = new UndirectedGraphNode(pos->label);
		table[pos] = new_pos;
		for(int i = 0; i < pos->neighbors.size(); i++){
			if(table.count(pos->neighbors[i]) == 0){
				q.push(pos->neighbors[i]);
			}
		}
	}
	q.push(node);
	visited.insert(node);
	while(!q.empty()){
		pos = q.front();
		q.pop();
		for(int i = 0; i < pos->neighbors.size(); i++){
			table[pos]->neighbors.push_back(table[pos->neighbors[i]]);
			if(visited.count(pos->neighbors[i]) == 0){
				q.push(pos->neighbors[i]);
				visited.insert(pos->neighbors[i]);
			}
		}
	}
	return table[node];
}
    \end{lstlisting}
\end{description}
