    
\qquad分治算法的基本思想是将一个规模为N的问题分解为K个规模较小的子问题,这些子问题相互独立且与原问题性质相同。求出子问题的解,就可得到原问题的解. \\
\qquad分治法解题的一般步骤:\\
(1)分解,将要解决的问题划分成若干规模较小的同类问题; \\
(2)求解,当子问题划分得足够小时,用较简单的方法解决; \\
(3)合并,按原问题的要求,将子问题的解逐层合并构成原问题的解。\\
\\
\qquad当我们求解某些问题时,由于这些问题要处理的数据相当多,或求解过程相当复杂,使得直接求解法在时间上相当长,或者根本无法直接求出。对于这类问题,我们往往先把它分解成几个子问题,找到求出这几个子问题的解法后,再找到合适的方法,把它们组合成求整个问题的解法。如果这些子问题还较大,难以解决,可以再把它们分成几个更小的子问题,以此类推,直至可以直接求出解为止。这就是分治策略的基本思想。
